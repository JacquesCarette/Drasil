\chapter{Background}
\section{Software Artifacts}
\label{sec:sfs}
 -  (Reference a lot of Parnas)
 - Rational designs (how/why to fake)
\section{Software Reuse and Software Families}
\subsection{Software/Program Families}
  - Bring up GNU/Linux and different distros as examples of software families
    - (Raspbian v Raspbian lite) = Debian--, etc.
\subsection{Reuse and Reproducible Research}
  - Touch on reuse areas like reproducible research - Gentleman and Lang 2012

Being able to reproduce results, is fundamental to the idea of good science.
When it comes to software projects, there are often many undocumented
assumptions or modifications (including hacks) involved in the finished product.
This can make replication impossible without the help of the original author,
and in some cases reveal errors in the original author's
work~\cite{IonescuAndJansson2013}.

Reproducible research has been used to mean embedding executable code in
research papers to allow readers to reproduce the results
described~\cite{SchulteEtAl2012}.

Combining research reports with relevant code, data, etc.\ is not necessarily
easy, especially when dealing with the publication versions of an author's work.
As such, the idea of \emph{compendia} were
introduced~\cite{GentlemanAndLang2012} to provide a means of encapsulating the
full scope of the work. Compendia allow readers to see computational details, as
well as re-run computations performed by the author. Gentleman and Lang proposed
that compendia should be used for peer review and distribution of scientific
work~\cite{GentlemanAndLang2012}.

Currently, several tools have been developed for reproducible research
including, but not limited to, Sweave~\cite{Leisch2002},
SASweave~\cite{LenthEtAl2007}, Statweave~\cite{Lenth2009},
Scribble~\cite{FlattEtAl2009}, and Org-mode~\cite{SchulteEtAl2012}. The most
popular of those being Sweave~\cite{SchulteEtAl2012}. The aforementioned tools
maintain a focus on code and certain computational details. Sweave,
specifically, allows for embedding code into a document which is run as the
document is being typeset so that up to date results are always included.
However, Sweave (along with many other tools), still maintains a focus on
producing a single, linear document. It is my hope that Drasil will outperform
these existing tools due to its flexibility and its ability to create multiple
artifacts from a knowledge base.

\section{Literate Approaches to Software Development}

There have been several approaches attempting to combine development of program 
code with documentation.\ds{Something something...}

\subsection{Literate Programming}

Literate Programming (LP) is a method for writing software introduced by Knuth 
that focuses on explaining to a human what we want a computer to do rather than 
simply writing a set of instructions for the computer on how to perform the 
task~\cite{Knuth1984}.

Developing literate programs involves breaking algorithms down into
\emph{chunks}~\cite{JohnsonAndJohnson1997} or \emph{sections}~\cite{Knuth1984}
which are small and easily understandable. The chunks are ordered to follow a 
``psychological order''~\cite{PieterseKourieAndBoake2004} if
you will, that promotes understanding. They do not have to be written in the 
same order that a computer would read them. It should also be noted that in a 
literate program, the code and documentation are kept together in one source. 
To extract runnable code, a process known as \emph{tangle} must be performed on 
the source. A similar process known as \emph{weave} is used to extract and 
typeset the documentation.

There are many advantages to LP beyond understandability. As a program is
developed and updated, the documentation surrounding the source code is more 
likely to be updated simultaneously. It has been experimentally found that 
using LP ends up with more consistent documentation and 
code~\cite{ShumAndCook1993}. There are many downsides to having inconsistent 
documentation while developing or maintaining 
code~\cite{Kotula2000,Thimbleby1986}, while the benefits of consistent 
documentation are numerous~\cite{Hyman1990, Kotula2000}. Keeping the advantages 
and disadvantages of good documentation in mind we can see that more effective, 
maintainable code can be produced if properly using 
LP~\cite{PieterseKourieAndBoake2004}.

Regardless of the benefits of LP, it has not been very popular with 
developers~\cite{ShumAndCook1993}. However, there are
several successful examples of LP's use in SC. Two such literate programs that 
come to mind are VNODE-LP~\cite{Nedialkov2006} and ``Physically Based 
Rendering: From Theory to Implementation''~\cite{PharrAndHumphreys2004} a 
literate program and textbook on the subject matter. Shum and 
Cook~\cite{ShumAndCook1993} discuss the main issues behind LP's lack of 
popularity which can be summed up as dependency on a 
particular output language or text processor, and the lack of flexibility on 
what should be presented or suppressed in the output.

There are several other factors which contribute to LP's lack of popularity and 
slow adoption thus far. While LP allows a developer to write their code and its 
documentation simultaneously, that documentation is comprised of a single 
artifact which may not cover the same material as standard artifacts software 
engineers expect (see Section~\ref{sec:sfs} for more details). LP also does not 
simplify the development process: documentation and code are written as usual, 
and there is the additional effort of re-ordering the chunks. The LP 
development process has some benefits such as allowing developers to follow a 
more natural flow in development by writing chunks in whichever order they 
wish, keep the documentation and code updated simultaneously (in theory) 
because of their co-location, and automatically incorporate code chunks into 
the documentation to reduce some information duplication.

There have been many attempts to increase LP's popularity by focusing on 
changing the output language or removing the text processor dependency. Several
new tools such as CWeb (for the C language), DOC++ (for C++), noweb 
(programming language independent), and others were developed. Other tools such 
as javadoc (for Java) and Doxygen (for multiple languages) were also influenced 
by LP, but differ in that they are merely document extraction tools. They do 
not contain the chunking features which allow for re-ordering algorithms.

With new tools came new features including, but not limited to, phantom
abstracting~\cite{ShumAndCook1993}, a ``What You See Is What You Get'' (WYSIWYG)
editor~\cite{FritzsonGunnarssonAndJirstrand2002}, and even movement away from 
the ``one source'' idea~\cite{Simonis2003}.

While LP is still not mainstream~\cite{Ramsey1994}, these more robust 
tools helped drive the understanding behind what exactly LP tools must 
do. In certain domains LP is becoming more standardized, for 
example: Agda, Haskell, and R support LP to some extent, even though it is not 
yet common practice. R has good tool support, with the most popular being
Sweave~\cite{Leisch2002}, however it is designed to dynamically create
up-to-date reports or manuals by running embedded code as opposed to being used
as part of the software development process. 

\subsection{Literate Software}

A combination of LP and Box Structure~\cite{Mills1986} was proposed as a new
method called ``Literate Software Development''
(LSD)~\cite{AlMatiiAndBoujarwah2002}. Box structure can be summarized as the
idea of different views which are abstractions that communicate the same
information in different levels of detail, for different purposes. Box
structures consist of black box, state machine, and clear box structures. The
black box gives an external (user) view of the system and consists of stimuli
and responses; the state machine makes the state data of the system visible (it
defines the data stored between stimuli); and the clear box gives an internal
(designer's) view describing how data are processed, typically referring to
smaller black boxes~\cite{Mills1986}. These three structures can be nested as
many times as necessary to describe a system.

LSD was developed with the intent to overcome the disadvantages of both LP and
box structure. It was intended to overcome LP's inability to specify interfaces
between modules, the inability to decompose boxes and implement the design
created by box structures, as well as the lack of tools to support box
structure~\cite{Deck1996}.

The framework developed for LSD, ``WebBox'', expanded LP and box structures in a
variety of ways. It included new chunk types, the ability to refine chunks, the
ability to specify interfaces and communication between boxes, and the ability
to decompose boxes at any level. However, literate software (and LSD) remains
primarily code-focused with very little support for creating other software
artifacts, in much the same way as LP.

\section{Generative Programming}
 - ?
