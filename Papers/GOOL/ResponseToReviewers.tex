\documentclass[12pt]{article}

\usepackage{url}

\title{Response to Reviewers}
\author{Spencer Smith, Brooks MacLachlan, Jacques Carette}
\date{\today} 

\begin{document}

\maketitle

The thoughtful and detailed comments of the reviewers are very much appreciated.
Their comments have been used to revise and improve the paper, as described in
the sections below.  Each section corresponds to one of the three reviewers.  In
each case, the reviewers comment are reproduced in italics, followed by the
author's response.

\section{Reviewer A}

\textit{The overall approach that the authors propose in the paper is 
understandable to some extent, but still leaves me a question about how 
developers will use the DSL.
	In general, this kind of DSLs has two possible usages.  The first one is to 
	write an entire program in the DSL, and generate into one or more concrete 
	languages.  In this case, the programmability of the DSL (how it is easy to 
	write general programs in the DSL) matters a lot.  Due to many constraints, 
	it will not be as easy as writing programs in the target languages 
	directly.  But it this case, since you don't need to worry about the 
	readability of the generated code, I suppose that the authors are not 
	thinking about this usage.
	The second one is to write skeletons of a program in the DSL, generate code 
	in a concrete language, and then write the rest of the application logic in 
	the target language.  In this case, the readability of the generated code 
	matters.  However, this usage requires the developer plan everything ahead 
	of code generation.  Otherwise, when the developer finds to extend or 
	modify some part of the program (like adding a field) after code 
	generation, there is no chance to use the DSL-based code generation.
	I would like to see discussion on this point in the paper.}

Expanded discussion about use cases have been added, including our concrete use 
case with \textit{Drasil}.

\textit{Sections 3.1, 3.2 and (first half of) 4 are not well-structured to 
understand the contributions.  While it is great to describe the design and 
implementation in detail, it was not easy to understand which parts of the 
design/implementation are novel, and which are mere results of some decisions.  
It would be great if the authors can extract the novel parts out, and explain 
them at the beginning of each section.}

Added an indication of novelty in the syntax section, still need to do similar 
for section 4.

\textit{Figure 2, the Y-axis of the graph should be started from zero.  The 
number of common methods are relative as they depend on how the authors 
decomposed the system into methods.  Therefore, relative numbers of common 
methods would be more useful, like "compared to Python/C++, Java/C++ has 33\% 
more common methods" rather  than "...Java/C++ has 50 more common methods"}

\section{Reviewer B}

\textit{From the snippets, GOOL looks a bit cumbersome to use and is probably 
less readable than the generated code. This could be because of the embedded 
DSL aspect, and the need to re-name all the embedded operators like equality, 
and so on.}

Added discussion of how the syntax might be such that it is harder to write a 
program in GOOL than traditional languages, but why that is not a big concern 
due to our anticipated use cases.

\textit{(Incidentally, the font formatting used also spaces letters too much.)}

Updated lstlistings settings to use flexible columns, leading to less spacing 
between characters/words.

\textit{Because the language is actually quite large, I didn't get an insight 
as to what the essence of OO is. The paper does not discuss inheritance at all, 
and does not delve into semantics of OO in general.}

Added very brief mention of inheritance, can possibly do more.

\textit{I think an interesting avenue for future work would be to allow the 
high-level design patterns to be described within GOOL itself.}

The paper discusses some design patterns that are encoded as language features 
of GOOL, and says we plan on doing more. So I am not sure what this comment is 
referring to.

\textit{Minor:
	- missing spaces after C++ or C+:
	- l. 215 "C++doesn’t"
	- l. 218 "C++templates"
	- l. 230 "C\#and"}

\section{Reviewer C}

\textit{The main problem I have with the paper is that I do not understand its 
motivation: who
would use such a language and what for? The paper does not give a specific use 
case
that needs or benefits from GOOL, nor is there an evaluation section that 
demonstrates
somehow the benefits of GOOL. The paper hints at there being such a use case 
involving
ODEs, but does not share it.}

See above comment for reviewer A's first point.

\textit{I also did not find an evaluation section where the paper evaluates to 
what extent
	GOOL succeeds in its aims and can be used to write practical programs in.}

Added brief discussion of the real-world examples from \textit{Drasil} we use 
to test that GOOL indeed works.

\textit{It would probably make the paper itself more appealing if less space 
were spent on the basic
	language features (expressions and statements) and more details were given 
	of the more advanced aspects like design patterns.}

It will likely be difficult to add more detail due to the 6-page constraint, 
but in cutting the paper down to 6 pages I will prioritize preservation of the 
higher-level patterns over the low-level syntax.
\end{document}  
