%-----------------------------------------------------------------------------
%
%               Template for sigplanconf LaTeX Class
%
% Name:         sigplanconf-template.tex
%
% Purpose:      A template for sigplanconf.cls, which is a LaTeX 2e class
%               file for SIGPLAN conference proceedings.
%
% Guide:        Refer to "Author's Guide to the ACM SIGPLAN Class,"
%               sigplanconf-guide.pdf
%
% Author:       Paul C. Anagnostopoulos
%               Windfall Software
%               978 371-2316
%               paul@windfall.com
%
% Created:      15 February 2005
%
%-----------------------------------------------------------------------------


\documentclass[10pt, preprint]{sigplanconf}

% The following \documentclass options may be useful:

% preprint      Remove this option only once the paper is in final form.
% 10pt          To set in 10-point type instead of 9-point.
% 11pt          To set in 11-point type instead of 9-point.
% authoryear    To obtain author/year citation style instead of numeric.

\usepackage{amsmath}


\begin{document}

\special{papersize=8.5in,11in}
\setlength{\pdfpageheight}{\paperheight}
\setlength{\pdfpagewidth}{\paperwidth}

\conferenceinfo{CONF 'yy}{Month d--d, 20yy, City, ST, Country} 
\copyrightyear{20yy} 
\copyrightdata{978-1-nnnn-nnnn-n/yy/mm} 
\doi{nnnnnnn.nnnnnnn}

% Uncomment one of the following two, if you are not going for the 
% traditional copyright transfer agreement.

%\exclusivelicense                % ACM gets exclusive license to publish, 
                                  % you retain copyright

%\permissiontopublish             % ACM gets nonexclusive license to publish
                                  % (paid open-access papers, 
                                  % short abstracts)

\titlebanner{banner above paper title}        % These are ignored unless
\preprintfooter{short description of paper}   % 'preprint' option specified.

\title{Learning to Cook 'ware}
\subtitle{Subtitle Text, if any} %D Something about prog families?

\authorinfo{Name1}
           {Affiliation1}
           {Email1}
\authorinfo{Name2\and Name3}
           {Affiliation2/3}
           {Email2/3}

\maketitle

\begin{abstract}
This is where we will put the abstract of our paper. It will be super-fantastic and make all the reviewers think that this should not only be accepted, but most definitely published.
\end{abstract}

\category{CR-number}{subcategory}{third-level}

% general terms are not compulsory anymore, 
% you may leave them out
\terms
term1, term2

\keywords
Program families, generative programming, documentation, scientific computing, cooking %D ^_^

\section{Introduction} %D Context goes here?

Problem solving can be simple or difficult depending on the expertise of the person attempting to solve the problem.

**segue**

The first step in solving a problem is understanding it. In SE that understanding is often described through a problem statement and SRS (which also helps focus the possible solution).

**segue**

In scientific computing (SC) / on the forefront of scientific knowledge where problems are still not well enough understood it is often difficult to communicate one's understanding of the problem and plan out a full solution. For Software Engineers, this results in difficulty during early phases of develop (i.e. requirements gathering).

--refWilson Difficulties of gathering reqs for SC dev and why it shouldn't be done.

Disagree with Wilson

**segue**

There is a distinct divide between SE and other scientists.

--refSegal Software engineers fail to meet scientists' expectations

    **Go into a bit of depth on this (not too longwinded, we're pandering to SE here so make it seem like it's still the scientists fault i.e. vague problem descriptions / expectations, lack of known (in/out)put pairs, etc.)

--refSegal Scientists fail to meet SE expectations

    **Go into a bit of depth on this (can talk at length to puff up SE -> "They fail to give us necessary information and expect us to somehow still make it work" <- something like that but less antagonistic. Talk about reqs gathering and maintainability, traceability, etc.

**segue** into Agile development

--refSegal Agile development is commonplace

Documentation typically loses traceability / stops being maintained as a project continues onward. (source?). Essentially most documentation would then have to be re-written in the end phases (source?). Some would likely choose not to bother if it is not considered necessary / no budget / etc. (source? Common sense?), but writing the documents is still worthwhile, even if they need to "fake" the traditional SE dev process (Parnas and Clements)

--refParnas More on benefits of faking

Wouldn't it be great if documentation could be generated throughout the development process alongside the source code with trivial effort?

**Drop the mic**

It can be, and that is the aim of this project. Particularly dealing w/ families of scientific software (more on that in a bit).

**Move to new section on background? What should come next?**

\section{Central Ideas}
Lots of text.

More text.

Lots of text.

More text.

Lots of text.

More text.

Lots of text.

More text.

Lots of text.

More text.

Lots of text.

More text.

Lots of text.

More text.

Lots of text.

More text.

Lots of text.

More text.

Lots of text.

More text.

\section{Historical Approaches}
Lots of text.

More text.

Lots of text.

More text.


Lots of text.

More text.

Lots of text.

More text.

Lots of text.

More text.

Lots of text.

More text.

Lots of text.

More text.

Lots of text.

More text.

Lots of text.

More text.

Lots of text.

More text.

\section{Our Approach} %D ?
Lots of text.

More text.

Lots of text.

More text.




Lots of text.

More text.

Lots of text.

More text.

Lots of text.

\section{A family meal} %D Our example
Lots of text.

More text.

Lots of text.

More text.

Lots of text.

More text.

Lots of text.

More text.

Lots of text.

More text.

Lots of text.

More text.

Lots of text.

More text.

Lots of text.

More text.

Lots of text.

More text.

Lots of text.

More text.

Lots of text.

More text.

Lots of text.

More text.

Lots of text.

\appendix
\section{Appendix Title}

This is the text of the appendix, if you need one.

\acks

Acknowledgments, if needed.

% We recommend abbrvnat bibliography style.

\bibliographystyle{abbrvnat}

% The bibliography should be embedded for final submission.

\begin{thebibliography}{}
\softraggedright

\bibitem[Smith et~al.(2009)Smith, Jones]{smith02}
P. Q. Smith, and X. Y. Jones. ...reference text...

\end{thebibliography}


\end{document}

%                       Revision History
%                       -------- -------
%  Date         Person  Ver.    Change
%  ----         ------  ----    ------

%  2013.06.29   TU      0.1--4  comments on permission/copyright notices

