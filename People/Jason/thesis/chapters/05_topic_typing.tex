\chapter{Typing the Expression Language}
\label{chap:typedExpr}

Mathematical expressions are one of the most prominent components of any
abstracted concept from scientific software artifacts. With a pencil and paper,
our mistakes might go unnoticed, because they are never ``hard'' validated by
any machine, but ``soft'' validated by us and other readers. In other words,
there is no clear \textit{validity assertion} when we traditionally write
expressions on paper. While \Cref{chap:modelkinds} focused on understanding how
mathematical expressions could be dissected and transformed into ``code,'' it
neglected to discuss which expressions it could even begin to dissect and
transform \textemdash{} that is to say, which expressions are ``valid?'' The
objective of this chapter is to create a system of type rules, and enforce
well-formedness/typedness\footnote{Please note that I will be using
      ``well-formed'' and ``well-typed'' interchangeably.} through them, for Drasils
mathematical expression languages.

\section{Recap of Drasils Math-related Expression Languages}

To recap, at this point, we have three (3) relevant and used ``mathematical
expression'' languages.

\subsection{One for \textquotedblleft{}Simple\textquotedblright{} Mathematics}

\Expr{} is a mathematical expression languages whose vocabulary is intended to
always have a definite value. In other words, with little to no extra work on
your end, you should be able to directly input these expressions in your
standard calculator (perhaps with a bit of work to handle vectors, functions,
etc.) to evaluate them.

\subsection{One For \textquotedblleft{}Code\textquotedblright{}}

\CodeExpr{} is a heavily mathematics-focused expression language with a few
extra features over \Expr{} for \acs{gool}/``code.'' The vocabulary should be
nearly directly usable in \acs{gool} for outputting to general-purpose
object-oriented programming languages. \CodeExpr{} is a superset of \Expr{}:

\[\Expr{} \subseteq{} \CodeExpr{}\]

\subsection{And One For General Mathematics}

\ModelExpr{} is the classical mathematics we know and love. It contains nearly
everything we know (up to what we've encoded thus far) and is intended to be a
descriptive language, with no particular restrictions on its terms (other than
that they should at least be describable on pencil and paper too). \ModelExpr{}
is also a superset of \Expr{}:

\[\Expr{} \subseteq{} \ModelExpr{}\]

However, \ModelExpr{}s terms are unlikely to appear in \CodeExpr{} due to their
indescribable nature in computable \acs{oo} ``code.''

As of right now, these languages have proven themselves to be effective
encodings, weakly proven through Drasils case studies being able to produce
working software artifacts. However, they are not without issue. Notably, at the
moment, Drasil does not have any readily-available type information about their
constructions. This lack of type information hampers the ``reliability'' aspect
of the code generator because the generator is unable to restrict its output
artifacts to those which are directly usable. In order for the generated
artifacts to be directly usable, they must be \textit{type safe} programs. In
other words, we need to make sure that the interpreters and compilers used on
the generated artifacts never get confused/stuck on the generated expressions.
To obtain this, we must add type information, and use it.

\section{Type Safety}

\intodo{Continue writing here!}

\intodo{Discuss what ``type safety'' is. Discuss: types, judgments, typing
      rules, well-typedness, preservation, and progress.}

\intodo{Discuss how we can add type information to Drasils expression languages,
      and the pros/cons of the solutions. For example, should type enforcement
      done at the construction-level or post-facto processed?}

\intodo{Discuss the properties of a good solution.}

\intodo{Re-write the typing rules without the Haskell code references (keep it
      mathematical/theoretical).}

\intodo{Re-write the typing rules with the Haskell code style.}

\intodo{Discuss our proposed and performed solution.}

\begin{enumerate}

      \item Ok, so we have these expression languages. They're great!

      \item However, they admit invalid expressions.

      \item Invalid expressions are bad because:

            \begin{enumerate}

                  \item they can't be properly dissected, for us to learn from
                        them.

                  \item they will cause problems later, in the artifacts we
                        generate.

                  \item they make no real discernible ``sense.'' They might make
                        sense to anyone in particular, but, broadly, they will
                        be gibberish, because they do not obey a globally agreed
                        upon rule set.

                  \item at the moment, to ensure that our generated artifacts
                        are directly usable in compilation and usage, we need to
                        manually ensure that they ``type-check.'' However,
                        Drasil has no clear understanding of what ``type-check''
                        means! For a few expressions, this manual methodology
                        might be ``okay,'' but it surely does not scale well
                        when add expression generation, nor human error through
                        scale in expression creation/usage.

            \end{enumerate}

      \item Ok, so what information are we missing? What do we need to ``teach''
            Drasil about in order for us to ensure that all of our expressions
            are well-formed (whatever that might mean)?

            \begin{enumerate}

                  \item First, we need to discuss: what does ``well-formedness''
                        mean?

                  \item What are ``typing rules?'', and how can we create a set
                        of typing rules?

                  \item What are our needed typing judgments?

            \end{enumerate}

      \item Now that we know what is needed of the expression language, how do
            we want to implement it in Drasil?

\end{enumerate}

\section{Background: Problem}

\begin{itemize}

      \item Writing invalid expressions is possible.
            \begin{itemize}

                  \item On paper, writing invalid expressions is as easy as
                        making a typo, but complete gibberish can also be
                        written. We rely on manually checking expressions to
                        ensure that they are ``correct''. As the number of
                        expressions grows, the cost of manually checking grows
                        rapidly, and changes result in costly setbacks. Imagine
                        systems with 10, 100, and 1000+ expressions, the cost
                        grows rapidly.

                  \item With computers, we can systematically check the validity
                        of expressions by imposing various kinds of
                        restrictions.

            \end{itemize}

      \item Mentally tracking expression creations to ensure they follow the
            implicit rules of the expression language is too difficult, and
            leads to mental strain.

      \item Compiling to ``lower languages'' requires special type checking
            before compiling to them. For example, the Swift code generator has
            to ensure that there are no ambiguously typed numerals as the types
            of numerics are not overloaded in Swift.

      \item Dynamically checking for invalid expression states is possible, but
            difficult and would result in increasingly difficult term tracking
            as terms in the expression language grow/are added.

      \item In general, being able to express invalid expressions causes large
            burden and mental overhead.

\end{itemize}


\section{Requirements \& properties of a good solution}

\begin{itemize}

      \item Invalid expressions should not be representable in the various
            expression languages (i.e., the expression types should strictly
            indicate valid expression constructions), without loss of
            generality.

      \item Invalid expression formation attempts should be statically found and
            reported by the compiler, at compile-time. This will move the
            previously runtime errors to compile-time.

      \item Invalid expression cases should not need to be considered when
            working (e.g., case-ing) with expressions.

      \item ``Safety = Preservation + Progress'' (\cite{Harper2016}, Ch.6)

\end{itemize}

\section{Solution}

\begin{itemize}

      \item Use TTF encodings of the smart constructors to lessen the cognitive
            load of handling at least 3 different expression languages.

      \item Statically type all 3 variants of Expr through GADTs.

\end{itemize}

\subsection{Syntax}

\subsubsection{Current}

An idealized version of the current syntax.

\startSyntaxTable
    \newsyntaxRow{Type}{\tau}{Integer}{\bb{Z}}{Integer numbers}
    \syntaxRow{Real}{\bb{R}}{Real numbers}
    \syntaxRow{String}{String}{Text}
    \syntaxRow{Bool}{\bb{B}}{Truth values (true/false)}
    \syntaxRow{Vector($\tau$)}{[\tau]}{Vectors}
     \\

    \newsyntaxRow{Literal}{l}{Integer[$n$]}{n}{Integer number}
    \syntaxRow{Real[$r$]}{r}{Real number}
    \syntaxRow{String[$s$]}{``s''}{Text}
    \syntaxRow{Bool[$b$]}{b}{Boolean value}
    \syntaxRow{Vector($l_1...l_n$)}{<l_1, ..., l_n>}{Vectors}
     \\

    \newsyntaxRow{UnaryOp}{u}{Not}{\lnot \_}{Logical negation}
    \syntaxRow{Neg}{- \_}{Numeric negation}
    \syntaxRow{...}{}{}
     \\

    \newsyntaxRow{BinaryOp}{\oplus}{Sub}{\_ - \_}{Subtraction}
    \syntaxRow{Pow}{\_^\_}{Powers}
    \syntaxRow{...}{}{}
     \\
    
    \newsyntaxRow{AssocBinOp}{\otimes}{Add}{\_ + \_}{Addition}
    \syntaxRow{Mul}{\_ \times \_}{Multiplication}
    \syntaxRow{...}{}{}
     \\

    \newsyntaxRow{Expr}{e}{Literal($l$)}{l}{Literal values}
    \syntaxRow{Vector($e_1...e_n$)}{<e_1, ..., e_n>}{Vectors}
    % TODO: Symbols : \syntaxRow{C(u)}
    % TODO: Function "Calls"
    \syntaxRow{UnaryOp($u$,$e$)}{u(e)}{Unary operations}
    \syntaxRow{BinaryOp($\oplus$,$e_1$,$e_2$)}{e_1 \oplus e_2}{Binary operations}
    \syntaxRow{AssocOp($\otimes$, $e_1...e_n$)}{e_1 \otimes ... \otimes e_n}{Associative binary operations}
    \syntaxRow{Case($e_{1c}e_{1e}...e_{nc}e_{ne}$)}{if\ e_{1c}\ then\ e_{2e}\ elif\ e_{2c}\ ...}{If-then-else-if-then-else (Switch-like statements)}
    % TODO: BigBinOp
    % TODO: RealI / Is In Interval

\closeSyntaxTable


\subsection{Typing Rules}

\subsubsection{Literal}

\begin{enumerate}

    \item Integers:
        \[ \infer{\ofTy{Integer[i]}{Literal Integer}}{\ofTy{i}{Integer}} \]

    \item Strings (Text):
        \[ \infer{\ofTy{Str[s]}{Literal String}}{\ofTy{s}{String}} \]

    \item Real numbers:
        \[ \infer{\ofTy{Dbl[d]}{Literal Real}}{\ofTy{d}{Double}} \]

    \item Whole numbered reals (\(\bb{Z} \subset{} \bb{R}\)): 
        \[ \infer{\ofTy{ExactDbl[d]}{Literal Real}}{\ofTy{d}{Integer}} \]

    \item Percentages:
        \[ \infer{\ofTy{Perc[n,d]}{Literal Real}}{\ofTy{n}{Integer} & \ofTy{d}{Integer}} \]

\end{enumerate}


\subsubsection{Miscellaneous}


\begin{enumerate}

    \item Completeness:
        \newrule{}
            {\ofTy{Complete[]}{Completeness}}
        
        \newrule{}
            {\ofTy{Incomplete[]}{Completeness}}

    \item AssocOp:
        \begin{enumerate}
            \item Numerics:
                \newrule{\numericTy{x}}
                    {\ofTy{Add[]}{AssocOp x}}
        
                \newrule{\numericTy{x}}
                    {\ofTy{Mul[]}{AssocOp x}}
    
            \item Bool:
                \newrule{}
                    {\ofTy{And[]}{AssocOp Bool}}
        
                \newrule{}
                    {\ofTy{Or[]}{AssocOp Bool}}
        \end{enumerate}

    \item UnaryOp:
        \begin{enumerate}
            \item Numerics:
                \todo{Discuss Numerics-($\Tau$) and Numerics-With-Negation-($\Tau$)}
                \newrule{\negNumericTy{x}}
                    {\ofTy{Neg[]}{UnaryOp x x}}

                \newrule{\negNumericTy{x}}
                    {\ofTy{Abs[]}{UnaryOp x x}}
                
                \newrule{\numericTy{x}}
                    {\ofTy{Exp[]}{UnaryOp x Real}}
                
                For Log, Ln, Sin, Cos, Tan, Sec, Csc, Cot, Arcsin, Arccos, Arctan, and Sqrt, please use the following template, replacing ``$\$TRG$'' with the desired operator:
                \newlblrule{}
                    {\ofTy{\$TRG[]}{UnaryOp Real Real}}{eqn:unOpTemplate}

                \newrule{}
                    {\ofTy{RtoI[]}{UnaryOp Real Integer}}
                
                \newrule{}
                    {\ofTy{ItoR[]}{UnaryOp Integer Real}}
            
                \newrule{}
                    {\ofTy{Floor[]}{UnaryOp Real Integer}}

                \newrule{}
                    {\ofTy{Ceil[]}{UnaryOp Real Integer}}
                
                \newrule{}
                    {\ofTy{Round[]}{UnaryOp Real Integer}}
                
                \newrule{}
                    {\ofTy{Trunc[]}{UnaryOp Real Integer}}
                
            \item Vectors:
                \newrule{\negNumericTy{x}}
                    {\ofTy{NegV[]}{UnaryOp [x] [x]}}

                \newrule{\numericTy{x}}
                    {\ofTy{Norm[]}{UnaryOp [x] Real}}

                \newrule{\ty{x}}
                    {\ofTy{Dim[]}{UnaryOp [x] Integer}}

            \item Booleans:
                \newrule{}
                    {\ofTy{Not[]}{UnaryOp Bool Bool}}

        \end{enumerate}

    \item BinaryOp:
        \begin{enumerate}
            \item Arithmetic: % TODO: Should we have FracI and FracR? 
                \newrule{}
                    {\ofTy{FracR[]}{BinaryOp Real Real Real}}

            \item Bool:
                \newrule{}
                    {\ofTy{Impl[]}{BinaryOp Bool Bool Bool}}
                
                \newrule{}
                    {\ofTy{Iff[]}{BinaryOp Bool Bool Bool}}
                
            \item Equality:
                \newrule{\ty{x}}
                    {\ofTy{Eq[]}{BinaryOp x x Bool}}
        
                \newrule{\ty{x}}
                    {\ofTy{NEq[]}{BinaryOp x x Bool}}
            
            \item Ordering:
                \newrule{\numericTy{x}}
                    {\ofTy{Lt[]}{BinaryOp x x Bool}}
        
                \newrule{\numericTy{x}}
                    {\ofTy{Gt[]}{BinaryOp x x Bool}}
            
                \newrule{\numericTy{x}}
                    {\ofTy{LEq[]}{BinaryOp x x Bool}}
        
                \newrule{\numericTy{x}}
                    {\ofTy{GEq[]}{BinaryOp x x Bool}}
            
            \item Indexing: % TODO: Everything related to vectors is up for debate, we can redesign them however we see fit. It shouldn't add much complexity.
                \newrule{\ty{x}}
                    {\ofTy{Index[]}{BinaryOp [x] Integer x}}
            
            \item Vectors: \todo{discuss vectors in general}
                \newrule{\numericTy{x}}
                    {\ofTy{Cross[]}{BinaryOp [x] [x] [x]}}
        
                \newrule{\numericTy{x}}
                    {\ofTy{Dot[]}{BinaryOp [x] [x] x}}
                
                \newrule{\numericTy{x}}
                    {\ofTy{Scale[]}{BinaryOp [x] x [x]}}

        \end{enumerate}
    
    \item RTopology:
        \newrule{}
            {\ofTy{Discrete[]}{RTopology}}

        \newrule{}
            {\ofTy{Continuous[]}{RTopology}}
    
    \item DomainDesc: % TODO: Why does the topology appear as a type constructor argument, and type signature argument?
        \newrule{\ofTy{top}{$\tau_1$} & \ofTy{bot}{$\tau_2$} & \ofTy{s}{Symbol} & \ofTy{rtop}{RTopology}}
            {\ofTy{BoundedDD[s, rtop, top, bot]}{DomainDesc Discrete $\tau_1$ $\tau_2$}}

        \newrule{\ty{topT} & \ty{botT} & \ofTy{s}{Symbol} & \ofTy{rtop}{RTopology}}
            {\ofTy{AllDD[s, rtop]}{DomainDesc Continuous topT botT}}

    \item Inclusive:
        \newrule{}
            {\ofTy{Inc[]}{Inclusive}}

        \newrule{}
            {\ofTy{Exc[]}{Inclusive}}

    \item RealInterval:
        \newrule{\ty{a} & \ty{b} & \ofTy{top}{(Inclusive, a)} & \ofTy{bot}{(Inclusive, b)}}
            {\ofTy{Bounded[top, bot]}{RealInterval a b}}

        \newrule{\ty{a} & \ty{b} & \ofTy{top}{(Inclusive, a)}}
            {\ofTy{UpTo[top]}{RealInterval a b}}

        \newrule{\ty{a} & \ty{b} & \ofTy{bot}{(Inclusive, b)}}
            {\ofTy{UpFrom[bot]}{RealInterval a b}}

\end{enumerate}


\subsubsection{Expr}

\begin{haskell}{Expression Language}{curExpr}{https://github.com/JacquesCarette/Drasil/blob/dc3674274edb00b1ae0d63e04ba03729e1dbc6f9/code/drasil-lang/lib/Language/Drasil/Expr/Lang.hs\#L81-L135}
-- | Expression language where all terms are supposed to be 'well understood'
--   (i.e., have a definite meaning). Right now, this coincides with
--   "having a definite value", but should not be restricted to that.
data Expr where
  -- | Brings a literal into the expression language.
  Lit :: Literal -> Expr
  -- | Takes an associative arithmetic operator with a list of expressions.
  AssocA   :: AssocArithOper -> [Expr] -> Expr
  -- | Takes an associative boolean operator with a list of expressions.
  AssocB   :: AssocBoolOper  -> [Expr] -> Expr
  -- | C stands for "Chunk", for referring to a chunk in an expression.
  --   Implicitly assumes that the chunk has a symbol.
  C        :: UID -> Expr
  -- | A function call accepts a list of parameters and a list of named parameters.
  --   For example
  --
  --   * F(x) is (FCall F [x] []).
  --   * F(x,y) would be (FCall F [x,y]).
  --   * F(x,n=y) would be (FCall F [x] [(n,y)]).
  FCall    :: UID -> [Expr] -> [(UID, Expr)] -> Expr
  -- | For multi-case expressions, each pair represents one case.
  Case     :: Completeness -> [(Expr, Relation)] -> Expr
  -- | Represents a matrix of expressions.
  Matrix   :: [[Expr]] -> Expr
  -- | Unary operation for most functions (eg. sin, cos, log, etc.).
  UnaryOp       :: UFunc -> Expr -> Expr
  -- | Unary operation for @Bool -> Bool@ operations.
  UnaryOpB      :: UFuncB -> Expr -> Expr
  -- | Unary operation for @Vector -> Vector@ operations.
  UnaryOpVV     :: UFuncVV -> Expr -> Expr
  -- | Unary operation for @Vector -> Number@ operations.
  UnaryOpVN     :: UFuncVN -> Expr -> Expr
  -- | Binary operator for arithmetic between expressions (fractional, power, and subtraction).
  ArithBinaryOp :: ArithBinOp -> Expr -> Expr -> Expr
  -- | Binary operator for boolean operators (implies, iff).
  BoolBinaryOp  :: BoolBinOp -> Expr -> Expr -> Expr
  -- | Binary operator for equality between expressions.
  EqBinaryOp    :: EqBinOp -> Expr -> Expr -> Expr
  -- | Binary operator for indexing two expressions.
  LABinaryOp    :: LABinOp -> Expr -> Expr -> Expr
  -- | Binary operator for ordering expressions (less than, greater than, etc.).
  OrdBinaryOp   :: OrdBinOp -> Expr -> Expr -> Expr
  -- | Binary operator for @Vector x Vector -> Vector@ operations (cross product).
  VVVBinaryOp   :: VVVBinOp -> Expr -> Expr -> Expr
  -- | Binary operator for @Vector x Vector -> Number@ operations (dot product).
  VVNBinaryOp   :: VVNBinOp -> Expr -> Expr -> Expr
  -- | Operators are generalized arithmetic operators over a 'DomainDesc'
  --   of an 'Expr'.  Could be called BigOp.
  --   ex: Summation is represented via 'Add' over a discrete domain.
  Operator :: AssocArithOper -> DiscreteDomainDesc Expr Expr -> Expr -> Expr
  -- | A different kind of 'IsIn'. A 'UID' is an element of an interval.
  RealI    :: UID -> RealInterval Expr Expr -> Expr
\end{haskell}


\subsubsection{ModelExpr}

\begin{haskell}{ModelExpr Language}{curModelExpr}{https://github.com/JacquesCarette/Drasil/blob/ab9e091dabd81685ddef86b0d218582c9f75cb20/code/drasil-lang/lib/Language/Drasil/ModelExpr/Lang.hs\#L82-L151}
-- | Expression language where all terms are supposed to have a meaning, but
--   that meaning may not be that of a definite value. For example,
--   specification expressions, especially with quantifiers, belong here.
data ModelExpr where
  -- | Brings a literal into the expression language.
  Lit       :: Literal -> ModelExpr
  
  -- | Introduce Space values into the expression language.
  Spc       :: Space -> ModelExpr
  
  -- | Takes an associative arithmetic operator with a list of expressions.
  AssocA    :: AssocArithOper -> [ModelExpr] -> ModelExpr
  -- | Takes an associative boolean operator with a list of expressions.
  AssocB    :: AssocBoolOper  -> [ModelExpr] -> ModelExpr
  -- | Derivative syntax is:
  --   Type ('Part'ial or 'Total') -> principal part of change -> with respect to
  --   For example: Deriv Part y x1 would be (dy/dx1).
  Deriv     :: Integer -> DerivType -> ModelExpr -> UID -> ModelExpr
  -- | C stands for "Chunk", for referring to a chunk in an expression.
  --   Implicitly assumes that the chunk has a symbol.
  C         :: UID -> ModelExpr
  -- | A function call accepts a list of parameters and a list of named parameters.
  --   For example
  --
  --   * F(x) is (FCall F [x] []).
  --   * F(x,y) would be (FCall F [x,y]).
  --   * F(x,n=y) would be (FCall F [x] [(n,y)]).
  FCall     :: UID -> [ModelExpr] -> [(UID, ModelExpr)] -> ModelExpr
  -- | For multi-case expressions, each pair represents one case.
  Case      :: Completeness -> [(ModelExpr, ModelExpr)] -> ModelExpr
  -- | Represents a matrix of expressions.
  Matrix    :: [[ModelExpr]] -> ModelExpr
  
  -- | Unary operation for most functions (eg. sin, cos, log, etc.).
  UnaryOp       :: UFunc -> ModelExpr -> ModelExpr
  -- | Unary operation for @Bool -> Bool@ operations.
  UnaryOpB      :: UFuncB -> ModelExpr -> ModelExpr
  -- | Unary operation for @Vector -> Vector@ operations.
  UnaryOpVV     :: UFuncVV -> ModelExpr -> ModelExpr
  -- | Unary operation for @Vector -> Number@ operations.
  UnaryOpVN     :: UFuncVN -> ModelExpr -> ModelExpr
  
  
  -- | Binary operator for arithmetic between expressions (fractional, power, and subtraction).
  ArithBinaryOp :: ArithBinOp -> ModelExpr -> ModelExpr -> ModelExpr
  -- | Binary operator for boolean operators (implies, iff).
  BoolBinaryOp  :: BoolBinOp -> ModelExpr -> ModelExpr -> ModelExpr
  -- | Binary operator for equality between expressions.
  EqBinaryOp    :: EqBinOp -> ModelExpr -> ModelExpr -> ModelExpr
  -- | Binary operator for indexing two expressions.
  LABinaryOp    :: LABinOp -> ModelExpr -> ModelExpr -> ModelExpr
  -- | Binary operator for ordering expressions (less than, greater than, etc.).
  OrdBinaryOp   :: OrdBinOp -> ModelExpr -> ModelExpr -> ModelExpr
  -- | Space-related binary operations.
  SpaceBinaryOp :: SpaceBinOp -> ModelExpr -> ModelExpr -> ModelExpr
  -- | Statement-related binary operations.
  StatBinaryOp  :: StatBinOp -> ModelExpr -> ModelExpr -> ModelExpr
  -- | Binary operator for @Vector x Vector -> Vector@ operations (cross product).
  VVVBinaryOp   :: VVVBinOp -> ModelExpr -> ModelExpr -> ModelExpr
  -- | Binary operator for @Vector x Vector -> Number@ operations (dot product).
  VVNBinaryOp   :: VVNBinOp -> ModelExpr -> ModelExpr -> ModelExpr
  
  
  -- | Operators are generalized arithmetic operators over a 'DomainDesc'
  --   of an 'Expr'.  Could be called BigOp.
  --   ex: Summation is represented via 'Add' over a discrete domain.
  Operator :: AssocArithOper -> DomainDesc t ModelExpr ModelExpr -> ModelExpr -> ModelExpr
  -- | A different kind of 'IsIn'. A 'UID' is an element of an interval.
  RealI    :: UID -> RealInterval ModelExpr ModelExpr -> ModelExpr
  
  -- | Universal quantification
  ForAll   :: UID -> Space -> ModelExpr -> ModelExpr
\end{haskell}


\subsubsection{CodeExpr}

\begin{enumerate}

    \item \[ \infer{A}{B & C} \]

\end{enumerate}

