\chapter{Conclusion}
\label{c:conclusion}

This thesis set out to address the persistent challenges of consistency,
traceability, and maintainability in scientific software and documentation.
The introduction established the need for a principled, automated approach to
knowledge capture and artifact generation, motivating the development and
evaluation of Drasil.

The results chapter provided concrete evidence that Drasil's single-source
architecture enforces consistency by construction, improves traceability
through explicit provenance, and enhances reproducibility via deterministic
artifact generation. Case studies demonstrated that local changes to domain
knowledge propagate automatically across all generated outputs, reducing
manual effort and the risk of inconsistencies. The results also highlighted
practical benefits such as the automatic maintenance of supporting materials
and the ease of creating new software family members by reusing and extending
the knowledge base.

The discussion chapter interpreted these findings, emphasizing that Drasil's
approach shifts the burden of correctness to the knowledge base, clarifies
implicit assumptions, and enables rapid, reliable adaptation to new
requirements. It also acknowledged tradeoffs, such as the increased upfront
modeling and onboarding effort, and identified the need for improved tooling
and formal validation.

The primary contribution of this thesis is the demonstration that a
centralized, recipe-driven knowledge base can systematically address many of
the recurring problems in scientific software engineering. By explicitly
capturing domain knowledge and automating artifact generation, Drasil
provides a scalable, extensible foundation for reliable, maintainable, and
reproducible scientific software. The lessons learned and directions for
future work outlined here position Drasil as both a practical tool and a
platform for further research in automated knowledge capture and software
generation.