\chapter{Future Work}
\label{chap:futureWork}

\intodo{Rewrite point form notes in Future Work chapter.}

While this work does contribute to the Drasil research project and the ideology
underpinning it, there are still many questions and concerns left unanswered.

\section{Chunks}

While \Cref{chap:storingChunks} answers questions regarding \textit{storing}
more kinds of chunks and has created a basic set of constraints that all chunks
must satisfy (\HasUID{} and \HasChunkRefs{}), we've embedded their solutions in
Haskell code rather than Drasil itself. In order for solutions to be included
``in Drasil itself,'' we need to encode it such that Drasil allows users to
interact with them, dynamically, without Template Haskell \todo{Cite something
about TemplateHaskell} or other things ``deep'' in Haskell. Similarly, we have
mysterious \Typeable{} usage left unknown, which should eventually be replaced
with something well-understood to us. In the future, we hope to improve chunk
building fundamentally, perhaps by using a \acs{dsl} instead of leaning on
built-in Haskell functionality. This would allow us to better analyze Drasil and
its projects.

Furthermore, while we've added requirements that \acsp{uid} be unique, we
haven't discussed how \UID{}s should be built (automatic or manual, and how?),
nor ensured that \UID{} references ultimately link to the chunks they were
intended to link to. We desire for them to fully be \textit{rigid designators}
\todo{Cite something about rigid designators.}. Perhaps these questions will
naturally resolve themselves when we try to switch to using a Drasil \acs{dsl}
to build chunks.

\intodo{Continue here!}

\section{Encodings}

\begin{itemize}

      \item With the above new definition of ``chunks'', they still remain a
            very vague idea, and still \textit{deeply embedded} (a place to
            recognize an encoding might be appropriate!) in Haskell.

      \item What are the kinds of chunks that can exist? What can be in a chunk,
            and what are we missing from the existing list of chunks?

      \item The problem with that is that we lose a lot of information by
            writing Haskell, and leaving the knowledge in the form of Haskell.

      \item We need to de-embed all chunks so that we can obtain a tangible
            understanding of them.

      \item Through de-embedding the chunks, we will also be forced to de-embed
            everything with it. This is including the ways in which we transform
            and generate ``new''-ish knowledge (not necessarily new types/kinds
            of knowledge, but new instances of types).

\end{itemize}

\section{Chunks}

\begin{enumerate}

      \item What is a ``chunk''?
            \begin{itemize}

                  \item A ``chunk instance'' is a single \textit{term} of a
                        language.
                        
                  \item A ``chunk type'' is a language itself.
                  
            \end{itemize}

      \item What is a ``transformer''?
            \begin{itemize}

                  \item A ``transformer'' is a conversion of a term written in a
                        language into another term, potentially in another
                        language.
                        
                  \item Transformers rely on a well-understood dissection of
                        knowledge (contained in a chunk type/language) in order
                        transform it (potentially with other terms/information
                        as well) into another term.
                        
            \end{itemize}

\end{enumerate}


\begin{itemize}

      \item This document will contain knowledge regarding the Expression language
            that is shown in Haskell code, but not quite in our encodings. To
            further improve Drasil, one of the next ``obvious'' steps is to
            transcribe the knowledge involved with writing any language down as
            well. An excruciating amount of knowledge is everywhere.

      \item The unit and dimension related to numbers is another project on its
            own. It will need to be added to calculate the units of operations and
            ensure that representations are appropriate for their (precision vs
            accuracy as Dr. Smith mentioned).

\end{itemize}
