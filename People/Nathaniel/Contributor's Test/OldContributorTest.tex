\documentclass[12pt,fleqn]{examtst}
\usepackage{graphicx}
\usepackage{amssymb}
\usepackage{amsmath}
\usepackage{listings}
\usepackage{multirow}
\usepackage{multicol}
\usepackage{hhline}
\usepackage{booktabs}

\begin{document}

\newcommand{\soln}{y} %y for yes and n for no

\lstset{language=c, basicstyle=\ttfamily, breaklines=true,
  showspaces=false, showstringspaces=false, breakatwhitespace=true, texcl=true,
  escapeinside={\%*}{*)}}

\newcommand{\codeit}[1]{\texttt{\textit{#1}}}

\begin{center}
  {\large \bf Drasil Future Contributors' Test}\\[1ex]
  {\large \bf McMaster University}\\[1ex]
  {\large Faculty of Engineering, Department of Computing and Software}
\end{center}

\medskip

\noindent
Future Drasil Contributors, \textbf{Version 1}  \hfill Dr.~S.~Smith \\
DURATION OF TEST: 1.5 hours - 2.0 hours (recommended) \\
MCMASTER UNIVERSITY TEST \hfill

\medskip

\noindent
\rule[3 mm]{\textwidth}{0.5mm}

\begin{minipage}[t]{1.0\textwidth}
\textbf{Please CLEARLY print}:\\[2mm]

NAME:\\[1ex]

\newsavebox{\bb}\newsavebox{\bbb}
\sbox{\bb}{\framebox[1cm]{\rule{0mm}{7mm}}}
\sbox{\bbb}{\usebox{\bb}\usebox{\bb}\usebox{\bb}\usebox{\bb}\usebox{\bb}\usebox{\bb}\usebox{\bb}\usebox{\bb}\usebox{\bb}}

\rule[3 mm]{\textwidth}{0.5mm}

This test paper includes \noofpages pages and 22 % VARIABILITY
questions. You are responsible for ensuring that your copy of the 
test paper is complete. Bring any discrepancy to the attention of 
the person administering this test.\\

\end{minipage}\\

\hspace{14cm}
\begin{minipage}[t]{0.2\textwidth}
\newcommand{\markheight}{\rule[-2mm]{0 mm}{7 mm}}
\begin{tabular}[t]{|c|p{1.5 cm}|r|}
\hline
1--17 & \markheight & 17\\
\hline
18--22 & \markheight & 10\\

\hline
Total & \markheight & 27 \\
\hline

\end{tabular}
\end{minipage}

\examheader{Drasil \ifthenelse{\equal{\soln}{y}} {\hfill QUESTIONS} }

\renewcommand{\labelenumi}{\Alph{enumi}.}

%%%%%%%%%%%%%%%%%%%%%%%%%%%%%%%%%%

\noindent
\begin{minipage}{\textwidth}

\question{1 mark}
What command is used to create a local version of a remote repository?

\begin{enumerate}
    \item \lstinline{git copy <link to repo>}
    \item \lstinline{git clone <link to repo>}
    \item \lstinline{clone <link to repo>}
    \item \lstinline{copy <link to repo>}
\end{enumerate}
\answer{Explain your answer here.}{0cm}{7cm}
\rule{0cm}{1cm}

\question{1 mark}
The \lstinline{git status} command displays:

\begin{enumerate}
    \item verified status of all repo files
    \item list of last 5 commits to the repo
    \item paths with differences between current state of repo and last commit
    \item changes made to files after last commit
\end{enumerate}
\answer{Explain your answer here.}{0cm}{7cm}
\rule{0cm}{1cm}

\question{1 mark}
If you are a Windows OS user working with Git Bash App, always run the command each time you open an instance of git bash:

\begin{enumerate}
    \item \lstinline{chcp.com 65001} 
    \item \lstinline{set encoding=utf-8}
    \item \lstinline{chcp 65001}
    \item \lstinline{A. or C.}
\end{enumerate}
\answer{Explain your answer here.}{0cm}{7cm}
\rule{0cm}{1cm}

\end{minipage}

%%%%%%%%%%%%%%%%%%%%%%%%%%%%%%%%%%

\noindent
\begin{minipage}{\textwidth}

\question{1 mark}
What is the correct order of the following \lstinline{git} commands?

\begin{enumerate}
    \item \lstinline{git push, git add, git clone, git commit}
    \item \lstinline{git add, git clone, git push, git commit}
    \item \lstinline{git clone, git add, git push, git commit}
    \item \lstinline{git clone, git add, git commit, git push}
\end{enumerate}
\answer{Explain your answer here.}{0cm}{7cm}
\rule{0cm}{1cm}

\question{1 mark}
Which of the below phrases can you use to link a relevant issue to a pull request on GitHub, without closing the issue once the PR is merged? (\#HASH is the issue hash \#)

\begin{enumerate}
    \item closes \#HASH
    \item contributes to \#HASH
\end{enumerate}
\answer{Explain your answer here.}{0cm}{7cm}
\rule{0cm}{1cm}

\question{1 mark}
The preferred coding style describes how lines should not be more than \_\_\_ characters wide.

\begin{enumerate}
    \item 60
    \item 90
    \item 80
    \item 70
\end{enumerate}
\answer{Explain your answer here.}{0cm}{7cm}
\rule{0cm}{1cm}

\end{minipage}

%%%%%%%%%%%%%%%%%%%%%%%%%%%%%%%%%%

\newpage
\noindent
\begin{minipage}{\textwidth}

\question{1 mark}
When making pull requests involving changes to multiple files (e.g. Haskell scripts and stable folder files), remember to:

\begin{enumerate}
    \item use multiple \lstinline{git add} to stage multiple files before doing a single commit
    \item update the stable files first and push them, then repeat with the scripts using \lstinline{git add}
    \item only update all scripts in one commit using multiple \lstinline{git add}
    \item only update all changed 'stable' folder files in one commit using multiple \lstinline{git add}
\end{enumerate}
\answer{Explain your answer here.}{0cm}{7cm}
\rule{0cm}{1cm}

\question{1 mark}
The \lstinline{git branch} command:

\begin{enumerate}
    \item shows the items under your current branch on your local repo
    \item shows a list of all your current branches on your local repo
    \item shows branches dependent on your current branch on your local repo
    \item shows a list of all current branches on the remote repo
\end{enumerate}
\answer{Explain your answer here.}{0cm}{7cm}
\rule{0cm}{1cm}

\question{1 mark}
When closing an issue, please provide:

\begin{enumerate}
    \item Rationale
    \item Relevant Links to other related issues
    \item Linked Pull Requests
    \item Any or all of the above
\end{enumerate}
\answer{Explain your answer here.}{0cm}{7cm}
\rule{0cm}{1cm}

\end{minipage}

%%%%%%%%%%%%%%%%%%%%%%%%%%%%%%%%%%

\newpage
\noindent
\begin{minipage}{\textwidth}

\question{1 mark}
To \textbf{only} build the 2D Rigid Body Physics Library example (gamephysics\_diff) using the Drasil framework, run the command:

\begin{enumerate}
    \item \lstinline{setup gamephysics_diff}
    \item \lstinline{make gamephysics_diff}
    \item \lstinline{stack exec gamephysics_diff}
    \item \lstinline{make}
\end{enumerate}
\answer{Explain your answer here.}{0cm}{7cm}
\rule{0cm}{1cm}

\question{1 mark}
The \lstinline{git pull} command is used to:

\begin{enumerate}
    \item sync your local version with remote version of the repo
    \item displays changes made to the remote version of the repo
    \item sync remote version with your local version of the repo
    \item updates the remote repo with other people's changes
\end{enumerate}
\answer{Explain your answer here.}{0cm}{7cm}
\rule{0cm}{1cm}

\question{1 mark}
To run the Glass-BR example (glassbr) using the Drasil framework (assume that the example has already been built), run the command:

\begin{enumerate}
    \item \lstinline{make glassbr}
    \item \lstinline{exec glassbr}
    \item \lstinline{stack glassbr}
    \item \lstinline{stack exec glassbr}
\end{enumerate}
\answer{Explain your answer here.}{0cm}{7cm}
\rule{0cm}{1cm}

\end{minipage}

%%%%%%%%%%%%%%%%%%%%%%%%%%%%%%%%%%

\newpage
\noindent
\begin{minipage}{\textwidth}

\question{1 mark}
What is the difference between a remote branch and a local branch?

\begin{enumerate}
    \item a local branch is stored on a server; a remote branch is stored on your computer
    \item a remote branch is stored on a server; a local branch is stored on your computer
    \item both remote and local branches are stored on a server
    \item both remote and local branches are stored on your computer
\end{enumerate}
\answer{Explain your answer here.}{0cm}{7cm}
\rule{0cm}{1cm}

\question{1 mark}
What is \lstinline{origin}?

\begin{enumerate}
    \item the latest branch created
    \item a basic template branch
    \item the master branch
    \item none of the above
\end{enumerate}
\answer{Explain your answer here.}{0cm}{7cm}
\rule{0cm}{1cm}

\question{1 mark}
What is \lstinline{HEAD}?

\begin{enumerate}
    \item the latest commit in your current branch
    \item the base/master branch of your remote repo
    \item the active item referenced by your current repo
    \item A. and C.
\end{enumerate}
\answer{Explain your answer here.}{0cm}{7cm}
\rule{0cm}{1cm}

\end{minipage}

%%%%%%%%%%%%%%%%%%%%%%%%%%%%%%%%%%

\newpage
\noindent
\begin{minipage}{\textwidth}

\question{1 mark}
Which branch are you pulling from when you execute the following commands:

\begin{lstlisting}
git branch
git switch master
git switch sample
git pull
\end{lstlisting}

\begin{enumerate}
    \item master
    \item sample
    \item both master and sample
    \item neither master nor sample
\end{enumerate}
\answer{Explain your answer here.}{0cm}{7cm}
\rule{0cm}{3cm}

\question{1 mark}
Suppose that your have created a new branch on your local repo \textbf{only}. Which of the following commands would you use to push your new branch onto the remote repo?

\begin{enumerate}
    \item \lstinline{git push --set-upstream <new branch name>}
    \item \lstinline{git push}
    \item \lstinline{git push <new branch name>}
    \item \lstinline{git push --set-upstream origin <new branch name>}
    \item none of the above
\end{enumerate}
\answer{Explain your answer here.}{0cm}{7cm}
\rule{0cm}{3cm}

\end{minipage}

%%%%%%%%%%%%%%%%%%%%%%%%%%%%%%%%%%

\newpage
\noindent
\begin{minipage}{\textwidth}

\question{2 marks} Pretend that you have made a commit on your local repo that you would like to undo. Describe the process that you would use to undo the commit if (choose one to answer):

\begin{enumerate}
    \item your commit is only on your local repo
    \item your commit has also been pushed to the remote repo
\end{enumerate}

Be sure to include any commands that you use to accomplish this task.

\answer{Answer here (please indicate which option you chose).}{0cm}{7cm}
\rule{0cm}{14cm}

\end{minipage}

%%%%%%%%%%%%%%%%%%%%%%%%%%%%%%%%%%

\newpage
\noindent
\begin{minipage}{\textwidth}

\question{2 marks}
When creating a new issue on GitHub, describe two tips to follow that help to ensure that the new issue includes enough information (context).

\answer{Answer here.}{0cm}{7cm}
\rule{0cm}{7cm}

\question{2 marks}
Explain the concept of cherry-picking in GitHub. Be sure to include what command(s) and information you would use to accomplish it.

\answer{Answer here.}{0cm}{7cm}
\rule{0cm}{7cm}

\end{minipage}

%%%%%%%%%%%%%%%%%%%%%%%%%%%%%%%%%%

\newpage
\noindent
\begin{minipage}{\textwidth}

\question{2 marks}
Describe Drasil (what is it, what does it do), and discuss its main goals.

\answer{Answer here.}{0cm}{7cm}
\rule{0cm}{7cm}

\question{2 marks}
Imagine that you have made changes to some files in the Drasil code on your local repository. Describe the process that you would take to update the remote repository for Drasil with your changes. List any commands that you would use here as well.

\answer{Answer here.}{0cm}{7cm}
\rule{0cm}{7cm}

\end{minipage}

\end{document}