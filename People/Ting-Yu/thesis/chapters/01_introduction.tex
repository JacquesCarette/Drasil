\chapter{Introduction}

Jupyter Notebook can be generated from codified knowledge to improve their 
reusability, usability, reliability, and modifiability. In many software 
engineering domains, like scientific computing, knowledge is duplicate across 
software artifacts, which sometimes make it difficult and tedious to write 
and maintain them manually. 

Generative programming is a technique to address this problem. It allows 
programmers to write the code or document at a higher abstract level, and the 
generator produces the desired outputs. With generative programming, we can do 
things more efficiently since automation increases the efficiency. Drasil is an 
application of generative programming, and it is the framework we used in this 
research. We are tyring to generate Jupyter Notebook in Drasil to extend the 
flexibility of Drasil and to acheive the reusability, usability, and 
maintainability of software artifacts. 

\section{Background}
\subsection{Drasil}
Drasil is a framework that can generate software artifacts, including Software 
Requirement Specification (SRS), code (C++, C\#, Java, and Python), README, and 
Makefile, from a stable knowledge base. The main goals are to reduce knowledge 
duplication and improve traceability. Drasil captures the knowledge through our 
hand-made case studies; it allows users to provide recipes of their scientific 
problems and generate code and documentation. Each piece of information is 
encoded in Drasil once and that information can be used wherever it is needed. 
Drasil is capable of generating SRS in document languages HTML and LaTeX. We 
are working on increasing our automated database and improving how we encode 
the recipes.

\subsection{Jupyter Notebook}
Jupyter Notebook is an interactive open-source web application for creating and 
sharing documents that contain text, executable code, equations, graphics, and 
visualizations. 
There are several advantages of Jupyter Notebook: sharable, all-in-one, and 
executable code and equations. First of all, the notebook is easy to share 
because it can be converted into other formats 
such as HTML, Markdown, and PDF. Another benefit is that 
it combines all aspects of data in one single document, making the document 
easier to visualize, maintain and modify. In addition, Juypyter Notebook 
provides an environment of executable code and equations. Programmers don't 
have to write the entire program before executing it, the notebook can execute 
a specific portion of the code. Jupyter Notebook is especially useful in data 
science based on its uses and the benefits mentioned above.

\section{Problem Statement}
Both Jupyter Notebook and Drasil focus on scientific computing, therefore, we 
are interested in generating Jupyter Notebook in Drasil. Following are the 
three main problems we are trying to solve:
\begin{enumerate}
	\item We need to generate document format that is readable by the 
	notebook. Jupyter Notebook is a simple JSON document with .ipynb file 
	extension. Drasil can only write in HTML and LaTeX before; we will
	need Drasil to be able to generate documents in JSON format so the 
	documents can be read and written in Jupyter Notebook.
	\item Not only being able to generate SRS in Drasil, we are trying to 
	expand the application of Drasil and increase the automated database by 
	generating a simple physics lesson plan. We need to capture the elements of 
	chapters, found the family of lesson plans, and classified the 
	knowledge in order to build a general structure in Drasil so the lesson 
	plan will be able to generalize to other lessons. 
	\item Jupyter Notebook is an interactive application for creating documents 
	that combine text and executable code. However, Drasil doesn't support 
	interactive receipes right now. We are looking for the possibility to 
	generate a document with a mix of text and code, increasing the ability of 
	Drasil and the potential to solve more scientific problems.
\end{enumerate}

By generating Jupyter Notebook in Drasil, we can extend the benefits of Jupyter 
Notebook, expand the flexibility and the kind of knowledge we can manipulate. 
In addition, we will be able to improve the reusability, usability, 
reliability, and modifiability of the generated software artifacts. 

\section{Thesis Outline}
Thesis outline here.