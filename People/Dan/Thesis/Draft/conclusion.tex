\chapter{Conclusion}
\label{c:conclusion}

This thesis set out to address the persistent challenges of consistency,
traceability, and maintainability in scientific software and documentation.
The introduction established the need for a principled, automated approach to
knowledge capture and artifact generation, motivating the development and
evaluation of Drasil.

The results chapter provided concrete evidence that Drasil's single-source
architecture enforces consistency by construction, improves traceability
through explicit provenance, and enhances reproducibility via deterministic
artifact generation. Case studies demonstrated that local changes to domain
knowledge propagate automatically across all generated outputs, reducing
manual effort and the risk of inconsistencies. The results also highlighted
practical benefits such as the automatic maintenance of supporting materials
and the ease of creating new software family members by reusing and extending
the knowledge base.

The discussion chapter interpreted these findings, emphasizing that Drasil's
approach shifts the burden of correctness to the knowledge base, clarifies
implicit assumptions, and enables rapid, reliable adaptation to new
requirements. It also acknowledged tradeoffs, such as the increased upfront
modeling and onboarding effort, and identified the need for improved tooling
and formal validation.

The primary contribution of this thesis is the demonstration that a
centralized, recipe-driven knowledge base can systematically address many of
the recurring problems in scientific software engineering. By explicitly
capturing domain knowledge and automating artifact generation, Drasil
provides a scalable, extensible foundation for reliable, maintainable, and
reproducible scientific software. The lessons learned and directions for
future work outlined here position Drasil as both a practical tool and a
platform for further research in automated knowledge capture and software
generation.

\section{Summary of Main Contributions}

This thesis presents the design and implementation of Drasil, a centralized,
recipe-driven knowledge base created as part of this research to
systematically address persistent challenges in scientific software
engineering. Through a systematic analysis of the commonalities and
differences among software artifacts, this work operationalizes our
understanding of software engineering principles, enabling their direct
application in automated artifact generation. By explicitly capturing domain
knowledge and automating the generation of software and documentation
artifacts, Drasil enforces consistency by construction, improves traceability
through explicit provenance, and enhances reproducibility via deterministic
outputs. Case studies presented in Chapter~\ref{c:results} show that local
changes to domain knowledge propagate automatically across all generated
outputs, reducing manual effort and minimizing the risk of inconsistencies.
The architecture described in Chapter~\ref{c:drasil}, refined through
iterative development and feedback, enables the reuse and
extension of the knowledge base, facilitating the creation of new software
family members and supporting the automatic maintenance of supporting
materials. Collectively, these contributions provide a scalable and
extensible foundation for reliable, maintainable, and reproducible
scientific software.


\section{Limitations and Open Questions}

While Drasil demonstrates significant progress toward automating knowledge
capture and artifact generation, several limitations remain. The approach
requires substantial upfront modeling effort and domain expertise, which may
pose a barrier to adoption for new users or projects. Tooling and usability
challenges persist, including the need for improved front-end interfaces and
developer experience plugins to streamline onboarding and knowledge capture.

Formal validation and empirical evaluation of Drasil's effectiveness are
ongoing, and further experiments are needed to establish its generalizability
across diverse scientific computing domains. At present, Drasil is primarily
limited to scientific computing software that adheres to an input-process-
output paradigm, where requirements and models are well-understood and
structured. As the knowledge base grows, scalability and efficient knowledge
management will become increasingly important challenges, particularly for
discovery, projection, and modularization of domain knowledge.

Extending Drasil to more dynamic or less formalized fields
remains an open question. These limitations and open questions, discussed 
in Chapter~\ref{c:discussion}, highlight important directions for future 
work, which are outlined in the following section.

\section{Future Work}

Building on the foundation established in this thesis, several avenues for
future work are apparent. As Drasil continues to evolve, its ongoing
development will be shaped by both practical experience and feedback from
real-world adoption. The directions outlined below reflect the need to
broaden Drasil's applicability, improve its usability, and deepen its
integration into scientific software engineering workflows. The following 
subsections highlight key areas where further research and development can
have the greatest impact.

\subsection{Expanding the Knowledge Base and Knowledge Capture Mechanisms}

As Drasil is adopted for new projects, the encoded chunk database will
naturally expand. This gradual growth is essential for increasing the
breadth and depth of reusable domain knowledge, which in turn enhances
the quality and variety of generated artifacts. Each new project
contributes additional chunks, models, and recipes, making it easier to
support related domains and reducing the effort required for future
knowledge capture. Over time, a richer knowledge base will enable Drasil
to generate more comprehensive documentation and software, facilitate
cross-project reuse, and improve consistency across scientific computing
applications.

Similarly, designing more flexible and extensible knowledge capture 
mechanisms is essential for broadening Drasil's applicability. Currently, 
the types of information encoded as chunks are fairly primitive, often 
limited to basic definitions, equations, and simple relationships. 
Future work should focus on advancing the sophistication of chunks to 
enable Drasil to generate higher-quality artifacts and address more nuanced 
requirements. Additionally, developing new methods and interfaces for knowledge
capture (such as guided authoring tools, semi-automated chunk inference, and 
support for alternative modeling paradigms) will allow domain experts to 
contribute knowledge more efficiently and facilitate extensibility into new 
scientific fields.

\subsection{Inference Improvements}
A key direction for future work is improving Drasil's ability to infer
knowledge chunks and their relationships directly from recipes and other
high-level specifications. Currently, much of the chunk structure and
dependency information must be defined explicitly, which can be tedious and
error-prone as projects scale. Automating chunk inference would reduce the
need for manual list definitions and make knowledge capture more efficient
and less repetitive. Achieving this will require advances in both recipe
design and the underlying inference passes. Progress in this area will
streamline the authoring process, lower the barrier to entry for new users,
and further enhance Drasil's scalability and maintainability.

\subsection{Typed Expression Language}
Developing a robust typed expression language within Drasil is a promising
direction for future work. Currently, expressions are represented in a way
that limits the ability to perform deep sanity checks or enforce domain-
specific constraints at the knowledge level. Introducing a typed expression
language would enable more rigorous validation of mathematical models,
formulae, and relationships before code or documentation is generated.
This would help catch errors early, ensure consistency across artifacts,
and support advanced features such as type-driven code generation and
automated reasoning about units and dimensions. Ultimately, a typed
expression language would increase the reliability and expressiveness of
Drasil's knowledge base, making it easier to support complex scientific
domains and reduce the risk of subtle errors in generated artifacts.

\subsection{Usability Enhancements}
Improving usability is critical for lowering barriers to adoption and
ensuring that domain experts can efficiently contribute and manage knowledge.
Future work should focus on developing intuitive tooling, such as visual
front-ends, drag-and-drop interfaces, and plugins for popular development
environments. Enhanced authoring tools will streamline the process of
defining and refining chunks, while improved onboarding resources and guided
workflows will help new users become productive more quickly. Tooling for
cross-referencing existing chunks will prevent duplication and facilitate
knowledge discovery, supporting both the expansion and sophistication of the
knowledge base. Collectively, these usability enhancements will make Drasil
more accessible and effective for a wider range of users and projects.

\subsection{Output Formats and Artifact Types}
Expanding the range of output formats and artifact types is essential for
making Drasil applicable to a wider array of audiences and use cases.
Future work should focus on supporting additional programming languages
(by expanding the existing GOOL backend) and document formats, 
such as docx or other widely used file types. 

Introducing new artifact types, (ex. Jupyter notebooks, journal papers), 
will further extend Drasil's utility beyond traditional software documentation.
These enhancements will enable Drasil to serve the diverse needs of scientific 
computing practitioners, educators, and researchers, and will help bridge the 
gap between code, documentation, and knowledge dissemination.

\subsection{Natural Language Variants}
Supporting multiple natural language variants and context-sensitive artifact
generation will significantly enhance Drasil's adaptability and accessibility.
This capability is especially important for international and interdisciplinary
teams, where documentation and software artifacts may need to be produced in
different languages tailored to specific audiences. Future work should
focus on developing mechanisms for encoding and selecting language variants
within the knowledge base, as well as strategies for generating artifacts
that reflect cultural and contextual differences. These improvements will
broaden Drasil's reach and facilitate collaboration across diverse scientific
communities.

\subsection{Test Case Generation}

Developing more robust automated test case generation is essential for
ensuring the reliability and correctness of artifacts produced by Drasil.
Future work should focus on enhancing the mechanisms for generating
comprehensive and meaningful test cases that reflect both domain knowledge
and software requirements. Improvements in this area will help verify
artifact integrity and support the adoption of Drasil in safety-critical
and high-assurance scientific computing projects.

\subsection{Experiments and Validation}

Formal experiments and empirical validation are needed to rigorously assess
Drasil's effectiveness and generalizability across different domains and
project types. Future work should include controlled studies measuring
onboarding metrics (such as time to first meaningful edit and mentoring
required), maintenance effort, and error rates across teams of varying
experience. Comparative experimental designs could evaluate Drasil-based
workflows against traditional manual approaches, quantifying differences in
initial productivity, long-term maintenance, and defect reduction. Gathering
quantitative evidence through these studies will help demonstrate the
practical benefits of Drasil's approach, identify limitations, and guide
further improvements. Feedback from such experiments will also inform the
development of improved authoring tools and onboarding resources, helping to
lower the initial ramp-up cost and make Drasil accessible to a broader range
of users. Demonstrating Drasil's impact through systematic experimentation
will be crucial for broader acceptance within the scientific computing 
and software engineering community.

\subsection{Extending Beyond Input-Process-Output}

Extending Drasil beyond the input-process-output paradigm represents a
significant challenge and opportunity for future work. Many scientific
and engineering applications involve interactive, event-driven, or
stateful behaviors that do not fit neatly into the current modeling
framework. Addressing these domains will require advances in flexible
modeling, chunk representation, and recipe design, enabling the capture
and projection of more complex knowledge structures. Progress in this
direction will broaden Drasil's applicability to a wider range of
software systems, including those with dynamic requirements and less
formalized processes, and will help realize the vision of fully
automated, knowledge-centric artifact generation across scientific
computing.

As Drasil continues to develop, ongoing research and collaboration will be
essential to realizing its full potential and advancing the practice of
automated, knowledge-centric scientific software engineering.

\section{Broader Impact and Personal Reflection}

The development of Drasil represents a step toward more principled and
automated scientific software engineering. By centralizing domain knowledge
and enabling systematic artifact generation, Drasil has the potential to
improve reliability, reproducibility, and maintainability across a wide
range of scientific computing projects. Its approach may influence future
tools and methodologies, encouraging the adoption of knowledge-centric
practices in both research and industry.

Drasil's evolution has been shaped by its iterative development process,
where feedback from case studies and practical use continually informed
refinement of both the knowledge base and framework itself. This
iterative approach has enabled Drasil to adapt to new requirements,
address unforeseen challenges, and incrementally improve its capabilities
while maintaining validated, working case study software projects.

On a personal level, this work has highlighted the importance of
interdisciplinary thinking and the value of bridging gaps between software
engineering theory and practical implementation. The challenges encountered
throughout the design and development of Drasil have reinforced my 
understanding of the need for flexibility, collaboration, and ongoing 
refinement. The experience has been both intellectually rewarding and 
formative, shaping my perspective on the future of scientific software. 

\section{Closing Remarks}

This thesis has demonstrated the feasibility and benefits of a knowledge-
centric approach to scientific software engineering. By operationalizing
software engineering principles and automating the generation of software
and documentation artifacts, Drasil offers a scalable solution to persistent
challenges of consistency, traceability, and maintainability. The work
presented here lays the groundwork for further advances in knowledge capture
and automated artifact generation, and invites continued research and
collaboration to realize the full potential of this approach.

As the scientific computing community evolves, tools like Drasil may play an
increasingly important role in supporting reliable, reproducible, and
adaptable software systems. The lessons learned from this work highlight the
value of centralized knowledge, systematic artifact generation, and the
importance of ongoing refinement. Continued investment in these principles
will help shape the future of scientific software engineering, ensuring that
software remains trustworthy, maintainable, and responsive to new challenges.
