\chapter{Reading Notes}
\label{chap:reading_notes}

Before reading this thesis, I encourage you to read through these notes, keeping
them in mind while reading.

\begin{itemize}

      \item Drasil's source code is publicly available on
            \porthref{GitHub}{https://github.com/JacquesCarette/Drasil}, and
            Drasil's documentation
            (\porthref{user-facing}{https://jacquescarette.github.io/Drasil/docs/index.html},
            and
            \porthref{internal}{https://jacquescarette.github.io/Drasil/docs/full/index.html})
            is available on the Drasil project
            \porthref{homepage}{https://jacquescarette.github.io/Drasil/}.
            Drasil's public wiki is hosted on the same \porthref{GitHub
                  repository}{https://github.com/JacquesCarette/Drasil/wiki},
            containing information on potential future Drasil projects,
            Drasil-related papers, a \porthref{developer workspace configuration
                  and ``quick start''
                  guide}{https://github.com/JacquesCarette/Drasil/wiki/New-Workspace-Setup},
            and a guide for \porthref{building your own project with
                  Drasil}{https://github.com/JacquesCarette/Drasil/wiki/Creating-Your-Project-in-Drasil}.
            Similarly, the \porthref{source
                  code}{https://github.com/JacquesCarette/Drasil/tree/main/People/Jason/thesis}
            for this thesis is also publicly available.

      \item ``Source Code'' snippets with ``Original'' in their title show code
            snippets as they appeared \textit{before} this work. All other
            snippets are either pseudocode or a view of the actual code at a
            particular git blob.

      \item The source code related to the prototyped \ChunkDB{} relevant to
            \Cref{chap:storingChunks} is also \porthref{publicly
                  available}{https://github.com/balacij/ProtoChunkDB}.

      \item Please note that blue coloured text in a monospaced font (such as
            \ExampleText{}) refers to names you can find in Drasil's source
            code.

      \item At times, we will refer to ``software artifacts'' as just
            ``artifacts.''

      \item When we refer to ``Haskell,'' we are referring to the Haskell 2010
            specification \cite{Haskell2010} and/or Haskell code compiled by
            \acs{ghc} 8.8.4 \cite{GHC884}.

      \item This report is available in two (2) flavours: one intended for
            viewing on a computer (the default), and one intended for printing.
            The one intended for viewing on a computer will have hyperlinks
            coloured in \textcolor{red}{red} and does not show website
            references explicitly. In particular, all ``Source Code'' content
            will link directly to a snippet of code. The copy intended for
            printing will show website references explicitly by adding footnotes
            to hyperlinks.

\end{itemize}
