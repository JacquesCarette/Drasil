\chapter{A look under the hood: \\ Our process}

\ds{Make sure we talk about continuous integration / git processes / etc}

The first step in removing unnecessary redundancy is identifying exactly what
that redundancy is and where it exists. To that end we need to understand what
each of our software artifacts is attempting to communicate, who their audience
is, and what information can be considered boilerplate versus system-specific.
Luckily, we have an excellent starting point thanks to the work of many smart
people - artifact templates.

Lots of work \ds{cite some people who did this} has been done to specify 
exactly what should be documented in a given artifact in an effort for 
standardization. Ironically, this has led to many different `standardized' 
templates. Through the examination of a number of different artifact templates, 
we have concluded they convey roughly the same overall information for a given 
artifact. Most differences are stylistic or related to content 
organization and naming conventions, as we will demonstrate in the following 
sections.

Once we understand our artifacts, we take a practical, example-driven approach
to identifying redundancy through the use of existing software system case
studies. For each of these case studies, we start by examining the source code
and existing software artifacts to understand exactly what problem they are
trying to solve. From there, we attempt to distill the system-specific knowledge
and generalize the boilerplate.

\section{A (very) brief introduction to our case study systems}
\ds{**NOTE: ensure each artifact has a 'who' (audience), 'what' (problem 
being solved), and 'how' (specific-knowledge vs boilerplate) - this last one 
may not be necessary}

To simplify the process of identifying redundancies and patterns, we have chosen
several case studies developed using common artifact templates, specifically 
those used by \smithea{} \ds{source\citep{SmithEtAlXXXX}} Also, as mentioned in 
\ds{[SCOPE]}, we have chosen software systems that follow the $`input' 
\rightarrow `process' \rightarrow `output'$ pattern. These systems cover a 
variety of use cases, to help avoid over-specializing into one particular 
system-type. 

The majority of the aforementioned case studies were developed to solve real
problems, with a few minor exceptions listed below their relevant cards.
  \ds{**NOTE: trying to find a good way to say `these are not just things we 
  cooked up to make Drasil look good'}
The cards below are meant to be used as a high-level reference to each case 
study, providing the general details at a glance. For the specifics of each 
system, all relevant case study artifacts can be found at \ds{Add a link here or
put in appendices?}.

\card{\gb}
{We need to efficiently and correctly predict whether a glass 
slab can withstand a blast under given conditions.}
{\ds{TODO}}

\card{\sw}
{Solar water heating systems incorporating phase change 
 material (PCM) use a renewable energy source and provide a novel way of 
 storing energy. A system is needed to investigate the effect of employing PCM
 within a solar water heating tank.}
{\ds{TODO}}

\card{\np}
{Solar water heating systems provide a novel way of 
heating water and storing renewable energy. A system is needed to investigate
the heating of water within a solar water heating tank.}
{\ds{TODO}}

The NoPCM case study was created as a software family member for the SWHS case
study. It was manually written, removing all references to PCM and thus 
remodeling the system.

\card{\sp}
{A slope of geological mass, composed of soil and rock 
 and sometimes water, is subject to the influence of gravity on the mass. 
 This can cause instability in the form of soil or rock movement which can
 be hazardous. A system is needed to evaluate the factor of safety of 
 a slope's slip surface and identify the critical slip surface of the slope, 
 as well as the interslice normal force and shear force along the critical 
 slip surface.}
{\ds{TODO}}

\card{\pr}
{A system is needed to efficiently and correctly predict
 the landing position of a projectile.}
{\ds{TODO}}

The Projectile case study, was the first example of a system 
created solely in Drasil, i.e. we did not have a manually created version to 
compare and contrast with through development. As such, it will not be 
referenced often until \ds{DRASILSECTION} since it did not inform Drasil's 
design or development until much further in our process. The Projectile case 
study was created post-facto to provide a simple, understandable example for a 
general audience as it requires, at most, a high-school level understanding of 
physics. 

\card{\gp}
{Many video games need physics libraries that simulate 
 objects acting under various physical conditions, while simultaneously being 
 fast and efficient enough to work in soft real-time during the game. 
 Developing a physics library from scratch takes a long period of time and is 
 very costly, presenting barriers of entry which make it difficult for game 
 developers to include physics in their products.}
{\ds{TODO}}

After carefully selecting our case studies, we went about a practical approach
to find and remove redundancies. The first step was to break down each artifact
type and understand exactly what they are trying to convey.


\section{Breaking down \sfs}
\label{sec:breakdown}

As noted earlier, for our approach to work we must understand exactly what each
of our artifacts are trying to say and to whom. By selecting our case studies 
from those developed using common artifact templates, we have given ourselves a
head start on that process, however, there is still much work to be done.

To start, we look at the Software Requirements Specification (SRS). The SRS
(or some incarnation of it) is one of the most important artifacts for any
software project as it specifies what problem the software is trying to solve.
There are many ways to state this problem, and the template from \smithea{} has 
given us a strong starting point. Figure~\ref{fig:SRSToC} shows the table of 
contents for an SRS using the \smithea{} template.

\fig{
  \begin{center}
    \emph{Figure showing the ToC of \smithea{} template}
  \end{center}
}{The Table of Contents from the \smithea{} template}{fig:SRSToC}

With the structure of the document in mind, let us look at several of our case
studies' SRS documents to get a deeper understanding of what each section truly
represents. Figure~\ref{fig:csRefSecs} shows the reference section of the SRS 
for \ds{TODO}. Each of the case studies' SRS contains a similar section so for 
brevity we will omit the others here, but they can be found at \ds{TODO}. 
% Provide a link / ref to relevant appendix
We will look into the case studies in more detail later, for now we will try to 
ignore the superficial differences in each of them while we look for 
commonality. \sout{We are strictly looking for patterns! Patterns will give 
us insight into the root of \emph{what} is being said in each section.}\ds{Not 
really true, will need to rework}

\fig{\begin{center}
\emph{Figure showing the Ref Section of one case 
study, split into multiple subfigures - case study TBD}\end{center}}{The 
reference sections of \ds{TBD} respectively}
{fig:csRefSecs}

Looking at the Table of Symbols, Table of Units, and Table of Abbreviations and 
Acronyms from Figure~\ref{fig:csRefSecs} we can see that, barring the table 
values themselves, they are almost identical. The Table of Symbols is simply a 
table of values, akin to a glossary, specific to the symbols that 
appear throughout the rest of the document. For each of those symbols, we see 
the symbol itself, a brief description of what that symbol represents, and the 
units it is measured in, if applicable. Similarly, the Table of Units lists the
Syst\`eme International d'Unit\'es (SI) Units used throughout the document, 
their descriptions, and the SI name. Finally, the table of Abbreviations and 
Acronyms lists the abbreviations and their full forms, which are 
essentially the symbols and their descriptions for each of the abbreviations.

While the reference material section should be fairly self-explanatory as to 
what it contains, other sections and subsections may not be so clear from their 
name alone. For example, it may not be clear offhand of what constitutes a 
theoretical model compared to a data definition or an instance model. One may 
argue that the author of the SRS, particularly if they chose to use the 
\smithea{} template, would need to understand that difference. However, it is 
not clear whether the intended audience would also have such an understanding. 
Who is that audience? We will explore that in Section~\ref{sec:sfsummary}.

Returning to our exercise of breaking down each section of the SRS to determine 
the subtleties of \emph{what} is contained therein (the details are omitted for 
brevity, although the overall process is very much akin to that of our 
breakdown of the Reference Material section) it should be unsurprising that 
each section maps to the definition provided in the \smithea{} template. 
However, as noted above, we can see distinct differences in the types of 
information contained in each section. Again we find some is boilerplate text 
meant to give a generic  (non-specific) overview, some is specific to the 
proposed system, and some is in-between: it is specific to the 
problem domain for the proposed system, but not necessarily specific to the 
system itself.

\section{\SF{} Summary}
\label{sec:sfsummary}

Parnas~\citep{Parnas2010} does an excellent job of defining the target audience 
for each of the most common \sfs{} and we extend that alongside our work. A 
summary can be found in Table~\ref{tab:sfsummary}.

\begin{table}[h]
\caption{A summary of the Audience for each of the most common \sfs{} and what 
problem that \sf{} is solving}
\label{tab:sfsummary}
\begin{tabular}{| P{.3\linewidth} | p{.3\linewidth}  | p{.3\linewidth} |}
\hline
 \SF{} & Who (Audience) & What (Problem)
\\ \hline
	SRS & 

	Software Architects \newline \newline
	QA analysts & 

	Define exactly what specification the software system must adhere to. 
\\ \hline
	Module Guide & 

	All developers \newline \newline
	QA analysts  & 

	\ds{TODO}
\\ \hline
	Module Interface Specifications & 

	Developers who implement the module \newline \newline
	Developers who use the module \newline \newline
	QA analysts & 

	\ds{TODO}
\\ \hline
	Verification and Validation Plan &
	
	Developers who implement the module \newline \newline
	QA analysts &

	\ds{TODO}
\\ \hline
\end{tabular}
\end{table}

\section{Patterns and repetition and patterns and repetition -- (OR -- 
Repeating patterns and patterns that repeat --)}
  \ds{**Make sure to mention "We actually found that some
  info is boilerplate, some is system-specific, and some is general to all
  members of a software family, but more specific than generic boilerplate"}

From the above sections, we see many emerging patterns in our \sfs{}. Ignoring, 
for now, the organizational patterns from the \smithea{} templates we can 
already see simple patterns emerging.

Returning to our example from Section~\ref{sec:breakdown}, looking only at the 
reference section of our SRS template, we have already found
three subsections that contain the majority of their information in the same
organizational structure: a table of symbols and general information relevant 
to those symbols. Additionally, we can see that the Table of Units and Table of 
Symbols have an introductory blurb preceding the tables themselves, whereas the 
Table of Abbreviations and Acronyms does not. Inspecting across case studies, 
we can see that the introduction to the Table of Units is simply boilerplate 
text dropped into each case study verbatim, thus is completely generic and 
applicable to \emph{any} software system using SI units. The introduction to 
the Table of Symbols appears to be boilerplate across several examples, 
however, it does have some variation (see: \gp{} compared to \gb).
These variations reveal the obvious: the variability between systems is greater 
than simply a difference in choice of symbols, and so there is some 
system-specific knowledge being encoded. We can intuit this conclusion based 
solely on each system solving a different problem, however, we have confirmed 
this by solely looking at the structure (patterns) of one section of the SRS.

The reference section of the SRS provides a lot of knowledge in a very 
straightforward and organized manner. The basic units provided in the table of 
units give a prime example of fundamental, global knowledge shared across 
domains. Nearly any system involving physical quantities will use some of these 
units. On the other hand, the table of symbols provides system/problem-domain 
specific knowledge that will not be useful across unrelated domains. For 
example, the stress distribution factor $J$ from GlassBR may appear in several 
related problems, but would be unlikely to be seen in something like SWHS, 
NoPCM, or Projectile. Finally, acronyms are very context-dependent. They are 
often specific to a given domain and, without a coinciding definition, it can 
be very difficult for even the target audience to understand what they refer 
to. Within one domain, there may be several acronyms meaning different things, 
for example: PM can refer to Product Manager, Project Manager, Program 
Manager, etc.

By continuing to breakdown the SRS and other \sfs{}, we are able to find many 
more patterns. For example, we see the same concept being introduced in 
multiple areas within a single artifact and across artifacts in a project. 
[Example from one of the figures in the previous section. Preferably something 
like a DD or TM that shows up within a single doc multiple times]. We also see 
patterns of commonality across software family members (The SWHS and NoPCM case 
studies) as they have been developed to solve similar, or in our case nearly 
identical, problems.

- inter-project (repetition throughout different views + other patterns.)
  vs intra-project knowledge (repetition across projects/family members,
  minor modifications, but fundamentally the same + other patterns.)
- Hint at chunkifying/parceling out the fundamental (system/view-agnostic)
knowledge vs the specific knowledge

\ds{Need to talk about knowledge projection/transformation in this section
- Use a relevant example from Case studies, maybe from PCM? Something like 
$\Delta{}U = Q - W$ is the same as ``Total energy within a closed system must 
be conserved" but transformed for / projects out what is relevant to a given 
audience.
- This will be useful for the following section}

\section{Organizing knowledge - a fluid approach}
  **Subsec roadmap:
    - We see the patterns above, we can generalize a lot of that
    - Direct repetition (copy-paste) vs indirect repetition (view-changes)
    require us to pull together knowledge from all artifacts into one place
    - Some can be derived automatically, the rest must be explicitly stated
    - We need to create a categorization system (hint at chunks) that is both
    robust and extensible to cover a wide variety of use cases.
    - Finally the templates give us structure

  **NOTE: Under the hood section should explain the process of how we determined
  what we needed to do. What we ended up doing should come in the following
  section(s) - no 'real' implementation details, only conceptual stuff here.
  
\section{The seeds of Drasil}
  **Subsec roadmap:
    -- Summarize the above subsections and lead into next section
    -- Add relevant information that doesn't quite fit above 
      and isn't implementation related
    -- 'Relevant buckshot section'