\chapter{Results}
\label{c:results}

In this chapter we will discuss our observations following the reimplementation 
of our case studies using Drasil. At present these observations are 
anecdotal in nature as we have not yet been afforded the time to design more 
rigorous experiments for data collection due to Drasil being very much in flux 
and undergoing constant development. We will discuss more about experimental 
data collection in ``Future Work"(Chapter~\ref{c:future}).

\section{``Pervasive" Bugs} 

One of the first, and most curious, observations made while using Drasil
was that of so called \emph{pervasive bugs}. While we usually consider bugs to 
be something we wish to avoid at all costs, this is a case where the 
pervasiveness of bugs themselves is beneficial. Since we are generating 
\emph{all} our \sfs{} from a single source, a bug in that source will 
result in a bug occurring through \emph{every single \sf{}}. The major 
consequence is that the bug now has increased visibility, so is more likely to 
be discovered.

\ds{Need a salient example of a pervasive bug we found here, or description of 
a handl of "we found these along the way just because they were so readily 
visible"}

Pervasive bugs have another unique selling point. Consider a piece of software 
developed using generally accepted processes like 
waterfall or agile. After the initial implementation is complete, any 
bugs found are typically fixed by updating the code and other pertinent 
\sfs{}. As mentioned earlier, there are many instances, especially those 
involving tight deadlines or where non-executable \sfs{} are not 
prioritized, where the \sfs{} can fall out of sync with the implemented 
solution. As such we may end up with inconsistent \sfs{} that are wrong in 
(potentially) different ways. The following example involves the equation for 
Non-Factored Load (NFL) taken from the GlassBR case study:
\[\mathit{NFL}=\frac{{\hat{q}_{\text{tol}}} E h^{4}}{\left(a b\right)^{2}}\]
and its code representation (in python): 
\lstinputlisting[language=Python, firstline=45, 
lastline=46]{code/Calculations.py}
At a glance, are the code and formula equivalent? 

It is difficult to say without confirming the value of $E$ is defined as 
$71.7\cdot{}10^{9}$, which is equivalent to the value used in the python code. 
However, if both \sfs{} were generated using Drasil then we have added 
confidence due to them being generated from the same source. Now if we 
determine there is a bug, we can look at either the formula or implementation 
-- whichever we are more comfortable debugging -- to determine how the source 
should be updated. As such, pervasive bugs give us peace of mind that our 
\sfs{} are consistent, even in the face of bugs.

\section{Originals vs. Reimplementations}
	Link to original case studies and reimplementations
	Highlight key (important) changes with explanations
	Show off major errors/oversights
	Original \sfs{} LoC vs number of Drasil LoC to reimplement (and compare the 
	output of Drasil - multiple languages, etc.)
	\ds{Kolmogorov complexity?}

- One major thing to point out here: The most common issues we ran into while 
attempting to generate our versions of the case study \sfs{} was that of 
implicit knowledge that was typically assumed to be ``understood" in the 
context of the domain (i.e. domain experts have tacit knowledge and there are 
many undocumented assumptions).

\section{Design for change}
	\ds{GlassBR /1000 example}

Designing for change can be difficult, especially when dealing with software 
with a long (10+ year) expected lifespan. Through our use of Drasil in updating 
the case studies, it has become obvious that Drasil expedites our ability to 
design for change.

Having only a single source to update accelerates implementation of desired 
changes, which we have demonstrated numerous times throughout Drasil's 
development and the reimplementation of our case studies. One salient example 
was in the GlassBR case study.

\ds{Fill in the example details here}

\section {Mundane Value}
\ds{Title should change since some of these may not be so "mundane", but I want 
to cover a few very specific results and they don't warrant their own sections}.

- Consistency by construction

- No undefined symbols - use Tau example?

- ``Free" sections - Ref mats can all be generated from "system information" 
with little effort, unlike manual creation.
	
\section{Usability}
One of Drasil's biggest issues is that of usability. Unless one reads the source
code or has a member of the Drasil team working with them, it can be incredibly
difficult, or even impossible, to create a new piece of software in Drasil.

As seen in the examples from [SECTION], while the recipe language is fairly
readable, the knowledge-capture mechanisms are arcane and determining which
knowledge has already been added to the database can be very difficult. As our
living knowledge-base expands, this will become even more difficult,
particularly for those concepts with many possible names.

  - As the above mentions, not great, but CS students / summer interns picked it 
    up fast enough to make meaningful changes in a short time period.
