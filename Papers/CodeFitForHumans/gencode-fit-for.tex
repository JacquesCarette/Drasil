% -*- coding: utf-8; -*-
% vim: set fileencoding=utf-8 :
\documentclass[english,submission]{programming}
%% First parameter: the language is 'english'.
%% Second parameter: use 'submission' for initial submission, remove it for camera-ready (see 5.1)

\usepackage[backend=biber]{biblatex}
\addbibresource{../Bib/drasil.bib}
\addbibresource{../Bib/codegen.bib}

%%%%%%%%%%%%%%%%%%
%% These data MUST be filled for your submission. (see 5.3)
\paperdetails{
  %% perspective options are: art, sciencetheoretical, scienceempirical, engineering.
  %% Choose exactly the one that best describes this work. (see 2.1)
  perspective=art,
  %% State one or more areas, separated by a comma. (see 2.2)
  %% Please see list of areas in http://programming-journal.org/cfp/
  %% The list is open-ended, so use other areas if yours is/are not listed.
  area={Modeling and Modularity, Generative Programming},
  %% You may choose the license for your paper (see 3.)
  %% License options include: cc-by (default), cc-by-nc
  license=cc-by
}
%%%%%%%%%%%%%%%%%%

%%%%%%%%%%%%%%%%%%
%% These data are provided by the editors. May be left out on submission.
%\paperdetails{
%  submitted=2016-08-10,
%  published=2016-10-11,
%  year=2016,
%  volume=1,
%  issue=1,
%  articlenumber=1,
%}
%%%%%%%%%%%%%%%%%%


\begin{document}

\title{Generating code fit for humans} % Working title
% \subtitle{}
\titlerunning{Generating code fit for humans}

\author[a]{Jacques Carette}
\author[a]{Spencer Smith}
\affiliation[a]{Department of Computing and Software, McMaster University, Canada}

\keywords{Code generation, documentation, layout, knowledge capture} % please provide 1--5 keywords


%%%%%%%%%%%%%%%%%%%%%%%%%%%%%
% Please go to https://dl.acm.org/ccs/ccs.cfm and generate your Classification
% System [view CCS TeX Code] stanz and copy _all of it_ to this place.
%% From HERE
% To HERE
%%%%%%%%%%%%%%%%%%%%%%%

\maketitle

% Please always include the abstract.
% The abstract MUST be written according to the directives stated in 
% http://programming-journal.org/submission/
% Failure to adhere to the abstract directives may result in the paper
% being returned to the authors.
\begin{abstract}
  Context:

  Inquiry:

  Approach:

  Knowledge:

  Grounding:

  Importance:

\end{abstract}

\section{Introduction}
\label{sec:intro}

Short. High-level of the problem.

\section{The problem}
\label{sec:problem}

Subsection of intro? But give examples, well illustrated. See outline.txt.

\section{Analysis}
\label{sec:analysis}

Analyse the actual content of the examples. See what makes them tick.

\section{Organize}

The content previously outlined is not a jumble.

\section{Recipes}

The programs that weave together things.

\section{Technology}

GOOL, pretty-printing.

\acks
Our zillions of students who built large parts of the software.

%\appendix
%\section{A Famous Filler Text}

\begin{comment}
\section{Submission Checklist}
\smaller
\begin{itemize}
\renewcommand*\labelitemi{\ensuremath{\square}}
\item All authors are listed, with affiliation
\item The abstract is not longer than 500 words.
\item The abstract states, in this order, \emph{Context},
  \emph{Inquiry}, \emph{Approach}, \emph{Knowledge}, \emph{Grounding}, and
  \emph{Importance} of the submission.
\item The ACM CCS 2012 classification is included.
\item A list of relevant keywords is provided.
\item One \emph{perspective} has been chosen.
\item One or more \emph{area(s) of submission} have been chosen.
\item The license has been acknowledged or changed to fit.
\item The title page does not spill onto the second page.
\item The running title on even pages' heads is not wider than the text block.
\item The running list of author on odd pages' heads is not wider than the text
  block.
\item The only text encoding used is \textsc{utf-8}.
\item The fonts, margins, and spacings are unchanged.
\item All \textsc{url}s are marked using the \lstinline|\url| command.
\item All units are separated from their values with a half space, consistently
  (see ~\cite{siunitx}).
\item Bold font is not used (or only in exceptional circumstances).
\item Underlining is not used (no exceptions).
\item Bold or bright colors are only ever used sparingly.
\item The captions of (a) tables, (b) listings, and (c) algorithms are
  \emph{above} the content, flush left.
\item The captions of (a) figures, (b) charts, (c) combined content, and
  similar are \emph{below} the content, flush left.
\item %
  \begin{minipage}[t]{.4\linewidth}
    Tables adhere to best practice layout~\cite{booktabs}. They do not contain
  vertical lines and only few horizontal lines.
  \end{minipage}
  {\renewcommand{\arraystretch}{.6}
    \hspace{2em} \textcolor{DarkGreen}{Yes} \hspace{1em}
      \begin{tabular}[t]{ll}\noalign{\vspace*{-2.2ex}}
        \toprule
        \emph{a} & \emph{b}\\
        \midrule
        x & abc \\
        y & def \\
        \bottomrule
      \end{tabular}
      \hspace{2em} \textcolor{DarkRed}{No} \hspace{1em}
      \begin{tabular}[t]{|l|l|}\noalign{\vspace*{-1.6ex}}
        \hline
        \emph{a} & \emph{b}\\
        \hline\hline
        x & abc \\\hline
        y & def \\\hline
      \end{tabular}}
\item All fonts are embedded. This includes the fonts for any included
  (\textsc{pdf}-)graphics.
\item All graphics are either vector graphics or have at least 300\,dpi.
  \item Acknowledgements, including statement of funding bodies or projects, is
  placed before the references, using the \lstinline|\acks| command.
\item Citations in the text are preceded by a non-breaking space.
  (\(\rightarrow\) Citations do not appear at the beginning of a line)
\item The bibliography lists all cited works, all authors are given (no
  \emph{et. al.}), the full names of the authors are used.
\item All bibliography entries have a \textsc{doi} (unless not applicable, such as for
  slides or unpublished work).
\item All bibliography entries show an \textsc{isbn}, if they have one assigned.
\item All bibliography entries that represent web sites have a \textsc{url} and
  an \emph{access date}.
\item No bibliography entries have duplicate information in \textsc{url} and
  \textsc{doi}.
\end{itemize}
\end{comment}
% 
\printbibliography

\end{document}

% Local Variables:
% TeX-engine: pdflatex
% End:
