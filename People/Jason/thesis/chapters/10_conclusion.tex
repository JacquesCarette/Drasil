\chapter{Conclusion}
\label{chap:conclusion}

In this thesis, we addressed 4 research questions, as stated in
\Cref{chap:intro:sec:research-questions}:

\begin{enumerate}

    \item[\textbf{RQ1}] \textit{Drasil's current encoding of ``theories'' are
          essentially black boxes. We would like to be able to use some
          structural information present in the short list of the ``kinds'' of
          theories that show up in scientific computing. How do we codify that?}

\end{enumerate}

In \Cref{chap:modelkinds}, we discussed how Drasil uses theories as part of an
abstracted \acs{srs} template \cite{SmithAndLai2005}. By having users ``fill
in'' the abstracted \acs{srs} templates information about scientific problems
where software can be somehow used, Drasil is able to generate that usable
software. However, when the depth of the scientific knowledge transcribed in the
\acs{srs} is too shallow, Drasil's developers struggled to make adequate use of
the knowledge transcribed. To mitigate these issues, we added structure (depth)
to the scientific theories. In doing so, we had opened up further opportunities
for domain-specific interpretation, such as analysis, flexible printing, and,
most importantly, code generation. Dong's work \cite{Chen2022MEng} has
anecdotally shown the success of this. Furthermore, in
\Cref{chap:more-theory-kinds}, we began dissecting Drasil's currently encoded
theories and replaced their encodings with structured variants, in hopes of
further usage in code generation in the future.

\begin{enumerate}
    
    \item[\textbf{RQ2}] \textit{Drasils theory encodings rely on a single
          mathematical expression language, and doesn't expose information about
          applicability to different contexts. In each context (e.g., code,
          theories, and common arithmetic), certain terms of the expression
          language should be treated differently, or are simply inapplicable.
          How can we restrict term usage by context?}

\end{enumerate}

In \Cref{chap:lang-division}, we analyzed Drasil's single, universal
mathematical expression language, divided it according to Drasil's current
needs, and created a means of using the divided variants seamlessly through
creating a \acs{ttf} encoding of it's smart constructors. In doing this, we were
able to restrict the mathematical expressions admitted in concrete theories to
only those with definite values, where we have confidence that we can generate
code for.

\begin{enumerate}

    \item[\textbf{RQ3}] \textit{How can we ensure that our mathematical
          expression language admits only valid expressions?}

\end{enumerate}

In \Cref{chap:typed-expr}, we had created a definition of what it means for
expressions to be ``valid'' following common mathematical conventions by
creating a system of type-rules that our expressions must obey. Additionally, we
implemented a bidirectional type-checker to catch all expressions before they
are used in code generation. The type-checker we built had exposed many small
typing issues in Drasil, and trying to fix them quickly spiralled into exposing
some deep-rooted problems in Drasil. However, we also built the type-checker in
a very flexible way, so that we can later extend it to further needs, but also
so that we can adjust the type rules as needed.

\begin{enumerate}

    \item[\textbf{RQ4}] \textit{Our current ``typed'' approach to collecting
          different kinds of data is difficult to extend. How can we make it
          easier to extend?}

\end{enumerate}

In \Cref{chap:storingChunks}, we discussed how ``knowledge'' (chunks) is encoded
and stored in Drasil, and the essential requirements Drasil actually has of its
chunks. By reducing chunks to those requirements, we were able to extend
Drasil's chunk database to allowing for any chunk that fulfills said essential
needs to be stored in Drasil.

While there is a considerable amount of future work to be done
(\Cref{chap:future-work}), I'm sure that leftover work will lead to more work by
Drasil's very nature. Thus, with those leftover tasks in mind, I reflect.

\subsection{What I Learned}
\label{chap:conclusion:what-i-learned}

The entirety of \Cref{chap:ideology} is one of the most important things I've
learned by conducting this research. Another important thing I've learned is
that \textit{actual} communication of knowledge from one to others is far more
complicated than it seems, and that any piece of knowledge has considerable
nuance to it. By codifying our understandings of things, we're able to test just
how ``well-understood'' \cite{well-understood} they truly are to us.
