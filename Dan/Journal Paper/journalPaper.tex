\documentclass[format=acmsmall, review=false, screen=true]{acmart}

\usepackage{booktabs} % For formal tables

\usepackage[ruled]{algorithm2e} % For algorithms
\renewcommand{\algorithmcfname}{ALGORITHM}
\SetAlFnt{\small}
\SetAlCapFnt{\small}
\SetAlCapNameFnt{\small}
\SetAlCapHSkip{0pt}
\IncMargin{-\parindent}


% Metadata Information
\acmJournal{TOSEM}
%\acmVolume{}
%\acmNumber{}
%\acmArticle{}
%\acmYear{}
%\acmMonth{}
\copyrightyear{2018}
%\acmArticleSeq{9}

% Copyright
%\setcopyright{acmcopyright}
\setcopyright{acmlicensed}
%\setcopyright{rightsretained}
%\setcopyright{usgov}
%\setcopyright{usgovmixed}
%\setcopyright{cagov}
%\setcopyright{cagovmixed}

% DOI
\acmDOI{0000001.0000001}

% Paper history
%\received{February 2007}
%\received[revised]{March 2009}
%\received[accepted]{June 2009}


% Document starts
\begin{document}
% Title portion. Note the short title for running heads
\title[headertitle]{TITLE}

\author{Dan Szymczak}
%\orcid{1234-5678-9012-3456}
\affiliation{%
  \institution{McMaster University}
  \streetaddress{1280 Main St. W.}
  \city{Hamilton}
  \state{ON}
  \postcode{L8S 4K1}
  \country{Canada}}
\email{szymczdm@mcmaster.ca}
\author{Jacques Carette}
%\affiliation{%
%  \institution{Inria Paris-Rocquencourt}
%  \city{Rocquencourt}
%  \country{France}
%}
%\email{beranger@inria.fr}
\author{Spencer Smith}
%\affiliation{%
% \institution{Rajiv Gandhi University}
% \streetaddress{Rono-Hills}
% \city{Doimukh}
% \state{Arunachal Pradesh}
% \country{India}}
%\email{aprna_patel@rguhs.ac.in}
%\author{Huifen Chan}
%\affiliation{%
%  \institution{Tsinghua University}
%  \streetaddress{30 Shuangqing Rd}
%  \city{Haidian Qu}
%  \state{Beijing Shi}
%  \country{China}
%}
%\email{chan0345@tsinghua.edu.cn}
%\author{Ting Yan}
%\affiliation{%
%  \institution{Eaton Innovation Center}
%  \city{Prague}
%  \country{Czech Republic}}
%\email{yanting02@gmail.com}
%\author{Tian He}
%\affiliation{%
%  \institution{University of Virginia}
%  \department{School of Engineering}
%  \city{Charlottesville}
%  \state{VA}
%  \postcode{22903}
%  \country{USA}
%}
%\affiliation{%
%  \institution{University of Minnesota}
%  \country{USA}}
%\email{tinghe@uva.edu}
%\author{Chengdu Huang}
%\author{John A. Stankovic}
%\author{Tarek F. Abdelzaher}
%\affiliation{%
%  \institution{University of Virginia}
%  \department{School of Engineering}
%  \city{Charlottesville}
%  \state{VA}
%  \postcode{22903}
%  \country{USA}
%}


\begin{abstract}

CONTEXT: Software (re-)certification requires the creation and maintenance of many different software artifacts. Manually creating and maintaining them is tedious and costly.

OBJECTIVE: Improve software (re-)certification efforts by automating as much of the artifact creation process as possible while maintaining full traceability within -- and between -- artifacts. Creation of our tool -- Drasil -- will facilitate this automation process using a knowledge-based approach to Software Engineering.
%DS Secondary objective -> Knowledge reuse -- Don't know if I want to focus here as it muddies the waters.

METHOD: Use grounded theory in the creation of a tool for software artifact generation. Generate all the things! Capture the underlying knowledge and apply transformations to create each of the requisite artifacts. Captured knowledge can be re-used across projects as it represents the ``science''. Maintenance will then involve updating the captured knowledge or transformations as necessary. 

RESULTS: Case studies -- GlassBR to show capture and transformation. SWHS and NoPCM for reuse (Something about Kolmogorov complexity / MDL here?). 

CONCLUSIONS: Automatically generating software artifacts from captured knowledge (fill in later)?????

\end{abstract}


%
% The code below should be generated by the tool at
% http://dl.acm.org/ccs.cfm
% Please copy and paste the code instead of the example below.
%
%\begin{CCSXML}
%<ccs2012>
% <concept>
%  <concept_id>10010520.10010553.10010562</concept_id>
%  <concept_desc>Computer systems organization~Embedded systems</concept_desc>
%  <concept_significance>500</concept_significance>
% </concept>
% <concept>
%  <concept_id>10010520.10010575.10010755</concept_id>
%  <concept_desc>Computer systems organization~Redundancy</concept_desc>
%  <concept_significance>300</concept_significance>
% </concept>
% <concept>
%  <concept_id>10010520.10010553.10010554</concept_id>
%  <concept_desc>Computer systems organization~Robotics</concept_desc>
%  <concept_significance>100</concept_significance>
% </concept>
% <concept>
%  <concept_id>10003033.10003083.10003095</concept_id>
%  <concept_desc>Networks~Network reliability</concept_desc>
%  <concept_significance>100</concept_significance>
% </concept>
%</ccs2012>
%\end{CCSXML}

%\ccsdesc[500]{Computer systems organization~Embedded systems}
%\ccsdesc[300]{Computer systems organization~Redundancy}
%\ccsdesc{Computer systems organization~Robotics}
%\ccsdesc[100]{Networks~Network reliability}

%
% End generated code
%


\keywords{??}




\maketitle

% The default list of authors is too long for headers.
\renewcommand{\shortauthors}{D. Szymczak et al.}

\section{Introduction}\label{S:Intro}

\subsection{Certification}

\subsection{Document-Driven Design} %?

\section{Background}

\subsection{Previous efforts}
Previous attempts at automating / reducing the artifact burden.

\subsection{}


\end{document}
