\documentclass[12pt]{article}

\usepackage[round]{natbib}
\usepackage{url}
\usepackage{hyperref}

\hypersetup{
bookmarks=true,     % show bookmarks bar?
colorlinks=true,    % false: boxed links; true: colored links
linkcolor=blue,     % color of internal links (change box color with linkbordercolor)
citecolor=blue,     % color of links to bibliography
filecolor=magenta,  % color of file links
urlcolor=cyan       % color of external links
}

\usepackage{authblk}

\begin{document}

\title{Drasil Onboarding New Project Members}

\author{Spencer Smith: \href{mailto:smiths@mcmaster.ca}{smiths@mcmaster.ca}}
\date{\today}

\maketitle

\tableofcontents

\newpage

This document describes the onboarding process for new members of the Drasil
team.  The new members could be summer students, Masters students, PhD students,
volunteers, etc.  If a contributor is keen to begin, some of these tasks can be
started before their official start date, although there is no expectation to do
so.

Below you will find the following: a summary of your work colleagues
(Section~\ref{SecIntroductions}), practical information for summer research
assistants (Section~\ref{SecPractInfo}), a summary of the relevant repositories
(Section~\ref{SecRepos}), a list of initial tasks
(Section~\ref{SecInitialTasks}), advice from previous contributors
(Section~\ref{SecAdvice}) and a list of ``to do''s for Drs.\ Carette and Smith
(Section~\ref{SecToDo}). 

The details of your initial tasks are in Section~\ref{SecInitialTasks}.  The
first tasks listed are related to GitHub and GitLab access.  As described in
Section~\ref{SecInitialTasks}, please do the following as your first tasks:

\begin{itemize}
\item send Drs.\ Carette and Smith your GitHub username
\item register for GitLab (if you haven't already done so)
\end{itemize}

The most important getting-started advice is to remind you of the importance of
communication.  We'll do our best to communicate our requirements and
expectations.  You should likewise do your best to communicate when you are
confused, frustrated, bored etc.  Our goal is to keep you busy with a fun and
rewarding experience.

\section{Introductions} \label{SecIntroductions}

In addition to \href{https://www.cas.mcmaster.ca/~carette/} {Dr.\ Jacques
Carette} and \href{https://www.cas.mcmaster.ca/~smiths/} {Dr.\ Spencer Smith}
(smiths@mcmaster.ca), here is a list of the current (as of the date of
generating this document) members of the Drasil team:

\begin{itemize}
\item Jason Balaci, PhD candidate
\item Samuel Crawford, MASc candidate
\item Jiaming (Levi) Shao, MEng candidate
\item Mohammad Bilal, Summer research assistant
\item Noah Cardoso, Summer research assistant
\item Brandon Bosman, Summer research assistant
\item Xinlu Yan, Mitacs research assistant (starts July 15, 2024)
\end{itemize}

\section{Summer Research Assistant Practical Information} \label{SecPractInfo}

\begin{description}

\item [Start date:] Monday, May 6, 2023

\item [End date:] Friday, Aug 23, 2023

\item [Workspace:] ITB/236. You will need a proximity card to access your
office.  You will need to go to JHE 216A (Engineering Support Services
(\href{https://www.eng.mcmaster.ca/engineering-support-services-hub/} {The
Hub})) to get your card.  A deposit is required for the card.  If there are any
problems, please coordinate with the Departmental Administrator (Ms.\ Laurie
LeBlanc).  

\item [First day:] Come to ITB/167 to see Dr.\ Smith at 9:30 am.  We won't have
a meeting at this time, just a brief exchange about the practical details of
your first day.  You will want to get your access card, claim a spot in your office,
and begin reviewing these onBoarding instructions.

\item [Working from home:] Once we get our summer work rhythm established, you
can potentially work from home on some days, but please discuss this with Drs.\
Carette and Smith.

\item [Hours:] 35 hours per week, 7 hours per day (Monday to Friday) with a
one-hour unpaid lunch.  You can take a half-hour lunch if you prefer, but a
lunch break is required. We will maintain a regular work day.  There is some
flexibility on the start time.  Any time between 8:30 am and 9:30 am is fine.
In some cases, you may need to alter your work schedule for personal reasons.
This is fine, but we need to discuss the proposed alternatives.

Please use a spreadsheet to keep track of your hours and the tasks that you
spend your time on.  For the first week, please e-mail the spreadsheet to Drs.\
Smith and Carette at the end of each day.  The purpose of communicating this
information is to help advise and understand; it is not to ``check up on you.''

\item [Overview meeting:] Monday, May 6 at 1:30 pm in ITB/225.  Dr.\ Smith,
Jason Balaci and Sam Crawford will
\href{https://github.com/JacquesCarette/Drasil/issues/3715} {present} an
overview of Drasil's purpose and its implementation.  Dr.\ Carette will attend
if available.

\item [``All hands'' meetings:] Monday, May 6 at 3:00 pm.
\href{https://github.com/JacquesCarette/Drasil/issues/3714} {First meeting} of
the full summer research Drasil team.

\end{description}
For meetings, we prefer in-person whenever feasible.

\section {Repos} \label{SecRepos}

We use several repos for our work.  They are listed below in roughly in order of
importance.  In some cases, you will need to create an account or access will
have to be given.  The specific access-related tasks are given in the next
section (Section~\ref{SecInitialTasks}).

\subsection{GitHub}

\subsubsection{Drasil}
\begin{itemize}
\item \href{https://github.com/JacquesCarette/Drasil} {Drasil Repo}
\item public repo
\item the source code and documentation for Drasil
\item any code or documentation you write on Drasil will be here
\item you cannot push to ``main''
\item all contributions will be done through pull requests
\item this is the repo where you will be doing most of your work
\item you will need to be added as a contributor (see Section~\ref{SecInitialTasks})
\end{itemize}

\subsubsection{Drasil Generated Case Studies and Documentation}
\begin{itemize}
\item \href{https://jacquescarette.github.io/Drasil/} {Drasil Case Studies and Documentation}
\item not actually a repo, but generated from Drasil
\item the automated versions of the Drasil case studies are built frequently and
pushed to this web-page
\item the Haskell dependency graphs are also provided here
\item the \href{https://jacquescarette.github.io/Drasil/docs/index.html}
{documentation} for Drasil is available
\item \href{https://jacquescarette.github.io/Drasil/} {package dependency
graphs} are available at the bottom of this web page
\end{itemize}

\subsection{GitLab (CAS server)}

\subsubsection{Publications}
\begin{itemize}
\item \href{https://gitlab.cas.mcmaster.ca/smiths/pub} {Publications}
\item private (within CAS) repo
\item bibliographic information (in BibTeX) for papers and other resources
relevant to our project
\item pdf versions of papers that are hard to find online
\item when you create a bib file, look here (in the \texttt{References.bib}
file) first to see if the bib data is already available
\item if you find a new reference, please add it to the References.bib file,
along with a pdf version, if you don't have a link to an online version
\item citations should be named using the Author Year style.  For one or two
authors their last names are listed and then the year.  For more than two
authors, the
\href{https://www.mcgill.ca/library/files/library/cse-name-year-citation-style-guide.pdf}
{first author's last name is listed followed by Et al.}
\item contributors can push to ``main''
\item you need to be added as a contributor to this repo
\end{itemize}

\subsubsection{Software Engineering Course Notes}
\begin{itemize}
\item \href{https://gitlab.cas.mcmaster.ca/smiths/se2aa4_cs2me3} {SE2AA4/CS2ME3 Course Notes}
\item public repo
\item on some occasions, we may refer to some of the concepts or technology from
software engineering; this repo might be referenced in those situations
\item you cannot push, but you can do a pull request, if necessary
\end{itemize}

\subsubsection{Software Engineering for Science}
\begin{itemize}
\item \href{https://gitlab.cas.mcmaster.ca/SEforSC/se4sc} {se4sc repo}
\item private repo
\item resources 
\item grad student and undergrad student work
\item paper drafts
\item research proposal drafts
\item any documents you write that aren't part of Drasil will be put in this repo
\item contributors can push to ``main''
\item depending on your work you might not need this resource; we will let you
know when it is relevant
\item in case it is relevant, you will be added as a contributor to this repo
\end{itemize}

\section{Initial Tasks} \label{SecInitialTasks}

If you have questions or challenges while completing the steps below, please
make a record of your challenge.  We are always working to improve our
onboarding instructions and contributor's guide.  Please let us know of any
problems with the documentation so that can address the problem in the future.
There is a good chance we'll ask you to update the documentation, so the better
your notes, the easier the task will be.

Once you get settled, you can begin with the tasks listed here.  These tasks
should be done in roughly the order listed.  Some of the later tasks do not need
to be done when you start (like learning LaTeX).  They are tasks you can return
to throughout the summer when you need something to do or when you are feeling
like a change of pace, or waiting for feedback on your work.

\begin{enumerate} 
    
\item Verify you can access all GitLab accounts on the CAS server.  You can
access GitLab at the \href{https://gitlab.cas.mcmaster.ca/users/sign_in} {sign
in page}.  For CAS students, you can follow the instructions on the screen to
create an account, if you haven't already.  For the nonCAS students, we'll work
on getting you added by asking Derek (Sys Admin for CAS).  Please let Dr.\ Smith
know if you need us to request an account for you.

\item GitHub account.  If you do not have one, please create one.  Send your
account username to Dr.\ Smith and Dr.\ Carette.  Verify that you can access all
of the GitHub repos listed above.

\item Consent to Provide Limited Personal Information about Highly Qualified
Personnel (HQP) to NSERC.  Dr.\ Smith will send a separate e-mail about this.

\item Connect with Drs.\ Carette and Smith over \href{https://www.linkedin.com/}
{Linked-In}

\item Familiarize yourself with Drasil, review the quick start guide and set up
your new Drasil workspace.  The relevant links are as follows:

\begin{itemize}

\item \href{https://github.com/JacquesCarette/Drasil/wiki} {wiki}

\item \href{https://github.com/JacquesCarette/Drasil/wiki#what-is-drasil-cont}
{wiki - what is Drasil?}

\item \href{https://github.com/JacquesCarette/Drasil#quick-start} {quick start}

\item \href{https://github.com/JacquesCarette/Drasil/wiki/New-Workspace-Setup}
{new workspace setup}

\end{itemize}

The last link is particularly practical and useful.  You'll want to follow the
new workspace setup instructions to have a sane build environment.  If you have
any problems setting up your Drasil workspace, post an issue on GitHub.  The
issue should include the details of your OS, what you have tried, and any
relevant screenshots.

\item Learn the basics of git (if you don't already know them).  An overview of
git can be found from the following resources:

\begin{itemize} \item
\href{https://mcmasteru365.sharepoint.com/:v:/r/sites/SummerDrasil/Shared\%20Documents/General/Recordings/Git\%20Tutorial-20220912_143247-Meeting\%20Recording.mp4?csf=1&web=1&e=AmDDmv}
{Capstone tutorial by Sam Crawford on Teams}

\item
\href{https://gitlab.cas.mcmaster.ca/courses/capstone/-/tree/main/Tutorials/T01_GitGitHub}
{Capstone Tutorial Info}

\item
\href{https://gitlab.cas.mcmaster.ca/courses/capstone/-/blob/main/Tutorials/T01_GitGitHub/Git\%20Cheatsheet.pdf}
{Capstone Cheat Sheet (also by Sam)}

\item
\href{https://gitlab.cas.mcmaster.ca/smiths/se2aa4_cs2me3/-/tree/master/Tutorials/T01a-VM-VersionControl/slides}
{2AA4/2ME3 Tutorial}

\item \href{https://github.com/JacquesCarette/Drasil/wiki/Git2Know-for-Drasil}
{Git2Know for Drasil} 
\end{itemize}

\item Review the \href{https://github.com/JacquesCarette/Drasil/wiki/Contributor's-Guide} {Contributor's Guide}

Issue tracking is also described in the Contributor's Guide.  Please follow the
issue tracking guidelines. You should also review the existing issues in Drasil,
especially those that are currently open.

\item Once you have reviewed the material on git and contributing, please
complete the
\href{https://github.com/JacquesCarette/Drasil/blob/main/doc/Contributor's\%20Test/ContributorTest.pdf}
{Contributor's Test}.  Complete the version without answers and then compare
your answers to the correct ones.  If you have any questions, please ask.

\item We will be using Haskell this summer.  A good introductory text for
Haskell is available at \href{https://learnyouahaskell.github.io/} {Learn You a Haskell
for Great Good}.  You should have installed Haskell as part of the
\href{https://github.com/JacquesCarette/Drasil/wiki/New-Workspace-Setup} {New
Workspace Setup}.  Other Haskell resources include:

\begin{itemize}
\item Coursera course on
\href{https://www.coursera.org/learn/programming-languages} {Programming
Languages, Part A} (The course isn't specifically on Haskell, but the languages
used are similar enough)

\item McMaster Univ
\href{https://www.youtube.com/watch?v=eGwR_MiIT_A&list=PLknslYp7IpnJYHyJd02cOsp0ZBKxWBXK9}
{Comp Sci 1JC3 online lectures} 

\item McMaster Univ
\href{https://www.youtube.com/watch?v=7WxbuAztuFs&list=PLHRF-X-NtQR4MZBvm05NshPIEI8ELID5m}
{Comp Sci 1JC3 online tutorials} 

\end{itemize}

\item Start working on any issues assigned to you.  We usually start with the
issues labelled ``newcomers''.  Please do not hesitate to ask for help with (a)
solving issues (b) finding new issues.  You are also welcome to explore the
issue tracker to find other issue that interest you.

\item Read and review the following Drasil-related papers.  One paper describes
the Drasil framework.  Another is a book chapter that gives an overview of
rational documentation for scientific software.  The final paper provides an
overview of GOOL (Generic Object Oriented Library).  Well Understood paper.

\begin{itemize}

\item
\href{https://gitlab.cas.mcmaster.ca/smiths/pub/-/blob/master/SzymczakEtAl2016.pdf}
{Drasil Position Paper} 

\item \href{https://arxiv.org/abs/2302.00740} {Generating Software for
Well-Understood Domains}

\item \href{https://arxiv.org/abs/1911.11824} {GOOL paper}, if relevant to
your work, you can look at the.

\end{itemize}

\item If you are new to software development tools, you will want to review the
lessons on \href{http://software-carpentry.org/lessons/} {Software Carpentry}.
In particular, the lessons on the Unix Shell and make would be a good start.

\item Learn basics of LaTeX.  We will be using LaTeX for our written documents.
You can find an overview in the
\href{https://gitlab.cas.mcmaster.ca/smiths/se2aa4_cs2me3/-/tree/master/Tutorials\%2FT02a-LaTeX}
{SE2AA4/CS2MED tutorial}.

\end{enumerate}

\section{Advice from Previous Year's Research Assistants} \label{SecAdvice}

\begin{itemize}

\item ``Don't be afraid to ask questions.''

\item ``You rarely have to implement at a low level, chances are many of the
things you want to implement have already been implemented (i.e. check existing
modules and packages there might be similar functions).''

\item
\href{https://github.com/JacquesCarette/Drasil/wiki/Contributor's-Guide#note-to-future-summer-research-students}
{Contributor's Guide notes for summer research students}

\end{itemize}

\section{Drs.\ Carette and Smith To Do for New Contributors} \label{SecToDo}

\begin{itemize}
\item add contributor to Drasil repo (GitHub)
\item add contributor to se4sc and pub repos (GitLab)
\item request CAS accounts for non-CAS contributors
\item add contributor to Discord 
\item schedule meetings
\item assign initial Drasil issues
\item request Consent to Provide Limited Personal Information about Highly
Qualified Personnel (HQP) to NSERC e-mail
\item add contributors to Teams (Rsch Stdnt Meeting Team)
\end{itemize}

\section{Offboarding}

When your ``official'' time with the Drasil project is wrapping up, please consider
the items in the following checklist:

\begin{itemize}
\item Delete any branches from the repo that do not contain useful code
\item Ensure your names is added to the
\href{https://github.com/JacquesCarette/Drasil/blob/main/CITATION.cff}
{CITATION.cff} file
\end{itemize}

When your time as a grad student or employee is over, we encourage you to stay
as engaged in the Drasil project as your time and interest allows.

\end{document}