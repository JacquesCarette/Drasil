\documentclass[12pt, titlepage]{article}

\usepackage{booktabs}
\usepackage{tabularx}
\usepackage{hyperref}
\hypersetup{
    colorlinks,
    citecolor=blue,
    filecolor=black,
    linkcolor=red,
    urlcolor=blue
}
\usepackage[version=4]{mhchem}

\newcounter{testnum} %NFR Number
\newcommand{\tthetestnum}{NFR\thetestnum}
\newcommand{\testref}[1]{T\ref{#1}}

%% Comments

\usepackage{color}

\newif\ifcomments\commentstrue %displays comments
%\newif\ifcomments\commentsfalse %so that comments do not display

\ifcomments
\newcommand{\authornote}[3]{\textcolor{#1}{[#3 ---#2]}}
\newcommand{\todo}[1]{\textcolor{red}{[TODO: #1]}}
\else
\newcommand{\authornote}[3]{}
\newcommand{\todo}[1]{}
\fi

\newcommand{\wss}[1]{\authornote{blue}{SS}{#1}} 
\newcommand{\plt}[1]{\authornote{magenta}{TPLT}{#1}} %For explanation of the template
\newcommand{\sjc}[1]{\authornote{cyan}{SC}{#1}}
%% Common Parts

\newcommand{\progname}{Master of Applied Science} 
\newcommand{\authname}{ChemCode
\\ Samuel Crawford}

\usepackage{hyperref}
    \hypersetup{colorlinks=true, linkcolor=blue, citecolor=blue, filecolor=blue,
                urlcolor=blue, unicode=false}
    \urlstyle{same}
                                

\begin{document}

\title{%Project Title: 
  System Verification and Validation Plan for \progname{}}
\author{\authname}
\date{\today}

\maketitle

\pagenumbering{roman}

\section{Revision History}

\begin{tabularx}{\textwidth}{llX}
  \toprule {\bf Date} & {\bf Version} & {\bf Notes}                       \\
  \midrule
  Feb. 5, 2023        & 0.0           & Create document and remove
  inapplicable content                                                    \\
  Feb. 7-8, 2023      & 0.1           & Add input tests                   \\
  Feb. 8, 2023        & 0.1.1         & Improve referencing of tests      \\
                      & 0.1.2         & Add matrix conversion tests and
  improve input tests, including rationale, labelling, and chemical
  equations                                                               \\
  Feb. 9, 2023        & 0.1.3         & Add tests for trivial equation    \\
                      & 0.1.4         & Add feasibility tests             \\
  Feb. 13, 2023       & 0.1.5         & Clarify notion of matrices having
  the same solution after swapping rows and/or columns                    \\
  Feb. 13-14, 2023    & 0.1.6         & Add balancing and output tests    \\
  Feb. 14, 2023       & 0.1.7         & Add accuracy test for balancing   \\
  \bottomrule
\end{tabularx}

\newpage

\tableofcontents

\listoftables
\wss{Remove this section if it isn't needed}

\listoffigures
\wss{Remove this section if it isn't needed}

\newpage

\section{Symbols, Abbreviations and Acronyms}

\renewcommand{\arraystretch}{1.2}
\begin{tabular}{l l}
  \toprule
  \textbf{symbol} & \textbf{description} \\
  \midrule
  T               & Test                 \\
  \bottomrule
\end{tabular}\\

\wss{symbols, abbreviations or acronyms --- you can simply reference the SRS
  \cite{SRS} tables, if appropriate}

\wss{Remove this section if it isn't needed}

\newpage

\pagenumbering{arabic}

This document ... \wss{provide an introductory blurb and roadmap of the
  Verification and Validation plan}

\section{General Information}

\subsection{Summary}

\wss{Say what software is being tested.  Give its name and a brief overview of
  its general functions.}

\subsection{Objectives}

\wss{State what is intended to be accomplished.  The objective will be around
  the qualities that are most important for your project.  You might have
  something like: ``build confidence in the software correctness,''
  ``demonstrate adequate usability.'' etc.  You won't list all of the qualities,
  just those that are most important.}

\subsection{Relevant Documentation}

\wss{Reference relevant documentation.  This will definitely include your SRS
  and your other project documents (design documents, like MG, MIS, etc).  You
  can include these even before they are written, since by the time the project
  is done, they will be written.}

\cite{SRS}

\section{Plan}

\wss{Introduce this section.   You can provide a roadmap of the sections to
  come.}

\subsection{Verification and Validation Team}

\wss{Your teammates.  Maybe your supervisor.
  You shoud do more than list names.  You should say what each person's role is
  for the project's verification.  A table is a good way to summarize this information.}

\subsection{SRS Verification Plan}

\wss{List any approaches you intend to use for SRS verification.  This may include
  ad hoc feedback from reviewers, like your classmates, or you may plan for
  something more rigorous/systematic.}

\wss{Maybe create an SRS checklist?}

\subsection{Design Verification Plan}

\wss{Plans for design verification}

\wss{The review will include reviews by your classmates}

\wss{Create a checklists?}

\subsection{Verification and Validation Plan Verification Plan}

\wss{The verification and validation plan is an artifact that should also be verified.}

\wss{The review will include reviews by your classmates}

\wss{Create a checklists?}

\subsection{Implementation Verification Plan}

\wss{You should at least point to the tests listed in this document and the unit
  testing plan.}

\wss{In this section you would also give any details of any plans for static verification of
  the implementation.  Potential techniques include code walkthroughs, code
  inspection, static analyzers, etc.}

\subsection{Automated Testing and Verification Tools}

\wss{What tools are you using for automated testing.  Likely a unit testing
  framework and maybe a profiling tool, like ValGrind.  Other possible tools
  include a static analyzer, make, continuous integration tools, test coverage
  tools, etc.  Explain your plans for summarizing code coverage metrics.
  Linters are another important class of tools.  For the programming language
  you select, you should look at the available linters.  There may also be tools
  that verify that coding standards have been respected, like flake9 for
  Python.}

\wss{If you have already done this in the development plan, you can point to
  that document.}

\wss{The details of this section will likely evolve as you get closer to the
  implementation.}

\subsection{Software Validation Plan}

\wss{If there is any external data that can be used for validation, you should
  point to it here.  If there are no plans for validation, you should state that
  here.}

\wss{You might want to use review sessions with the stakeholder to check that
  the requirements document captures the right requirements.  Maybe task based
  inspection?}

\wss{This section might reference back to the SRS verification section.}

\section{System Test Description}

\subsection{Tests for Functional Requirements}

The tests in each section are given in order of increasing complexity/likelihood
of the situation arising during use of \progname{}.

\wss{Include a blurb here to explain why the subsections below
  cover the requirements.  References to the SRS would be good here.}

\subsubsection{Input Testing}

In order for \progname{} to be useful, it needs to be able to receive a
chemical equation from the user and store it to balance it later. This
section defines tests for inputting chemical equations from R1 of the SRS
\sjc{Add link}. \sjc{Justify the choice of these specific tests.}

\begin{enumerate}

  \item[T\refstepcounter{testnum}\thetestnum \label{test_small_valid_input}:]
    \textbf{Test for Small Valid Input}

    Control: Manual

    Initial State: \progname{} is started.

    Input: A representation of the equation $\ce{O2} \rightarrow \ce{O3}$
    \cite{fahey_twenty_2011}.

    Output: The inputted chemical equation is stored in \progname{}.

    Test Case Derivation: The inputted chemical equation is
    valid and trivial.

    How test will be performed: A debug statement will be added to display the
    stored chemical equation and this representation will be manually compared to
    the given input.

  \item[T\refstepcounter{testnum}\thetestnum \label{test_valid_input}:]
    \textbf{Test for Valid Input}

    Control: Manual

    Initial State: \progname{} is started.

    Input: A representation of the equation
    $\ce{C2H6} + \ce{O2} \rightarrow \ce{CO2} + \ce{H2O}$
    \cite{hamid_balancing_2019}.

    Output: The inputted chemical equation is stored in \progname{}.

    Test Case Derivation: The inputted chemical equation is valid and
    relatively small, but larger than the trivial one from
    \testref{test_small_valid_input}.

    How test will be performed: A debug statement will be added to display the
    stored chemical equation and this representation will be manually compared to
    the given input.

  \item[T\refstepcounter{testnum}\thetestnum \label{test_large_valid_input}:]
    \textbf{Test for Large Valid Input}

    Control: Manual

    Initial State: \progname{} is started.

    Input: A representation of the following equation from
    \cite{taylor_balancing_2021}.
    $$\ce{KMnO4} + \ce{HCl} \rightarrow \ce{MnCl2} + \ce{KCl} + \ce{Cl2} +
      \ce{H2O}$$

    Output: The inputted chemical equation is stored in \progname{}.

    Test Case Derivation: The inputted chemical equation is
    valid and larger than the one from \testref{test_valid_input}.

    How test will be performed: A debug statement will be added to display the
    stored chemical equation and this representation will be manually compared to
    the given input.

  \item[T\refstepcounter{testnum}\thetestnum \label{test_inf_over_valid_input}:]
    \textbf{Test for Valid Input that is Infeasible due to Overconstrained
      System}

    Control: Manual

    Initial State: \progname{} is started.

    Input: A representation of the following equation (modified from
    \cite{hamid_balancing_2019}).
    $$\ce{K4FeC6N6} + \ce{K2S2O3} \rightarrow \ce{CO2} + \ce{K2SO4} + \ce{NO2} +
      \ce{FeS}$$

    Output: The inputted chemical equation is stored in \progname{}.

    Test Case Derivation: The inputted chemical equation is infeasible since
    each compound has more than one element, so changing any coefficient
    affects the number of some other element, causing a chain reaction that
    does not converge. There is no solution to this system other than the
    trivial solution ($\mathbf{0}$) \cite{hamid_balancing_2019}.

    How test will be performed: A debug statement will be added to display the
    stored chemical equation and this representation will be manually compared to
    the given input.

  \item[T\refstepcounter{testnum}\thetestnum \label{test_inf_cons_mass_valid_input}:]
    \textbf{Test for Valid Input that is Infeasible due to Conservation of Mass
      Violation}

    Control: Manual

    Initial State: \progname{} is started.

    Input: A representation of the equation
    $\ce{C2H6} \rightarrow \ce{CO2} + \ce{H2O}$.

    Output: The inputted chemical equation is stored in \progname{}.

    Test Case Derivation: The inputted chemical equation is infeasible since
    every element does not exist on both sides of the equation, which violates
    the Law of Conservation of Mass (TM1 from SRS \sjc{add link}).

    How test will be performed: A debug statement will be added to display the
    stored chemical equation and this representation will be manually compared to
    the given input.

  \item[T\refstepcounter{testnum}\thetestnum \label{test_nonstoich_valid_input}:]
    \textbf{Test for Valid Input with Nonstoichiometric Compound}

    Control: Manual

    Initial State: \progname{} is started.

    Input: A representation of the equation
    $\ce{Fe_{0.95}O} + \ce{O2} \rightarrow \ce{Fe2O3}$
    \cite{doubtnut_when_nodate}.

    Output: The inputted chemical equation is stored in \progname{}.

    Test Case Derivation: The inputted chemical equation contains a
    nonstoichiometric compound (i.e., one with a fractional subscript).

    How test will be performed: A debug statement will be added to display the
    stored chemical equation and this representation will be manually compared to
    the given input.

\end{enumerate}

\subsubsection{Matrix Conversion Testing}

To solve a system of linear equations, \progname{} must first convert the
inputted chemical equation into matrix form. This section defines tests for
the matrix form conversion of chemical equations' stored representations from
R2 of the SRS \sjc{Add link}. \sjc{Justify the choice of these specific tests.}
Note that since swapping rows and/or columns in a matrix doesn't change its
solution, the matrices outputted may have their rows and/or columns in a
different order.

\begin{enumerate}

  \item[T\refstepcounter{testnum}\thetestnum \label{test_convert_small_valid}:]
    \textbf{Test for Converting Small Valid Chemical Equation}

    Control: Automated

    Initial State: \progname{} is started with a representation of the equation
    $\ce{O2} \rightarrow \ce{O3}$ \cite{fahey_twenty_2011} stored in memory.

    Input: --

    Output: The matrix
    $\begin{bmatrix}
        2 & -3
      \end{bmatrix}$ or
    $\begin{bmatrix}
        -3 & 2
      \end{bmatrix}$.

    Test Case Derivation: The stored chemical equation is valid and trivial.

    How test will be performed: The outputted matrix form of the stored
    chemical equation will be automatically compared to the expected output.

  \item[T\refstepcounter{testnum}\thetestnum \label{test_convert_valid}:]
    \textbf{Test for Converting Valid Chemical Equation}

    Control: Automated

    Initial State: \progname{} is started with a representation of the equation
    $\ce{C2H6} + \ce{O2} \rightarrow \ce{CO2} + \ce{H2O}$
    \cite{hamid_balancing_2019} stored in memory.

    Input: --

    Output: Some matrix with the rows and columns of the following matrix from
    \cite{hamid_balancing_2019}, potentially in a different order:
    $$\begin{bmatrix}
        2 & 0 & -1 & 0  \\
        6 & 0 & 0  & -2 \\
        0 & 2 & -2 & -1
      \end{bmatrix}$$

    Test Case Derivation: The stored chemical equation is valid and relatively
    small, but larger than the trivial one from
    \testref{test_convert_small_valid}.

    How test will be performed: The outputted matrix form of the stored
    chemical equation will be automatically compared to the expected output.

  \item[T\refstepcounter{testnum}\thetestnum \label{test_convert_large_valid}:]
    \textbf{Test for Converting Large Valid Chemical Equation}

    Control: Automated

    Initial State: \progname{} is started with a representation of the equation
    $\ce{KMnO4} + \ce{HCl} \rightarrow \ce{MnCl2} + \ce{KCl} + \ce{Cl2} +
      \ce{H2O}$ \cite{taylor_balancing_2021} stored in memory.

    Input: --

    Output: Some matrix with the rows and columns of the following matrix,
    potentially in a different order:
    $$\begin{bmatrix}
        1 & 0 & 0  & -1 & 0  & 0  \\
        1 & 0 & -1 & 0  & 0  & 0  \\
        4 & 0 & 0  & 0  & 0  & -1 \\
        0 & 1 & 0  & 0  & 0  & -2 \\
        0 & 1 & -2 & -1 & -2 & 0
      \end{bmatrix}$$

    Test Case Derivation: The stored chemical equation is
    valid and larger than the one from \testref{test_convert_valid}.

    How test will be performed: The outputted matrix form of the stored
    chemical equation will be automatically compared to the expected output.

  \item[T\refstepcounter{testnum}\thetestnum \label{test_convert_inf_over_valid}:]
    \textbf{Test for Converting Valid Chemical Equation that is Infeasible due
      to Overconstrained System}

    Control: Automated

    Initial State: \progname{} is started with a representation of the equation
    $\ce{K4FeC6N6} + \ce{K2S2O3} \rightarrow \ce{CO2} + \ce{K2SO4} + \ce{NO2} +
      \ce{FeS}$ \cite{hamid_balancing_2019} stored in memory.

    Input: --

    Output: Some matrix with the rows and columns of the following
    matrix\footnote{While
      \cite{hamid_balancing_2019} does not include a matrix representation, the
      system of equations it provides has a typo: the equation for $\ce{N}$
      should be $6x_1 = x_5$, since $\ce{NO2}$ only has one $\ce{N}$.},
    potentially in a different order:
    $$\begin{bmatrix}
        4 & 2 & 0  & -2 & 0  & 0  \\
        1 & 0 & 0  & 0  & 0  & -1 \\
        6 & 0 & -1 & 0  & 0  & 0  \\
        6 & 0 & 0  & 0  & -1 & 0  \\
        0 & 2 & 0  & -1 & 0  & -1 \\
        0 & 3 & -2 & -4 & -2 & 0
      \end{bmatrix}$$

    Test Case Derivation: The stored chemical equation is infeasible since
    each compound has more than one element, so changing any coefficient
    affects the number of some other element, causing a chain reaction that
    does not converge. There is no solution to this system other than the
    trivial solution ($\mathbf{0}$) \cite{hamid_balancing_2019}.

    How test will be performed: The outputted matrix form of the stored
    chemical equation will be automatically compared to the expected output.

  \item[T\refstepcounter{testnum}\thetestnum \label{test_convert_inf_cons_mass_valid}:]
    \textbf{Test for Converting Valid Chemical Equation that is Infeasible due
      to Conservation of Mass Violation}

    Control: Automated

    Initial State: \progname{} is started with a representation of the equation
    $\ce{C2H6} \rightarrow \ce{CO2} + \ce{H2O}$ stored in memory.

    Input: --

    Output: Some matrix with the rows and columns of the following matrix,
    potentially in a different order:
    $$\begin{bmatrix}
        2 & -1 & 0  \\
        6 & 0  & -2 \\
        0 & -2 & -1
      \end{bmatrix}$$

    Test Case Derivation: The stored chemical equation is infeasible since
    every element does not exist on both sides of the equation, which violates
    the Law of Conservation of Mass (TM1 from SRS \sjc{add link}).

    How test will be performed: The outputted matrix form of the stored
    chemical equation will be automatically compared to the expected output.

  \item[T\refstepcounter{testnum}\thetestnum \label{test_convert_nonstoich_valid}:]
    \textbf{Test for Converting Valid Chemical Equation with Nonstoichiometric
      Compound}

    Control: Automated

    Initial State: \progname{} is started with a representation of the equation
    $\ce{Fe_{0.95}O} + \ce{O2} \rightarrow \ce{Fe2O3}$
    \cite{doubtnut_when_nodate} stored in memory.

    Input: --

    Output: Some matrix with the rows and columns of the following matrix,
    potentially in a different order:
    $$\begin{bmatrix}
        0.95 & 0 & -2 \\
        1    & 2 & -3
      \end{bmatrix}$$

    Test Case Derivation: The stored chemical equation contains a
    nonstoichiometric compound (i.e., one with a fractional subscript).

    How test will be performed: The outputted matrix form of the stored
    chemical equation will be automatically compared to the expected output.

\end{enumerate}

\subsubsection{Feasibility Checking Testing}

To balance a chemical equation, the equation must be able to be balanced. This
section defines tests for determining if a chemical equation is feasible from
R3 of the SRS \sjc{Add link}. \sjc{Justify the choice of these specific tests.}

\begin{enumerate}

  \item[T\refstepcounter{testnum}\thetestnum \label{test_small_valid_feas}:]
    \textbf{Test for Feasibility of Small Valid Chemical Equation}

    Control: Automated

    Initial State: \progname{} is started.

    Input: $\begin{bmatrix}
        2 & -3
      \end{bmatrix}$.

    Output: $\textsc{T}$

    Test Case Derivation: The matrix represents a chemical equation that is
    valid and trivial.

    How test will be performed: The outputted Boolean will be automatically
    compared to the expected output.

  \item[T\refstepcounter{testnum}\thetestnum \label{test_valid_feas}:]
    \textbf{Test for Feasibility of Valid Chemical Equation}

    Control: Automated

    Initial State: \progname{} is started.

    Input:
    $\begin{bmatrix}
        2 & 0 & -1 & 0  \\
        6 & 0 & 0  & -2 \\
        0 & 2 & -2 & -1
      \end{bmatrix}$ from \cite{hamid_balancing_2019}.

    Output: $\textsc{T}$

    Test Case Derivation: The matrix represents a chemical equation that is
    valid and relatively small, but larger than the trivial one from
    \testref{test_small_valid_feas}.

    How test will be performed: The outputted Boolean will be automatically
    compared to the expected output.

  \item[T\refstepcounter{testnum}\thetestnum \label{test_large_valid_feas}:]
    \textbf{Test for Feasibility of Large Valid Chemical Equation}

    Control: Automated

    Initial State: \progname{} is started.

    Input: $\begin{bmatrix}
        1 & 0 & 0  & -1 & 0  & 0  \\
        1 & 0 & -1 & 0  & 0  & 0  \\
        4 & 0 & 0  & 0  & 0  & -1 \\
        0 & 1 & 0  & 0  & 0  & -2 \\
        0 & 1 & -2 & -1 & -2 & 0
      \end{bmatrix}$

    Output: $\textsc{T}$

    Test Case Derivation: The matrix represents a chemical equation that is
    valid and larger than the one from \testref{test_valid_feas}.

    How test will be performed: The outputted Boolean will be automatically
    compared to the expected output.

  \item[T\refstepcounter{testnum}\thetestnum \label{test_over_valid_inf}:]
    \textbf{Test for Feasibility of Valid Chemical Equation that is Infeasible
      due to Overconstrained System}

    Control: Automated

    Initial State: \progname{} is started.

    Input:
    $\begin{bmatrix}
        4 & 2 & 0  & -2 & 0  & 0  \\
        1 & 0 & 0  & 0  & 0  & -1 \\
        6 & 0 & -1 & 0  & 0  & 0  \\
        6 & 0 & 0  & 0  & -1 & 0  \\
        0 & 2 & 0  & -1 & 0  & -1 \\
        0 & 3 & -2 & -4 & -2 & 0
      \end{bmatrix}$ based on a linear system from
    \cite{hamid_balancing_2019}\footnote{While \cite{hamid_balancing_2019}
      does not include a matrix representation, the
      system of equations it provides has a typo: the equation for $\ce{N}$
      should be $6x_1 = x_5$, since $\ce{NO2}$ only has one $\ce{N}$.}.

    Output: $\textsc{F}$

    Test Case Derivation: The matrix represents a chemical equation that is
    infeasible since
    each compound has more than one element, so changing any coefficient
    affects the number of some other element, causing a chain reaction that
    does not converge. There is no solution to this system other than the
    trivial solution ($\mathbf{0}$) \cite{hamid_balancing_2019}.

    How test will be performed: The outputted Boolean will be automatically
    compared to the expected output.

  \item[T\refstepcounter{testnum}\thetestnum \label{test_cons_mass_valid_inf}:]
    \textbf{Test for Feasibility of  Valid Chemical Equation that is Infeasible
      due to Conservation of Mass Violation}

    Control: Automated

    Initial State: \progname{} is started.

    Input: $\begin{bmatrix}
        2 & -1 & 0  \\
        6 & 0  & -2 \\
        0 & -2 & -1
      \end{bmatrix}$

    Output: $\textsc{F}$

    Test Case Derivation: The matrix represents a chemical equation is
    infeasible since
    every element does not exist on both sides of the equation, which violates
    the Law of Conservation of Mass (TM1 from SRS \sjc{add link}).

    How test will be performed: The outputted Boolean will be automatically
    compared to the expected output.

  \item[T\refstepcounter{testnum}\thetestnum \label{test_nonstoich_valid_feas}:]
    \textbf{Test for Feasibility of Valid Chemical Equation with
      Nonstoichiometric Compound}

    Control: Automated

    Initial State: \progname{} is started.

    Input:
    $\begin{bmatrix}
        0.95 & 0 & -2 \\
        1    & 2 & -3
      \end{bmatrix}$

    Output: $\textsc{T}$

    Test Case Derivation: The matrix represents a chemical equation that
    contains a nonstoichiometric compound (i.e., one with a fractional
    subscript).

    How test will be performed: The outputted Boolean will be automatically
    compared to the expected output.

\end{enumerate}

\subsubsection{Balancing Testing}

The main purpose of \progname{} is to balance chemical equations, so this
functionality must be verified. This section defines tests for balancing a
chemical equation from
R4 of the SRS \sjc{Add link}. \sjc{Justify the choice of these specific tests.}
Note that since swapping rows and/or columns in a matrix doesn't change its
solution, the matrices outputted may have their rows and/or columns in a
different order, although swapping columns of the input matrix (e.g., columns
three and four) means that the associated rows of the output matrix (e.g., rows
three and four) must also be swapped.

\begin{enumerate}

  \item[T\refstepcounter{testnum}\thetestnum \label{test_bal_small_valid}:]
    \textbf{Test for Balancing Small Valid Chemical Equation}

    Control: Automated

    Initial State: \progname{} is started.

    Input: $\begin{bmatrix}
        2 & -3
      \end{bmatrix}$.

    Output: $\begin{bmatrix}
        3 \\
        2
      \end{bmatrix}$

    Test Case Derivation: The matrix represents a chemical equation that is
    valid and trivial.

    How test will be performed: The outputted matrix will be automatically
    compared to the expected output.

  \item[T\refstepcounter{testnum}\thetestnum \label{test_bal_valid}:]
    \textbf{Test for Balancing Valid Chemical Equation}

    Control: Automated

    Initial State: \progname{} is started.

    Input:
    $\begin{bmatrix}
        2 & 0 & -1 & 0  \\
        6 & 0 & 0  & -2 \\
        0 & 2 & -2 & -1
      \end{bmatrix}$ from \cite{hamid_balancing_2019}

    Output: $\begin{bmatrix}
        2 \\
        7 \\
        4 \\
        6
      \end{bmatrix}$ based on \cite{hamid_balancing_2019}

    Test Case Derivation: The matrix represents a chemical equation that is
    valid and relatively small, but larger than the trivial one from
    \testref{test_bal_small_valid}.

    How test will be performed: The outputted matrix will be automatically
    compared to the expected output.

  \item[T\refstepcounter{testnum}\thetestnum \label{test_bal_large_valid}:]
    \textbf{Test for Balancing Large Valid Chemical Equation}

    Control: Automated

    Initial State: \progname{} is started.

    Input: $\begin{bmatrix}
        1 & 0 & 0  & -1 & 0  & 0  \\
        1 & 0 & -1 & 0  & 0  & 0  \\
        4 & 0 & 0  & 0  & 0  & -1 \\
        0 & 1 & 0  & 0  & 0  & -2 \\
        0 & 1 & -2 & -1 & -2 & 0
      \end{bmatrix}$

    Output: $\begin{bmatrix}
        2  \\
        16 \\
        2  \\
        2  \\
        5  \\
        8
      \end{bmatrix}$

    Test Case Derivation: The matrix represents a chemical equation that is
    valid and larger than the one from \testref{test_bal_valid}.

    How test will be performed: The outputted matrix will be automatically
    compared to the expected output.

  \item[T\refstepcounter{testnum}\thetestnum \label{test_bal_nonstoich_valid}:]
    \textbf{Test for Balancing Valid Chemical Equation with
      Nonstoichiometric Compound}

    Control: Automated

    Initial State: \progname{} is started.

    Input:
    $\begin{bmatrix}
        0.95 & 0 & -2 \\
        1    & 2 & -3
      \end{bmatrix}$

    Output:  $\begin{bmatrix}
        80 \\
        17 \\
        38
      \end{bmatrix}$

    Test Case Derivation: The matrix represents a chemical equation that
    contains a nonstoichiometric compound (i.e., one with a fractional
    subscript).

    How test will be performed: The outputted matrix will be automatically
    compared to the expected output.

\end{enumerate}

\subsubsection{Infeasible Reaction Output Testing}

If a chemical equation is infeasible, it is important to notify the user as to
why, in case it is a mistake on the part of the user. This section defines
tests for outputting a descriptive message if an equation is infeasible from
R5 of the SRS \sjc{Add link}. \sjc{Justify the choice of these specific tests.}
The test cases here were chosen to represent different reason that a chemical
reaction could be infeasible.

\begin{enumerate}

  \item[T\refstepcounter{testnum}\thetestnum \label{test_over_valid_out}:]
    \textbf{Test for Output for Valid Chemical Equation that is Infeasible
      due to Overconstrained System}

    Control: Manual/Automated \sjc{TBD}

    Initial State: \progname{} is started and the following matrix from
    \cite{hamid_balancing_2019}\footnote{While \cite{hamid_balancing_2019}
      does not include a matrix representation, the
      system of equations it provides has a typo: the equation for $\ce{N}$
      should be $6x_1 = x_5$, since $\ce{NO2}$ only has one $\ce{N}$.} is
    stored in memory and has been determined to be infeasible:
    $$\begin{bmatrix}
        4 & 2 & 0  & -2 & 0  & 0  \\
        1 & 0 & 0  & 0  & 0  & -1 \\
        6 & 0 & -1 & 0  & 0  & 0  \\
        6 & 0 & 0  & 0  & -1 & 0  \\
        0 & 2 & 0  & -1 & 0  & -1 \\
        0 & 3 & -2 & -4 & -2 & 0
      \end{bmatrix}$$

    Input: --

    Output: A descriptive message stating that the inputted chemical reaction
    is infeasible because the system is overconstrained.

    Test Case Derivation: The matrix represents a chemical equation that is
    infeasible since
    each compound has more than one element, so changing any coefficient
    affects the number of some other element, causing a chain reaction that
    does not converge. There is no solution to this system other than the
    trivial solution ($\mathbf{0}$) \cite{hamid_balancing_2019}.

    How test will be performed: The output will be verified to ensure that an
    appropriate message is displayed to the user.

  \item[T\refstepcounter{testnum}\thetestnum \label{test_cons_mass_valid_out}:]
    \textbf{Test for Output for Valid Chemical Equation that is Infeasible
      due to Conservation of Mass Violation}

    Control: Manual/Automated \sjc{TBD}

    Initial State: \progname{} is started, and the following matrix is
    stored in memory and has been determined to be infeasible:
    $$\begin{bmatrix}
        2 & -1 & 0  \\
        6 & 0  & -2 \\
        0 & -2 & -1
      \end{bmatrix}$$

    Input: --

    Output: A descriptive message stating that the inputted chemical reaction
    is infeasible because an element is present on one side of the equation but
    not the other. \sjc{Is this test meaningful? The output should likely give
      the specific elements to the user, which isn't possible to ascertain just
      from the matrix representation.}

    Test Case Derivation: The matrix represents a chemical equation is
    infeasible since
    every element does not exist on both sides of the equation, which violates
    the Law of Conservation of Mass (TM1 from SRS \sjc{add link}).

    How test will be performed: The output will be verified to ensure that an
    appropriate message is displayed to the user.
\end{enumerate}

\subsubsection{Feasible Reaction Output Testing}

Once a chemical reaction is balanced, it must be given to the user for them
to be able to use it. This section defines tests for outputting a
representation of the balanced chemical equation from
R6 of the SRS \sjc{Add link}. \sjc{Justify the choice of these specific tests.}

\begin{enumerate}

  \item[T\refstepcounter{testnum}\thetestnum \label{test_small_valid_out}:]
    \textbf{Test for Output for Small Valid Chemical Equation}

    Control: Manual/Automated \sjc{TBD}

    Initial State: \progname{} is started, and some form of the original
    equation $\ce{O2} \rightarrow \ce{O3}$ and the coefficient matrix
    $\begin{bmatrix}
        3 \\
        2
      \end{bmatrix}$ are stored in memory.

    Input: --

    Output: A representation of the equation $\ce{3O2} \rightarrow \ce{2O3}$
    \cite[p.~6]{fahey_twenty_2011}.

    Test Case Derivation: This chemical equation is valid and trivial.

    How test will be performed: \sjc{TBD}

  \item[T\refstepcounter{testnum}\thetestnum \label{test_valid_out}:]
    \textbf{Test for Output for Valid Chemical Equation}

    Control: Manual/Automated \sjc{TBD}

    Initial State: \progname{} is started, and some form of the original
    equation $\ce{C2H6} + \ce{O2} \rightarrow \ce{CO2} + \ce{H2O}$
    \cite{hamid_balancing_2019} and the following coefficient matrix
    based on \cite{hamid_balancing_2019} are stored in memory:
    $$\begin{bmatrix}
        2 \\
        7 \\
        4 \\
        6
      \end{bmatrix}$$

    Input: --

    Output: A representation of the following equation from
    \cite[p.~523]{hamid_balancing_2019}:
    $$\ce{2C2H6} + \ce{7O2} \rightarrow \ce{4CO2} + \ce{6H2O}$$

    Test Case Derivation: This chemical equation is
    valid and relatively small, but larger than the trivial one from
    \testref{test_small_valid_out}.

    How test will be performed: \sjc{TBD}

  \item[T\refstepcounter{testnum}\thetestnum \label{test_large_valid_out}:]
    \textbf{Test for Output for Large Valid Chemical Equation}

    Control: Manual/Automated \sjc{TBD}

    Initial State: \progname{} is started, and some form of the original
    equation $\ce{KMnO4} + \ce{HCl} \rightarrow \ce{MnCl2} + \ce{KCl} +
      \ce{Cl2} + \ce{H2O}$ \cite{taylor_balancing_2021} and the following
    coefficient matrix are stored in memory:
    $$\begin{bmatrix}
        2  \\
        16 \\
        2  \\
        2  \\
        5  \\
        8
      \end{bmatrix}$$

    Input: --

    Output: A representation of the following equation from
    \cite{taylor_balancing_2021}:
    $$\ce{2KMnO4} + \ce{16HCl} \rightarrow \ce{2MnCl2} + \ce{2KCl} +
      \ce{5Cl2} + \ce{8H2O}$$


    Test Case Derivation: This chemical equation is
    valid and larger than the one from \testref{test_valid_out}.

    How test will be performed: \sjc{TBD}

  \item[T\refstepcounter{testnum}\thetestnum \label{test_nonstoich_valid_out}:]
    \textbf{Test for Output for Valid Chemical Equation with
      Nonstoichiometric Compound}

    Control: Manual/Automated \sjc{TBD}

    Initial State: \progname{} is started, and some form of the original
    equation $\ce{Fe_{0.95}O} + \ce{O2} \rightarrow \ce{Fe2O3}$
    \cite{doubtnut_when_nodate} and the following
    coefficient matrix are stored in memory:
    $$\begin{bmatrix}
        80 \\
        17 \\
        38
      \end{bmatrix}$$

    Input: --

    Output: A representation of the following equation:
    $$\ce{80Fe_{0.95}O} + \ce{17O2} \rightarrow \ce{38Fe2O3}$$

    Test Case Derivation: This chemical equation
    contains a nonstoichiometric compound (i.e., one with a fractional
    subscript).

    How test will be performed: \sjc{TBD}

\end{enumerate}

\subsection{Tests for Nonfunctional Requirements}

\subsubsection{Accuracy Testing}

\begin{enumerate}

  \item[T\refstepcounter{testnum}\thetestnum \label{test_bal_accuracy}:]
    \textbf{Test for Accuracy of Balancing}

    Control: Automated

    Initial State: \progname{} is started, and the following test cases are
    run: \testref{test_bal_small_valid}, \testref{test_bal_valid},
    \testref{test_bal_large_valid}, \testref{test_bal_nonstoich_valid}.

    Input: See the relevant test cases above for their input.

    Output: See the relevant test cases above for their output.

    Test Case Derivation: Chemical equations are only useful if they are
    balanced \cite{lund_introduction_2023}, so computed coefficients should be
    exact. Since these coefficients should be as small as possible
    \cite{lund_introduction_2023}, there is exactly one possible set of
    coefficients that for each feasible chemical equation, and since they are
    whole numbers \cite{lund_introduction_2023}, they can be compared for
    equality by a computer.

    How test will be performed: For each test case, the actual output will be
    compared to the expected output to ensure that they are exactly equal
    (e.g., by using integer comparison).

\end{enumerate}

\wss{Tests related to usability could include conducting a usability test and
  survey.  The survey will be in the Appendix.}

\wss{Static tests, review, inspections, and walkthroughs, will not follow the
  format for the tests given below.}

\subsubsection{Area of Testing1}

\paragraph{Title for Test}

\begin{enumerate}

  \item[T\refstepcounter{testnum}\thetestnum \label{test_label}:]
    \textbf{Title of Test}

    Type: Functional, Dynamic, Manual, Static etc.

    Initial State:

    Input/Condition:

    Output/Result:

    How test will be performed:

  \item[T\refstepcounter{testnum}\thetestnum \label{test_label2}:]
    \textbf{Title of Test}

    Type: Functional, Dynamic, Manual, Static etc.

    Initial State:

    Input:

    Output:

    How test will be performed:

\end{enumerate}

\subsection{Traceability Between Test Cases and Requirements}

\wss{Provide a table that shows which test cases are supporting which
  requirements.}

\section{Unit Test Description}

\wss{Reference your MIS (detailed design document) and explain your overall
  philosophy for test case selection.}
\wss{This section should not be filled in until after the MIS (detailed design
  document) has been completed.}

\subsection{Unit Testing Scope}

\wss{What modules are outside of the scope.  If there are modules that are
  developed by someone else, then you would say here if you aren't planning on
  verifying them.  There may also be modules that are part of your software, but
  have a lower priority for verification than others.  If this is the case,
  explain your rationale for the ranking of module importance.}

\subsection{Tests for Functional Requirements}

\wss{Most of the verification will be through automated unit testing.  If
  appropriate specific modules can be verified by a non-testing based
  technique.  That can also be documented in this section.}

\subsubsection{Module 1}

\wss{Include a blurb here to explain why the subsections below cover the module.
  References to the MIS would be good.  You will want tests from a black box
  perspective and from a white box perspective.  Explain to the reader how the
  tests were selected.}

\begin{enumerate}

  \item{test-id1\\}

  Type: \wss{Functional, Dynamic, Manual, Automatic, Static etc. Most will
    be automatic}

  Initial State:

  Input:

  Output: \wss{The expected result for the given inputs}

  Test Case Derivation: \wss{Justify the expected value given in the Output field}

  How test will be performed:

  \item{test-id2\\}

  Type: \wss{Functional, Dynamic, Manual, Automatic, Static etc. Most will
    be automatic}

  Initial State:

  Input:

  Output: \wss{The expected result for the given inputs}

  Test Case Derivation: \wss{Justify the expected value given in the Output field}

  How test will be performed:

  \item{...\\}

\end{enumerate}

\subsubsection{Module 2}

...

\subsection{Tests for Nonfunctional Requirements}

\wss{If there is a module that needs to be independently assessed for
  performance, those test cases can go here.  In some projects, planning for
  nonfunctional tests of units will not be that relevant.}

\wss{These tests may involve collecting performance data from previously
  mentioned functional tests.}

\subsubsection{Module ?}

\begin{enumerate}

  \item{test-id1\\}

  Type: \wss{Functional, Dynamic, Manual, Automatic, Static etc. Most will
    be automatic}

  Initial State:

  Input/Condition:

  Output/Result:

  How test will be performed:

  \item{test-id2\\}

  Type: Functional, Dynamic, Manual, Static etc.

  Initial State:

  Input:

  Output:

  How test will be performed:

\end{enumerate}

\subsubsection{Module ?}

...

\subsection{Traceability Between Test Cases and Modules}

\wss{Provide evidence that all of the modules have been considered.}

\bibliographystyle{ieeetr}

\bibliography{../sources}

\newpage

\section{Appendix}

This is where you can place additional information.

\subsection{Symbolic Parameters}

The definition of the test cases will call for SYMBOLIC\_CONSTANTS.
Their values are defined in this section for easy maintenance.

\subsection{Usability Survey Questions?}

\wss{This is a section that would be appropriate for some projects.}

\end{document}