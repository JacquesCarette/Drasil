\documentclass[12pt]{article}
\usepackage{fontspec}
\usepackage{fullpage}
\usepackage{hyperref}
\hypersetup{bookmarks=true,colorlinks=true,linkcolor=red,citecolor=blue,filecolor=magenta,urlcolor=cyan}
\usepackage{amsmath}
\usepackage{amssymb}
\usepackage{mathtools}
\usepackage{unicode-math}
\usepackage{tabu}
\usepackage{longtable}
\usepackage{booktabs}
\usepackage{caption}
\usepackage{graphics}
\usepackage{svg}
\usepackage{enumitem}
\usepackage{filecontents}
\usepackage[backend=bibtex]{biblatex}
\usepackage{url}
\setmathfont{Latin Modern Math}
\newcommand{\gt}{\ensuremath >}
\newcommand{\lt}{\ensuremath <}
\global\tabulinesep=1mm
\newlist{symbDescription}{description}{1}
\setlist[symbDescription]{noitemsep, topsep=0pt, parsep=0pt, partopsep=0pt}
\bibliography{bibfile}
\title{Software Requirements Specification for Pendulum}
\author{Dong Chen}
\begin{document}
\maketitle
\tableofcontents
\newpage
\section{Reference Material}
\label{Sec:RefMat}
This section records information for easy reference.

\subsection{Table of Units}
\label{Sec:ToU}
The unit system used throughout is SI (Système International d'Unités). In addition to the basic units, several derived units are also used. For each unit, the \hyperref[Table:ToU]{Table of Units} lists the symbol, a description and the SI name.

\begin{longtable}{l l l}
\toprule
\textbf{Symbol} & \textbf{Description} & \textbf{SI Name}
\\
\midrule
\endhead
${{}^{\circ}}$ & angle & degree
\\
${\text{kg}}$ & mass & kilogram
\\
${\text{m}}$ & length & metre
\\
${\text{N}}$ & force & newton
\\
${\text{rad}}$ & angle & radian
\\
${\text{s}}$ & time & second
\\
\bottomrule
\caption{Table of Units}
\label{Table:ToU}
\end{longtable}
\subsection{Table of Symbols}
\label{Sec:ToS}
The symbols used in this document are summarized in the \hyperref[Table:ToS]{Table of Symbols} along with their units. Throughout the document, symbols in bold will represent vectors, and scalars otherwise. The symbols are listed in alphabetical order. For vector quantities, the units shown are for each component of the vector.

\begin{longtable}{l l l}
\toprule
\textbf{Symbol} & \textbf{Description} & \textbf{Units}
\\
\midrule
\endhead
${a_{\text{x}1}}$ & The horizontal acceleration of the first object & $\frac{\text{m}}{\text{s}^{2}}$
\\
${a_{\text{x}2}}$ & The horizontal acceleration of the second object & $\frac{\text{m}}{\text{s}^{2}}$
\\
${a_{\text{y}1}}$ & The vertical acceleration of the first object & $\frac{\text{m}}{\text{s}^{2}}$
\\
${a_{\text{y}2}}$ & The vertical acceleration of the second object & $\frac{\text{m}}{\text{s}^{2}}$
\\
$\symbf{a}$ & Acceleration & $\frac{\text{m}}{\text{s}^{2}}$
\\
$\symbf{F}$ & Force & ${\text{N}}$
\\
$\symbf{g}$ & Gravitational acceleration & $\frac{\text{m}}{\text{s}^{2}}$
\\
$\symbf{\hat{i}}$ & Unit vector & --
\\
${L_{1}}$ & The length of the first rod & ${\text{m}}$
\\
${L_{2}}$ & The length of the second rod & ${\text{m}}$
\\
$m$ & Mass & ${\text{kg}}$
\\
${m_{1}}$ & The mass of the first object & ${\text{kg}}$
\\
${m_{2}}$ & The mass of the second object & ${\text{kg}}$
\\
${p_{\text{x}1}}$ & The horizontal position of the first object & ${\text{m}}$
\\
${p_{\text{x}2}}$ & The horizontal position of the second object & ${\text{m}}$
\\
${p_{\text{y}1}}$ & The vertical position of the first object & ${\text{m}}$
\\
${p_{\text{y}2}}$ & The vertical position of the second object & ${\text{m}}$
\\
$\symbf{p}$ & Position & ${\text{m}}$
\\
$\symbf{T}$ & Tension & ${\text{N}}$
\\
${\symbf{T}_{1}}$ & The tension of the first object & ${\text{N}}$
\\
${\symbf{T}_{2}}$ & The tension of the second object & ${\text{N}}$
\\
$t$ & Time & ${\text{s}}$
\\
${v_{\text{x}1}}$ & The horizontal velocity of the first object & $\frac{\text{m}}{\text{s}}$
\\
${v_{\text{x}2}}$ & The horizontal velocity of the second object & $\frac{\text{m}}{\text{s}}$
\\
${v_{\text{y}1}}$ & The vertical velocity of the first object & $\frac{\text{m}}{\text{s}}$
\\
${v_{\text{y}2}}$ & The vertical velocity of the second object & $\frac{\text{m}}{\text{s}}$
\\
$\symbf{v}$ & Velocity & $\frac{\text{m}}{\text{s}}$
\\
${w_{1}}$ & The angular velocity of the first object & $\frac{\text{rad}}{\text{s}}$
\\
${w_{2}}$ & The angular velocity of the second object & $\frac{\text{rad}}{\text{s}}$
\\
${α_{1}}$ & The angular acceleration of the first object & $\frac{\text{rad}}{\text{s}^{2}}$
\\
${α_{2}}$ & The angular acceleration of the second object & $\frac{\text{rad}}{\text{s}^{2}}$
\\
${θ_{1}}$ & The angle of the first rod & ${{}^{\circ}}$
\\
${θ_{2}}$ & The angle of the second rod & ${{}^{\circ}}$
\\
$π$ & Ratio of circumference to diameter for any circle & --
\\
\bottomrule
\caption{Table of Symbols}
\label{Table:ToS}
\end{longtable}
\subsection{Abbreviations and Acronyms}
\label{Sec:TAbbAcc}
\begin{longtable}{l l}
\toprule
\textbf{Abbreviation} & \textbf{Full Form}
\\
\midrule
\endhead
2D & Two-Dimensional
\\
A & Assumption
\\
DD & Data Definition
\\
GD & General Definition
\\
GS & Goal Statement
\\
IM & Instance Model
\\
PS & Physical System Description
\\
R & Requirement
\\
SRS & Software Requirements Specification
\\
TM & Theoretical Model
\\
Uncert. & Typical Uncertainty
\\
\bottomrule
\caption{Abbreviations and Acronyms}
\label{Table:TAbbAcc}
\end{longtable}
\section{Introduction}
\label{Sec:Intro}
A pendulum consists of mass attached to the end of a rod and its moving curve is highly sensitive to initial conditions. Therefore, it is useful to have a program to simulate the motion of the pendulum to exhibit its chaotic characteristics. The program documented here is called Pendulum.

The following section provides an overview of the Software Requirements Specification (SRS) for Pendulum. This section explains the purpose of this document, the scope of the requirements, the characteristics of the intended reader, and the organization of the document.

\subsection{Purpose of Document}
\label{Sec:DocPurpose}
The primary purpose of this document is to record the requirements of the Pendulum. Goals, assumptions, theoretical models, definitions, and other model derivation information are specified, allowing the reader to fully understand and verify the purpose and scientific basis of DblPendulum. With the exception of \hyperref[Sec:SysConstraints]{system constraints}, this SRS will remain abstract, describing what problem is being solved, but not how to solve it.

This document will be used as a starting point for subsequent development phases, including writing the design specification and the software verification and validation plan. The design document will show how the requirements are to be realized, including decisions on the numerical algorithms and programming environment. The verification and validation plan will show the steps that will be used to increase confidence in the software documentation and the implementation. Although the SRS fits in a series of documents that follow the so-called waterfall model, the actual development process is not constrained in any way. Even when the waterfall model is not followed, as Parnas and Clements point out \cite{parnasClements1986}, the most logical way to present the documentation is still to ``fake'' a rational design process.

\subsection{Scope of Requirements}
\label{Sec:ReqsScope}
The scope of the requirements includes the analysis of a two-dimensional (2D) pendulum motion problem with various initial conditions.

\subsection{Characteristics of Intended Reader}
\label{Sec:ReaderChars}
Reviewers of this documentation should have an understanding of undergraduate level 2 physics, undergraduate level 1 calculus, and ordinary differential equations. The users of DblPendulum can have a lower level of expertise, as explained in \hyperref[Sec:UserChars]{Sec:User Characteristics}.

\subsection{Organization of Document}
\label{Sec:DocOrg}
The organization of this document follows the template for an SRS for scientific computing software proposed by \cite{koothoor2013} and \cite{smithLai2005}. The presentation follows the standard pattern of presenting goals, theories, definitions, and assumptions. For readers that would like a more bottom up approach, they can start reading the \hyperref[Sec:IMs]{instance models} and trace back to find any additional information they require.

The \hyperref[Sec:GoalStmt]{goal statements} are refined to the theoretical models and the \hyperref[Sec:TMs]{theoretical models} to the \hyperref[Sec:IMs]{instance models}.

\section{General System Description}
\label{Sec:GenSysDesc}
This section provides general information about the system. It identifies the interfaces between the system and its environment, describes the user characteristics, and lists the system constraints.

\subsection{System Context}
\label{Sec:SysContext}
\hyperref[Figure:sysCtxDiag]{Fig:sysCtxDiag} shows the system context. A circle represents an entity external to the software, the user in this case. A rectangle represents the software system itself (DblPendulum). Arrows are used to show the data flow between the system and its environment.

\begin{figure}
\begin{center}
\includegraphics[width=\textwidth]{../../../../datafiles/dblpendulum/SystemContextFigure.png}
\caption{System Context}
\label{Figure:sysCtxDiag}
\end{center}
\end{figure}
The interaction between the product and the user is through an application programming interface. The responsibilities of the user and the system are as follows:

\begin{itemize}
\item{User Responsibilities}
\begin{itemize}
\item{Provide initial conditions of the physical state of the motion and the input data related to the Pendulum, ensuring no errors in the data entry.}
\item{Ensure that consistent units are used for input variables.}
\item{Ensure required \hyperref[Sec:Assumps]{software assumptions} are appropriate for any particular problem input to the software.}
\end{itemize}
\item{DblPendulum Responsibilities}
\begin{itemize}
\item{Detect data type mismatch, such as a string of characters input instead of a floating point number.}
\item{Determine if the inputs satisfy the required physical and software constraints.}
\item{Calculate the required outputs.}
\item{Generate the required graphs.}
\end{itemize}
\end{itemize}
\subsection{User Characteristics}
\label{Sec:UserChars}
The end user of DblPendulum should have an understanding of high school physics, high school calculus and ordinary differential equations.

\subsection{System Constraints}
\label{Sec:SysConstraints}
There are no system constraints.

\section{Specific System Description}
\label{Sec:SpecSystDesc}
This section first presents the problem description, which gives a high-level view of the problem to be solved. This is followed by the solution characteristics specification, which presents the assumptions, theories, and definitions that are used.

\subsection{Problem Description}
\label{Sec:ProbDesc}
A system is needed to efficiently and correctly to predict the motion of a pendulum.

\subsubsection{Terminology and Definitions}
\label{Sec:TermDefs}
This subsection provides a list of terms that are used in the subsequent sections and their meaning, with the purpose of reducing ambiguity and making it easier to correctly understand the requirements.

\begin{itemize}
\item{Gravity: The force that attracts one physical body with mass to another.}
\item{Cartesian coordinate system: A coordinate system that specifies each point uniquely in a plane by a set of numerical coordinates, which are the signed distances to the point from two fixed perpendicular oriented lines, measured in the same unit of length (from \cite{cartesianWiki}).}
\end{itemize}
\subsubsection{Physical System Description}
\label{Sec:PhysSyst}
The physical system of DblPendulum, as shown in \hyperref[Figure:dblpendulum]{Fig:dblpendulum}, includes the following elements:

\begin{itemize}
\item[PS1:]{The first rod (with the length of the first rod ${L_{1}}$).}
\item[PS2:]{The second rod (with the length of the second rod ${L_{2}}$).}
\item[PS3:]{The first object.}
\item[PS4:]{The second object.}
\end{itemize}
\begin{figure}
\begin{center}
\includegraphics[width=0.6\textwidth]{../../../../datafiles/dblpendulum/dblpendulum.png}
\caption{The physical system}
\label{Figure:dblpendulum}
\end{center}
\end{figure}
\subsubsection{Goal Statements}
\label{Sec:GoalStmt}
Given the masses, length of the rods, initial angle of the masses and the gravitational constant, the goal statements are:

\begin{itemize}
\item[motionMass:\phantomsection\label{motionMass}]{Calculate the motion of the masses.}
\end{itemize}
\subsection{Solution Characteristics Specification}
\label{Sec:SolCharSpec}
The instance models that govern DblPendulum are presented in the \hyperref[Sec:IMs]{Instance Model Section}. The information to understand the meaning of the instance models and their derivation is also presented, so that the instance models can be verified.

\subsubsection{Assumptions}
\label{Sec:Assumps}
This section simplifies the original problem and helps in developing the theoretical models by filling in the missing information for the physical system. The assumptions refine the scope by providing more detail.

\begin{itemize}
\item[twoDMotion:\phantomsection\label{twoDMotion}]{The pendulum motion is two-dimensional (2D).}
\item[cartSys:\phantomsection\label{cartSys}]{A Cartesian coordinate system is used.}
\item[cartSysR:\phantomsection\label{cartSysR}]{The Cartesian coordinate system is right-handed where positive $x$-axis and $y$-axis point right up.}
\item[yAxisDir:\phantomsection\label{yAxisDir}]{The direction of the $y$-axis is directed opposite to gravity.}
\item[startOrigin:\phantomsection\label{startOrigin}]{The first rod is attached to the origin.}
\item[firstPend:\phantomsection\label{firstPend}]{The first rod has two sides. One side attaches to the origin. Another side attaches to the first object.}
\item[secondPend:\phantomsection\label{secondPend}]{The second rod has two sides. One side attaches to the first object. Another side attaches to the second object.}
\end{itemize}
\subsubsection{Theoretical Models}
\label{Sec:TMs}
This section focuses on the general equations and laws that DblPendulum is based on.

\vspace{\baselineskip}
\noindent
\begin{minipage}{\textwidth}
\begin{tabular}{>{\raggedright}p{0.13\textwidth}>{\raggedright\arraybackslash}p{0.82\textwidth}}
\toprule \textbf{Refname} & \textbf{TM:acceleration}
\phantomsection 
\label{TM:acceleration}
\\ \midrule \\
Label & Acceleration
        
\\ \midrule \\
Equation & \begin{displaymath}
           \symbf{a}=\frac{\,d\symbf{v}}{\,dt}
           \end{displaymath}
\\ \midrule \\
Description & \begin{symbDescription}
              \item{$\symbf{a}$ is the acceleration ($\frac{\text{m}}{\text{s}^{2}}$)}
              \item{$t$ is the time (${\text{s}}$)}
              \item{$\symbf{v}$ is the velocity ($\frac{\text{m}}{\text{s}}$)}
              \end{symbDescription}
\\ \midrule \\
Source & \cite{accelerationWiki}
         
\\ \midrule \\
RefBy & 
\\ \bottomrule
\end{tabular}
\end{minipage}
\vspace{\baselineskip}
\noindent
\begin{minipage}{\textwidth}
\begin{tabular}{>{\raggedright}p{0.13\textwidth}>{\raggedright\arraybackslash}p{0.82\textwidth}}
\toprule \textbf{Refname} & \textbf{TM:velocity}
\phantomsection 
\label{TM:velocity}
\\ \midrule \\
Label & Velocity
        
\\ \midrule \\
Equation & \begin{displaymath}
           \symbf{v}=\frac{\,d\symbf{p}}{\,dt}
           \end{displaymath}
\\ \midrule \\
Description & \begin{symbDescription}
              \item{$\symbf{v}$ is the velocity ($\frac{\text{m}}{\text{s}}$)}
              \item{$t$ is the time (${\text{s}}$)}
              \item{$\symbf{p}$ is the position (${\text{m}}$)}
              \end{symbDescription}
\\ \midrule \\
Source & \cite{velocityWiki}
         
\\ \midrule \\
RefBy & 
\\ \bottomrule
\end{tabular}
\end{minipage}
\vspace{\baselineskip}
\noindent
\begin{minipage}{\textwidth}
\begin{tabular}{>{\raggedright}p{0.13\textwidth}>{\raggedright\arraybackslash}p{0.82\textwidth}}
\toprule \textbf{Refname} & \textbf{TM:NewtonSecLawMot}
\phantomsection 
\label{TM:NewtonSecLawMot}
\\ \midrule \\
Label & Newton's second law of motion
        
\\ \midrule \\
Equation & \begin{displaymath}
           \symbf{F}=m \symbf{a}
           \end{displaymath}
\\ \midrule \\
Description & \begin{symbDescription}
              \item{$\symbf{F}$ is the force (${\text{N}}$)}
              \item{$m$ is the mass (${\text{kg}}$)}
              \item{$\symbf{a}$ is the acceleration ($\frac{\text{m}}{\text{s}^{2}}$)}
              \end{symbDescription}
\\ \midrule \\
Notes & The net force $\symbf{F}$ on a body is proportional to the acceleration $\symbf{a}$ of the body, where $m$ denotes the mass of the body as the constant of proportionality.
        
\\ \midrule \\
Source & --
         
\\ \midrule \\
RefBy & 
\\ \bottomrule
\end{tabular}
\end{minipage}
\subsubsection{General Definitions}
\label{Sec:GDs}
This section collects the laws and equations that will be used to build the instance models.

\vspace{\baselineskip}
\noindent
\begin{minipage}{\textwidth}
\begin{tabular}{>{\raggedright}p{0.13\textwidth}>{\raggedright\arraybackslash}p{0.82\textwidth}}
\toprule \textbf{Refname} & \textbf{GD:velocityX1}
\phantomsection 
\label{GD:velocityX1}
\\ \midrule \\
Label & The $x$-component of velocity of the first object
        
\\ \midrule \\
Units & $\frac{\text{m}}{\text{s}}$
        
\\ \midrule \\
Equation & \begin{displaymath}
           {v_{\text{x}1}}={w_{1}} {L_{1}} \cos\left({θ_{1}}\right)
           \end{displaymath}
\\ \midrule \\
Description & \begin{symbDescription}
              \item{${v_{\text{x}1}}$ is the the horizontal velocity of the first object ($\frac{\text{m}}{\text{s}}$)}
              \item{${w_{1}}$ is the the angular velocity of the first object ($\frac{\text{rad}}{\text{s}}$)}
              \item{${L_{1}}$ is the the length of the first rod (${\text{m}}$)}
              \item{${θ_{1}}$ is the the angle of the first rod (${{}^{\circ}}$)}
              \end{symbDescription}
\\ \midrule \\
Source & --
         
\\ \midrule \\
RefBy & 
\\ \bottomrule
\end{tabular}
\end{minipage}
\paragraph{Detailed derivation of the $x$-component of velocity:}
\label{GD:velocityX1Deriv}
At a given point in time, velocity is defined in \hyperref[DD:positionGDD]{DD:positionGDD}

\begin{displaymath}
\symbf{v}=\frac{\,d\symbf{p}}{\,dt}
\end{displaymath}
We also know the horizontal position that is defined in \hyperref[DD:positionXDD1]{DD:positionXDD1}

\begin{displaymath}
{p_{\text{x}1}}={L_{1}} \sin\left({θ_{1}}\right)
\end{displaymath}
Applying this,

\begin{displaymath}
{v_{\text{x}1}}=\frac{\,d{L_{1}} \sin\left({θ_{1}}\right)}{\,dt}
\end{displaymath}
${L_{1}}$ is constant with respect to time, so

\begin{displaymath}
{v_{\text{x}1}}={L_{1}} \frac{\,d\sin\left({θ_{1}}\right)}{\,dt}
\end{displaymath}
Therefore, using the chain rule,

\begin{displaymath}
{v_{\text{x}1}}={w_{1}} {L_{1}} \cos\left({θ_{1}}\right)
\end{displaymath}
\vspace{\baselineskip}
\noindent
\begin{minipage}{\textwidth}
\begin{tabular}{>{\raggedright}p{0.13\textwidth}>{\raggedright\arraybackslash}p{0.82\textwidth}}
\toprule \textbf{Refname} & \textbf{GD:velocityY1}
\phantomsection 
\label{GD:velocityY1}
\\ \midrule \\
Label & The $y$-component of velocity of the first object
        
\\ \midrule \\
Units & $\frac{\text{m}}{\text{s}}$
        
\\ \midrule \\
Equation & \begin{displaymath}
           {v_{\text{y}1}}={w_{1}} {L_{1}} \sin\left({θ_{1}}\right)
           \end{displaymath}
\\ \midrule \\
Description & \begin{symbDescription}
              \item{${v_{\text{y}1}}$ is the the vertical velocity of the first object ($\frac{\text{m}}{\text{s}}$)}
              \item{${w_{1}}$ is the the angular velocity of the first object ($\frac{\text{rad}}{\text{s}}$)}
              \item{${L_{1}}$ is the the length of the first rod (${\text{m}}$)}
              \item{${θ_{1}}$ is the the angle of the first rod (${{}^{\circ}}$)}
              \end{symbDescription}
\\ \midrule \\
Source & --
         
\\ \midrule \\
RefBy & 
\\ \bottomrule
\end{tabular}
\end{minipage}
\paragraph{Detailed derivation of the $y$-component of velocity:}
\label{GD:velocityY1Deriv}
At a given point in time, velocity is defined in \hyperref[DD:positionGDD]{DD:positionGDD}

\begin{displaymath}
\symbf{v}=\frac{\,d\symbf{p}}{\,dt}
\end{displaymath}
We also know the vertical position that is defined in \hyperref[DD:positionYDD1]{DD:positionYDD1}

\begin{displaymath}
{p_{\text{y}1}}=-{L_{1}} \cos\left({θ_{1}}\right)
\end{displaymath}
Applying this,

\begin{displaymath}
{v_{\text{y}1}}=-\left(\frac{\,d{L_{1}} \cos\left({θ_{1}}\right)}{\,dt}\right)
\end{displaymath}
${L_{1}}$ is constant with respect to time, so

\begin{displaymath}
{v_{\text{y}1}}=-{L_{1}} \frac{\,d\cos\left({θ_{1}}\right)}{\,dt}
\end{displaymath}
Therefore, using the chain rule,

\begin{displaymath}
{v_{\text{y}1}}={w_{1}} {L_{1}} \sin\left({θ_{1}}\right)
\end{displaymath}
\vspace{\baselineskip}
\noindent
\begin{minipage}{\textwidth}
\begin{tabular}{>{\raggedright}p{0.13\textwidth}>{\raggedright\arraybackslash}p{0.82\textwidth}}
\toprule \textbf{Refname} & \textbf{GD:velocityX2}
\phantomsection 
\label{GD:velocityX2}
\\ \midrule \\
Label & The $x$-component of velocity of the second object
        
\\ \midrule \\
Units & $\frac{\text{m}}{\text{s}}$
        
\\ \midrule \\
Equation & \begin{displaymath}
           {v_{\text{x}2}}={v_{\text{x}1}}+{w_{2}} {L_{2}} \cos\left({θ_{2}}\right)
           \end{displaymath}
\\ \midrule \\
Description & \begin{symbDescription}
              \item{${v_{\text{x}2}}$ is the the horizontal velocity of the second object ($\frac{\text{m}}{\text{s}}$)}
              \item{${v_{\text{x}1}}$ is the the horizontal velocity of the first object ($\frac{\text{m}}{\text{s}}$)}
              \item{${w_{2}}$ is the the angular velocity of the second object ($\frac{\text{rad}}{\text{s}}$)}
              \item{${L_{2}}$ is the the length of the second rod (${\text{m}}$)}
              \item{${θ_{2}}$ is the the angle of the second rod (${{}^{\circ}}$)}
              \end{symbDescription}
\\ \midrule \\
Source & --
         
\\ \midrule \\
RefBy & 
\\ \bottomrule
\end{tabular}
\end{minipage}
\paragraph{Detailed derivation of the $x$-component of velocity:}
\label{GD:velocityX2Deriv}
At a given point in time, velocity is defined in \hyperref[DD:positionGDD]{DD:positionGDD}

\begin{displaymath}
\symbf{v}=\frac{\,d\symbf{p}}{\,dt}
\end{displaymath}
We also know the horizontal position that is defined in \hyperref[DD:positionXDD2]{DD:positionXDD2}

\begin{displaymath}
{p_{\text{x}2}}={p_{\text{x}1}}+{L_{2}} \sin\left({θ_{2}}\right)
\end{displaymath}
Applying this,

\begin{displaymath}
{v_{\text{x}2}}=\frac{\,d{p_{\text{x}1}}+{L_{2}} \sin\left({θ_{2}}\right)}{\,dt}
\end{displaymath}
${L_{1}}$ is constant with respect to time, so

\begin{displaymath}
{v_{\text{x}2}}={v_{\text{x}1}}+{w_{2}} {L_{2}} \cos\left({θ_{2}}\right)
\end{displaymath}
\vspace{\baselineskip}
\noindent
\begin{minipage}{\textwidth}
\begin{tabular}{>{\raggedright}p{0.13\textwidth}>{\raggedright\arraybackslash}p{0.82\textwidth}}
\toprule \textbf{Refname} & \textbf{GD:velocityY2}
\phantomsection 
\label{GD:velocityY2}
\\ \midrule \\
Label & The $y$-component of velocity of the second object
        
\\ \midrule \\
Units & $\frac{\text{m}}{\text{s}}$
        
\\ \midrule \\
Equation & \begin{displaymath}
           {v_{\text{y}2}}={v_{\text{y}1}}+{w_{2}} {L_{2}} \sin\left({θ_{2}}\right)
           \end{displaymath}
\\ \midrule \\
Description & \begin{symbDescription}
              \item{${v_{\text{y}2}}$ is the the vertical velocity of the second object ($\frac{\text{m}}{\text{s}}$)}
              \item{${v_{\text{y}1}}$ is the the vertical velocity of the first object ($\frac{\text{m}}{\text{s}}$)}
              \item{${w_{2}}$ is the the angular velocity of the second object ($\frac{\text{rad}}{\text{s}}$)}
              \item{${L_{2}}$ is the the length of the second rod (${\text{m}}$)}
              \item{${θ_{2}}$ is the the angle of the second rod (${{}^{\circ}}$)}
              \end{symbDescription}
\\ \midrule \\
Source & --
         
\\ \midrule \\
RefBy & 
\\ \bottomrule
\end{tabular}
\end{minipage}
\paragraph{Detailed derivation of the $y$-component of velocity:}
\label{GD:velocityY2Deriv}
At a given point in time, velocity is defined in \hyperref[DD:positionGDD]{DD:positionGDD}

\begin{displaymath}
\symbf{v}=\frac{\,d\symbf{p}}{\,dt}
\end{displaymath}
We also know the vertical position that is defined in \hyperref[DD:positionYDD2]{DD:positionYDD2}

\begin{displaymath}
{p_{\text{y}2}}={p_{\text{y}1}}-{L_{2}} \cos\left({θ_{2}}\right)
\end{displaymath}
Applying this,

\begin{displaymath}
{v_{\text{y}2}}=-\left(\frac{\,d{p_{\text{y}1}}-{L_{2}} \cos\left({θ_{2}}\right)}{\,dt}\right)
\end{displaymath}
Therefore, using the chain rule,

\begin{displaymath}
{v_{\text{y}2}}={v_{\text{y}1}}+{w_{2}} {L_{2}} \sin\left({θ_{2}}\right)
\end{displaymath}
\vspace{\baselineskip}
\noindent
\begin{minipage}{\textwidth}
\begin{tabular}{>{\raggedright}p{0.13\textwidth}>{\raggedright\arraybackslash}p{0.82\textwidth}}
\toprule \textbf{Refname} & \textbf{GD:accelerationX1}
\phantomsection 
\label{GD:accelerationX1}
\\ \midrule \\
Label & The $x$-component of acceleration of the first object
        
\\ \midrule \\
Units & $\frac{\text{m}}{\text{s}^{2}}$
        
\\ \midrule \\
Equation & \begin{displaymath}
           {a_{\text{x}1}}=-{w_{1}}^{2} {L_{1}} \sin\left({θ_{1}}\right)+{α_{1}} {L_{1}} \cos\left({θ_{1}}\right)
           \end{displaymath}
\\ \midrule \\
Description & \begin{symbDescription}
              \item{${a_{\text{x}1}}$ is the the horizontal acceleration of the first object ($\frac{\text{m}}{\text{s}^{2}}$)}
              \item{${w_{1}}$ is the the angular velocity of the first object ($\frac{\text{rad}}{\text{s}}$)}
              \item{${L_{1}}$ is the the length of the first rod (${\text{m}}$)}
              \item{${θ_{1}}$ is the the angle of the first rod (${{}^{\circ}}$)}
              \item{${α_{1}}$ is the the angular acceleration of the first object ($\frac{\text{rad}}{\text{s}^{2}}$)}
              \end{symbDescription}
\\ \midrule \\
Source & --
         
\\ \midrule \\
RefBy & \hyperref[IM:calOfAngularAcceleration2]{IM:calOfAngularAcceleration2}
        
\\ \bottomrule
\end{tabular}
\end{minipage}
\paragraph{Detailed derivation of the $x$-component of acceleration:}
\label{GD:accelerationX1Deriv}
Our acceleration is:

\begin{displaymath}
\symbf{a}=\frac{\,d\symbf{v}}{\,dt}
\end{displaymath}
Earlier, we found the horizontal velocity to be

\begin{displaymath}
{v_{\text{x}1}}={w_{1}} {L_{1}} \cos\left({θ_{1}}\right)
\end{displaymath}
Applying this to our equation for acceleration

\begin{displaymath}
{a_{\text{x}1}}=\frac{\,d{w_{1}} {L_{1}} \cos\left({θ_{1}}\right)}{\,dt}
\end{displaymath}
By the product and chain rules, we find

\begin{displaymath}
{a_{\text{x}1}}=\frac{\,d{w_{1}}}{\,dt} {L_{1}} \cos\left({θ_{1}}\right)-{w_{1}} {L_{1}} \sin\left({θ_{1}}\right) \frac{\,d{θ_{1}}}{\,dt}
\end{displaymath}
Simplifying,

\begin{displaymath}
{a_{\text{x}1}}=-{w_{1}}^{2} {L_{1}} \sin\left({θ_{1}}\right)+{α_{1}} {L_{1}} \cos\left({θ_{1}}\right)
\end{displaymath}
\vspace{\baselineskip}
\noindent
\begin{minipage}{\textwidth}
\begin{tabular}{>{\raggedright}p{0.13\textwidth}>{\raggedright\arraybackslash}p{0.82\textwidth}}
\toprule \textbf{Refname} & \textbf{GD:accelerationY1}
\phantomsection 
\label{GD:accelerationY1}
\\ \midrule \\
Label & The $y$-component of acceleration of the first object
        
\\ \midrule \\
Units & $\frac{\text{m}}{\text{s}^{2}}$
        
\\ \midrule \\
Equation & \begin{displaymath}
           {a_{\text{y}1}}={w_{1}}^{2} {L_{1}} \cos\left({θ_{1}}\right)+{α_{1}} {L_{1}} \sin\left({θ_{1}}\right)
           \end{displaymath}
\\ \midrule \\
Description & \begin{symbDescription}
              \item{${a_{\text{y}1}}$ is the the vertical acceleration of the first object ($\frac{\text{m}}{\text{s}^{2}}$)}
              \item{${w_{1}}$ is the the angular velocity of the first object ($\frac{\text{rad}}{\text{s}}$)}
              \item{${L_{1}}$ is the the length of the first rod (${\text{m}}$)}
              \item{${θ_{1}}$ is the the angle of the first rod (${{}^{\circ}}$)}
              \item{${α_{1}}$ is the the angular acceleration of the first object ($\frac{\text{rad}}{\text{s}^{2}}$)}
              \end{symbDescription}
\\ \midrule \\
Source & --
         
\\ \midrule \\
RefBy & \hyperref[IM:calOfAngularAcceleration2]{IM:calOfAngularAcceleration2}
        
\\ \bottomrule
\end{tabular}
\end{minipage}
\paragraph{Detailed derivation of the $y$-component of acceleration:}
\label{GD:accelerationY1Deriv}
Our acceleration is:

\begin{displaymath}
\symbf{a}=\frac{\,d\symbf{v}}{\,dt}
\end{displaymath}
Earlier, we found the vertical velocity to be

\begin{displaymath}
{v_{\text{y}1}}={w_{1}} {L_{1}} \sin\left({θ_{1}}\right)
\end{displaymath}
Applying this to our equation for acceleration

\begin{displaymath}
{a_{\text{y}1}}=\frac{\,d{w_{1}} {L_{1}} \sin\left({θ_{1}}\right)}{\,dt}
\end{displaymath}
By the product and chain rules, we find

\begin{displaymath}
{a_{\text{y}1}}=\frac{\,d{w_{1}}}{\,dt} {L_{1}} \sin\left({θ_{1}}\right)+{w_{1}} {L_{1}} \cos\left({θ_{1}}\right) \frac{\,d{θ_{1}}}{\,dt}
\end{displaymath}
Simplifying,

\begin{displaymath}
{a_{\text{y}1}}={w_{1}}^{2} {L_{1}} \cos\left({θ_{1}}\right)+{α_{1}} {L_{1}} \sin\left({θ_{1}}\right)
\end{displaymath}
\vspace{\baselineskip}
\noindent
\begin{minipage}{\textwidth}
\begin{tabular}{>{\raggedright}p{0.13\textwidth}>{\raggedright\arraybackslash}p{0.82\textwidth}}
\toprule \textbf{Refname} & \textbf{GD:accelerationX2}
\phantomsection 
\label{GD:accelerationX2}
\\ \midrule \\
Label & The $x$-component of acceleration of the second object
        
\\ \midrule \\
Units & $\frac{\text{m}}{\text{s}^{2}}$
        
\\ \midrule \\
Equation & \begin{displaymath}
           {a_{\text{x}2}}={a_{\text{x}1}}-{w_{2}}^{2} {L_{2}} \sin\left({θ_{2}}\right)+{α_{2}} {L_{2}} \cos\left({θ_{2}}\right)
           \end{displaymath}
\\ \midrule \\
Description & \begin{symbDescription}
              \item{${a_{\text{x}2}}$ is the the horizontal acceleration of the second object ($\frac{\text{m}}{\text{s}^{2}}$)}
              \item{${a_{\text{x}1}}$ is the the horizontal acceleration of the first object ($\frac{\text{m}}{\text{s}^{2}}$)}
              \item{${w_{2}}$ is the the angular velocity of the second object ($\frac{\text{rad}}{\text{s}}$)}
              \item{${L_{2}}$ is the the length of the second rod (${\text{m}}$)}
              \item{${θ_{2}}$ is the the angle of the second rod (${{}^{\circ}}$)}
              \item{${α_{2}}$ is the the angular acceleration of the second object ($\frac{\text{rad}}{\text{s}^{2}}$)}
              \end{symbDescription}
\\ \midrule \\
Source & --
         
\\ \midrule \\
RefBy & \hyperref[IM:calOfAngularAcceleration2]{IM:calOfAngularAcceleration2}
        
\\ \bottomrule
\end{tabular}
\end{minipage}
\paragraph{Detailed derivation of the $x$-component of acceleration:}
\label{GD:accelerationX2Deriv}
Our acceleration is:

\begin{displaymath}
\symbf{a}=\frac{\,d\symbf{v}}{\,dt}
\end{displaymath}
Earlier, we found the horizontal velocity to be

\begin{displaymath}
{v_{\text{x}2}}={v_{\text{x}1}}+{w_{2}} {L_{2}} \cos\left({θ_{2}}\right)
\end{displaymath}
Applying this to our equation for acceleration

\begin{displaymath}
{a_{\text{x}2}}=\frac{\,d{v_{\text{x}1}}+{w_{2}} {L_{2}} \cos\left({θ_{2}}\right)}{\,dt}
\end{displaymath}
By the product and chain rules, we find

\begin{displaymath}
{a_{\text{x}2}}={a_{\text{x}1}}-{w_{2}}^{2} {L_{2}} \sin\left({θ_{2}}\right)+{α_{2}} {L_{2}} \cos\left({θ_{2}}\right)
\end{displaymath}
\vspace{\baselineskip}
\noindent
\begin{minipage}{\textwidth}
\begin{tabular}{>{\raggedright}p{0.13\textwidth}>{\raggedright\arraybackslash}p{0.82\textwidth}}
\toprule \textbf{Refname} & \textbf{GD:accelerationY2}
\phantomsection 
\label{GD:accelerationY2}
\\ \midrule \\
Label & The $y$-component of acceleration of the second object
        
\\ \midrule \\
Units & $\frac{\text{m}}{\text{s}^{2}}$
        
\\ \midrule \\
Equation & \begin{displaymath}
           {a_{\text{y}2}}={a_{\text{y}1}}+{w_{2}}^{2} {L_{2}} \cos\left({θ_{2}}\right)+{α_{2}} {L_{2}} \sin\left({θ_{2}}\right)
           \end{displaymath}
\\ \midrule \\
Description & \begin{symbDescription}
              \item{${a_{\text{y}2}}$ is the the vertical acceleration of the second object ($\frac{\text{m}}{\text{s}^{2}}$)}
              \item{${a_{\text{y}1}}$ is the the vertical acceleration of the first object ($\frac{\text{m}}{\text{s}^{2}}$)}
              \item{${w_{2}}$ is the the angular velocity of the second object ($\frac{\text{rad}}{\text{s}}$)}
              \item{${L_{2}}$ is the the length of the second rod (${\text{m}}$)}
              \item{${θ_{2}}$ is the the angle of the second rod (${{}^{\circ}}$)}
              \item{${α_{2}}$ is the the angular acceleration of the second object ($\frac{\text{rad}}{\text{s}^{2}}$)}
              \end{symbDescription}
\\ \midrule \\
Source & --
         
\\ \midrule \\
RefBy & \hyperref[IM:calOfAngularAcceleration2]{IM:calOfAngularAcceleration2}
        
\\ \bottomrule
\end{tabular}
\end{minipage}
\paragraph{Detailed derivation of the $y$-component of acceleration:}
\label{GD:accelerationY2Deriv}
Our acceleration is:

\begin{displaymath}
\symbf{a}=\frac{\,d\symbf{v}}{\,dt}
\end{displaymath}
Earlier, we found the horizontal velocity to be

\begin{displaymath}
{v_{\text{y}2}}={v_{\text{y}1}}+{w_{2}} {L_{2}} \sin\left({θ_{2}}\right)
\end{displaymath}
Applying this to our equation for acceleration

\begin{displaymath}
{a_{\text{y}2}}=\frac{\,d{v_{\text{y}1}}+{w_{2}} {L_{2}} \sin\left({θ_{2}}\right)}{\,dt}
\end{displaymath}
By the product and chain rules, we find

\begin{displaymath}
{a_{\text{y}2}}={a_{\text{y}1}}+{w_{2}}^{2} {L_{2}} \cos\left({θ_{2}}\right)+{α_{2}} {L_{2}} \sin\left({θ_{2}}\right)
\end{displaymath}
\vspace{\baselineskip}
\noindent
\begin{minipage}{\textwidth}
\begin{tabular}{>{\raggedright}p{0.13\textwidth}>{\raggedright\arraybackslash}p{0.82\textwidth}}
\toprule \textbf{Refname} & \textbf{GD:xForce1}
\phantomsection 
\label{GD:xForce1}
\\ \midrule \\
Label & Horizontal force on the first object
        
\\ \midrule \\
Units & ${\text{N}}$
        
\\ \midrule \\
Equation & \begin{displaymath}
           \symbf{F}=m \symbf{a}=-{\symbf{T}_{1}} \sin\left({θ_{1}}\right)+{\symbf{T}_{2}} \sin\left({θ_{2}}\right)
           \end{displaymath}
\\ \midrule \\
Description & \begin{symbDescription}
              \item{$\symbf{F}$ is the force (${\text{N}}$)}
              \item{$m$ is the mass (${\text{kg}}$)}
              \item{$\symbf{a}$ is the acceleration ($\frac{\text{m}}{\text{s}^{2}}$)}
              \item{${\symbf{T}_{1}}$ is the the tension of the first object (${\text{N}}$)}
              \item{${θ_{1}}$ is the the angle of the first rod (${{}^{\circ}}$)}
              \item{${\symbf{T}_{2}}$ is the the tension of the second object (${\text{N}}$)}
              \item{${θ_{2}}$ is the the angle of the second rod (${{}^{\circ}}$)}
              \end{symbDescription}
\\ \midrule \\
Source & --
         
\\ \midrule \\
RefBy & \hyperref[IM:calOfAngularAcceleration2]{IM:calOfAngularAcceleration2}
        
\\ \bottomrule
\end{tabular}
\end{minipage}
\paragraph{Detailed derivation of force on the first object:}
\label{GD:xForce1Deriv}
\begin{displaymath}
\symbf{F}=m \symbf{a}=-{\symbf{T}_{1}} \sin\left({θ_{1}}\right)+{\symbf{T}_{2}} \sin\left({θ_{2}}\right)
\end{displaymath}
\vspace{\baselineskip}
\noindent
\begin{minipage}{\textwidth}
\begin{tabular}{>{\raggedright}p{0.13\textwidth}>{\raggedright\arraybackslash}p{0.82\textwidth}}
\toprule \textbf{Refname} & \textbf{GD:yForce1}
\phantomsection 
\label{GD:yForce1}
\\ \midrule \\
Label & Vertical force on the first object
        
\\ \midrule \\
Units & ${\text{N}}$
        
\\ \midrule \\
Equation & \begin{displaymath}
           \symbf{F}=m \symbf{a}={\symbf{T}_{1}} \cos\left({θ_{1}}\right)-{\symbf{T}_{2}} \cos\left({θ_{2}}\right)-{m_{1}} \symbf{g}
           \end{displaymath}
\\ \midrule \\
Description & \begin{symbDescription}
              \item{$\symbf{F}$ is the force (${\text{N}}$)}
              \item{$m$ is the mass (${\text{kg}}$)}
              \item{$\symbf{a}$ is the acceleration ($\frac{\text{m}}{\text{s}^{2}}$)}
              \item{${\symbf{T}_{1}}$ is the the tension of the first object (${\text{N}}$)}
              \item{${θ_{1}}$ is the the angle of the first rod (${{}^{\circ}}$)}
              \item{${\symbf{T}_{2}}$ is the the tension of the second object (${\text{N}}$)}
              \item{${θ_{2}}$ is the the angle of the second rod (${{}^{\circ}}$)}
              \item{${m_{1}}$ is the the mass of the first object (${\text{kg}}$)}
              \item{$\symbf{g}$ is the gravitational acceleration ($\frac{\text{m}}{\text{s}^{2}}$)}
              \end{symbDescription}
\\ \midrule \\
Source & --
         
\\ \midrule \\
RefBy & \hyperref[IM:calOfAngularAcceleration2]{IM:calOfAngularAcceleration2}
        
\\ \bottomrule
\end{tabular}
\end{minipage}
\paragraph{Detailed derivation of force on the first object:}
\label{GD:yForce1Deriv}
\begin{displaymath}
\symbf{F}=m \symbf{a}={\symbf{T}_{1}} \cos\left({θ_{1}}\right)-{\symbf{T}_{2}} \cos\left({θ_{2}}\right)-{m_{1}} \symbf{g}
\end{displaymath}
\vspace{\baselineskip}
\noindent
\begin{minipage}{\textwidth}
\begin{tabular}{>{\raggedright}p{0.13\textwidth}>{\raggedright\arraybackslash}p{0.82\textwidth}}
\toprule \textbf{Refname} & \textbf{GD:xForce2}
\phantomsection 
\label{GD:xForce2}
\\ \midrule \\
Label & Horizontal force on the second object
        
\\ \midrule \\
Units & ${\text{N}}$
        
\\ \midrule \\
Equation & \begin{displaymath}
           \symbf{F}=m \symbf{a}=-{\symbf{T}_{2}} \sin\left({θ_{2}}\right)
           \end{displaymath}
\\ \midrule \\
Description & \begin{symbDescription}
              \item{$\symbf{F}$ is the force (${\text{N}}$)}
              \item{$m$ is the mass (${\text{kg}}$)}
              \item{$\symbf{a}$ is the acceleration ($\frac{\text{m}}{\text{s}^{2}}$)}
              \item{${\symbf{T}_{2}}$ is the the tension of the second object (${\text{N}}$)}
              \item{${θ_{2}}$ is the the angle of the second rod (${{}^{\circ}}$)}
              \end{symbDescription}
\\ \midrule \\
Source & --
         
\\ \midrule \\
RefBy & \hyperref[IM:calOfAngularAcceleration2]{IM:calOfAngularAcceleration2}
        
\\ \bottomrule
\end{tabular}
\end{minipage}
\paragraph{Detailed derivation of force on the second object:}
\label{GD:xForce2Deriv}
\begin{displaymath}
\symbf{F}=m \symbf{a}=-{\symbf{T}_{2}} \sin\left({θ_{2}}\right)
\end{displaymath}
\vspace{\baselineskip}
\noindent
\begin{minipage}{\textwidth}
\begin{tabular}{>{\raggedright}p{0.13\textwidth}>{\raggedright\arraybackslash}p{0.82\textwidth}}
\toprule \textbf{Refname} & \textbf{GD:yForce2}
\phantomsection 
\label{GD:yForce2}
\\ \midrule \\
Label & Vertical force on the second object
        
\\ \midrule \\
Units & ${\text{N}}$
        
\\ \midrule \\
Equation & \begin{displaymath}
           \symbf{F}=m \symbf{a}={\symbf{T}_{2}} \cos\left({θ_{2}}\right)-{m_{2}} \symbf{g}
           \end{displaymath}
\\ \midrule \\
Description & \begin{symbDescription}
              \item{$\symbf{F}$ is the force (${\text{N}}$)}
              \item{$m$ is the mass (${\text{kg}}$)}
              \item{$\symbf{a}$ is the acceleration ($\frac{\text{m}}{\text{s}^{2}}$)}
              \item{${\symbf{T}_{2}}$ is the the tension of the second object (${\text{N}}$)}
              \item{${θ_{2}}$ is the the angle of the second rod (${{}^{\circ}}$)}
              \item{${m_{2}}$ is the the mass of the second object (${\text{kg}}$)}
              \item{$\symbf{g}$ is the gravitational acceleration ($\frac{\text{m}}{\text{s}^{2}}$)}
              \end{symbDescription}
\\ \midrule \\
Source & --
         
\\ \midrule \\
RefBy & \hyperref[IM:calOfAngularAcceleration2]{IM:calOfAngularAcceleration2}
        
\\ \bottomrule
\end{tabular}
\end{minipage}
\paragraph{Detailed derivation of force on the second object:}
\label{GD:yForce2Deriv}
\begin{displaymath}
\symbf{F}=m \symbf{a}={\symbf{T}_{2}} \cos\left({θ_{2}}\right)-{m_{2}} \symbf{g}
\end{displaymath}
\subsubsection{Data Definitions}
\label{Sec:DDs}
This section collects and defines all the data needed to build the instance models.

\vspace{\baselineskip}
\noindent
\begin{minipage}{\textwidth}
\begin{tabular}{>{\raggedright}p{0.13\textwidth}>{\raggedright\arraybackslash}p{0.82\textwidth}}
\toprule \textbf{Refname} & \textbf{DD:positionGDD}
\phantomsection 
\label{DD:positionGDD}
\\ \midrule \\
Label & Velocity
        
\\ \midrule \\
Symbol & $\symbf{v}$
         
\\ \midrule \\
Units & $\frac{\text{m}}{\text{s}}$
        
\\ \midrule \\
Equation & \begin{displaymath}
           \symbf{v}=\frac{\,d\symbf{p}}{\,dt}
           \end{displaymath}
\\ \midrule \\
Description & \begin{symbDescription}
              \item{$\symbf{v}$ is the velocity ($\frac{\text{m}}{\text{s}}$)}
              \item{$t$ is the time (${\text{s}}$)}
              \item{$\symbf{p}$ is the position (${\text{m}}$)}
              \end{symbDescription}
\\ \midrule \\
Source & --
         
\\ \midrule \\
RefBy & \hyperref[GD:velocityY2]{GD:velocityY2}, \hyperref[GD:velocityY1]{GD:velocityY1}, \hyperref[GD:velocityX2]{GD:velocityX2}, and \hyperref[GD:velocityX1]{GD:velocityX1}
        
\\ \bottomrule
\end{tabular}
\end{minipage}

\vspace{\baselineskip}
\noindent
\begin{minipage}{\textwidth}
\begin{tabular}{>{\raggedright}p{0.13\textwidth}>{\raggedright\arraybackslash}p{0.82\textwidth}}
\toprule \textbf{Refname} & \textbf{DD:positionXDD1}
\phantomsection 
\label{DD:positionXDD1}
\\ \midrule \\
Label & The horizontal position of the first object
        
\\ \midrule \\
Symbol & ${p_{\text{x}1}}$
         
\\ \midrule \\
Units & ${\text{m}}$
        
\\ \midrule \\
Equation & \begin{displaymath}
           {p_{\text{x}1}}={L_{1}} \sin\left({θ_{1}}\right)
           \end{displaymath}
\\ \midrule \\
Description & \begin{symbDescription}
              \item{${p_{\text{x}1}}$ is the the horizontal position of the first object (${\text{m}}$)}
              \item{${L_{1}}$ is the the length of the first rod (${\text{m}}$)}
              \item{${θ_{1}}$ is the the angle of the first rod (${{}^{\circ}}$)}
              \end{symbDescription}
\\ \midrule \\
Notes & ${p_{\text{x}1}}$ is the horizontal position
        
        ${p_{\text{x}1}}$ is shown in \hyperref[Figure:dblpendulum]{Fig:dblpendulum}.
        
\\ \midrule \\
Source & --
         
\\ \midrule \\
RefBy & \hyperref[GD:velocityX1]{GD:velocityX1}
        
\\ \bottomrule
\end{tabular}
\end{minipage}

\vspace{\baselineskip}
\noindent
\begin{minipage}{\textwidth}
\begin{tabular}{>{\raggedright}p{0.13\textwidth}>{\raggedright\arraybackslash}p{0.82\textwidth}}
\toprule \textbf{Refname} & \textbf{DD:positionYDD1}
\phantomsection 
\label{DD:positionYDD1}
\\ \midrule \\
Label & The vertical position of the first object
        
\\ \midrule \\
Symbol & ${p_{\text{y}1}}$
         
\\ \midrule \\
Units & ${\text{m}}$
        
\\ \midrule \\
Equation & \begin{displaymath}
           {p_{\text{y}1}}=-{L_{1}} \cos\left({θ_{1}}\right)
           \end{displaymath}
\\ \midrule \\
Description & \begin{symbDescription}
              \item{${p_{\text{y}1}}$ is the the vertical position of the first object (${\text{m}}$)}
              \item{${L_{1}}$ is the the length of the first rod (${\text{m}}$)}
              \item{${θ_{1}}$ is the the angle of the first rod (${{}^{\circ}}$)}
              \end{symbDescription}
\\ \midrule \\
Notes & ${p_{\text{y}1}}$ is the vertical position
        
        ${p_{\text{y}1}}$ is shown in \hyperref[Figure:dblpendulum]{Fig:dblpendulum}.
        
\\ \midrule \\
Source & --
         
\\ \midrule \\
RefBy & \hyperref[GD:velocityY1]{GD:velocityY1}
        
\\ \bottomrule
\end{tabular}
\end{minipage}

\vspace{\baselineskip}
\noindent
\begin{minipage}{\textwidth}
\begin{tabular}{>{\raggedright}p{0.13\textwidth}>{\raggedright\arraybackslash}p{0.82\textwidth}}
\toprule \textbf{Refname} & \textbf{DD:positionXDD2}
\phantomsection 
\label{DD:positionXDD2}
\\ \midrule \\
Label & The horizontal position of the second object
        
\\ \midrule \\
Symbol & ${p_{\text{x}2}}$
         
\\ \midrule \\
Units & ${\text{m}}$
        
\\ \midrule \\
Equation & \begin{displaymath}
           {p_{\text{x}2}}={p_{\text{x}1}}+{L_{2}} \sin\left({θ_{2}}\right)
           \end{displaymath}
\\ \midrule \\
Description & \begin{symbDescription}
              \item{${p_{\text{x}2}}$ is the the horizontal position of the second object (${\text{m}}$)}
              \item{${p_{\text{x}1}}$ is the the horizontal position of the first object (${\text{m}}$)}
              \item{${L_{2}}$ is the the length of the second rod (${\text{m}}$)}
              \item{${θ_{2}}$ is the the angle of the second rod (${{}^{\circ}}$)}
              \end{symbDescription}
\\ \midrule \\
Notes & ${p_{\text{x}2}}$ is the horizontal position
        
        ${p_{\text{x}2}}$ is shown in \hyperref[Figure:dblpendulum]{Fig:dblpendulum}.
        
\\ \midrule \\
Source & --
         
\\ \midrule \\
RefBy & \hyperref[GD:velocityX2]{GD:velocityX2}
        
\\ \bottomrule
\end{tabular}
\end{minipage}

\vspace{\baselineskip}
\noindent
\begin{minipage}{\textwidth}
\begin{tabular}{>{\raggedright}p{0.13\textwidth}>{\raggedright\arraybackslash}p{0.82\textwidth}}
\toprule \textbf{Refname} & \textbf{DD:positionYDD2}
\phantomsection 
\label{DD:positionYDD2}
\\ \midrule \\
Label & The vertical position of the second object
        
\\ \midrule \\
Symbol & ${p_{\text{y}2}}$
         
\\ \midrule \\
Units & ${\text{m}}$
        
\\ \midrule \\
Equation & \begin{displaymath}
           {p_{\text{y}2}}={p_{\text{y}1}}-{L_{2}} \cos\left({θ_{2}}\right)
           \end{displaymath}
\\ \midrule \\
Description & \begin{symbDescription}
              \item{${p_{\text{y}2}}$ is the the vertical position of the second object (${\text{m}}$)}
              \item{${p_{\text{y}1}}$ is the the vertical position of the first object (${\text{m}}$)}
              \item{${L_{2}}$ is the the length of the second rod (${\text{m}}$)}
              \item{${θ_{2}}$ is the the angle of the second rod (${{}^{\circ}}$)}
              \end{symbDescription}
\\ \midrule \\
Notes & ${p_{\text{y}2}}$ is the vertical position
        
        ${p_{\text{y}2}}$ is shown in \hyperref[Figure:dblpendulum]{Fig:dblpendulum}.
        
\\ \midrule \\
Source & --
         
\\ \midrule \\
RefBy & \hyperref[GD:velocityY2]{GD:velocityY2}
        
\\ \bottomrule
\end{tabular}
\end{minipage}

\vspace{\baselineskip}
\noindent
\begin{minipage}{\textwidth}
\begin{tabular}{>{\raggedright}p{0.13\textwidth}>{\raggedright\arraybackslash}p{0.82\textwidth}}
\toprule \textbf{Refname} & \textbf{DD:accelerationGDD}
\phantomsection 
\label{DD:accelerationGDD}
\\ \midrule \\
Label & Acceleration
        
\\ \midrule \\
Symbol & $\symbf{a}$
         
\\ \midrule \\
Units & $\frac{\text{m}}{\text{s}^{2}}$
        
\\ \midrule \\
Equation & \begin{displaymath}
           \symbf{a}=\frac{\,d\symbf{v}}{\,dt}
           \end{displaymath}
\\ \midrule \\
Description & \begin{symbDescription}
              \item{$\symbf{a}$ is the acceleration ($\frac{\text{m}}{\text{s}^{2}}$)}
              \item{$t$ is the time (${\text{s}}$)}
              \item{$\symbf{v}$ is the velocity ($\frac{\text{m}}{\text{s}}$)}
              \end{symbDescription}
\\ \midrule \\
Source & --
         
\\ \midrule \\
RefBy & 
\\ \bottomrule
\end{tabular}
\end{minipage}

\vspace{\baselineskip}
\noindent
\begin{minipage}{\textwidth}
\begin{tabular}{>{\raggedright}p{0.13\textwidth}>{\raggedright\arraybackslash}p{0.82\textwidth}}
\toprule \textbf{Refname} & \textbf{DD:forceGDD}
\phantomsection 
\label{DD:forceGDD}
\\ \midrule \\
Label & Force
        
\\ \midrule \\
Symbol & $\symbf{F}$
         
\\ \midrule \\
Units & ${\text{N}}$
        
\\ \midrule \\
Equation & \begin{displaymath}
           \symbf{F}=m \symbf{a}
           \end{displaymath}
\\ \midrule \\
Description & \begin{symbDescription}
              \item{$\symbf{F}$ is the force (${\text{N}}$)}
              \item{$m$ is the mass (${\text{kg}}$)}
              \item{$\symbf{a}$ is the acceleration ($\frac{\text{m}}{\text{s}^{2}}$)}
              \end{symbDescription}
\\ \midrule \\
Source & --
         
\\ \midrule \\
RefBy & 
\\ \bottomrule
\end{tabular}
\end{minipage}

\subsubsection{Instance Models}
\label{Sec:IMs}
This section transforms the problem defined in the \hyperref[Sec:ProbDesc]{problem description} into one which is expressed in mathematical terms. It uses concrete symbols defined in the \hyperref[Sec:DDs]{data definitions} to replace the abstract symbols in the models identified in \hyperref[Sec:TMs]{theoretical models} and \hyperref[Sec:GDs]{general definitions}.

\vspace{\baselineskip}
\noindent
\begin{minipage}{\textwidth}
\begin{tabular}{>{\raggedright}p{0.13\textwidth}>{\raggedright\arraybackslash}p{0.82\textwidth}}
\toprule \textbf{Refname} & \textbf{IM:calOfAngularAcceleration1}
\phantomsection 
\label{IM:calOfAngularAcceleration1}
\\ \midrule \\
Label & Calculation of angular acceleration
        
\\ \midrule \\
Input & ${L_{1}}$, ${L_{2}}$, ${m_{1}}$, ${m_{2}}$, ${θ_{1}}$, ${θ_{2}}$
        
\\ \midrule \\
Output & ${α_{1}}$
         
\\ \midrule \\
Input Constraints & \begin{displaymath}
                    {L_{1}}\gt{}0
                    \end{displaymath}
                    \begin{displaymath}
                    {L_{2}}\gt{}0
                    \end{displaymath}
                    \begin{displaymath}
                    {m_{1}}\gt{}0
                    \end{displaymath}
                    \begin{displaymath}
                    {m_{2}}\gt{}0
                    \end{displaymath}
\\ \midrule \\
Output Constraints & \begin{displaymath}
                     {α_{1}}\gt{}0
                     \end{displaymath}
\\ \midrule \\
Equation & \begin{displaymath}
           {α_{1}}\left({θ_{1}},{θ_{2}},{w_{1}},{w_{2}}\right)=\frac{-\symbf{g} \left(2 {m_{1}}+{m_{2}}\right) \sin\left({θ_{1}}\right)-{m_{2}} \symbf{g} \sin\left({θ_{1}}-2 {θ_{2}}\right)-2 \sin\left({θ_{1}}-{θ_{2}}\right) {m_{2}} \left({w_{2}}^{2} {L_{2}}+{w_{1}}^{2} {L_{1}} \cos\left({θ_{1}}-{θ_{2}}\right)\right)}{{L_{1}} \left(2 {m_{1}}+{m_{2}}-{m_{2}} \cos\left(2 {θ_{1}}-2 {θ_{2}}\right)\right)}
           \end{displaymath}
\\ \midrule \\
Description & \begin{symbDescription}
              \item{${α_{1}}$ is the the angular acceleration of the first object ($\frac{\text{rad}}{\text{s}^{2}}$)}
              \item{${θ_{1}}$ is the the angle of the first rod (${{}^{\circ}}$)}
              \item{${θ_{2}}$ is the the angle of the second rod (${{}^{\circ}}$)}
              \item{${w_{1}}$ is the the angular velocity of the first object ($\frac{\text{rad}}{\text{s}}$)}
              \item{${w_{2}}$ is the the angular velocity of the second object ($\frac{\text{rad}}{\text{s}}$)}
              \item{$\symbf{g}$ is the gravitational acceleration ($\frac{\text{m}}{\text{s}^{2}}$)}
              \item{${m_{1}}$ is the the mass of the first object (${\text{kg}}$)}
              \item{${m_{2}}$ is the the mass of the second object (${\text{kg}}$)}
              \item{${L_{2}}$ is the the length of the second rod (${\text{m}}$)}
              \item{${L_{1}}$ is the the length of the first rod (${\text{m}}$)}
              \end{symbDescription}
\\ \midrule \\
Source & --
         
\\ \midrule \\
RefBy & \hyperref[outputValues]{FR:Output-Values}, \hyperref[calcAngPos]{FR:Calculate-Angular-Position-Of-Mass}, and \hyperref[IM:calOfAngularAcceleration2]{IM:calOfAngularAcceleration2}
        
\\ \bottomrule
\end{tabular}
\end{minipage}

\vspace{\baselineskip}
\noindent
\begin{minipage}{\textwidth}
\begin{tabular}{>{\raggedright}p{0.13\textwidth}>{\raggedright\arraybackslash}p{0.82\textwidth}}
\toprule \textbf{Refname} & \textbf{IM:calOfAngularAcceleration2}
\phantomsection 
\label{IM:calOfAngularAcceleration2}
\\ \midrule \\
Label & Calculation of angular acceleration
        
\\ \midrule \\
Input & ${L_{1}}$, ${L_{2}}$, ${m_{1}}$, ${m_{2}}$, ${θ_{1}}$, ${θ_{2}}$
        
\\ \midrule \\
Output & ${α_{2}}$
         
\\ \midrule \\
Input Constraints & \begin{displaymath}
                    {L_{1}}\gt{}0
                    \end{displaymath}
                    \begin{displaymath}
                    {L_{2}}\gt{}0
                    \end{displaymath}
                    \begin{displaymath}
                    {m_{1}}\gt{}0
                    \end{displaymath}
                    \begin{displaymath}
                    {m_{2}}\gt{}0
                    \end{displaymath}
\\ \midrule \\
Output Constraints & \begin{displaymath}
                     {α_{2}}\gt{}0
                     \end{displaymath}
\\ \midrule \\
Equation & \begin{displaymath}
           {α_{2}}\left({θ_{1}},{θ_{2}},{w_{1}},{w_{2}}\right)=\frac{2 \sin\left({θ_{1}}-{θ_{2}}\right) \left({w_{1}}^{2} {L_{1}} \left({m_{1}}+{m_{2}}\right)+\symbf{g} \left({m_{1}}+{m_{2}}\right) \cos\left({θ_{1}}\right)+{w_{2}}^{2} {L_{2}} {m_{2}} \cos\left({θ_{1}}-{θ_{2}}\right)\right)}{{L_{2}} \left(2 {m_{1}}+{m_{2}}-{m_{2}} \cos\left(2 {θ_{1}}-2 {θ_{2}}\right)\right)}
           \end{displaymath}
\\ \midrule \\
Description & \begin{symbDescription}
              \item{${α_{2}}$ is the the angular acceleration of the second object ($\frac{\text{rad}}{\text{s}^{2}}$)}
              \item{${θ_{1}}$ is the the angle of the first rod (${{}^{\circ}}$)}
              \item{${θ_{2}}$ is the the angle of the second rod (${{}^{\circ}}$)}
              \item{${w_{1}}$ is the the angular velocity of the first object ($\frac{\text{rad}}{\text{s}}$)}
              \item{${w_{2}}$ is the the angular velocity of the second object ($\frac{\text{rad}}{\text{s}}$)}
              \item{${L_{1}}$ is the the length of the first rod (${\text{m}}$)}
              \item{${m_{1}}$ is the the mass of the first object (${\text{kg}}$)}
              \item{${m_{2}}$ is the the mass of the second object (${\text{kg}}$)}
              \item{$\symbf{g}$ is the gravitational acceleration ($\frac{\text{m}}{\text{s}^{2}}$)}
              \item{${L_{2}}$ is the the length of the second rod (${\text{m}}$)}
              \end{symbDescription}
\\ \midrule \\
Source & --
         
\\ \midrule \\
RefBy & \hyperref[outputValues]{FR:Output-Values}, \hyperref[calcAngPos]{FR:Calculate-Angular-Position-Of-Mass}, and \hyperref[IM:calOfAngularAcceleration2]{IM:calOfAngularAcceleration2}
        
\\ \bottomrule
\end{tabular}
\end{minipage}
\paragraph{Detailed derivation of the angle of the second rod:}
\label{IM:calOfAngularAcceleration2Deriv}
By solving equations \hyperref[GD:xForce2]{GD:xForce2} and \hyperref[GD:yForce2]{GD:yForce2} for ${\symbf{T}_{2}} \sin\left({θ_{2}}\right)$ and ${\symbf{T}_{2}} \cos\left({θ_{2}}\right)$ and then substituting into eqaution \hyperref[GD:xForce1]{GD:xForce1} and \hyperref[GD:yForce1]{GD:yForce1} , We can get equations 1 and 2:

\begin{displaymath}
{m_{1}} {a_{\text{x}1}}=-{\symbf{T}_{1}} \sin\left({θ_{1}}\right)-{m_{2}} {a_{\text{x}2}}
\end{displaymath}

\begin{displaymath}
{m_{1}} {a_{\text{y}1}}={\symbf{T}_{1}} \cos\left({θ_{1}}\right)-{m_{2}} {a_{\text{y}2}}-{m_{2}} \symbf{g}-{m_{1}} \symbf{g}
\end{displaymath}
Multiply the equation 1 by $\cos\left({θ_{1}}\right)$ and the equation 2 by $\sin\left({θ_{1}}\right)$ and rearrange to get:

\begin{displaymath}
{\symbf{T}_{1}} \sin\left({θ_{1}}\right) \cos\left({θ_{1}}\right)=-\cos\left({θ_{1}}\right) \left({m_{1}} {a_{\text{x}1}}+{m_{2}} {a_{\text{x}2}}\right)
\end{displaymath}

\begin{displaymath}
{\symbf{T}_{1}} \sin\left({θ_{1}}\right) \cos\left({θ_{1}}\right)=\sin\left({θ_{1}}\right) \left({m_{1}} {a_{\text{y}1}}+{m_{2}} {a_{\text{y}2}}+{m_{2}} \symbf{g}+{m_{1}} \symbf{g}\right)
\end{displaymath}
This leads to the equation 3

\begin{displaymath}
\sin\left({θ_{1}}\right) \left({m_{1}} {a_{\text{y}1}}+{m_{2}} {a_{\text{y}2}}+{m_{2}} \symbf{g}+{m_{1}} \symbf{g}\right)=-\cos\left({θ_{1}}\right) \left({m_{1}} {a_{\text{x}1}}+{m_{2}} {a_{\text{x}2}}\right)
\end{displaymath}
Next, multiply equation \hyperref[GD:xForce2]{GD:xForce2} by $\cos\left({θ_{2}}\right)$ and equation \hyperref[GD:yForce2]{GD:yForce2} by $\sin\left({θ_{2}}\right)$ and rearrange to get:

\begin{displaymath}
{\symbf{T}_{2}} \sin\left({θ_{2}}\right) \cos\left({θ_{2}}\right)=-\cos\left({θ_{2}}\right) {m_{2}} {a_{\text{x}2}}
\end{displaymath}

\begin{displaymath}
{\symbf{T}_{1}} \sin\left({θ_{2}}\right) \cos\left({θ_{2}}\right)=\sin\left({θ_{2}}\right) \left({m_{2}} {a_{\text{y}2}}+{m_{2}} \symbf{g}\right)
\end{displaymath}
which leads to equation 4

\begin{displaymath}
\sin\left({θ_{2}}\right) \left({m_{2}} {a_{\text{y}2}}+{m_{2}} \symbf{g}\right)=-\cos\left({θ_{2}}\right) {m_{2}} {a_{\text{x}2}}
\end{displaymath}
By giving equations \hyperref[GD:accelerationX1]{GD:accelerationX1} and \hyperref[GD:accelerationX2]{GD:accelerationX2} and \hyperref[GD:accelerationY1]{GD:accelerationY1} and \hyperref[GD:accelerationY2]{GD:accelerationY2} plus additional two equations, 3 and 4, we can get \hyperref[IM:calOfAngularAcceleration1]{IM:calOfAngularAcceleration1} and \hyperref[IM:calOfAngularAcceleration2]{IM:calOfAngularAcceleration2} via a computer algebra program:

\subsubsection{Data Constraints}
\label{Sec:DataConstraints}
The \hyperref[Table:InDataConstraints]{Data Constraints Table} shows the data constraints on the input variables. The column for physical constraints gives the physical limitations on the range of values that can be taken by the variable. The uncertainty column provides an estimate of the confidence with which the physical quantities can be measured. This information would be part of the input if one were performing an uncertainty quantification exercise. The constraints are conservative, to give the user of the model the flexibility to experiment with unusual situations. The column of typical values is intended to provide a feel for a common scenario.

\begin{longtable}{l l l l}
\toprule
\textbf{Var} & \textbf{Physical Constraints} & \textbf{Typical Value} & \textbf{Uncert.}
\\
\midrule
\endhead
${L_{1}}$ & ${L_{1}}\gt{}0$ & $1.0$ ${\text{m}}$ & 10$\%$
\\
${L_{2}}$ & ${L_{2}}\gt{}0$ & $1.0$ ${\text{m}}$ & 10$\%$
\\
${m_{1}}$ & ${m_{1}}\gt{}0$ & $0.5$ ${\text{kg}}$ & 10$\%$
\\
${m_{2}}$ & ${m_{2}}\gt{}0$ & $0.5$ ${\text{kg}}$ & 10$\%$
\\
${θ_{1}}$ & ${θ_{1}}\gt{}0$ & $30.0$ ${{}^{\circ}}$ & 10$\%$
\\
${θ_{2}}$ & ${θ_{2}}\gt{}0$ & $30.0$ ${{}^{\circ}}$ & 10$\%$
\\
\bottomrule
\caption{Input Data Constraints}
\label{Table:InDataConstraints}
\end{longtable}
\subsubsection{Properties of a Correct Solution}
\label{Sec:CorSolProps}
The \hyperref[Table:OutDataConstraints]{Data Constraints Table} shows the data constraints on the output variables. The column for physical constraints gives the physical limitations on the range of values that can be taken by the variable.

\begin{longtable}{l l}
\toprule
\textbf{Var} & \textbf{Physical Constraints}
\\
\midrule
\endhead
${α_{1}}$ & ${α_{1}}\gt{}0$
\\
${α_{2}}$ & ${α_{2}}\gt{}0$
\\
\bottomrule
\caption{Output Data Constraints}
\label{Table:OutDataConstraints}
\end{longtable}
\section{Requirements}
\label{Sec:Requirements}
This section provides the functional requirements, the tasks and behaviours that the software is expected to complete, and the non-functional requirements, the qualities that the software is expected to exhibit.

\subsection{Functional Requirements}
\label{Sec:FRs}
This section provides the functional requirements, the tasks and behaviours that the software is expected to complete.

\begin{itemize}
\item[Input-Values:\phantomsection\label{inputValues}]{Input the values from \hyperref[Table:ReqInputs]{Tab:ReqInputs}.}
\item[Verify-Input-Values:\phantomsection\label{verifyInptVals}]{Check the entered input values to ensure that they do not exceed the \hyperref[Sec:DataConstraints]{data constraints}. If any of the input values are out of bounds, an error message is displayed and the calculations stop.}
\item[Calculate-Angular-Position-Of-Mass:\phantomsection\label{calcAngPos}]{Calculate the following values: ${α_{1}}$ and ${α_{2}}$ (from \hyperref[IM:calOfAngularAcceleration1]{IM:calOfAngularAcceleration1}) (from \hyperref[IM:calOfAngularAcceleration2]{IM:calOfAngularAcceleration2}).}
\item[Output-Values:\phantomsection\label{outputValues}]{Output ${α_{1}}$ and ${α_{2}}$ (from \hyperref[IM:calOfAngularAcceleration1]{IM:calOfAngularAcceleration1} and \hyperref[IM:calOfAngularAcceleration2]{IM:calOfAngularAcceleration2}).}
\end{itemize}
\begin{longtable}{l l l}
\toprule
\textbf{Symbol} & \textbf{Description} & \textbf{Units}
\\
\midrule
\endhead
${L_{1}}$ & The length of the first rod & ${\text{m}}$
\\
${L_{2}}$ & The length of the second rod & ${\text{m}}$
\\
${m_{1}}$ & The mass of the first object & ${\text{kg}}$
\\
${m_{2}}$ & The mass of the second object & ${\text{kg}}$
\\
${θ_{1}}$ & The angle of the first rod & ${{}^{\circ}}$
\\
${θ_{2}}$ & The angle of the second rod & ${{}^{\circ}}$
\\
\bottomrule
\caption{Required Inputs following \hyperref[inputValues]{FR:Input-Values}}
\label{Table:ReqInputs}
\end{longtable}
\subsection{Non-Functional Requirements}
\label{Sec:NFRs}
This section provides the non-functional requirements, the qualities that the software is expected to exhibit.

\begin{itemize}
\item[Correct:\phantomsection\label{correct}]{The outputs of the code have the \hyperref[Sec:CorSolProps]{properties of a correct solution}.}
\item[Portable:\phantomsection\label{portable}]{The code is able to be run in different environments.}
\end{itemize}
\section{Traceability Matrices and Graphs}
\label{Sec:TraceMatrices}
The purpose of the traceability matrices is to provide easy references on what has to be additionally modified if a certain component is changed. Every time a component is changed, the items in the column of that component that are marked with an ``X'' should be modified as well. \hyperref[Table:TraceMatAvsA]{Tab:TraceMatAvsA} shows the dependencies of assumptions on the assumptions. \hyperref[Table:TraceMatAvsAll]{Tab:TraceMatAvsAll} shows the dependencies of data definitions, theoretical models, general definitions, instance models, requirements, likely changes, and unlikely changes on the assumptions. \hyperref[Table:TraceMatRefvsRef]{Tab:TraceMatRefvsRef} shows the dependencies of data definitions, theoretical models, general definitions, and instance models with each other. \hyperref[Table:TraceMatAllvsR]{Tab:TraceMatAllvsR} shows the dependencies of requirements, goal statements on the data definitions, theoretical models, general definitions, and instance models.

\begin{longtable}{l l l l l l l l}
\toprule
\textbf{} & \textbf{\hyperref[twoDMotion]{A:twoDMotion}} & \textbf{\hyperref[cartSys]{A:cartSys}} & \textbf{\hyperref[cartSysR]{A:cartSysR}} & \textbf{\hyperref[yAxisDir]{A:yAxisDir}} & \textbf{\hyperref[startOrigin]{A:startOrigin}} & \textbf{\hyperref[firstPend]{A:firstPend}} & \textbf{\hyperref[secondPend]{A:secondPend}}
\\
\midrule
\endhead
\hyperref[twoDMotion]{A:twoDMotion} &  &  &  &  &  &  & 
\\
\hyperref[cartSys]{A:cartSys} &  &  &  &  &  &  & 
\\
\hyperref[cartSysR]{A:cartSysR} &  &  &  &  &  &  & 
\\
\hyperref[yAxisDir]{A:yAxisDir} &  &  &  &  &  &  & 
\\
\hyperref[startOrigin]{A:startOrigin} &  &  &  &  &  &  & 
\\
\hyperref[firstPend]{A:firstPend} &  &  &  &  &  &  & 
\\
\hyperref[secondPend]{A:secondPend} &  &  &  &  &  &  & 
\\
\bottomrule
\caption{Traceability Matrix Showing the Connections Between Assumptions and Other Assumptions}
\label{Table:TraceMatAvsA}
\end{longtable}
\begin{longtable}{l l l l l l l l}
\toprule
\textbf{} & \textbf{\hyperref[twoDMotion]{A:twoDMotion}} & \textbf{\hyperref[cartSys]{A:cartSys}} & \textbf{\hyperref[cartSysR]{A:cartSysR}} & \textbf{\hyperref[yAxisDir]{A:yAxisDir}} & \textbf{\hyperref[startOrigin]{A:startOrigin}} & \textbf{\hyperref[firstPend]{A:firstPend}} & \textbf{\hyperref[secondPend]{A:secondPend}}
\\
\midrule
\endhead
\hyperref[DD:positionGDD]{DD:positionGDD} &  &  &  &  &  &  & 
\\
\hyperref[DD:positionXDD1]{DD:positionXDD1} &  &  &  &  &  &  & 
\\
\hyperref[DD:positionYDD1]{DD:positionYDD1} &  &  &  &  &  &  & 
\\
\hyperref[DD:positionXDD2]{DD:positionXDD2} &  &  &  &  &  &  & 
\\
\hyperref[DD:positionYDD2]{DD:positionYDD2} &  &  &  &  &  &  & 
\\
\hyperref[DD:accelerationGDD]{DD:accelerationGDD} &  &  &  &  &  &  & 
\\
\hyperref[DD:forceGDD]{DD:forceGDD} &  &  &  &  &  &  & 
\\
\hyperref[TM:acceleration]{TM:acceleration} &  &  &  &  &  &  & 
\\
\hyperref[TM:velocity]{TM:velocity} &  &  &  &  &  &  & 
\\
\hyperref[TM:NewtonSecLawMot]{TM:NewtonSecLawMot} &  &  &  &  &  &  & 
\\
\hyperref[GD:velocityX1]{GD:velocityX1} &  &  &  &  &  &  & 
\\
\hyperref[GD:velocityY1]{GD:velocityY1} &  &  &  &  &  &  & 
\\
\hyperref[GD:velocityX2]{GD:velocityX2} &  &  &  &  &  &  & 
\\
\hyperref[GD:velocityY2]{GD:velocityY2} &  &  &  &  &  &  & 
\\
\hyperref[GD:accelerationX1]{GD:accelerationX1} &  &  &  &  &  &  & 
\\
\hyperref[GD:accelerationY1]{GD:accelerationY1} &  &  &  &  &  &  & 
\\
\hyperref[GD:accelerationX2]{GD:accelerationX2} &  &  &  &  &  &  & 
\\
\hyperref[GD:accelerationY2]{GD:accelerationY2} &  &  &  &  &  &  & 
\\
\hyperref[GD:xForce1]{GD:xForce1} &  &  &  &  &  &  & 
\\
\hyperref[GD:yForce1]{GD:yForce1} &  &  &  &  &  &  & 
\\
\hyperref[GD:xForce2]{GD:xForce2} &  &  &  &  &  &  & 
\\
\hyperref[GD:yForce2]{GD:yForce2} &  &  &  &  &  &  & 
\\
\hyperref[IM:calOfAngularAcceleration1]{IM:calOfAngularAcceleration1} &  &  &  &  &  &  & 
\\
\hyperref[IM:calOfAngularAcceleration2]{IM:calOfAngularAcceleration2} &  &  &  &  &  &  & 
\\
\hyperref[inputValues]{FR:Input-Values} &  &  &  &  &  &  & 
\\
\hyperref[verifyInptVals]{FR:Verify-Input-Values} &  &  &  &  &  &  & 
\\
\hyperref[calcAngPos]{FR:Calculate-Angular-Position-Of-Mass} &  &  &  &  &  &  & 
\\
\hyperref[outputValues]{FR:Output-Values} &  &  &  &  &  &  & 
\\
\hyperref[correct]{NFR:Correct} &  &  &  &  &  &  & 
\\
\hyperref[portable]{NFR:Portable} &  &  &  &  &  &  & 
\\
\bottomrule
\caption{Traceability Matrix Showing the Connections Between Assumptions and Other Items}
\label{Table:TraceMatAvsAll}
\end{longtable}
\begin{longtable}{l l l l l l l l l l l l l l l l l l l l l l l l l}
\toprule
\textbf{} & \textbf{\hyperref[DD:positionGDD]{DD:positionGDD}} & \textbf{\hyperref[DD:positionXDD1]{DD:positionXDD1}} & \textbf{\hyperref[DD:positionYDD1]{DD:positionYDD1}} & \textbf{\hyperref[DD:positionXDD2]{DD:positionXDD2}} & \textbf{\hyperref[DD:positionYDD2]{DD:positionYDD2}} & \textbf{\hyperref[DD:accelerationGDD]{DD:accelerationGDD}} & \textbf{\hyperref[DD:forceGDD]{DD:forceGDD}} & \textbf{\hyperref[TM:acceleration]{TM:acceleration}} & \textbf{\hyperref[TM:velocity]{TM:velocity}} & \textbf{\hyperref[TM:NewtonSecLawMot]{TM:NewtonSecLawMot}} & \textbf{\hyperref[GD:velocityX1]{GD:velocityX1}} & \textbf{\hyperref[GD:velocityY1]{GD:velocityY1}} & \textbf{\hyperref[GD:velocityX2]{GD:velocityX2}} & \textbf{\hyperref[GD:velocityY2]{GD:velocityY2}} & \textbf{\hyperref[GD:accelerationX1]{GD:accelerationX1}} & \textbf{\hyperref[GD:accelerationY1]{GD:accelerationY1}} & \textbf{\hyperref[GD:accelerationX2]{GD:accelerationX2}} & \textbf{\hyperref[GD:accelerationY2]{GD:accelerationY2}} & \textbf{\hyperref[GD:xForce1]{GD:xForce1}} & \textbf{\hyperref[GD:yForce1]{GD:yForce1}} & \textbf{\hyperref[GD:xForce2]{GD:xForce2}} & \textbf{\hyperref[GD:yForce2]{GD:yForce2}} & \textbf{\hyperref[IM:calOfAngularAcceleration1]{IM:calOfAngularAcceleration1}} & \textbf{\hyperref[IM:calOfAngularAcceleration2]{IM:calOfAngularAcceleration2}}
\\
\midrule
\endhead
\hyperref[DD:positionGDD]{DD:positionGDD} &  &  &  &  &  &  &  &  &  &  &  &  &  &  &  &  &  &  &  &  &  &  &  & 
\\
\hyperref[DD:positionXDD1]{DD:positionXDD1} &  &  &  &  &  &  &  &  &  &  &  &  &  &  &  &  &  &  &  &  &  &  &  & 
\\
\hyperref[DD:positionYDD1]{DD:positionYDD1} &  &  &  &  &  &  &  &  &  &  &  &  &  &  &  &  &  &  &  &  &  &  &  & 
\\
\hyperref[DD:positionXDD2]{DD:positionXDD2} &  &  &  &  &  &  &  &  &  &  &  &  &  &  &  &  &  &  &  &  &  &  &  & 
\\
\hyperref[DD:positionYDD2]{DD:positionYDD2} &  &  &  &  &  &  &  &  &  &  &  &  &  &  &  &  &  &  &  &  &  &  &  & 
\\
\hyperref[DD:accelerationGDD]{DD:accelerationGDD} &  &  &  &  &  &  &  &  &  &  &  &  &  &  &  &  &  &  &  &  &  &  &  & 
\\
\hyperref[DD:forceGDD]{DD:forceGDD} &  &  &  &  &  &  &  &  &  &  &  &  &  &  &  &  &  &  &  &  &  &  &  & 
\\
\hyperref[TM:acceleration]{TM:acceleration} &  &  &  &  &  &  &  &  &  &  &  &  &  &  &  &  &  &  &  &  &  &  &  & 
\\
\hyperref[TM:velocity]{TM:velocity} &  &  &  &  &  &  &  &  &  &  &  &  &  &  &  &  &  &  &  &  &  &  &  & 
\\
\hyperref[TM:NewtonSecLawMot]{TM:NewtonSecLawMot} &  &  &  &  &  &  &  &  &  &  &  &  &  &  &  &  &  &  &  &  &  &  &  & 
\\
\hyperref[GD:velocityX1]{GD:velocityX1} & X & X &  &  &  &  &  &  &  &  &  &  &  &  &  &  &  &  &  &  &  &  &  & 
\\
\hyperref[GD:velocityY1]{GD:velocityY1} & X &  & X &  &  &  &  &  &  &  &  &  &  &  &  &  &  &  &  &  &  &  &  & 
\\
\hyperref[GD:velocityX2]{GD:velocityX2} & X &  &  & X &  &  &  &  &  &  &  &  &  &  &  &  &  &  &  &  &  &  &  & 
\\
\hyperref[GD:velocityY2]{GD:velocityY2} & X &  &  &  & X &  &  &  &  &  &  &  &  &  &  &  &  &  &  &  &  &  &  & 
\\
\hyperref[GD:accelerationX1]{GD:accelerationX1} &  &  &  &  &  &  &  &  &  &  &  &  &  &  &  &  &  &  &  &  &  &  &  & 
\\
\hyperref[GD:accelerationY1]{GD:accelerationY1} &  &  &  &  &  &  &  &  &  &  &  &  &  &  &  &  &  &  &  &  &  &  &  & 
\\
\hyperref[GD:accelerationX2]{GD:accelerationX2} &  &  &  &  &  &  &  &  &  &  &  &  &  &  &  &  &  &  &  &  &  &  &  & 
\\
\hyperref[GD:accelerationY2]{GD:accelerationY2} &  &  &  &  &  &  &  &  &  &  &  &  &  &  &  &  &  &  &  &  &  &  &  & 
\\
\hyperref[GD:xForce1]{GD:xForce1} &  &  &  &  &  &  &  &  &  &  &  &  &  &  &  &  &  &  &  &  &  &  &  & 
\\
\hyperref[GD:yForce1]{GD:yForce1} &  &  &  &  &  &  &  &  &  &  &  &  &  &  &  &  &  &  &  &  &  &  &  & 
\\
\hyperref[GD:xForce2]{GD:xForce2} &  &  &  &  &  &  &  &  &  &  &  &  &  &  &  &  &  &  &  &  &  &  &  & 
\\
\hyperref[GD:yForce2]{GD:yForce2} &  &  &  &  &  &  &  &  &  &  &  &  &  &  &  &  &  &  &  &  &  &  &  & 
\\
\hyperref[IM:calOfAngularAcceleration1]{IM:calOfAngularAcceleration1} &  &  &  &  &  &  &  &  &  &  &  &  &  &  &  &  &  &  &  &  &  &  &  & 
\\
\hyperref[IM:calOfAngularAcceleration2]{IM:calOfAngularAcceleration2} &  &  &  &  &  &  &  &  &  &  &  &  &  &  & X & X & X & X & X & X & X & X & X & X
\\
\bottomrule
\caption{Traceability Matrix Showing the Connections Between Items and Other Sections}
\label{Table:TraceMatRefvsRef}
\end{longtable}
\begin{longtable}{l l l l l l l l l l l l l l l l l l l l l l l l l l l l l l l}
\toprule
\textbf{} & \textbf{\hyperref[DD:positionGDD]{DD:positionGDD}} & \textbf{\hyperref[DD:positionXDD1]{DD:positionXDD1}} & \textbf{\hyperref[DD:positionYDD1]{DD:positionYDD1}} & \textbf{\hyperref[DD:positionXDD2]{DD:positionXDD2}} & \textbf{\hyperref[DD:positionYDD2]{DD:positionYDD2}} & \textbf{\hyperref[DD:accelerationGDD]{DD:accelerationGDD}} & \textbf{\hyperref[DD:forceGDD]{DD:forceGDD}} & \textbf{\hyperref[TM:acceleration]{TM:acceleration}} & \textbf{\hyperref[TM:velocity]{TM:velocity}} & \textbf{\hyperref[TM:NewtonSecLawMot]{TM:NewtonSecLawMot}} & \textbf{\hyperref[GD:velocityX1]{GD:velocityX1}} & \textbf{\hyperref[GD:velocityY1]{GD:velocityY1}} & \textbf{\hyperref[GD:velocityX2]{GD:velocityX2}} & \textbf{\hyperref[GD:velocityY2]{GD:velocityY2}} & \textbf{\hyperref[GD:accelerationX1]{GD:accelerationX1}} & \textbf{\hyperref[GD:accelerationY1]{GD:accelerationY1}} & \textbf{\hyperref[GD:accelerationX2]{GD:accelerationX2}} & \textbf{\hyperref[GD:accelerationY2]{GD:accelerationY2}} & \textbf{\hyperref[GD:xForce1]{GD:xForce1}} & \textbf{\hyperref[GD:yForce1]{GD:yForce1}} & \textbf{\hyperref[GD:xForce2]{GD:xForce2}} & \textbf{\hyperref[GD:yForce2]{GD:yForce2}} & \textbf{\hyperref[IM:calOfAngularAcceleration1]{IM:calOfAngularAcceleration1}} & \textbf{\hyperref[IM:calOfAngularAcceleration2]{IM:calOfAngularAcceleration2}} & \textbf{\hyperref[inputValues]{FR:Input-Values}} & \textbf{\hyperref[verifyInptVals]{FR:Verify-Input-Values}} & \textbf{\hyperref[calcAngPos]{FR:Calculate-Angular-Position-Of-Mass}} & \textbf{\hyperref[outputValues]{FR:Output-Values}} & \textbf{\hyperref[correct]{NFR:Correct}} & \textbf{\hyperref[portable]{NFR:Portable}}
\\
\midrule
\endhead
\hyperref[motionMass]{GS:motionMass} &  &  &  &  &  &  &  &  &  &  &  &  &  &  &  &  &  &  &  &  &  &  &  &  &  &  &  &  &  & 
\\
\hyperref[inputValues]{FR:Input-Values} &  &  &  &  &  &  &  &  &  &  &  &  &  &  &  &  &  &  &  &  &  &  &  &  &  &  &  &  &  & 
\\
\hyperref[verifyInptVals]{FR:Verify-Input-Values} &  &  &  &  &  &  &  &  &  &  &  &  &  &  &  &  &  &  &  &  &  &  &  &  &  &  &  &  &  & 
\\
\hyperref[calcAngPos]{FR:Calculate-Angular-Position-Of-Mass} &  &  &  &  &  &  &  &  &  &  &  &  &  &  &  &  &  &  &  &  &  &  & X & X &  &  &  &  &  & 
\\
\hyperref[outputValues]{FR:Output-Values} &  &  &  &  &  &  &  &  &  &  &  &  &  &  &  &  &  &  &  &  &  &  & X & X &  &  &  &  &  & 
\\
\hyperref[correct]{NFR:Correct} &  &  &  &  &  &  &  &  &  &  &  &  &  &  &  &  &  &  &  &  &  &  &  &  &  &  &  &  &  & 
\\
\hyperref[portable]{NFR:Portable} &  &  &  &  &  &  &  &  &  &  &  &  &  &  &  &  &  &  &  &  &  &  &  &  &  &  &  &  &  & 
\\
\bottomrule
\caption{Traceability Matrix Showing the Connections Between Requirements, Goal Statements and Other Items}
\label{Table:TraceMatAllvsR}
\end{longtable}
The purpose of the traceability graphs is also to provide easy references on what has to be additionally modified if a certain component is changed. The arrows in the graphs represent dependencies. The component at the tail of an arrow is depended on by the component at the head of that arrow. Therefore, if a component is changed, the components that it points to should also be changed. \hyperref[Figure:TraceGraphAvsA]{Fig:TraceGraphAvsA} shows the dependencies of assumptions on the assumptions. \hyperref[Figure:TraceGraphAvsAll]{Fig:TraceGraphAvsAll} shows the dependencies of data definitions, theoretical models, general definitions, instance models, requirements, likely changes, and unlikely changes on the assumptions. \hyperref[Figure:TraceGraphRefvsRef]{Fig:TraceGraphRefvsRef} shows the dependencies of data definitions, theoretical models, general definitions, and instance models with each other. \hyperref[Figure:TraceGraphAllvsR]{Fig:TraceGraphAllvsR} shows the dependencies of requirements, goal statements on the data definitions, theoretical models, general definitions, and instance models. \hyperref[Figure:TraceGraphAllvsAll]{Fig:TraceGraphAllvsAll} shows the dependencies of dependencies of assumptions, models, definitions, requirements, goals, and changes with each other.

\begin{figure}
\begin{center}
\includesvg[width=\textwidth, inkscapelatex = false]{../../../../traceygraphs/dblpendulum/avsa}
\caption{TraceGraphAvsA}
\label{Figure:TraceGraphAvsA}
\end{center}
\end{figure}
\begin{figure}
\begin{center}
\includesvg[width=\textwidth, inkscapelatex = false]{../../../../traceygraphs/dblpendulum/avsall}
\caption{TraceGraphAvsAll}
\label{Figure:TraceGraphAvsAll}
\end{center}
\end{figure}
\begin{figure}
\begin{center}
\includesvg[width=\textwidth, inkscapelatex = false]{../../../../traceygraphs/dblpendulum/refvsref}
\caption{TraceGraphRefvsRef}
\label{Figure:TraceGraphRefvsRef}
\end{center}
\end{figure}
\begin{figure}
\begin{center}
\includesvg[width=\textwidth, inkscapelatex = false]{../../../../traceygraphs/dblpendulum/allvsr}
\caption{TraceGraphAllvsR}
\label{Figure:TraceGraphAllvsR}
\end{center}
\end{figure}
\begin{figure}
\begin{center}
\includesvg[width=\textwidth, inkscapelatex = false]{../../../../traceygraphs/dblpendulum/allvsall}
\caption{TraceGraphAllvsAll}
\label{Figure:TraceGraphAllvsAll}
\end{center}
\end{figure}
For convenience, the following graphs can be found at the links below:

\begin{itemize}
\item{\hyperref{../../../../traceygraphs/dblpendulum/avsa.svg}{}{}{TraceGraphAvsA}}
\item{\hyperref{../../../../traceygraphs/dblpendulum/avsall.svg}{}{}{TraceGraphAvsAll}}
\item{\hyperref{../../../../traceygraphs/dblpendulum/refvsref.svg}{}{}{TraceGraphRefvsRef}}
\item{\hyperref{../../../../traceygraphs/dblpendulum/allvsr.svg}{}{}{TraceGraphAllvsR}}
\item{\hyperref{../../../../traceygraphs/dblpendulum/allvsall.svg}{}{}{TraceGraphAllvsAll}}
\end{itemize}
\section{Values of Auxiliary Constants}
\label{Sec:AuxConstants}
There are no auxiliary constants.

\section{References}
\label{Sec:References}
\begin{filecontents*}{bibfile.bib}
@book{hibbeler2004,
author={Hibbeler, R. C.},
title={Engineering Mechanics: Dynamics},
publisher={Pearson Prentice Hall},
year={2004}}
@mastersthesis{koothoor2013,
author={Koothoor, Nirmitha},
title={A document drive approach to certifying scientific computing software},
school={McMaster University},
year={2013},
address={Hamilton, ON, Canada}}
@article{parnasClements1986,
author={Parnas, David L. and Clements, P. C.},
title={A rational design process: How and why to fake it},
journal={IEEE Transactions on Software Engineering},
year={1986},
month=feb,
volume={12},
number={2},
pages={251--257},
address={Washington, USA}}
@inproceedings{smithLai2005,
author={Smith, W. Spencer and Lai, Lei},
title={A new requirements template for scientific computing},
booktitle={Proceedings of the First International Workshop on Situational Requirements Engineering Processes - Methods, Techniques and Tools to Support Situation-Specific Requirements Engineering Processes, SREP'05},
year={2005},
editor={Agerfalk, PJ and Kraiem, N. and Ralyte, J.},
address={Paris, France},
pages={107--121},
note={In conjunction with 13th IEEE International Requirements Engineering Conference,}}
@misc{accelerationWiki,
author={Wikipedia Contributors},
title={Acceleration},
howpublished={\url{https://en.wikipedia.org/wiki/Acceleration}},
month=jun,
year={2019}}
@misc{cartesianWiki,
author={Wikipedia Contributors},
title={Cartesian coordinate system},
howpublished={\url{https://en.wikipedia.org/wiki/Cartesian\_coordinate\_system}},
month=jun,
year={2019}}
@misc{velocityWiki,
author={Wikipedia Contributors},
title={Velocity},
howpublished={\url{https://en.wikipedia.org/wiki/Velocity}},
month=jun,
year={2019}}
\end{filecontents*}
\nocite{*}
\bibstyle{ieeetr}
\printbibliography[heading=none]
\end{document}
