\documentclass[10pt,conference]{IEEEtran}
\bibliographystyle{plainurl}% the recommnded bibstyle

\usepackage{listings}
\usepackage{amsmath}
\usepackage{hyperref}
\usepackage{cite}

\begin{document}
\title{The Contents of (Scientific) Software}

%\titlerunning{Dummy short title}%optional, please use if title is longer than one line

%\author{Jacques Carette}{Department of Computing and Software, McMaster University, Canada}{carette@mcmaster.ca}{orcid}{[funding]}

%\author{W. Spencer Smith}{Department of Computing and Software, McMaster University, Canada}{smiths@mcmaster.ca}{orcid}{[funding]}

\author{\IEEEauthorblockN{Jacques Carette}
\IEEEauthorblockA{Department of Computing and Software\\
McMaster University, Canada\\
Email: carette@mcmaster.ca}
\and
\IEEEauthorblockN{W. Spencer Smith}
\IEEEauthorblockA{Department of Computing and Software\\
McMaster University, Canada\\
Email: smiths@mcmaster.ca}
}
%\authorrunning{J. Carette and S. Smith}

% \Copyright{Jacques Carette and W. Spencer Smith}

%\subjclass{Dummy classification}% mandatory: Please choose ACM 2012 classifications from https://www.acm.org/publications/class-2012 or https://dl.acm.org/ccs/ccs_flat.cfm . E.g., cite as "General and reference $\rightarrow$ General literature" or \ccsdesc[100]{General and reference~General literature}. 

%\keywords{Dummy keyword}%mandatory

%\category{}%optional, e.g. invited paper

%\relatedversion{}%optional, e.g. full version hosted on arXiv, HAL, or other respository/website

%\supplement{}%optional, e.g. related research data, source code, ... hosted on a repository like zenodo, figshare, GitHub, ...

%\funding{}%optional, to capture a funding statement, which applies to all authors. Please enter author specific funding statements as fifth argument of the \author macro.

%\acknowledgements{I want to thank \dots}%optional

\maketitle

\begin{abstract}
To do when the paper is mostly written.
\end{abstract}

\section{Main Ideas}
\begin{itemize}
\item Software is way more than just code
\item Let's call the collection of artifacts a 'software product' [better name?]
\item (Scientific) software products are 'made up' of distinct pieces of knowledge
\item From a working prototype of a "knowlege ---assemble--> software product", we can witness what the ingredients are
\item Here's an explained list of all these pieces, why they occur, and some design ideas related to them 
\end{itemize}

\section{Introduction}
\begin{enumerate}
  \item Software is more than just code.
  \item The collection of artifacts is highly redundant -- on purpose. Different views of the same thing.
  \item Some software is better understood than others (scientific), so easier to analyze
  \item "Better understood" ~~ DSL + generator(s)
  \item Traceability, coherence, parsimony, quantum of re-use 
\end{enumerate}

\section{What is 'Software'}
\begin{enumerate}
  \item Basically a list of all (potential) artifacts that make up a 'software product'
  \item Some explanation of each (most?), how they differ, how they are the same/linked
    (some emphasis again on the 'views' idea)
  \item Natural explanation that the "knowlege ---assemble--> software product" picture is more of a sequence of trees
  \item Reminder that 'assemble' is (procedural) knowledge, but knowledge nevertheless 
\end{enumerate}

\section{Background}
\begin{enumerate}
  \item Drasil as a prototype.
  \item Examples in Drasil.
  \item Drasil as itself a source of knowledge. 
\end{enumerate}

\section{The Contents}
  A bottom-up analysis of the 'content'. [incomplete list]
  \begin{enumerate}
  \item People, Space, UID, ShortName, Label, Symbol, Stage, Expr, UnitLang, Sentence, NounPhrase, Date, Citation, Reference, Uncertainty, ReasValue, Constraint
  \item HasUID (Chunk), NamedIdea, Idea, CommonIdea, Concept, Quantity, Unitary, Equation, DataDefinition, UnitaryConcept,
    Theory, General Definition, Defined Quantity, Instance Model, Unital, Constrained, UncertainQuantity
  [need to analyze drasil-code, -data, -docLang, -gen and -printers too. -example?]
  \item generic information about units, concepts in physics (etc) that recur
  \item some CS knowledge 
  \end{enumerate}

\section{Discussion}
\begin{enumerate}
  \item Representing software-related knowledge in a (re)usable way
  \item Traceability, coherence, parsimony, quantum of re-use
  \item Ultimate: self-representation 
\end{enumerate}

\section{Conclusion}

Take home:
\begin{itemize}
\item Software is way more than just code
\item ``knowledge'' is a better unit of re-use than code
\item There is some 'unexpected' (but easily explained) knowledge in Software 
\end{itemize}

%%
%% Bibliography
%%

\bibliography{cites}


\end{document}
