\documentclass{article}

\usepackage{tabularx}
\usepackage{booktabs}

\title{Problem Statement and Goals\\\progname}

\author{\authname}

\date{January 20, 2023}

%% Comments

\usepackage{color}

\newif\ifcomments\commentstrue %displays comments
%\newif\ifcomments\commentsfalse %so that comments do not display

\ifcomments
\newcommand{\authornote}[3]{\textcolor{#1}{[#3 ---#2]}}
\newcommand{\todo}[1]{\textcolor{red}{[TODO: #1]}}
\else
\newcommand{\authornote}[3]{}
\newcommand{\todo}[1]{}
\fi

\newcommand{\wss}[1]{\authornote{blue}{SS}{#1}} 
\newcommand{\plt}[1]{\authornote{magenta}{TPLT}{#1}} %For explanation of the template
\newcommand{\sjc}[1]{\authornote{cyan}{SC}{#1}}
%% Common Parts

\newcommand{\progname}{Master of Applied Science} 
\newcommand{\authname}{ChemCode
\\ Samuel Crawford}

\usepackage{hyperref}
    \hypersetup{colorlinks=true, linkcolor=blue, citecolor=blue, filecolor=blue,
                urlcolor=blue, unicode=false}
    \urlstyle{same}
                                

\begin{document}

\maketitle

\begin{table}[hp]
\caption{Revision History} \label{TblRevisionHistory}
\begin{tabularx}{\textwidth}{llX}
\toprule
\textbf{Date} & \textbf{Developer(s)} & \textbf{Change}\\
\midrule
Jan. 18, 2023 & Sam & Create document and fill in Problem, Stakeholders, and
Goals sections\\
Jan. 19, 2023 & Sam & Format for Drasil upload, fill in Inputs and Outputs
and Environment sections, update Stakeholders and Goals sections, and move
Environment section\\
\bottomrule
\end{tabularx}
\end{table}

\section{Problem Statement}

\wss{You should check your problem statement with the
\href{https://github.com/smiths/capTemplate/blob/main/docs/Checklists/ProbState-Checklist.pdf}
{problem statement checklist}.}
\wss{You can change the section headings, as long as you include the required
information.}

\subsection{Problem}

Chemistry is a broad field that studies matter and its interactions
\cite{gordon_chm101_2023}, primarily through chemical reactions.
During a chemical reaction, bonds between some substances break and new ones are
formed to create new substances; these reactions are often written as chemical
equations \cite{lund_introduction_2023}. Despite new chemicals being created,
all atoms from the initial substances, or ``reactants", must be present in the
final substances, or ``products" because of the Law of Conservation of Matter
\cite{lund_introduction_2023}. This means that for a chemical equation to be
useful, it must be balanced by changing the coefficients of the substances
involved in the reaction \cite{lund_introduction_2023}. Additionally, since
molecules only exist in whole numbers (since dividing a molecule changes its
composition into new types of molecules), these coefficients must be whole
numbers, and by convention should be as small as possible
\cite{lund_introduction_2023}.

While these equations can be balanced by hand through the process of
``balancing by inspection" \cite{lund_introduction_2023}, this can be
time-consuming, prone
to error, and inefficient, especially for more complicated chemical reactions.
For each element present in the reaction, an equality can be written for the
number of elements in each substance, with the reactants on one side and the
products on the other, using the coefficients of each substance as the
variables \cite{hamid_balancing_2019}. These equalities then form a system of
linear
equations that can be solved through various methods to yield a relation
between each coefficient, which can then be manipulated to find the require whole
numbers \cite{lund_introduction_2023, hamid_balancing_2019}. This method can
also identify reactions that are ``infeasible" and balance reactions involving
fractional oxidation states \cite{hamid_balancing_2019}, which ``are used to
describe the distribution of electrons in a molecule"
\cite{unacademy_fractional_2023}.

\subsection{Inputs and Outputs}

\wss{Characterize the problem in terms of ``high level'' inputs and outputs.
Use abstraction so that you can avoid details.}
\null\newline
\null\newline
\noindent Input:

\begin{itemize}
	\item A representation of a chemical equation
\end{itemize}

\noindent Output:

\begin{itemize}
\item A representation of the inputted chemical equation in its balanced form
with the smallest whole number coefficients possible
\end{itemize}

\subsection{Environment}

\wss{Hardware and software}
\null\newline
\null\newline
\noindent\progname~will be developed using Drasil
\cite{carette_drasil_2021}, ``a framework for generating
high-quality documentation and code for Scientific Computing Software''
\cite[p. iii]{maclachlan_design_2020} by encapsulating scientific knowledge as
``chunks'' to be reused among projects \cite{maclachlan_design_2020}. By 
building this project in Drasil, relevant concepts about chemistry and systems
of linear equations must first be added, along with the capability to solve
these systems. Therefore, a byproduct of this project is that other programs
that use chemistry and/or systems of linear equations can be made using
Drasil. The implementation in Drasil places some constraints on this project.

Since Drasil is built on the idea of reusability, external libraries will be
used to solve these systems of linear equations. This was previously done with
ordinary differential equation (ODE) solvers, since ``creating a complete ODE
solver in Drasil would take considerable time, and there are already many
reliable external libraries \dots tested by long use''
\cite[p. 24]{chen_solving_2022}.

Additionally, Drasil can currently generate code in Python, C++, C\#, Java, and
Swift \cite{chen_solving_2022}. The scope of this project will be limited to 
generating code Python since it is the language in which I \sjc{First person?}
have the most experience. Python also supports the SciPy library
\cite{2020SciPy-NMeth}, which is already used by Drasil to solve ODEs
\cite{chen_solving_2022} and supports solving systems of linear equations \cite{the_scipy_community_scipylinalgsolve_2023}.

%While using external ODE solvers in
%Drasil, the developers ``did not find a suitable
%library for Swift'' \cite[p.24]{chen_solving_2022}; a similar problem may
%arise when using external system of linear equations solvers, meaning that
%\progname~may not be generated in all five languages.

Since both \progname~and Drasil are purely software systems, the only
hardware involved is the user's computer used to run \progname.

\subsection{Stakeholders}
The main stakeholder of this project is Dr.~Spencer Smith, the instructor for
the CAS 741 Development of Scientific Computing Software course for which this
project is being completed. Dr.~Smith and Dr.~Jacques Carette are in charge of
the Drasil project that \progname~seeks to extend, so the implementation and
development process are of significance to them. Likewise, any future
developers of Drasil, including myself \sjc{Can I use the first person?}, are
potential stakeholders of this project, since they may use features added to
Drasil, such as ideas about chemistry or systems of linear equations. Jason
Balaci, a fellow CAS 741 student and Drasil contributor, is of particular 
mention, since
there may be some overlap between our projects so we may be collaborating
throughout this project. I am also a stakeholder of \progname~as the
developer.

More generally, anyone in the field of chemistry in at least a high-school level
may be a stakeholder of this project, as they may use this tool in their work.

\section{Goals}

The goals of this project are to develop a program that...

\begin{itemize}
    \item can balance chemical equations (including ones with fractional
    oxidation states).
    \item can determine if a given chemical reaction is ``infeasible'' (i.e.,
    not able to be balanced).
    \item is generated by Drasil (along with relevant documentation).
    \item extends Drasil by introducing the concepts from chemistry necessary
    to balance equations, such as elements, compounds, and reactions.
    \item generates code that uses appropriate external libraries to solve
    systems of linear equations (see \cite[Ch. 4]{chen_solving_2022}).
\end{itemize}

\section{Stretch Goals}

\bibliographystyle{ieeetr}
\bibliography{sources}

\end{document}