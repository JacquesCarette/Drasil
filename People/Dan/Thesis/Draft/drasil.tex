\chapter{Drasil}
\label{c:drasil}

\ds{**Section Roadmap:
    -- This is where the real meat of Drasil is discussed (implementation details)
    -- Intro to our knowledge-capture mechanisms
      - Chunks/hierarchy
      - Break down each with examples from the case studies.
      - Look for 'interesting' examples (synonyms, acronyms, complexity, etc.)
    -- Intro to the DSL
      - Captured knowledge is useless without the transformations/rendering engine
      - DSL for each \sf{}}

In this chapter we introduce the Drasil framework and some details of its 
implementation including knowledge-capture mechanisms, the domain specific 
languages used throughout, and how all of these pieces are brought together in 
a human-usable way to generate \sfs{}. The name Drasil, derived from Norse 
Mythology's world tree Yggdrasil whose branches spread across the 
many realms, is representative of how our framework spreads across the many 
domains and contexts relevant to software generation.
      
\section{Drasil in a nutshell}

Manually writing and maintaining a full range of \sfs{} is redundant, tedious, 
and often leads to divergent \sfs{}. Drasil is a purpose-built framework 
created to tackle these problems.

Contrary to documentation generators like Doxygen, Javadoc, and Pandoc
which take a code-centric view of the problem and rely on manual redundancy -- 
i.e. natural-language explanations written as specially delimited
comments which can then be weaved into API documentation alongside code -- 
Drasil takes a knowledge-centric, redundancy-limiting, fully traceable, single 
source approach to generating \sfs{}.

However, Drasil is not, nor is it intended to be, a panacea for all the woes of 
software development. Even the seemingly well-defined problems of unnecessary 
redundancy and manual duplication turn out to be large, many-headed beasts 
which exist across a multitude of software domains; each with their own 
benefits, drawbacks, and challenges.

To reiterate: Drasil has not been designed as a silver-bullet. It is a 
specialized tool meant to be used in well-understood domains for software that 
will undergo frequent maintenance and/or changes. In deciding whether Drasil 
would be useful for developing software to tackle a given problem, we recommend 
identifying those projects that are long-lived (10+ years) with \sfs{} 
relevant to multiple stakeholders. For our purposes, as mentioned earlier, we 
have focused on SC software that follows the input $\rightarrow$ process 
$\rightarrow$ output pattern. SC software has the benefit of being relatively 
slow to change, so models used today may not be updated or invalidated for some 
time, if ever. Should that happen, the models will likely still be applicable 
given a set of assumptions or assuming certain acceptable margins for error.

With Drasil being built around this specific class of problems, we remain aware 
that there are likely many in-built assumptions in its current state that could 
affect its applicability to other domains. As such, we consider expanding 
Drasil's reach as an avenue for future work.

The Drasil framework relies on a knowledge-centric approach to software 
specification and development. We attempt to codify the foundational theory 
behind the problems we are attempting to solve and operationalize it through 
the use of generative technologies. By doing so, we can reuse common knowledge 
across projects and maintain a single source of truth in our knowledge database.

Given how important knowledge is to Drasil, one might think we are building 
ontologies or ontology generators. We must make it clear this is \emph{not} the 
case. We are not attempting to create a source for all knowledge and 
relationships inside a given field. We are merely using the information we have 
available to build up knowledge as needed to solve problems. Over time, this 
may take on the appearance of an ontology, but Drasil does not currently 
enforce any strict rules on how knowledge should be captured, outside of its 
type system and some best practice recommendations. We will explore knowledge 
capture in more depth in Section~\ref{sec:kc}.

\section{Our Ingredients: Organizing and Capturing Knowledge}
\label{sec:kc}

For Drasil to function as intended, we need a means of capturing and organizing 
the underlying knowledge of the software systems we are trying to build 
alongside common knowledge that would be relevant across our entire target 
domain of SC software. This knowledge capture method must also be robust enough 
to be operationalized by a multi-faceted generation framework.

Before we can design a knowledge capture mechanism, we must first define what 
exactly we believe we need to capture. It is nearly impossible to consider 
every case of knowledge that could be used in the domain of SC software, not to 
mention an extremely large undertaking to begin with ``all the things". As 
such we have decided to take an iterative, progressive refinement approach to 
our knowledge capture mechanisms. This should come as no surprise and follows 
from our general process for developing the Drasil framework.

\subsection{Capturing Knowledge via Chunks}

We begin by borrowing and re-purposing the \emph{chunk} term from Literate 
Programming (LP) and use it to create a simplified, extensible, and ever 
expanding hierarchy  of chunks based on the requirements for capturing a single 
piece of knowledge. Our base chunk, \codeH{NamedChunk} is defined in 
Figure~\ref{fig:namedchunk} and can be thought of as any uniquely identifiable 
term. It is composed of a \codeH{UID}, or unique identifier, and a 
\codeH{NP}, or noun phrase, representing the term.

\fig{
\lstinputlisting[language=Haskell, firstline=33, lastline=43, 
firstnumber=33]{code/NamedIdea.hs}}{NamedChunk Definition}{fig:namedchunk}

A single node does not a hierarchy make, but now with the root 
\codeH{NamedChunk} defined, we can begin to progressively extend it to cover 
any new chunk types we may need for our knowledge capture requirements. For 
example, if we want to capture a simple term with its definition, we need more 
than just a \codeH{NamedChunk}. We need to extend \codeH{NamedChunk} to include 
a definition, and we refer to this particular variant chunk as a 
\codeH{ConceptChunk}. For ease of creation, we define a number of semantic, 
so-called \emph{smart constructors} for each chunk type which allow us to more 
simply define our chunks and their intermediary datatypes (ie: \codeH{UID} and 
\codeH{NP} from the \codeH{NamedChunk} example). An example of such smart 
constructors being used to create some simple \codeH{ConceptChunk} instances 
can be seen in Figure~\ref{fig:conceptchunk}. Note there are smart constructors 
for each datatype when multiple variants exist and could be used in a given 
place. Looking at the \codeH{ConceptChunk} example, we can see two different 
smart constructors (\codeH{cnIES} and \codeH{cn'}) being used to create the 
\codeH{NP} instance. Both smart constructors create an instance of \codeH{NP}, 
but in this case the smart constructor used defines given properties (ie. 
pluralization rules of the noun phrase used as a term) for the instance to 
simplify construction.

\fig{
\lstinputlisting[language=Haskell, firstline=76, 
lastline=82, firstnumber=76]{code/Math.hs}}{Some example instances of 
\codeH{ConceptChunk} 
using the \codeH{dcc} smart constructor}{fig:conceptchunk}

These simple chunks are fine as a starting point, however, knowing the SC 
domain we can already foresee the need for a variety of other chunks. For 
brevity, we will only expand upon the chunks needed for the examples used in 
this paper. More information, including detailed definitions of all the types 
and current state of Drasil can be found in the wiki 
(\href{https://github.com/JacquesCarette/Drasil/wiki/}
{https://github.com/JacquesCarette/Drasil/wiki/}) or the Haddock documentation 
which can be generated from the source code or found online at 
\href{https://jacquescarette.github.io/Drasil/docs/full/}
{https://jacquescarette.github.io/Drasil/docs/full/}

\subsection{Chunk Combinatorics}
\label{sec:chunky}
\ds{Would we call our chunk hierarchy a DSL? Or would the DSL be more along the 
lines of Expr and Sentence which make up our chunks?}

Even with extremely simple chunks we have seen the need for multiple chunk 
types and a number of ways to construct instances of those types. The more we 
extend our chunk hierarchy, the larger the number of possible combinations we 
will need to account for. This is not only relevant to chunks themselves, but 
also to the information they encode. Take, for instance, a term definition that 
relies on other terminology that has been captured. In our effort to reduce 
duplication and maintain a single source of information, we intend to be able 
to create chunks with references to other chunks such that the definition can 
use the known terminology.

\ds{Should I work in something about how we use lenses in here, as all complex 
chunks are instances of the simpler chunks they've been built from? Or is that 
getting too into the Haskell-specific details?}

To define a term with references to other terms, we need to ensure we are 
creating our definitions by projecting relevant knowledge into the definition, 
but also ensuring we have the flexibility to extend that knowledge. With that 
in mind, we created a Domain Specific Language (DSL) for creating and combining 
(English language) sentences aptly called \codeH{Sentence}. In its most basic 
form, \codeH{Sentence} will simply wrap a string. However, by defining a number 
of helper functions and other useful utilities, our \codeH{Sentence} DSL can be 
used to combine, change case, pluralize, and more. A simple example of a series 
of chunks that utilize the \codeH{Sentence} DSL to derive their definitions can 
be seen in Figure~\ref{fig:acceleration}. We use the \codeH{phrase} helper 
function to pull the appropriate sentence out of the \codeH{velocity} chunk and 
the combinator (\codeH{+:+}) to concatenate the sentences. Note that position 
uses a different smart constructor (for non-derived definitions) and so doesn't 
require the sentence constructor (\codeH{S}) around its definition.

\fig{
\lstinputlisting[language=Haskell, linerange={59-60,135-136,174-175}, 
numbers=none]{code/Physics.hs}
}{Projecting knowledge into a chunk's definition}{fig:acceleration}

Looking back at our examples of the types of knowledge we are interested in 
from Figure~\ref{fig:gbrddq} we can see that the simple chunks we've defined so 
far do not even start to cover the full spectrum of our needs. We can see a lot 
of information missing such as a symbolic representation ($\hat{q}$) and 
equation ($\hat{q} = \frac{q(ab)^2}{Eh^4\text{GTF}}$). As that is a relatively
complex example, we will start by looking at something far simpler: Newton's 
2nd Law of motion\citep{Newton1687} which, roughly translated, states ``the net 
force on a body at any instant of time is equal to the body's acceleration 
multiplied by its mass or, equivalently, the rate at which the body's momentum 
is changing with time"  or as most physics students recognize it $F=ma$ 
provided we have definitions for $F$, $m$, and $a$.

To understand what exactly is missing, let us look at how we achieve an 
operational definition of \emph{Force} as a chunk in Drasil. To start, we can 
see we need a symbolic representation ($F$) and some way to define an equation. 
Also, each of force, mass, and acceleration are measured in some form of units 
($N$, $kg$, and $m/s^2$ respectively) which we will likely care to capture and 
keep track of. 

\fig{
\lstinputlisting[language=Haskell, linerange={91-92}, 
firstnumber=91]{code/Physics.hs}
}{The force \codeH{ConceptChunk}}{fig:forceCC}

\fig{
\lstinputlisting[language=Haskell, linerange={47-47,57-57}, 
numbers=none]{code/PhysicsQuantities.hs}
}{The force and acceleration \codeH{UnitalChunk}s}{fig:forceUC}

\fig{
\lstinputlisting[language=Haskell, linerange={21-30}, 
firstnumber=21]{code/SI_Units.hs}
}{The fundamental SI Units as Captured in Drasil}{fig:Data.Drasil.SI}

We start by building a \codeH{ConceptChunk} for the Force concept as seen in 
Figure~\ref{fig:forceCC}. It is built from a common noun (``force") and a 
definition. Next we need to capture the units of measurement to the force 
concept by defining what we call a \codeH{UnitalChunk}. This definition for 
force can be seen in Figure~\ref{fig:forceUC}. Note that we are not re-defining 
force in this instance. We are instead using a smart constructor to build atop 
the existing force chunk (\codeH{CP.force}) and adding a symbolic 
representation (\codeH{vec cF}, which is shorthand for the selected vector 
representation\footnote{More on this in Section~\ref{sec:recipes}} of the 
capital letter \emph{F}). We also must capture the units force is measured with 
(\codeH{newton}) which are defined in \codeH{Data.Drasil.SI_Units} and can be 
seen in Figure~\ref{fig:Data.Drasil.SI}. The SI Units are captured using 
another type of chunk known as a \codeH{UnitDefn} which are built atop a 
\codeH{ConceptChunk} and \codeH{UnitSymbol}. The \codeH{UnitSymbol} is also a 
chunk that can be one of several other chunk types.\footnote{For brevity we are 
glossing over many chunk definitions and the differences between them as there 
are chunks all the way down. The full technical documentation can be found on 
our github repository.}. Returning to the force \codeH{UnitalChunk}, the smart 
constructor \codeH{uc} we are using will actually capture even more knowledge 
than just what we've fed it as we want it to assume we are working in the real 
number space. There are other smart constructors for other spaces, which are 
also captured using other chunks elsewhere in Drasil (see: 
\codeH{Language.Drasil.Space}).

\fig{
\lstinputlisting[language=Haskell, linerange={11-20}, 
firstnumber=11]{code/PhysicsUnits.hs}
}{Examples of chunks for units derived from other SI Unit chunks}{fig:accelU}

Now we approach the terms mass and acceleration in a similar manner. We capture 
each as a \codeH{ConceptChunk} (see Figure~\ref{fig:acceleration} for a 
reminder of how we captured acceleration's definition relative to velocity and 
thus relative to position). Then construct a \codeH{UnitalChunk} for each. 
Acceleration can be seen in Figure~\ref{fig:forceUC} and mass has a similar 
\codeH{UnitalChunk} that uses the \codeH{metre UnitDefn} chunk. More 
interestingly, the units for acceleration are derived from other units, and can 
be seen in Figure~\ref{fig:accelU} with some additional derived unit types 
(taken from \codeH{Data.Drasil.Units.Physics}).

Now we have almost everything we need to finish our definition of Newton's 
second law. Everything thus far has been defined in natural language using our 
\codeH{Sentence} DSL, but it is not well-suited to the one piece we are 
currently missing: a way to define expressions relating chunks in a 
mathematics context.

\subsection{Relating Chunks via Expressions}
\label{sec:expr}

Continuing our Newton's Second Law example from Section~\ref{sec:chunky}, we 
need a means to define how force is calculated. We know it is calculated 
relative to mass and acceleration, so we need a way to capture that, preferably 
in an operational manner. We also know acceleration is itself derived from 
time and velocity, which is derived from time and position.

\ds{TODO: Explain the Expr language -- it is essentially an expression language
that captures a host of mathematical operations and allows them to be applied 
to chunks as a whole, rather than just the symbolic representations, so we can 
preserve that captured relationship and describe it anywhere the chunk is used 
throughout the software we are building with Drasil.}

\section{Recipes: Codifying Structure}
\label{sec:recipes}
  - Organized knowledge is fine, but is essentially just a collection of (collections of) definitions. Pretty meaningless on its own so we need the structure (in our case from the templates / case studies) to have meaning.
  - Each \sf{} has its own recipe for combining knowledge
  - As we consider \sfs{} "views" of the knowledge, we need to 
  combine/transform/manipulate the knowledge into a meaningful form for the 
  given view - ex: Math formula for human-readable doc, Function/method for 
  code (show examples).
  - Recipes define the "how" and "where" of putting together the knowledge. The rendering engine reads the recipe and follows its instructions.

\section{Cooking it all up: Generation/Rendering}
  - Recipes are “little programs”
  - Each recipe can be rendered a number of ways, based on parameters fed to the generator.
  -  Implicit parameters vs explicit: Ex. an SRS will always be rendered based on the recipe, but its output will either be LaTeX or HTML based on an explicit choice. Implicit params fed to gen table of symbols/A\&A / ToU.


\section{Iteration and refinement}
\ds{TODO: Figure out what belongs here and what belongs in results}

  - Practical approach to iron out kinks / find holes in Drasil

  - Find places to improve upon the existing case studies - ‘update as you go’ 
  mindset

  - Observe the amount of effort required to correct errors - show examples

  - Most of the code we’ll show off should be in here.

  - Tau example (see issue 348) and its implications - symbols and definitions 
  didn't match. -> implicit 1m depth into the page (means we may need to change 
  the equations). Resistive and mobilizing shear switched throughout the 
  original docs -- impossible with Drasil.

  - \ds{This next one might belong in future work} Implicit assumptions -> 
  Issue 91. We take for granted things are "physical 
  materials", but this is an assumption that could be codified and made 
  explicit to the system (which would allow us some more flexibility).