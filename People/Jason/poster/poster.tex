% Modified from https://www.overleaf.com/latex/templates/stockholm-university-poster-template/xsrhnggqmcsx
% The author, Qi Dang, based their work on that of Jiyu Chen, as mentioned above.
% Modified by Jason Balaci
% Last Modified: 2022-04-12

\documentclass[20pt,margin=1in,innermargin=-1in,blockverticalspace=-0.1in]{tikzposter}
\geometry{paperwidth=36in,paperheight=24in}

\usepackage{xparse}

\usepackage[canadian]{babel}

\usepackage[utf8]{inputenc}
\usepackage[T1]{fontenc}

\usepackage{blindtext}

\usepackage{amsmath}
\usepackage{amsfonts}
\usepackage{amsthm}
\usepackage{amssymb}
\usepackage{mathrsfs}
\usepackage{graphicx}
\usepackage{adjustbox}
\usepackage{enumitem}
\usepackage{csquotes}
\usepackage[backend=biber,style=numeric]{biblatex}

\usepackage{tikz}
\usetikzlibrary{cd}
\usepackage{quiver}
\usetikzlibrary{babel} % Make sure quiver/tikz uses babel

\usepackage{mwe} % for placeholder images

\addbibresource{references.bib}

\usepackage{McMasterTheme}
\tikzposterlatexaffectionproofoff
\usetheme{McMasterTheme}
\usecolorstyle{McMasterStyle}
\usetitlestyle{Filled}

\usepackage[scaled]{helvet}
\renewcommand\familydefault{\sfdefault} 
\usepackage[T1]{fontenc}

\title{Capturing Mathematical Knowledge in Drasil: the Case of Theories}
\author{Jason Balaci (balacij), Jacques Carette (carette)}
\institute{Department of Computing and Software, McMaster University}
\titlegraphic{\includegraphics[width=0.16\textwidth]{assets/mcm-cas_left-wht_eps.eps}}
\projectlogo{\includegraphics[width=0.06\textwidth]{assets/Drasil Logo.png}}

% begin document
\begin{document}

\maketitle

\centering
\begin{columns}
    %----------------------------------
    %- First Column
    %----------------------------------
    \column{0.32}

    \block{What is Drasil?}{ Drasil is a framework for generating families of
        software artifacts from a coherent knowledge base, following it's
        mantra; ``Generate All The Things!''. Drasil uses a series of variably
        sized Domain-Specific Languages (DSLs) to describe various fragments of
        knowledge that domain experts and users alike may use to piece together
        fragments of knowledge into a coherent ``story''. Through forming some
        coherent ``story'' in a domain captured by Drasil, a representational
        software artifact may be generated. Drasil currently focuses on
        Scientific Computing Software (SCS), following Smith and Lai's Software
        Requirements Specifications (SRS) template as described in
        \cite{SmithAndLai2005}. Behind the scenes of the SRS, a mathematical
        language is used to describe various theories, and have representational
        software constructed via compiling to Generic Object-Oriented Language
        (GOOL) \cite{Carette2020GOOL}. Through encoding knowledge in Drasil, an
        increase in productivity (and maintainability) in building reliable and
        traceable software artifacts is observed \cite{SzymczakEtAl2016},
        specifically in SCS \cite{Smith2018}. Drasil's source code, case
        studies, and documentation studies can be found on it's website;
        \url{https://jacquescarette.github.io/Drasil/}.}

    % Currently focused on increasing productivity and maintainability in
    % scientific software development by generating reliable and traceable
    % software artifacts with a single Generic Object-Oriented Language (GOOL)
    % \cite{Carette2020GOOL}, using a mathematical language to describe
    % problem domains, and various staged generators to systematically
    % generate artifacts conforming to the Smith and Lai's ``Software
    % Requirements Specifications'' (SRS) template.

    \block{Research Motivation/Problem}{
        \begin{itemize}
            \item Only a handful of the case studies (?) generate code, because...
                \begin{itemize}
                    \item ODEs
                    \item Reliability in terms
                    \item Quantification
                \end{itemize}
        \end{itemize}
    }

    %----------------------------------
    %- Second Column
    %----------------------------------
    \column{0.36}

    \block{Mathematical Knowledge Flow}{
        \vspace{-1em}
        % Originally created with https://q.uiver.app/?q=WzAsNyxbOCw0XSxbMSwwLCJUaGVvcmllcyAoUmVsYXRpb25Db25jZXB0KSJdLFswLDAsIkV4cHJlc3Npb25zIChFeHByKSJdLFsyLDAsIk1hdGhlbWF0aWNhbCBLbm93bGVkZ2UiXSxbMiwxLCJRdWFudGl0eSBEZWZpbml0aW9ucyAoUURlZm5zKSJdLFsyLDIsIkNvZGUgKEdPT0wpIl0sWzEsMSwiU2NpZW50aWZpYyBSZXF1aXJlbWVudHMgU3BlY2lmaWNhdGlvbnMgKFNSUykiXSxbMiwxXSxbMSwzXSxbMyw0XSxbNCw1XSxbMSw2XV0=
        % But requires some post-processing
        \[\begin{tikzcd}[cramped,sep=small,align=center,ampersand replacement=\&]
                {\parbox{0.2\linewidth}{\centering Expression (\textit{Expr})}}
                \& {\parbox{0.24\linewidth}{\centering Theory \\(\textit{RelationConcept})}}
                \& |[draw=orange, cloud, aspect=2.8, inner sep=0pt]| {\parbox{0.2\linewidth}{\centering Mathematical Knowledge}} \\

                {\parbox{0.25\linewidth}{\centering Scientific Requirements Specifications (SRS)}}
                \& {\parbox{0.1\linewidth}{\centering Code (GOOL)}}
                \& {\parbox{0.2\linewidth}{\centering Quantity Definition (\textit{QDefn})}} \\
                \&\& \\
                \arrow[color=green, from=1-1, to=1-2]
                \arrow[squiggly, color=red, from=1-2, to=1-3]
                \arrow[squiggly, color=red, from=1-3, to=2-3]
                \arrow[squiggly, color=red, from=2-3, to=2-2]
                \arrow[color=green, from=1-2, to=2-1]
            \end{tikzcd}\]
        \vspace{-2.5em}
    }

    \block{Capturing Mathematical Knowledge}{
        % Originally created with https://q.uiver.app/?q=WzAsNixbMSwxLCJNYXRoZW1hdGljYWwgS25vd2xlZGdlIChNb2RlbEtpbmRzKSJdLFsxLDIsIkNvZGUgKEdPT0wpIl0sWzAsMCwiRXF1YXRpb25hbE1vZGVsIChRRGVmbikiXSxbMSwwLCJFcXVhdGlvbmFsUmVhbG0gKE11bHRpRGVmbikiXSxbMiwwLCJldGMuIl0sWzAsMiwiU1JTIl0sWzAsMSwiT25seSBFcXVhdGlvbmFsIl0sWzIsMF0sWzMsMF0sWzQsMF0sWzAsNV1d
        % But requires some post-processing
        \[\begin{tikzcd}[cramped,sep=small,align=center,ampersand replacement=\&]
                {\parbox{0.25\linewidth}{\centering EquationalModel (\textit{QDefn Expr})}}
                \& {\parbox{0.25\linewidth}{\centering EquationalRealm (\textit{MultiDefn Expr/ModelExpr})}}
                \& {\parbox{0.25\linewidth}{\centering etc.}} \\

                {\parbox{0.1\linewidth}{\centering SRS}}
                \& {\parbox{0.25\linewidth}{\centering Mathematical Knowledge (\textit{ModelKinds})}}
                \& {\parbox{0.1\linewidth}{\centering Code (GOOL)}} \\

                \arrow[squiggly, color=blue, from=2-2, to=2-3]
                \arrow[color=green, from=1-1, to=2-2]
                \arrow[color=green, from=1-2, to=2-2]
                \arrow[color=green, from=1-3, to=2-2]
                \arrow[color=green, from=2-2, to=2-1]
            \end{tikzcd}\]
        \vspace{-2em}
    }

    \block{What changed?}{
        \begin{itemize}
            \item +ModelKinds (concrete representation of known theories)
            \item \(\textit{Expr} \longrightarrow \textit{Expr}\ \cup\ \textit{ModelExpr}\ \cup\ \textit{CodeExpr}\)
            \item +TTF Encodings of Expressions
            \item Static checking of validity of theory models, ensuring reliable generation
            \item Ease of creation; improving productivity
            \item Creating a classification system (can be utilized more later)
                \begin{itemize}
                    \item \textit{EquationalModel}
                    \item \textit{EquationalRealm}
                    \item \textit{EquationalConstraints}
                    \item \textit{DEModel}
                    \item \textit{OthModel}
                \end{itemize}
        \end{itemize}
    }

    %----------------------------------
    %- Third Column
    %----------------------------------
    \column{0.32}

    % \block{}{\blindtext}

    \block{Conclusions \& Future Work}{
        \begin{itemize}
            \item Something
        \end{itemize}
    }

    \block{References}{
        \vspace{-1em}
        \renewcommand*{\bibfont}{\small}
        \printbibliography[heading=none]
    }
\end{columns}

\end{document}
