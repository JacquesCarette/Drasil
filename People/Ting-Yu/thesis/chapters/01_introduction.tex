\chapter{Introduction} \label{intro}
Scientific computing (SC) is an intersection of computer science, mathematics, 
and science. It is a field that solves complex scientific problems by using 
computing techniques and tools. Writing documentation is a process of 
developing scientific software. The role of documentation is to help 
people better understand the software and to ``communicate information to its 
audience and instil knowledge of the system it describes" 
\cite{forward2002software}. The significance of software documentation has 
been presented in many papers by previous researchers \cite{parnas2011precise}, 
\cite{chomal2014significance}, \cite{kipyegen2013importance}. It is further 
shown by Smith et al. \cite{SmithandKoothoor2016}, \cite{SmithandYu2007} that 
developing scientific computing software (SCS) in a document-driven methodology 
improves the quality of the software . 

Jupyter Notebook is a system for creating and sharing data science and 
scientific computing documentation. It is a nonprofit, open-source application 
born out in 2014, providing interactive computing across multiple programming 
languages, such as Python, Javascript, Matlab, and R. A Jupyter Notebook 
integrates text, live code, equations, computational outputs, visualizations, 
and multimedia resources, including images and videos. Jupyter Notebook is one 
of the most widely used interactive systems among scientists. Its popularity 
has grown from 200,000 to 2.5 million public Jupyter Notebooks on GitHub in 
three years from 2015 to 2018 \cite{Jeffrey2018}. It is used in a variety of 
areas and ways because of its flexibility and added values. For example, the 
notebook can be used as an educational tool in engineering courses, enhancing 
teaching and learning efficiency \cite{cardoso2019using}, \cite{zhao2019use}.

Even though the importance of documentation is widely recognized, it is often 
missing or poorly documented in SCS because: i) scientists are not aware of 
the why, how, and what of documentation \cite{hermann2022documenting}, 
\cite{chang2022understanding}; ii) it is time-consuming to produce 
\cite{sanders2008dealing}; iii) scientists generally believe that writing 
documentation demands more work and effort than they would likely yield in 
terms of the benefits of it \cite{smith2016advantages}.

We are trying to increase the efficiency of documentation development by 
adopting generative programming. Generative programming is a technique that 
allows programmers to write the code or document at a higher abstraction level, 
and the generator produces the desired outputs. Drasil is an application of 
generative programming, and it is the framework we use to conduct this 
research. Drasil saves us more time in the documentation development process by 
letting us encode each piece of information of our scientific problems once and 
generating the document automatically.


\section{Background}
\subsection{Drasil}
Drasil is a framework that can generate software artifacts, including Software 
Requirement Specifications (SRS), code (C++, C\#, Java, and Python), README, 
and Makefile, from a stable knowledge base. The goals of Drasil are reducing 
knowledge duplication and improving traceability \cite{drasil}. Drasil captures 
the knowledge through our hand-made case studies. We currently have 10 case 
studies that cover different physics problems, such as Projectile and Pendulum. 
Recipes for scientific problems are encoded in Drasil, and it generates code 
and documentation for us. Each piece of information only needs to be provided 
to Drasil once, and that information can be used wherever it is needed (note: 
add an example). SRS is a template for designing and documenting scientific 
computing software requirement decisions created by Smith et al 
\cite{smith2005new}. Drasil is capable of generating SRS in document languages 
HTML and LaTeX. We are looking to extend the capability of Drasil by generating 
Jupyter Notebook in Drasil.

\subsection{Jupyter Notebook}
Jupyter Notebook is an interactive open-source web application for creating and 
sharing computational science documentation that contains text, executable 
code, mathematical equations, graphics, and visualizations.

\subsubsection{Structure of a notebook document}
A Jupyter Notebook has two components: front-end ``cells" and back-end 
``kernels". The notebook consists of a sequence of cells: code cells, markdown 
cells, and raw cells. A cell is a multiline text input field. The notebook 
works by users entering a piece of information (text or programming code) in 
cells from the web page user interface. That information is then passed to the 
back-end kernels which execute the code and return the results 
\cite{notebookdoc}.

\subsubsection{Values of Jupyter Notebook}
There are several values of Jupyter Notebook: sharable, all-in-one, and live 
code. First of all, the notebook is easy to share because it can be converted 
into other formats such as HTML, Markdown, and PDF. Secondly, it combines all 
aspects of data in one single document, making the document easy to visualize, 
maintain and modify. In addition, Jupyter Notebook provides an environment of 
live code and computational equations. Usually, when programmers are running 
code on some other IDEs, they have to write the entire program before executing 
it. However, the notebook allows programmers to execute a specific portion of 
the code without running the whole program. The ability to run a snippet of 
code and integrate with text highlight the usability of the notebook.

\section{Problem Statement}
Since both Jupyter Notebook and Drasil focus on creating and generating 
scientific computing documentation, we are interested in extending the values 
of Jupyter Notebook to Drasil and the kind of knowledge we can manipulate. 
Following are the three main problems we are trying to solve with Drasil in 
this paper:

\begin{enumerate}
	\item Generate documents in JSON format. Jupyter Notebook is a simple JSON 
	document with a .ipynb file extension. A Drasil-generated document needs to 
	have a format that is readable by the notebook. Drasil can only write in 
	HTML and LaTeX; we will need Drasil to be able to generate documents in 
	JSON format so that the documents can be read and written in Jupyter 
	Notebook.
	\item Develop the structure of lesson plans and generate them. Not only can 
	Drasil generate SRS, but we would also like to expand its application by 
	generating lesson plans. As mentioned, Jupyter Notebook is used as an 
	educational tool for teaching engineering courses. We are interested in 
	teaching Drasil a ``textbook" structure by starting with generating a 
	simple physics lesson plan. We are trying to capture the elements of 
	textbook chapters, find the family of lesson plans, and classify the 
	knowledge in order to build a general structure in Drasil so that the 
	lesson plan will be able to generalize to a variety of lessons.
	\item Generate an interactive notebook. Jupyter Notebook is an interactive 
	application for creating documents that contain formattable text and 
	executable code. However, Drasil doesn't support interactive recipes. We 
	are looking for the possibility of generating a notebook document with a 
	mix of text and code, increasing the ability of Drasil and its potential to 
	solve more scientific problems.
\end{enumerate}

\section{Thesis Outline}
Thesis outline here.