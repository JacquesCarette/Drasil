\documentclass[12pt,oneside]{book}

% Extra functionality for command parsing, colouring elements, and commentary
\usepackage{xparse}
\usepackage[table]{xcolor}
\usepackage{comment}

% Configure font and file encodings, and language as Canadian English
\usepackage[utf8]{inputenc}
\usepackage[T1]{fontenc}
\usepackage[canadian]{babel}

% Math-related, but also generally helpful
\usepackage{proof}
\usepackage{amsmath}
\usepackage{amsfonts}
\usepackage{amsthm}
\usepackage{amssymb}
\usepackage{mathrsfs}
\usepackage{graphicx}
\usepackage{lmodern}
\usepackage{longtable}

% Tiny package for easily grabbing the page count of the "main matter"
\usepackage{lastpage}

% For quotes, I wanted to put the "left bar" style. For implementation, Gonzalo
% Medina was very kind to create an example. It is based on:
% https://tex.stackexchange.com/a/50623
\usepackage{framed}
\usepackage[framemethod=TikZ]{mdframed}
\newmdenv[topline=false, rightline=false, bottomline=false,%
  linewidth=2pt, innerrightmargin=0pt, leftmargin=0pt,%
  innerleftmargin=5pt, skipabove=8pt, skipbelow=8pt]{mdleftbar}

\newmdenv[linewidth=2pt, linecolor=green, backgroundcolor=green!8, roundcorner=10pt,
  skipabove=8pt, skipbelow=8pt]{mdwritingdirectives}

% For nice captions and floating environments, such as for my code snippets
\usepackage{caption}
\usepackage{float}

% For code highlighting
\usepackage[newfloat]{minted}
% Credits to Arash Esbati (https://tex.stackexchange.com/a/254177) for the
% listings-related component of minted usage.

\usemintedstyle{colorful}
% TODO: Figure out how to make a decent custom `minted' style:
%       <https://tex.stackexchange.com/a/584212>

% If I want to _just_ change the font of the minted code.
% \usepackage{fontspec}
% \setmonofont{mononoki}

% Make \today only show month and year
\usepackage[useregional]{datetime2}
\DTMlangsetup[en-CA]{showdayofmonth=false}

% Configure page shape
\usepackage[
  a4paper,
  top=3.8cm,
  bottom=2.5cm,
  inner=3.8cm,
  outer=2.5cm,
  headheight=15pt,
]{geometry}
\usepackage{afterpage}

% Enable links within the document
\usepackage{hyperref}
\hypersetup{
  colorlinks=true,
  linkcolor=red,
  breaklinks=true
}
\usepackage[nameinlink]{cleveref} % Fixes capitalization of internal references

% Make footnote counter reset for each new page.
\usepackage{footnpag}

% Required for biblatex, but also adds functionality for quotation
\usepackage{csquotes}

% Credit to Gabriel Devenyi for this bibliography cfg:
% github.com/gdevenyi/mcmaster.latex
\usepackage[
  style=numeric-comp,
  backend=biber,
  sorting=none,
  backref=true,
  maxnames=99,
  alldates=iso,
  seconds=true
]{biblatex} % bibliography
\addbibresource{references.bib}

% Fancy Headers
\usepackage{fancyhdr}

% For abbreviations, we use "acro" package, and mfirstuc to help capitalize long
% versions normally
\usepackage{array}
\usepackage{mfirstuc}
\MFUhyphentrue % tell mfirstuc to capitalize hyphenated words
\usepackage{acro}
% For one-offs,
% \DeclareAcronym{acronym}{short=short-version,long=long-version}

\newcommand{\newacr}[2]{\DeclareAcronym{#1}{short=\uppercase{#1},long=#2}}

% Alphabetically sorted list of acronyms
\newacr{gadt}{Generalized Algebraic Datatypes}
\newacr{gool}{Generic Object-Oriented Language}
\newacr{tt}{Typed Tagless}
\newacr{uid}{Unique Identifier}


% Manifest data
\newcommand{\thesisTitle}{Capturing Mathematical Knowledge In Drasil}
\newcommand{\thesisHalfTitle}{\thesisTitle}

\newcommand{\thesisAuthorName}{Jason Balaci}
\newcommand{\thesisAuthorCredentials}{B.Sc.}
\newcommand{\thesisSupervisor}{Dr. Jacques Carette}

% TODO: Add more information about the university, and date of submission
\newcommand{\thesisInstitution}{McMaster University}

% QUESTION-DIRECTED WRITING SWITCH
\newif\ifshowwritingdirectives
\showwritingdirectivestrue % Show questions
% \showwritingdirectivesfalse % Don't show questions


% General Utility Functions
%%%%%%%%%%%%%%%%%%%%%%%%%%%%%%%%%%%%%%%%%%%%%%%%%%%%%%%%%%%%%%%%%%%%%%%%%%%%%%%
% QUESTION DIRECTED WRITING
\ifshowwritingdirectives
  \newenvironment{writingdirectives}{\begin{mdwritingdirectives}\centering\textbf{Writing Directives}\begin{itemize}}{\end{itemize}\end{mdwritingdirectives}}
\else
  \excludecomment{writingdirectives}
\fi

\newcommand{\wqanswer}[1]{\textit{#1}}

%%%%%%%%%%%%%%%%%%%%%%%%%%%%%%%%%%%%%%%%%%%%%%%%%%%%%%%%%%%%%%%%%%%%%%%%%%%%%%%
% CASE STUDIES
\newcommand{\caseStudy}[1]{\ACL{#1} (\textit{\acs{#1}})}

%%%%%%%%%%%%%%%%%%%%%%%%%%%%%%%%%%%%%%%%%%%%%%%%%%%%%%%%%%%%%%%%%%%%%%%%%%%%%%%
% JUDGMENTS

\newcommand{\newrule}[2]{\begin{equation} \infer{#2}{#1} \end{equation}} % Adds to automatically numbered equations
\newcommand{\newlblrule}[3]{\begin{equation} \infer{#2}{#1} \label{#3}\end{equation}} % Adds to automatically numbered equations, with a label
\newcommand{\exampleRule}[2]{\[ \infer{#2}{#1} \]} % Does not add a number to equations

\newcommand{\Tau}{\mathrm{T}}
\newcommand{\ty}[1]{\texttt{#1} : \tau}

\newcommand{\ofTy}[2]{#1 : \texttt{#2}}

\newcommand{\numericTy}[1]{\ofTy{#1}{\texttt{Numerics($\Tau$)}}}
\newcommand{\negNumericTy}[1]{\ofTy{#1}{\texttt{NumericsWithNegation($\Tau$)}}}

%%%%%%%%%%%%%%%%%%%%%%%%%%%%%%%%%%%%%%%%%%%%%%%%%%%%%%%%%%%%%%%%%%%%%%%%%%%%%%%
% MATH

\newcommand{\bb}[1]{\mathbb{#1}}

%%%%%%%%%%%%%%%%%%%%%%%%%%%%%%%%%%%%%%%%%%%%%%%%%%%%%%%%%%%%%%%%%%%%%%%%%%%%%%%
% SYNTAX CHARTS
\newcommand{\startSyntaxTable}{\begin{longtable}{ r c c l c l }}
\newcommand{\newsyntaxRow}[5]{#1 & \( #2 \) & $::=$ & \texttt{#3} & $#4$ & #5 \\}
\newcommand{\syntaxRow}[3]{& & $\vert$ & \texttt{#1} & $#2$ & #3 \\}
\newcommand{\closeSyntaxTable}{\end{longtable}}

%%%%%%%%%%%%%%%%%%%%%%%%%%%%%%%%%%%%%%%%%%%%%%%%%%%%%%%%%%%%%%%%%%%%%%%%%%%%%%%
% Footnotes that only show "when compiling for printing"

\ifcompilingforprinting
  \newcommand{\printOnlyFootnote}[1]{\footnote{#1}}
  \newcommand{\printOnlyFootnoteText}[1]{\footnotetext{#1}}
  \newcommand{\printOnlyFootnoteMark}{\footnotemark}
\else
  \newcommand{\printOnlyFootnote}[1]{}
  \newcommand{\printOnlyFootnoteText}[1]{}
  \newcommand{\printOnlyFootnoteMark}{}
\fi

%%%%%%%%%%%%%%%%%%%%%%%%%%%%%%%%%%%%%%%%%%%%%%%%%%%%%%%%%%%%%%%%%%%%%%%%%%%%%%%
% Portable HREFs

% Common variant
\newcommand{\porthref}[2]{\href{#2}{#1}\printOnlyFootnote{\url{#2}}}
% Custom URLs
\newcommand{\porthreft}[3]{\href{#3}{#1}\printOnlyFootnote{\href{#3}{#2}}}
% Inside of some environments, footnote marks aren't registered properly, so we
% need to manually write the "text" part
\newcommand{\porthreftm}[2]{\href{#2}{#1\printOnlyFootnoteMark}}

%%%%%%%%%%%%%%%%%%%%%%%%%%%%%%%%%%%%%%%%%%%%%%%%%%%%%%%%%%%%%%%%%%%%%%%%%%%%%%%
% Inlined TODOs
\newcommand{\intodo}[1]{\todo[inline]{#1}}

%%%%%%%%%%%%%%%%%%%%%%%%%%%%%%%%%%%%%%%%%%%%%%%%%%%%%%%%%%%%%%%%%%%%%%%%%%%%%%%
% Haskell snippet
\newenvironment{code}{\captionsetup{type=listing,skip=14pt}}{}
\SetupFloatingEnvironment{listing}{name=Source Code, listname=List of Source Codes}
\crefname{listing}{source code}{source codes}
\Crefname{listing}{Source Code}{Source Codes}
\newenvironment{haskell}[4]
    {\VerbatimEnvironment\singlespacing\begin{code}\captionof{listing}[#1]{\protect\porthreftm{#1}{#4}}\printOnlyFootnoteText{\protect\href{#4}{#3}}\label{lst:#2}\begin{minted}[frame=lines,framerule=2pt,breaklines]{haskell}}
    {\end{minted}\end{code}\doublespacing}

\newcommand{\inlineHs}[1]{\mintinline{haskell}|#1|}

% TODO: Look into 'xurl' instead of the above hacky solution: <https://tex.stackexchange.com/questions/54946/how-to-break-a-long-url>

% Haskell env original caption style: {\VerbatimEnvironment\begin{code}\caption[#1]{\protect\porthreftm{#1}{#4}}\footnotetext{#3}\label{lst:#2}\begin{minted}[frame=lines,framerule=2pt]{haskell}}

% TODO: Figure out how I can properly space the label between "Source Code X.X:
% X" and the minted code itself. "\vspace*{-3mm}" barely works, and doesn't
% really solve the real problem here.


% General Assets
%------------------------------------------------------------------------------
% Folders
%------------------------------------------------------------------------------

\newcommand{\assets}{assets}
\newcommand{\figureAssets}{\assets/figures}
\newcommand{\imageAssets}{\assets/images}
\newcommand{\tableAssets}{\assets/tables}
\newcommand{\codeAssets}{\assets/code}
\newcommand{\currentCode}{\codeAssets/current}
\newcommand{\originalCode}{\codeAssets/original}

%------------------------------------------------------------------------------
% Code
%------------------------------------------------------------------------------

% Command based on: https://tex.stackexchange.com/questions/266811/define-a-new-command-with-parameters-inside-newcommand
\newcommand{\codeName}[1]{\expandafter\newcommand\csname #1\endcsname{\inlineHs{#1}}}

\codeName{Chunk}
\codeName{ChunkDB}
\codeName{CodeExpr}
\codeName{ConceptChunk}
\codeName{ConstraintKinds}
\codeName{ConstraintSet}
\codeName{DEModel}
\codeName{DefiningExpr}
\codeName{EquationalConstraints}
\codeName{EquationalModel}
\codeName{EquationalRealm}
\codeName{ExistentialQuantification}
\codeName{Expr}
\codeName{Express}
\codeName{HasChunkRefs}
\codeName{HasUID}
\codeName{Literal}
\codeName{ModelExpr}
\codeName{ModelKind}
\codeName{ModelKinds}
\codeName{MultiDefn}
\codeName{NewDEModel}
\codeName{ODEInfo}
\codeName{OthModel}
\codeName{QDefinition}
\codeName{QuantityDict}
\codeName{Relation}
\codeName{RelationConcept}
\codeName{relToQD}
\codeName{Typeable}
\codeName{TypeRep}
\codeName{UID}


% "Current"
%--------------------------------------

% expr
\newcommand{\currentExprHaskell}{\begin{haskell}{Expression Language}{curExpr}{https://github.com/JacquesCarette/Drasil/blob/dc3674274edb00b1ae0d63e04ba03729e1dbc6f9/code/drasil-lang/lib/Language/Drasil/Expr/Lang.hs\#L81-L135}
-- | Expression language where all terms are supposed to be 'well understood'
--   (i.e., have a definite meaning). Right now, this coincides with
--   "having a definite value", but should not be restricted to that.
data Expr where
  -- | Brings a literal into the expression language.
  Lit :: Literal -> Expr
  -- | Takes an associative arithmetic operator with a list of expressions.
  AssocA   :: AssocArithOper -> [Expr] -> Expr
  -- | Takes an associative boolean operator with a list of expressions.
  AssocB   :: AssocBoolOper  -> [Expr] -> Expr
  -- | C stands for "Chunk", for referring to a chunk in an expression.
  --   Implicitly assumes that the chunk has a symbol.
  C        :: UID -> Expr
  -- | A function call accepts a list of parameters and a list of named parameters.
  --   For example
  --
  --   * F(x) is (FCall F [x] []).
  --   * F(x,y) would be (FCall F [x,y]).
  --   * F(x,n=y) would be (FCall F [x] [(n,y)]).
  FCall    :: UID -> [Expr] -> [(UID, Expr)] -> Expr
  -- | For multi-case expressions, each pair represents one case.
  Case     :: Completeness -> [(Expr, Relation)] -> Expr
  -- | Represents a matrix of expressions.
  Matrix   :: [[Expr]] -> Expr
  -- | Unary operation for most functions (eg. sin, cos, log, etc.).
  UnaryOp       :: UFunc -> Expr -> Expr
  -- | Unary operation for @Bool -> Bool@ operations.
  UnaryOpB      :: UFuncB -> Expr -> Expr
  -- | Unary operation for @Vector -> Vector@ operations.
  UnaryOpVV     :: UFuncVV -> Expr -> Expr
  -- | Unary operation for @Vector -> Number@ operations.
  UnaryOpVN     :: UFuncVN -> Expr -> Expr
  -- | Binary operator for arithmetic between expressions (fractional, power, and subtraction).
  ArithBinaryOp :: ArithBinOp -> Expr -> Expr -> Expr
  -- | Binary operator for boolean operators (implies, iff).
  BoolBinaryOp  :: BoolBinOp -> Expr -> Expr -> Expr
  -- | Binary operator for equality between expressions.
  EqBinaryOp    :: EqBinOp -> Expr -> Expr -> Expr
  -- | Binary operator for indexing two expressions.
  LABinaryOp    :: LABinOp -> Expr -> Expr -> Expr
  -- | Binary operator for ordering expressions (less than, greater than, etc.).
  OrdBinaryOp   :: OrdBinOp -> Expr -> Expr -> Expr
  -- | Binary operator for @Vector x Vector -> Vector@ operations (cross product).
  VVVBinaryOp   :: VVVBinOp -> Expr -> Expr -> Expr
  -- | Binary operator for @Vector x Vector -> Number@ operations (dot product).
  VVNBinaryOp   :: VVNBinOp -> Expr -> Expr -> Expr
  -- | Operators are generalized arithmetic operators over a 'DomainDesc'
  --   of an 'Expr'.  Could be called BigOp.
  --   ex: Summation is represented via 'Add' over a discrete domain.
  Operator :: AssocArithOper -> DiscreteDomainDesc Expr Expr -> Expr -> Expr
  -- | A different kind of 'IsIn'. A 'UID' is an element of an interval.
  RealI    :: UID -> RealInterval Expr Expr -> Expr
\end{haskell}
}
\newcommand{\refCurrentExprHaskell}{\Cref{lst:curExpr}}

% express
\newcommand{\currentExpressHaskell}{\begin{haskell}{Express Typeclass}{curExpress}{https://github.com/JacquesCarette/Drasil/blob/dc3674274edb00b1ae0d63e04ba03729e1dbc6f9/code/drasil-lang/lib/Language/Drasil/ExprClasses.hs\#L9-L11}
-- | Data that can be expressed using 'ModelExpr'.
class Express c where
  express :: c -> ModelExpr
\end{haskell}
}
\newcommand{\refCurrentExpressHaskell}{\Cref{lst:curExpress}}

% exprTTF
\newcommand{\currentExprTTFHaskell}{\begin{haskell}{Expr Constructor Encoding (TTF)}{curExprTTF}{https://github.com/JacquesCarette/Drasil/blob/f979892d095cceed4d062afb17bd9654bcda4fe2/code/drasil-lang/lib/Language/Drasil/Expr/Class.hs\#L86-L209}
class ExprC r where
  infixr 8 $^
  infixl 7 $/
  infixr 4 $=
  infixr 9 $&&
  infixr 9 $||
  
  lit :: Literal -> r

  -- * Binary Operators
  
  ($=), ($!=) :: r -> r -> r
  
  -- | Smart constructor for ordering two equations.
  ($<), ($>), ($<=), ($>=) :: r -> r -> r
  
  -- | Smart constructor for the dot product of two equations.
  ($.) :: r -> r -> r
  
  -- | Add two expressions (Integers).
  addI :: r -> r -> r
  
  -- | Add two expressions (Real numbers).
  addRe :: r -> r -> r
  
  -- | Multiply two expressions (Integers).
  mulI :: r -> r -> r
  
  -- | Multiply two expressions (Real numbers).
  mulRe :: r -> r -> r
  
  ($-), ($/), ($^) :: r -> r -> r
  
  ($=>), ($<=>) :: r -> r -> r
  
  ($&&), ($||) :: r -> r -> r
  
  -- | Smart constructor for taking the absolute value of an expression.
  abs_ :: r -> r
  
  -- | Smart constructor for negating an expression.
  neg :: r -> r 
  
  -- | Smart constructor to take the log of an expression.
  log :: r -> r
  
  -- | Smart constructor to take the ln of an expression.
  ln :: r -> r
  
  -- | Smart constructor to take the square root of an expression.
  sqrt :: r -> r
  
  -- | Smart constructor to apply sin to an expression.
  sin :: r -> r
  
  -- | Smart constructor to apply cos to an expression.
  cos :: r -> r 
  
  -- | Smart constructor to apply tan to an expression.
  tan :: r -> r
  
  -- | Smart constructor to apply sec to an expression.
  sec :: r -> r 
  
  -- | Smart constructor to apply csc to an expression.
  csc :: r -> r
  
  -- | Smart constructor to apply cot to an expression.
  cot :: r -> r 
  
  -- | Smart constructor to apply arcsin to an expression.
  arcsin :: r -> r 
  
  -- | Smart constructor to apply arccos to an expression.
  arccos :: r -> r 
  
  -- | Smart constructor to apply arctan to an expression.
  arctan :: r -> r 
  
  -- | Smart constructor for the exponential (base e) function.
  exp :: r -> r
  
  -- | Smart constructor for calculating the dimension of a vector.
  dim :: r -> r
  
  -- | Smart constructor for calculating the normal form of a vector.
  norm :: r -> r
  
  -- | Smart constructor for negating vectors.
  negVec :: r -> r
  
  -- | Smart constructor for applying logical negation to an expression.
  not_ :: r -> r
  
  -- | Smart constructor for indexing.
  idx :: r -> r -> r
  
  -- | Smart constructor for the summation, product, and integral functions over an interval.
  defint, defsum, defprod :: Symbol -> r -> r -> r -> r
  
  -- | Smart constructor for 'real interval' membership.
  realInterval :: HasUID c => c -> RealInterval r r -> r
  
  -- | Euclidean function : takes a vector and returns the sqrt of the sum-of-squares.
  euclidean :: [r] -> r
  
  -- | Smart constructor to cross product two expressions.
  cross :: r -> r -> r
  
  -- | Smart constructor for case statements with a complete set of cases.
  completeCase :: [(r, r)] -> r
  
  -- | Smart constructor for case statements with an incomplete set of cases.
  incompleteCase :: [(r, r)] -> r
  
  -- | Create a matrix.
  matrix :: [[r]] -> r


  -- | Applies a given function with a list of parameters.
  apply :: (HasUID f, HasSymbol f) => f -> [r] -> r
   
  -- Note how |sy| 'enforces' having a symbol
  -- | Create an 'Expr' from a 'Symbol'ic Chunk.
  sy :: (HasUID c, HasSymbol c) => c -> r
\end{haskell}
}
\newcommand{\refCurrentExprTTFHaskell}{\Cref{lst:curExprTTF}}

% modelexpr
\newcommand{\currentModelExprHaskell}{\begin{haskell}{ModelExpr Language}{curModelExpr}{https://github.com/JacquesCarette/Drasil/blob/ab9e091dabd81685ddef86b0d218582c9f75cb20/code/drasil-lang/lib/Language/Drasil/ModelExpr/Lang.hs\#L82-L151}
-- | Expression language where all terms are supposed to have a meaning, but
--   that meaning may not be that of a definite value. For example,
--   specification expressions, especially with quantifiers, belong here.
data ModelExpr where
  -- | Brings a literal into the expression language.
  Lit       :: Literal -> ModelExpr
  
  -- | Introduce Space values into the expression language.
  Spc       :: Space -> ModelExpr
  
  -- | Takes an associative arithmetic operator with a list of expressions.
  AssocA    :: AssocArithOper -> [ModelExpr] -> ModelExpr
  -- | Takes an associative boolean operator with a list of expressions.
  AssocB    :: AssocBoolOper  -> [ModelExpr] -> ModelExpr
  -- | Derivative syntax is:
  --   Type ('Part'ial or 'Total') -> principal part of change -> with respect to
  --   For example: Deriv Part y x1 would be (dy/dx1).
  Deriv     :: Integer -> DerivType -> ModelExpr -> UID -> ModelExpr
  -- | C stands for "Chunk", for referring to a chunk in an expression.
  --   Implicitly assumes that the chunk has a symbol.
  C         :: UID -> ModelExpr
  -- | A function call accepts a list of parameters and a list of named parameters.
  --   For example
  --
  --   * F(x) is (FCall F [x] []).
  --   * F(x,y) would be (FCall F [x,y]).
  --   * F(x,n=y) would be (FCall F [x] [(n,y)]).
  FCall     :: UID -> [ModelExpr] -> [(UID, ModelExpr)] -> ModelExpr
  -- | For multi-case expressions, each pair represents one case.
  Case      :: Completeness -> [(ModelExpr, ModelExpr)] -> ModelExpr
  -- | Represents a matrix of expressions.
  Matrix    :: [[ModelExpr]] -> ModelExpr
  
  -- | Unary operation for most functions (eg. sin, cos, log, etc.).
  UnaryOp       :: UFunc -> ModelExpr -> ModelExpr
  -- | Unary operation for @Bool -> Bool@ operations.
  UnaryOpB      :: UFuncB -> ModelExpr -> ModelExpr
  -- | Unary operation for @Vector -> Vector@ operations.
  UnaryOpVV     :: UFuncVV -> ModelExpr -> ModelExpr
  -- | Unary operation for @Vector -> Number@ operations.
  UnaryOpVN     :: UFuncVN -> ModelExpr -> ModelExpr
  
  
  -- | Binary operator for arithmetic between expressions (fractional, power, and subtraction).
  ArithBinaryOp :: ArithBinOp -> ModelExpr -> ModelExpr -> ModelExpr
  -- | Binary operator for boolean operators (implies, iff).
  BoolBinaryOp  :: BoolBinOp -> ModelExpr -> ModelExpr -> ModelExpr
  -- | Binary operator for equality between expressions.
  EqBinaryOp    :: EqBinOp -> ModelExpr -> ModelExpr -> ModelExpr
  -- | Binary operator for indexing two expressions.
  LABinaryOp    :: LABinOp -> ModelExpr -> ModelExpr -> ModelExpr
  -- | Binary operator for ordering expressions (less than, greater than, etc.).
  OrdBinaryOp   :: OrdBinOp -> ModelExpr -> ModelExpr -> ModelExpr
  -- | Space-related binary operations.
  SpaceBinaryOp :: SpaceBinOp -> ModelExpr -> ModelExpr -> ModelExpr
  -- | Statement-related binary operations.
  StatBinaryOp  :: StatBinOp -> ModelExpr -> ModelExpr -> ModelExpr
  -- | Binary operator for @Vector x Vector -> Vector@ operations (cross product).
  VVVBinaryOp   :: VVVBinOp -> ModelExpr -> ModelExpr -> ModelExpr
  -- | Binary operator for @Vector x Vector -> Number@ operations (dot product).
  VVNBinaryOp   :: VVNBinOp -> ModelExpr -> ModelExpr -> ModelExpr
  
  
  -- | Operators are generalized arithmetic operators over a 'DomainDesc'
  --   of an 'Expr'.  Could be called BigOp.
  --   ex: Summation is represented via 'Add' over a discrete domain.
  Operator :: AssocArithOper -> DomainDesc t ModelExpr ModelExpr -> ModelExpr -> ModelExpr
  -- | A different kind of 'IsIn'. A 'UID' is an element of an interval.
  RealI    :: UID -> RealInterval ModelExpr ModelExpr -> ModelExpr
  
  -- | Universal quantification
  ForAll   :: UID -> Space -> ModelExpr -> ModelExpr
\end{haskell}
}
\newcommand{\refCurrentModelExprHaskell}{\Cref{lst:curModelExpr}}

% modelexprTTF
\newcommand{\currentModelExprTTFHaskell}{\begin{haskell}{ModelExpr Constructor Encoding (TTF)}{curModelExprTTF}{https://github.com/JacquesCarette/Drasil/blob/ab9e091dabd81685ddef86b0d218582c9f75cb20/code/drasil-lang/lib/Language/Drasil/ModelExpr/Class.hs\#L30-L51}
class ModelExprC r where
  -- This also wants a symbol constraint.
  -- | Gets the derivative of an 'ModelExpr' with respect to a 'Symbol'.
  deriv, pderiv :: (HasUID c, HasSymbol c) => r -> c -> r
  
  -- | Gets the nthderivative of an 'ModelExpr' with respect to a 'Symbol'.
  nthderiv, nthpderiv :: (HasUID c, HasSymbol c) => Integer -> r -> c -> r
  
  -- | One expression is "defined" by another.
  defines :: r -> r -> r
  
  -- | Space literals.
  space :: Space -> r
  
  -- | Check if a value belongs to a Space.
  isIn :: r -> Space -> r
  
  -- | Binary associative "Equivalence".
  equiv :: [r] -> r
  
  -- | Smart constructor for the summation, product, and integral functions over all Real numbers.
  intAll, sumAll, prodAll :: Symbol -> r -> r
\end{haskell}
}
\newcommand{\refCurrentModelExprTTFHaskell}{\Cref{lst:curModelExprTTF}}

% modelkinds
\newcommand{\currentModelKindsHaskell}{\begin{haskell}{ModelKinds}{curModelKinds}{https://github.com/JacquesCarette/Drasil/blob/dc3674274edb00b1ae0d63e04ba03729e1dbc6f9/code/drasil-theory/lib/Theory/Drasil/ModelKinds.hs\#L25-L47}
-- | Models can be of different kinds: 
--
--     * 'NewDEModel's represent differential equations as 'DifferentialModel's
--     * 'DEModel's represent differential equations as 'RelationConcept's
--     * 'EquationalConstraint's represent invariants that will hold in a system of equations.
--     * 'EquationalModel's represent quantities that are calculated via a single definition/'QDefinition'.
--     * 'EquationalRealm's represent MultiDefns; quantities that may be calculated using any one of many 'DefiningExpr's (e.g., 'x = A = ... = Z')
--     * 'FunctionalModel's represent quantity-resulting function definitions.
--     * 'OthModel's are placeholders for models. No new 'OthModel's should be created, they should be using one of the other kinds.
data ModelKinds e where
  NewDEModel            :: DifferentialModel -> ModelKinds e
  DEModel               :: RelationConcept   -> ModelKinds e -- TODO: Split into ModelKinds Expr and ModelKinds ModelExpr resulting variants. The Expr variant should carry enough information that it can be solved properly.
  EquationalConstraints :: ConstraintSet e   -> ModelKinds e
  EquationalModel       :: QDefinition e     -> ModelKinds e
  EquationalRealm       :: MultiDefn e       -> ModelKinds e
  OthModel              :: RelationConcept   -> ModelKinds e -- TODO: Remove (after having removed all instances of it).


-- | 'ModelKinds' carrier, used to carry commonly overwritten information from the IMs/TMs/GDs.
data ModelKind e = MK {
  _mk     :: ModelKinds e,
  _mkUID  :: UID,
  _mkTerm :: NP
}
\end{haskell}
}
\newcommand{\refCurrentModelKindsHaskell}{\Cref{lst:curModelKinds}}

% qdefinition
\newcommand{\currentQDefinitionHaskell}{\begin{haskell}{QDefinition Encoding}{curQDefinition}{https://github.com/JacquesCarette/Drasil/blob/ab9e091dabd81685ddef86b0d218582c9f75cb20/code/drasil-lang/lib/Language/Drasil/Chunk/Eq.hs\#L34-L35}
data QDefinition e where
  QD :: DefinedQuantityDict -> [UID] -> e -> QDefinition e
\end{haskell}
}
\newcommand{\refCurrentQDefinitionHaskell}{\Cref{lst:curQDefinition}}

% relationconcept
\newcommand{\currentRelationConceptHaskell}{\begin{haskell}{Definition of RelationConcept}{curRelConcept}{https://github.com/JacquesCarette/Drasil/blob/dc3674274edb00b1ae0d63e04ba03729e1dbc6f9/code/drasil-lang/lib/Language/Drasil/Chunk/Relation.hs\#L19-L24}
-- | For a concept ('ConceptChunk') that also has a 'Relation' ('ModelExpr') attached.
--
-- Ex. We can describe a pendulum arm and then apply an associated equation so that we know its behaviour.
data RelationConcept = RC { _conc :: ConceptChunk
                          , _rel  :: ModelExpr
                          }
\end{haskell}
}
\newcommand{\refCurrentRelationConceptHaskell}{\Cref{lst:curRelConcept}}

% "Original"
%--------------------------------------

% chunkdb
\newcommand{\originalChunkDBHaskell}{\begin{haskell}{Original Chunk Database (ChunkDB)}{origChunkDB}{https://github.com/JacquesCarette/Drasil/blob/9c26b43d3e30c3f618e534a3f176a5152729af74/code/drasil-database/Database/Drasil/ChunkDB.hs\#L130-L144}
-- | Our chunk databases. Should contain all the maps we will need.
data ChunkDB = CDB { symbolTable :: SymbolMap
                   , termTable :: TermMap 
                   , defTable  :: ConceptMap
                   , _unitTable :: UnitMap
                   , _traceTable :: TraceMap
                   , _refbyTable :: RefbyMap
                   , _dataDefnTable  :: DatadefnMap
                   , _insmodelTable   :: InsModelMap
                   , _gendefTable   :: GendefMap
                   , _theoryModelTable :: TheoryModelMap
                   , _conceptinsTable :: ConceptInstanceMap
                   , _sectionTable :: SectionMap
                   , _labelledcontentTable :: LabelledContentMap
                   } --TODO: Expand and add more databases
\end{haskell}
}
\newcommand{\refOriginalChunkDBHaskell}{\Cref{lst:origChunkDB}}

% chunkdb type maps
\newcommand{\originalChunkDBTypeMapsHaskell}{\begin{haskell}{Original ChunkDB Type Maps}{origConceptDBTypeMaps}{https://github.com/JacquesCarette/Drasil/blob/9c26b43d3e30c3f618e534a3f176a5152729af74/code/drasil-database/Database/Drasil/ChunkDB.hs\#L18-L47}
-- The misnomers below are not actually a bad thing, we want to ensure data can't
-- be added to a map if it's not coming from a chunk, and there's no point confusing
-- what the map is for. One is for symbols + their units, and the others are for
-- what they state.
type UMap a = Map.Map UID (a, Int)

-- | A bit of a misnomer as it's really a map of all quantities, for retrieving
-- symbols and their units.
type SymbolMap  = UMap QuantityDict

-- | A map of all concepts, normally used for retrieving definitions.
type ConceptMap = UMap ConceptChunk

-- | A map of all the units used. Should be restricted to base units/synonyms.
type UnitMap = UMap UnitDefn

-- | Again a bit of a misnomer as it's really a map of all NamedIdeas.
-- Until these are built through automated means, there will
-- likely be some 'manual' duplication of terms as this map will contain all
-- quantities, concepts, etc.
type TermMap = UMap IdeaDict
type TraceMap = Map.Map UID [UID]
type RefbyMap = Map.Map UID [UID]
type DatadefnMap = UMap DataDefinition
type InsModelMap = UMap InstanceModel
type GendefMap = UMap GenDefn
type TheoryModelMap = UMap TheoryModel
type ConceptInstanceMap = UMap ConceptInstance
type SectionMap = UMap Section
type LabelledContentMap = UMap LabelledContent
\end{haskell}
}
\newcommand{\refOriginalChunkDBTypeMapsHaskell}{\Cref{lst:origConceptDBTypeMaps}}

% conceptchunk
\newcommand{\originalConceptChunkHaskell}{\begin{haskell}{Original ConceptChunk}{origConceptChunk}{https://github.com/JacquesCarette/Drasil/blob/9c26b43d3e30c3f618e534a3f176a5152729af74/code/drasil-lang/Language/Drasil/Chunk/Concept/Core.hs\#L25-L29}
-- | The ConceptChunk datatype is a Concept
data ConceptChunk = ConDict { _idea :: IdeaDict
                            , _defn' :: Sentence
                            , cdom' :: [UID]
                            }
\end{haskell}
}
\newcommand{\refOriginalConceptChunkHaskell}{\Cref{lst:origConceptChunk}}

% expr
\newcommand{\originalExprHaskell}{\begin{haskell}{Expression Language}{curExpr}{https://github.com/JacquesCarette/Drasil/blob/dc3674274edb00b1ae0d63e04ba03729e1dbc6f9/code/drasil-lang/lib/Language/Drasil/Expr/Lang.hs\#L81-L135}
-- | Expression language where all terms are supposed to be 'well understood'
--   (i.e., have a definite meaning). Right now, this coincides with
--   "having a definite value", but should not be restricted to that.
data Expr where
  -- | Brings a literal into the expression language.
  Lit :: Literal -> Expr
  -- | Takes an associative arithmetic operator with a list of expressions.
  AssocA   :: AssocArithOper -> [Expr] -> Expr
  -- | Takes an associative boolean operator with a list of expressions.
  AssocB   :: AssocBoolOper  -> [Expr] -> Expr
  -- | C stands for "Chunk", for referring to a chunk in an expression.
  --   Implicitly assumes that the chunk has a symbol.
  C        :: UID -> Expr
  -- | A function call accepts a list of parameters and a list of named parameters.
  --   For example
  --
  --   * F(x) is (FCall F [x] []).
  --   * F(x,y) would be (FCall F [x,y]).
  --   * F(x,n=y) would be (FCall F [x] [(n,y)]).
  FCall    :: UID -> [Expr] -> [(UID, Expr)] -> Expr
  -- | For multi-case expressions, each pair represents one case.
  Case     :: Completeness -> [(Expr, Relation)] -> Expr
  -- | Represents a matrix of expressions.
  Matrix   :: [[Expr]] -> Expr
  -- | Unary operation for most functions (eg. sin, cos, log, etc.).
  UnaryOp       :: UFunc -> Expr -> Expr
  -- | Unary operation for @Bool -> Bool@ operations.
  UnaryOpB      :: UFuncB -> Expr -> Expr
  -- | Unary operation for @Vector -> Vector@ operations.
  UnaryOpVV     :: UFuncVV -> Expr -> Expr
  -- | Unary operation for @Vector -> Number@ operations.
  UnaryOpVN     :: UFuncVN -> Expr -> Expr
  -- | Binary operator for arithmetic between expressions (fractional, power, and subtraction).
  ArithBinaryOp :: ArithBinOp -> Expr -> Expr -> Expr
  -- | Binary operator for boolean operators (implies, iff).
  BoolBinaryOp  :: BoolBinOp -> Expr -> Expr -> Expr
  -- | Binary operator for equality between expressions.
  EqBinaryOp    :: EqBinOp -> Expr -> Expr -> Expr
  -- | Binary operator for indexing two expressions.
  LABinaryOp    :: LABinOp -> Expr -> Expr -> Expr
  -- | Binary operator for ordering expressions (less than, greater than, etc.).
  OrdBinaryOp   :: OrdBinOp -> Expr -> Expr -> Expr
  -- | Binary operator for @Vector x Vector -> Vector@ operations (cross product).
  VVVBinaryOp   :: VVVBinOp -> Expr -> Expr -> Expr
  -- | Binary operator for @Vector x Vector -> Number@ operations (dot product).
  VVNBinaryOp   :: VVNBinOp -> Expr -> Expr -> Expr
  -- | Operators are generalized arithmetic operators over a 'DomainDesc'
  --   of an 'Expr'.  Could be called BigOp.
  --   ex: Summation is represented via 'Add' over a discrete domain.
  Operator :: AssocArithOper -> DiscreteDomainDesc Expr Expr -> Expr -> Expr
  -- | A different kind of 'IsIn'. A 'UID' is an element of an interval.
  RealI    :: UID -> RealInterval Expr Expr -> Expr
\end{haskell}
}
\newcommand{\refOriginalExprHaskell}{\Cref{lst:origExpr}}

% few expr smart constructors
\newcommand{\originalFewExprSmartConstructorsHaskell}{\begin{haskell}{Original: Snapshot of a few of Exprs Smart Constructors}{origFewExprSmartConstructors}{https://github.com/JacquesCarette/Drasil/blob/9c26b43d3e30c3f618e534a3f176a5152729af74/code/drasil-lang/Language/Drasil/Expr/Math.hs}
-- | Smart constructor to apply tan to an expression
tan :: Expr -> Expr
tan = UnaryOp Tan

-- | Smart constructor to apply sec to an expression
sec :: Expr -> Expr 
sec = UnaryOp Sec

-- | Smart constructor to apply csc to an expression
csc :: Expr -> Expr
csc = UnaryOp Csc

-- | Smart constructor to apply cot to an expression
cot :: Expr -> Expr 
cot = UnaryOp Cot
\end{haskell}
}
\newcommand{\refOriginalFewExprSmartConstructorsHaskell}{\Cref{lst:origFewExprSmartConstructors}}

% qdefinition
\newcommand{\originalQDefinitionHaskell}{\begin{haskell}{QDefinition Encoding}{curQDefinition}{https://github.com/JacquesCarette/Drasil/blob/ab9e091dabd81685ddef86b0d218582c9f75cb20/code/drasil-lang/lib/Language/Drasil/Chunk/Eq.hs\#L34-L35}
data QDefinition e where
  QD :: DefinedQuantityDict -> [UID] -> e -> QDefinition e
\end{haskell}
}
\newcommand{\refOriginalQDefinitionHaskell}{\Cref{lst:origQDefinition}}

% relation
\newcommand{\originalRelation}{\begin{haskell}{Original Relation}{origRelation}{https://github.com/JacquesCarette/Drasil/blob/9c26b43d3e30c3f618e534a3f176a5152729af74/code/drasil-lang/Language/Drasil/Expr.hs\#L14}
type Relation = Expr
\end{haskell}
}
\newcommand{\refOriginalRelation}{\Cref{lst:origRelation}}

% relationconcept
\newcommand{\originalRelationConcept}{\begin{haskell}{Definition of RelationConcept}{curRelConcept}{https://github.com/JacquesCarette/Drasil/blob/dc3674274edb00b1ae0d63e04ba03729e1dbc6f9/code/drasil-lang/lib/Language/Drasil/Chunk/Relation.hs\#L19-L24}
-- | For a concept ('ConceptChunk') that also has a 'Relation' ('ModelExpr') attached.
--
-- Ex. We can describe a pendulum arm and then apply an associated equation so that we know its behaviour.
data RelationConcept = RC { _conc :: ConceptChunk
                          , _rel  :: ModelExpr
                          }
\end{haskell}
}
\newcommand{\refOriginalRelationConcept}{\Cref{lst:origRelationConcept}}

%------------------------------------------------------------------------------
% Figures
%------------------------------------------------------------------------------

% language division
\newcommand{\languageDivision}{% https://en.wikibooks.org/wiki/LaTeX/Floats,_Figures_and_Captions#Figures
% The "[H]" means ~"place it precisely here".

\begin{figure}[H]
    \centering
    \caption{Mathematical Language Division}
    \label{fig:mathLanguageDivision}

    \begin{center}
        \Expr{} \Rightarrow{} \Expr{} \cup{} \ModelExpr{} \cup{} \CodeExpr{}
    \end{center}
\end{figure}
}
\newcommand{\refLanguageDivision}{\Cref{fig:mathLanguageDivision}}

% knowledge flow in theories without modelkinds
\newcommand{\theoriesWithoutModelKinds}{\begin{figure}
    \centering
    \caption{Original Theories}
    \label{fig:theoriesWithoutModelKinds}
    % https://q.uiver.app/?q=WzAsNixbMCwwLCJFeHByIl0sWzEsMSwiUmVsYXRpb25Db25jZXB0Il0sWzEsMiwiU1JTIl0sWzIsMiwiTWF0aGVtYXRpY2FsXFxuZXdsaW5le30gS25vd2xlZGdlIl0sWzIsMywiUURlZmluaXRpb24iXSxbMiw0LCJDb2RlIl0sWzAsMV0sWzEsMl0sWzEsM10sWzMsNF0sWzQsNV1d
    \[\begin{tikzcd}[align=center]
        Expr \\
        & RelationConcept \\
        & SRS & \parbox{3cm}{\centering Mathematical Knowledge} \\
        && QDefinition \\
        && Code
        \arrow[from=1-1, to=2-2]
        \arrow[from=2-2, to=3-2]
        \arrow[from=2-2, to=3-3]
        \arrow[from=3-3, to=4-3]
        \arrow[from=4-3, to=5-3]
    \end{tikzcd}\]
\end{figure}
}
\newcommand{\refTheoriesWithoutModelKinds}{\Cref{fig:theoriesWithoutModelKinds}}

% knowledge flow in theories with modelkinds
\newcommand{\theoriesWithModelKinds}{% https://en.wikibooks.org/wiki/LaTeX/Floats,_Figures_and_Captions#Figures
% The "[H]" means ~"place it precisely here".

\begin{figure}[H]
	\centering
	\caption[Mathematical Knowledge Flow, with Formal Capture]{Mathematical Knowledge Flow, with Formal Capture\footnotemark{}}
	\label{fig:theoriesWithModelKinds}

	% Originally created with https://q.uiver.app/?q=WzAsNixbMSwxLCJNYXRoZW1hdGljYWwgS25vd2xlZGdlIChNb2RlbEtpbmRzKSJdLFsxLDIsIkNvZGUgKEdPT0wpIl0sWzAsMCwiRXF1YXRpb25hbE1vZGVsIChRRGVmbikiXSxbMSwwLCJFcXVhdGlvbmFsUmVhbG0gKE11bHRpRGVmbikiXSxbMiwwLCJldGMuIl0sWzAsMiwiU1JTIl0sWzAsMSwiT25seSBFcXVhdGlvbmFsIl0sWzIsMF0sWzMsMF0sWzQsMF0sWzAsNV1d
	% But requires some post-processing
	\[\begin{tikzcd}[cramped,sep=small,align=center,ampersand replacement=\&]
			{\parbox{0.25\linewidth}{\centering EquationalModel (\textit{QDefn Expr/ModelExpr})}}
			\& {\parbox{0.25\linewidth}{\centering EquationalRealm (\textit{MultiDefn Expr/ModelExpr})}}
			\& {\parbox{0.25\linewidth}{\centering etc.}} \\

			{\parbox{0.1\linewidth}{\centering \acs{srs}}}
			\& |[draw=green, cloud, aspect=2.8, inner sep=0pt, line width=2]| {\parbox{0.25\linewidth}{\centering Mathematical Knowledge (\textit{ModelKinds})}}
			\& {\parbox{0.1\linewidth}{\centering Code (\acs{gool})}} \\

			\arrow[line width=1, squiggly, color=blue, from=2-2, to=2-3]
			\arrow[line width=1, color=green, from=1-1, to=2-2]
			\arrow[line width=1, color=green, from=1-2, to=2-2]
			\arrow[line width=1, color=green, from=1-3, to=2-2]
			\arrow[line width=1, color=green, from=2-2, to=2-1]
		\end{tikzcd}\]
	\vspace{-2em}

	\footnotesize
	\begin{tabular}{llllll}
		\textcolor{green}{$\rightarrow$}                           & Stable transformation    &
		\textcolor{blue}{$\rightsquigarrow$}                       & Imperfect transformation &
		\tikz{\node[cloud, aspect=3, draw=green] (c) at (0,0) {};} & Formally captured          \\ \\
	\end{tabular}
\end{figure}

\footnotetext{This is an updated version of the diagram from my project poster \cite{Balaci2021Poster}.}
}
\newcommand{\refTheoriesWithModelKinds}{\Cref{fig:theoriesWithModelKinds}}

%------------------------------------------------------------------------------
% Images
%------------------------------------------------------------------------------

\newcommand{\drasilLogo}{\imageAssets/Drasil Logo.png}

%------------------------------------------------------------------------------
% Tables
%------------------------------------------------------------------------------

% List of case studies, and their goals/focus
\newcommand{\caseStudiesTable}{\begin{longtable}[c]{|>{\raggedright}p{0.3\linewidth}|>{\raggedright\arraybackslash}p{0.54\linewidth}|}
    \caption{Drasil Case Studies}
    \label{tab:drasilCaseStudies}                                              \\

    \hline

    \rowcolor{McMasterMediumGrey}
    \textbf{Case Study}      & \textbf{Focus}
    \\ \hline

    \caseStudy{glassbr}      & {Predicting likelihood of a glass slab
            resisting a specified blast.}
    \\ \hline

    \caseStudy{projectile}   & {Determining if a launched projectile hits a
            target, assuming no flight collisions.}
    \\ \hline

    \caseStudy{sglpendulum}  & {Observing the motion of a single pendulum.}
    \\ \hline

    \caseStudy{dblpendulum}  & {Observing the motion of a double pendulum.}
    \\ \hline

    \caseStudy{gamephysics}  & {Modelling of an open source 2D rigid body
            physics library used for games.}
    \\ \hline

    \caseStudy{hghc}         & {Examining the heat transfer coefficients
            related to clad.}
    \\ \hline

    \caseStudy{pdcontroller} & {Examining the output of a ``Power Plant''
            (Process Variable) over time.}
    \\ \hline

    \caseStudy{swhs}         & {Modelling of a solar water heating system with
            phase change material, predicting temperatures and change in heat energy
            of water and the PCM over time.}
    \\ \hline

    \caseStudy{nopcm}        & {Modelling of a solar water heating system
            without phase change material, predicting temperatures and change in heat
            energy of water and the PCM over time.}
    \\ \hline

    \caseStudy{ssp}          & {Assessment of the safety of a slope (composed
            of rock and soil) subject to gravity, identifying the surface
            most likely to experience slip and an index of its relative
            stability (factor of safety).}
    \\ \hline
\end{longtable}
}
\newcommand{\refCaseStudiesTable}{\Cref{tab:drasilCaseStudies}}

% List of case studies, and information about their code generation
\newcommand{\caseStudiesCodeTable}{\begin{longtable}[c]{|l|c|c|c|c|c|c|}
    \caption{Drasil Case Studies Artifacts Generated}
    \label{tab:drasilCaseStudiesCode}

    \\

    \hline

    \rowcolor{McMasterMediumGrey}
    \textbf{Case Study} & \textbf{SRS} & \textbf{C/C++} & \textbf{Java} & \textbf{C\#} & \textbf{Python} & \textbf{Swift}
    \\ \hline

    \acs{glassbr}       & {\checkmark} & {\checkmark}   & {\checkmark}  & {\checkmark} & {\checkmark}    & {\checkmark}
    \\ \hline

    \acs{projectile}    & {\checkmark} & {\checkmark}   & {\checkmark}  & {\checkmark} & {\checkmark}    & {\checkmark}
    \\ \hline

    \acs{sglpendulum}   & {\checkmark} & {}             & {}            & {}           & {}              & {}
    \\ \hline

    \acs{dblpendulum}   & {\checkmark} & {}             & {}            & {}           & {}              & {}
    \\ \hline

    \acs{gamephysics}   & {\checkmark} & {}             & {}            & {}           & {}              & {}
    \\ \hline

    \acs{hghc}          & {\checkmark} & {}             & {}            & {}           & {}              & {}
    \\ \hline

    \acs{pdcontroller}  & {\checkmark} & {}             & {}            & {}           & {\checkmark}    & {}
    \\ \hline

    \acs{swhs}          & {\checkmark} & {}             & {}            & {}           & {}              & {}
    \\ \hline

    \acs{nopcm}         & {\checkmark} & {\checkmark}   & {\checkmark}  & {\checkmark} & {\checkmark}    & {}
    \\ \hline

    \acs{ssp}           & {\checkmark} & {}             & {}            & {}           & {}              & {}
    \\ \hline
\end{longtable}
}
\newcommand{\refCaseStudiesCodeTable}{\Cref{tab:drasilCaseStudiesCode}}

% Drasil personification
\newcommand{\drasilPersonification}{\begin{longtable}[c]{|>{\raggedright}p{0.3\linewidth}|>{\raggedright\arraybackslash}p{0.54\linewidth}|}
    \caption{Drasil Logo Personification}
    \label{tab:drasilPersonification}
    \\

    \hline

    \rowcolor{McMasterMediumGrey}
    \textbf{Component}                 & \textbf{Conceptual Counterpart/Personification}

    \\ \hline

    {Roots}                            & {The roots are where the information of
            the seed influences the earth, and makes it comfortable for the tree
            to grow tall and firm. Information influences and encourages
            re-evaluation and structural change of the ground.}

    \\ \hline

    {Ground / Foundation}              & {The most important component, it is
            where the tree stands tall, and all knowledge relies on. It contains
            the definitions of the encodings, and is required to be strong or
            else a seed will be insufficient, irrelevant of how much topsoil is
            provided. The ground is irreplaceable, and difficult to ``fake''.}

    \\ \hline

    {Seed}                             & {The initial bundle of information,
            from which everything originates and derives from. You provide the
            bare minimum information to describe your problem, and use nutrients
            and care to carefully grow the seed into something else.}

    \\ \hline

    {Nutrients (topsoil and sunlight)} & {Encouraged growth through
            hinting/providing extra information. This is where you configure
            growth and encourage further growth externally, artificially.}

    \\ \hline

    {Trunk}                            & {The initial display of growth of the
            tree, building a wide knowledge-base. Sometimes requires maintenance
            (trimming, or, extra information/nutrients) to grow further and
            become a strong basis for the crown.}

    \\ \hline

    {Crown + Fruits}                   & {The fruits of your labour, standing on
            the shoulder of giants, where the final product (software artifacts)
            are realized.}

    \\ \hline
\end{longtable}
}
\newcommand{\refDrasilPersonification}{\Cref{tab:drasilPersonification}}


% McMaster Colours
\input{mcmasterColours}

% For fancy pictures
\usepackage{tikz}
\usetikzlibrary{shapes,arrows,cd}
\usepackage{quiver}
\usetikzlibrary{babel} % Make sure quiver/tikz uses babel

% Make sure the floating lists (figures, source codes, and abbreviations) are
% shown in the toc.
\usepackage{tocbibind}

% Set double spacing
\usepackage{setspace}
\doublespacing

\title{
    {Title}\\
    {McMaster University}
}
\author{Jason Balaci}
\date{\today}

% START : TODO LIST SETUP
% TODO Notes!
\ifshowwritingdirectives
  \usepackage{todonotes}
  \setlength{\marginparwidth}{3.5cm}
  \reversemarginpar % place on left-hand side
\else
  \usepackage[disable]{todonotes}
\fi
% END   : TODO LIST SETUP

\begin{document}

\frontmatter

% START : TODO LIST
% Only show it if the "show writing directives" flag is enabled
\ifshowwritingdirectives
  \todototoc
  \listoftodos
  \newpage
\fi
% END   : TODO LIST


%------------------------------------------------------------------------------
% Half Title page
%------------------------------------------------------------------------------

% First two pages will have a slightly different margin than the rest of the
% document, for the sake of making the document look slightly nicer.
\newgeometry{margin=1in}

\pagenumbering{gobble} % page 0
\null % Needed or else vspace doesn't work
\vspace{0.25\textheight}

\begin{center}
  \MakeUppercase{\thesisHalfTitle}
\end{center}

\newpage

%------------------------------------------------------------------------------
% Title page
%------------------------------------------------------------------------------

\null % Needed, or else vspace doesn't work
\vspace{0.2\textheight}

\begin{center}
  \MakeUppercase{\thesisTitle}

  \vspace{2cm}

  By \MakeUppercase{\thesisAuthorName}, \thesisAuthorCredentials{}

  \vfill

  A Thesis Submitted to the School of Graduate Studies in Partial Fulfillment of
  the Requirements for the Degree \thesisTargetDegree{}

  \vspace{2cm}

  \thesisInstitution{} \textcopyright{} Copyright by \thesisAuthorName{}, \thesisSubmissionMonthYear{}

\end{center}

\newpage

% restore the required margin details for the rest of the document.
\restoregeometry

% START : HEADER %
\pagestyle{fancy}
\fancyhead{}
\fancyfoot{}
\fancyfoot[C]{\thepage}
\renewcommand{\headrulewidth}{0pt} % Remove underline of header
% END   : HEADER %

%------------------------------------------------------------------------------
% Descriptive Note
%------------------------------------------------------------------------------

\pagenumbering{roman}
\setcounter{page}{2} % page 2
\noindent
\thesisInstitution{} \\
\thesisTargetDegreeName{} (\thesisSubmissionYear{}) \\
\thesisCityProvince{} (\thesisInstitutionDepartment{}) 

\vspace{2cm}

\noindent
TITLE: \thesisTitle{} \\
AUTHOR: \thesisAuthorName{}, \thesisAuthorCredentials{} \\
SUPERVISOR: \thesisSupervisor{} \\
PAGES: \pageref{lastOfFrontMatter}, \pageref{LastPage}

\newpage

%------------------------------------------------------------------------------
% Lay Abstract
%------------------------------------------------------------------------------

\chapter*{Lay Abstract} % Hide from TOC
A lay abstract, with a maximum of 150 words, explaining the key goals and
contributions. See
\url{https://gs.mcmaster.ca/app/uploads/2019/10/Prep_Guide_Masters_and_Doctoral_Theses_August-2021.pdf}
for an explanation of the requirements.

\intodo{lay abstract}


%------------------------------------------------------------------------------
% Abstract
%------------------------------------------------------------------------------

\chapter{Abstract}
\chapter{Abstract}
\label{chap:abstract}

Drasil~\cite{Drasil2021} is a software suite for generating software, with
particular focus for generating \ACF{scs} following the requirements described
in an abstract \ACF{srs} template. The template breaks up scientific knowledge
into various categories, and the abstracted variant of the template makes it
digestible for Drasil. A series of \acsp{dsl} are used to ``fill in'' the
template, from which Drasil is able to interpret an instance of, and configure a
generation procedure to generate usable software. The depth of knowledge
contained in the template's theories is key to understanding to what end we can
interpret and use the instantiated theories. To strengthen this depth, we create
a system for extending Drasil's known theory kinds and clearly explaining which
ones are usable in any specific context. Similarly, each theory kind contains a
particular subset of mathematical language that is relevant to them, and we act
on this information to restrict usable expression terms to their related
contexts. To further enrich the admissibility of expressions, we also make one
of the most critical subsets, that for concrete theory transcription,
\textit{type-safe} by building a \textit{bidirectional} type-checker that
catches all existing typing issues. In doing so, we were able to uncover many
typing issues in both the case studies generating code and those not. Finally,
as Drasil relies on a plethora of different \textit{types} of knowledge, it
needs a place to store them. Thus, we create a system to store any instance of
any type of knowledge in Drasils memory bank of knowledge, and enforce a basic
set of requirements that each type of knowledge must satisfy.


%------------------------------------------------------------------------------
% Acknowledgements
%------------------------------------------------------------------------------

\chapter{Acknowledgements}
\prefacesection{Acknowledgements}

I am deeply grateful to my supervisors, Dr. Spencer Smith and Dr. Jacques 
Carette, for their support and guidance throughout this project. Their 
expertise and insights have been invaluable in shaping my research, and I could 
not have completed this work without their guidance. Their constructive 
feedback, encouragement, and patience have been truly appreciated, and I feel 
fortunate to have had the opportunity to work with them.

I would also like to express my gratitude to my colleagues, Jason Balaci, Sam 
Crawford, and Don Chen, for their willingness to share their expertise and 
knowledge. Their dedication and talent have inspired and motivated me, and I am 
grateful to have worked alongside such amazing colleagues.

Lastly, I would like to thank my parents for their unconditional love and 
unwavering support throughout my academic journey. Their encouragement have 
given me the confidence to pursue my dreams, and I am forever grateful for 
their belief in me.

%------------------------------------------------------------------------------
% Table of Content
%------------------------------------------------------------------------------

\tableofcontents

%------------------------------------------------------------------------------
% List of Figures
%------------------------------------------------------------------------------

% \renewcommand{\listfigurename}{List of Illustrations, Charts, and Diagrams}
\listoffigures

%------------------------------------------------------------------------------
% List of Tables
%------------------------------------------------------------------------------

\listoftables

%------------------------------------------------------------------------------
% Table of Source Codes
%------------------------------------------------------------------------------

\listoflistings

%------------------------------------------------------------------------------
% List of Abbreviations and Symbols
%------------------------------------------------------------------------------

% TODO: make sure that all acronyms are actually used.
\printacronyms[
  display=all,
  template=longtable,
  heading=chapter,
  name=List of Abbreviations and Symbols
]

%------------------------------------------------------------------------------
% Declaration of Academic Achievement
%------------------------------------------------------------------------------

\chapter{Declaration of Academic Achievement}
\chapter{Declaration of Academic Achievement}
\label{chap:declaration_of_academic_achievement} % NOTE: If you're going to change this label, you might need to also update it in your manifest.tex file.

\imptodo{Declaration of academic achievement. See McMaster's thesis guidelines
    imptodo a more descriptive instruction.}
% https://gs.mcmaster.ca/app/uploads/2019/10/Prep_Guide_Masters_and_Doctoral_Theses_August-2021.pdf

\label{lastOfFrontMatter}

%------------------------------------------------------------------------------
% Main Chapters
%------------------------------------------------------------------------------

\mainmatter

% START : HEADER %
\fancyhead[R]{McMaster University — Computing and Software}
\fancyhead[L]{M.Sc. Thesis — Jason Balaci}
% END   : HEADER %

\chapter{Introduction}
\label{chap:introduction}
\chapter{Introduction} \label{chap:intro}
Scientific computing (SC) is an intersection of computer science, mathematics, 
and science. It is a field that solves complex scientific problems by using 
computing techniques and tools. Writing documentation is a part of the process 
of developing scientific software. The role of documentation is to help 
people better understand the software and to ``communicate information to its 
audience and instil knowledge of the system it describes" 
\cite{forward2002software}. The significance of software documentation has 
been presented in many papers by previous researchers \cite{parnas2011precise}, 
\cite{chomal2014significance}, \cite{kipyegen2013importance}. It is further 
shown by Smith et al. \cite{SmithandKoothoor2016}, \cite{SmithandYu2007} that 
developing scientific computing software (SCS) in a document-driven methodology 
improves the quality of the software . 

Jupyter Notebook is a system for creating and sharing data science and 
scientific computing documentation. It is a nonprofit, open-source application 
born out in 2014, providing interactive computing across multiple programming 
languages, such as Python, Javascript, Matlab, and R. A Jupyter Notebook 
integrates text, live code, equations, computational outputs, visualizations, 
and multimedia resources, including images and videos. Jupyter Notebook is one 
of the most widely used interactive systems among scientists. Its popularity 
has grown from 200,000 to 2.5 million public Jupyter Notebooks on GitHub in 
three years from 2015 to 2018 \cite{Jeffrey2018}. It is used in a variety of 
areas and ways because of its flexibility and added values. For example, the 
notebook can be used as an educational tool in engineering courses, enhancing 
teaching and learning efficiency \cite{cardoso2019using}, \cite{zhao2019use}.

Even though the importance of documentation is widely recognized, it is often 
missing or poorly documented in SCS because: i) scientists are not aware of 
the why, how, and what of documentation \cite{hermann2022documenting}, 
\cite{chang2022understanding}; ii) it is time-consuming to produce 
\cite{sanders2008dealing}; iii) scientists generally believe that writing 
documentation demands more work and effort than they would likely yield in 
terms of the benefits of it \cite{smith2016advantages}.

We are trying to increase the efficiency of documentation development by 
adopting generative programming. Generative programming is a technique that 
allows programmers to write the code or document at a higher abstraction level, 
and the generator produces the desired outputs. Drasil is an application of 
generative programming, and it is the framework we use to conduct this 
research. Drasil saves us more time in the documentation development process by 
letting us encode each piece of information of our scientific problems once and 
generating the document automatically.


\section{Background}
\subsection{Drasil}
Drasil is a framework that can generate software artifacts, including Software 
Requirement Specifications (SRS), code (C++, C\#, Java, and Python), README, 
and Makefile, from a stable knowledge base. The goals of Drasil are reducing 
knowledge duplication and improving traceability \cite{drasil}. Drasil captures 
the knowledge through our hand-made case studies. We currently have 10 case 
studies that cover different physics problems, such as Projectile and Pendulum. 
Recipes for scientific problems are encoded in Drasil, and it generates code 
and documentation for us. Each piece of information only needs to be provided 
to Drasil once, and that information can be used wherever it is needed. By 
defining and storing common concepts in a central repository, and case-specific 
concepts in their own packages, Drasil enables the reuse of information across 
different engineering domains and applications. This feature significantly 
reduces the time and effort required for software development and 
documentation, while also improving the consistency and accuracy of the 
information being used. More information and an example of how knowledge is 
reused can be found in Chapter~\ref{chap:projMotion}. 

Drasil is currently capable of generating SRS, a template for designing and 
documenting scientific computing software requirement decisions created by 
Smith et al \cite{smith2005new}, in document languages HTML and LaTeX. We are 
looking to extend the capability of Drasil by generating Jupyter Notebook in 
Drasil.

\subsection{Jupyter Notebook}
Jupyter Notebook is an interactive open-source web application for creating and 
sharing computational science documentation that contains text, executable 
code, mathematical equations, graphics, and visualizations.

\subsubsection{Structure of a notebook document}
A Jupyter Notebook has two components: front-end ``cells" and back-end 
``kernels". The notebook consists of a sequence of cells: code cells, markdown 
cells, and raw cells. A cell is a multiline text input field. The notebook 
works by users entering a piece of information (text or programming code) in 
cells from the web page user interface. That information is then passed to the 
back-end kernels which execute the code and return the results 
\cite{notebookdoc}.

\subsubsection{The Value of Jupyter Notebook}
There are several advantages of Jupyter Notebook: sharable, all-in-one, and 
live code. First of all, the notebook is easy to share because it can be 
converted into other formats such as HTML, Markdown, and PDF. Secondly, it 
combines all aspects of data in one single document, making the document easy 
to visualize, maintain and modify. In addition, Jupyter Notebook provides an 
environment of live code and computational equations. Usually, when programmers 
are running code on some other IDEs, they have to write the entire program 
before executing it. However, the notebook allows programmers to execute a 
specific portion of the code without running the whole program. The ability to 
run a snippet of code and integrate with text highlight the usability of the 
notebook.

\section{Problem Statement}
Since both Jupyter Notebook and Drasil focus on creating and generating 
scientific computing documentation, we are interested in extending the values 
of Jupyter Notebook to Drasil and the kind of knowledge we can manipulate. 
Following are the three main problems we are trying to solve with Drasil in 
this paper:

\begin{enumerate}
	\item Generate Jupyter Notebooks. To acheive this, we will have 
	to generate documents in notebook format. Jupyter Notebook is a simple 
	JSON document with a .ipynb file extension. Notebook contents are either 
	code or Markdown. Therefore, non-code contents must be in Markdown format 
	with JSON layout. Drasil can only write in HTML and LaTeX. We are building 
	a notebook printer in Drasil for generating documents that are 
	readable and writable in Jupyter Notebook.
	\item Develop the structure of lesson plans and generate them. As 
	mentioned, Jupyter Notebook is used as an educational tool for teaching 
	engineering courses. When it comes to teaching, lesson plans are often 
	brought up because they help teachers to organize the daily activities 
	in each class time. We are interested in teaching Drasil a ``textbook" 
	structure by starting with generating a simple physics lesson plan and 
	expanding Drasil's application. We aim to capture the elements of 
	textbook chapters, identify the family of lesson plans, and classify the 
	knowledge to build a general structure in Drasil, which will enable the 
	lesson plan to generalize to a variety of lessons.
	\item Generate notebooks that mix text and code. Jupyter Notebook is 
	an interactive application for creating documents that contain formattable 
	text and executable code. However, Drasil doesn't support interactive 
	recipes. There is no code in SRS documents, and text and code are generated 
	separately in Drasil. We are looking for the possibility of generating a 
	notebook document that incorporate both text and code, thereby enhancing 
	the capabilities of Drasil and its potential to solve more scientific 
	problems.
\end{enumerate}

\section{Thesis Outline}
Thesis outline here.

\chapter{Ideology}
\label{chap:ideology}
The focus of this work is fundamentally based on the idea of ``generation.''
However, unlike GitHub and OpenAI's Copilot \cite{Copilot}, the ideology does
not delve into artificial intelligence, and does not focus on ``autocompleting
code'' by natural language. Instead of working with the set of natural language,
the ideology focuses on formalizing the meaning of specific subsets into
principled stories. The principled stories are then intertwined and mixed until
a coherent ``whole story'' is formed, where we deterministically understand what
can and cannot be done with the knowledge described (including, but not limited
to, generating representational software artifacts and snippets).

\section{On Developing Software}
\label{sec:idlgy:on_developing_software}

As a software developer working to build a new piece of software, one might
sense that they are writing ``a lot of the same code'' as their or other
pre-existing software projects. One might consider building a library, shared
between all projects for some common functionality/tooling, this is a large
improvement for their program --- being able to reuse their code is great for
debugging and removing the possibility of bugs when stable and tested code is
used. The library will surely save them time in the future once they've
stabilized it to a reasonable degree, after which they will not need to worry
about repeating the same errors they made when they were originally writing it.
They will have made significant gains in their development environment and
workflow. However, the library might not be portable across machines, the
library might not make sense to those more or less familiar with certain ideas
touched upon by the library, others might question the validity of the knowledge
it pulls from, and the library might not be readily accessible to those not
using the same programming language the library was written in. They can write
an \acs{ffi}, but this is a fairly complex task that many are unfamiliar with,
and which requires meticulous analysis to ensure compatibility and creates time
expensive update procedures when foreign libraries are updated.

Commonly, one looks to use mature libraries and frameworks to underpin their
projects, occasionally without guarantee that connecting these libraries is
safe. A familiar example of this failure is the sinking of the Vasa
\cite{wiki:Vasa_ship}, partially caused by different teams working together but
using different ``feet'' units (the Swedish foot is 12'' while the Amsterdam
foot is 11'') resulting in unintended weight distribution. As general purpose
programming languages are often also used, misunderstandings of tacit project
knowledge may also cause errors. Unfortunately, despite building on the
shoulders of giants, this programming methodology takes significant time and
stress until a working product is formed with minimal bugs.

To resolve this, we believe we need to revisit the original sensation felt when
code re-creation/duplication was recognized -- why does this sensation occur?

The reason is obvious (ignoring the more obvious fact that they are aware of
similarities between their code and that previously written)! They feel this
sensation because they already understood it and had written it previously.
Implicitly, when writing the software, the developer already had a mental
connection between some algorithm and the code they had written (at times,
specific to the programming language they had chosen to use) -- it might have
already been \textit{well-understood} to them.

Unfortunately, a non-trivial amount of time is spent manually recreating the
same piece of code from the same implicit knowledge-base, and the developer is
likely left with less than their preferred amount of time to work on the
components that are interesting/important that they might never have built
before or that they might not have connected together in the past.

\section{Thoughts of Generation}
\label{sec:idlgy:thoughts_of_generation}

The ideology foundational to this work is predicated on this simple idea that if
we understand how a \textit{thing} works, and we can create a working model
describing it, then we should be able to encode it, describing everything about
this model. Furthermore, if we understand how this \textit{thing} can be
transformed into another \textit{thing}, then we should be able to describe the
process of transformation as well. Finally, if we are able to do this en masse
to a pool of \textit{various things}, we should be able to generate whole pools
of \textit{other various things} from empty, or minimal seedling pools.

Applying this idea to the world of software, where we can create and model
\textit{things} (henceforth referred to as ``knowledge''), we can take pools of
knowledge and generate other pools of knowledge (including whole software
artifacts). \todo{Discuss triform theories} With a weak breadth and low depth of
captured knowledge, the originating pool of knowledge may approximately appear
as a description of the software artifacts generated. However, with a strong
breadth and high depth of captured knowledge, the original pool of knowledge may
appear far removed from the software artifacts generated (it may contain little
to no discussion of the desired software artifacts at all, or a precise
description). In this case, the stress of software development is alleviated
entirely, and a ``perfect'' software artifact (or series of ``perfect'' software
artifacts) is (are) constructed.

\section{The Goal}
\label{sec:idlgy:the_goal}

If one originally sets out to build a program that does something they
understand very well and each component of every step of the grand
scheme/algorithm of the program was understood, the development of their related
software should be a ``clear matter of principled engineering'' \todo{Cite
    GitHub README? I really don't remember where I read this now\ldots{}}. However,
with the existing programming methodology, it is not yet simple enough to
reliably produce error-free programs with no trade-offs, and which precisely
satisfy a set of requirements and follow a formal design.

Optimally, they would use their natural language to perfectly describe their
problem to their computer and have it magically give them a program that does
exactly what they wanted. Unfortunately, it is difficult to have computers
systematically understand and act on natural language as well as us humans can.
However, specific subsets of natural language may be usable.

\section{Reconciliation — ``Generate Everything''}
\label{sec:idlgy:generate_everything}

While it is difficult to have a computer act on a huge set of natural language,
we do have well-understood thoughts on specific \textit{terms} and how they can
be interpreted. More specifically, there are sub-languages of natural language
that we can use to describe specific kinds of knowledge, and problems and
solutions. Here, we recognize that selecting or creating \aclp{dsl} for specific
buckets of knowledge and creating \aclp{dsl} that connect buckets of other
\aclp{dsl}, we can effectively create rigid sub-languages of our natural
language used to describe programs. These rigid sub-languages being encoded as
\aclp{dsl} will have a well-defined and formalized \acs{ast}, which allows us to
write interpreters that can take them and transform them into other fragments of
knowledge (us being most interested in ultimately generating software
artifacts). Thus, with enough effort, and through sequencing and connecting
terms of \aclp{dsl}, we can effectively model what the discussed software
developers are trying to build, allowing the computer to be able to better
understand what they are trying to build.

Note, however, that this idea likely only thrives in domains of knowledge that
is ``well-understood'' \cite{well-understood}.

In theory, this should appear as a \ACF{kms}, where generation phases are passed
through the knowledge registry until a final knowledge registry is constructed
such that it satisfies the requirements of the user (e.g., generating some
desired software artifacts).

\section{Prospective Workflow \& Roles}
\label{sec:idlgy:prospective_workflow}

\intodo{Discuss ``idealized'' workflow.}

Following this ideology, there will be at least 3 key roles: the
\textbf{knowledge encoder}, the \textbf{knowledge user}, and the
\textbf{end-user} of the produced software artifact.

\subsection{Knowledge Encoder (Domain Expert)}

The knowledge encoder should be a master of a particular domain. They are
expected to encode the knowledge discussed in their respective domain in such a
way that is accessible to those without knowledge of their domain. Additionally,
they should encode information about the ways in which the knowledge can be
transformed into other forms of knowledge (including that which is
interdisciplinary). The knowledge they would be encoding should be as well-known
and globally standardized as possible. As discussed in
\autoref{sec:idlgy:generate_everything}, it is likely that the knowledge encoder
will focus on writing a series of highly specific \aclp{dsl}, where the
languages may be restricted to as specific as one term or a handful.

\subsection{Knowledge / Domain User \& Orchestrator}

The knowledge user/orchestrator should at least be familiar with a particular
domain, and have goals in mind for information that they would like to ``seed''
into their \acl{kms}. They should also have a working understanding of what the
end-user needs. From this, they should be able to connect the work of the domain
expert into ``plug-n-play'' stories (arguably, compilers for the end-users to
use), or be able to encode/reduce friction between knowledge encoded by domain
experts.

\subsection{End-user}

The final end-user should find the most delight from this ideology. They are the
actual users of the software artifacts, perhaps tweaking the final build of the
software artifacts to be accustomed to their workflow. If the tasks assigned to
the knowledge encoders and the knowledge users are performed correctly, then the
end-user should have strong confidence in the artifacts as they were built with
strict adherence to the knowledge captured at \textit{every step of the way}. As
such, one should confidently expect the final software artifacts to be
completely devoid of unexpected things (including errors, unconformities to
specifications, etc.).

\intodo{Figure out where this below paragraph should belong}
As a domain expert transcribing knowledge encodings of some well-understood
domain, one will largely be discussing the ways in which pieces of knowledge are
\textit{constructed} and \textit{relate to each other}. In order for this
abstract knowledge encodings to be \textit{usable} in some way, it is vital to
have ``names'' (\textit{types}) for the knowledge encodings. In working to
capture the working knowledge of a domain, it's of utmost importance to ensure
that all ``instances'' of your ``names'' (types) are \textit{always} usable in
some meaningful way and that the knowledge is exposed in a usable way (e.g.,
sufficiently through some sort of \acs{api}). In other words, all knowledge
encodings should create a stringent, explicit set of rules for which all
``instances'' should conform to, and, arguably, also creates a justification for
the need to create that particular knowledge/data type. As such, optimally, a
domain expert would write their knowledge encodings and renderers in a general
purpose programming language with a sound type system (e.g., Haskell
\cite{Haskell2010}, Agda \cite{Norell2007}, etc.) — preferring ones with a type
system based on formal type theories for their feature richness.

\section{Feasibility}
\label{sec:idlgy:feasibility}

In order for us to discuss feasibility of this idealized prospective workflow,
we must discuss the depth \& breadth of knowledge we need to make this feasible.
Depth of knowledge refers to the vertical knowledge understood about a specific
fragment of knowledge, and it's preciseness. For example, we may have a
low-depth of knowledge \& claim that English sentences are a sequence of
characters, or we might have a slightly ``deeper'' depth/understanding  of
sentences by describing them as a language that follows a specific syntax rule
set and using a specific set of words. Breadth of knowledge is the horizontal
domain of knowledge, it is the various kinds of knowledge we have in a wide
variety of subjects and domains.

In low-depth areas, we may observe that this ideology is very practical, and
heavily used. Widely used \ACFP{cms}, such as WordPress \cite{WordPress} and
Drupal \cite{Drupal}, and web frameworks, such as Django \cite{Django} and
Laravel \cite{Laravel}, are arguably also following similar ideals as this
ideology. They all typically provide a basic understanding of ``users'' of a
hosted website, facilities to write HTML content in one way or another, plugins,
and more. While some might be, these listed above are not specific to one
specific use-case. They are very versatile and highly reusable for a wide
variety of use-cases because they ship with low but sufficient depth of
knowledge (though they might call it ``features''). Out of the box, these web
technologies listed come with simple, common, functionality (features) and
powerful extensibility through either plugins or through software extension and
usage. With the basic tooling provided, users are able to rapidly deploy
websites with content. Through extending the website's knowledge-base (e.g.,
plugins or software extension), they are able to obtain a wider breadth and
deeper depth to the knowledge contained within them. Through this, the end-users
are able to encode increasingly complex and different kinds of data into the
systems to ultimately obtain increasingly specialized websites, such as
technical blogs, eCommerce websites, online accounting software, online
discussion forums, and more. In these technologies, knowledge depth typically
remains ``shallow'', but through increasing breadth of knowledge, increasingly
interesting websites may be created. The mechanized generation-related
components of the ideology is also fairly shallow in this area, but, still,
highly feasible.

\intodo{Realistically, almost any area of generation can be a ``low-depth''/breadth area too}

\intodo{Add to high knowledge density examples: \url{https://en.wikipedia.org/wiki/Algebraic_modeling_language}}
\intodo{Add to high knowledge density examples: \url{https://github.com/McMasterU/HashedExpression}}

In areas of high knowledge density, as long as developers have infinite patience
and can invest infinite time into transcribing knowledge and information into
the software, anything is possible, and this ideology is very practical.
Unfortunately, that situation is not quite realistic. As such, we should
restrict our scope in high knowledge density areas to only those
``well-understood'' \cite{well-understood}. \todo{Define well-understood}. In
projects with high knowledge density, it's crucial to have clear and concise
knowledge fragment encodings that allow users to directly interact with the
decomposed simplicity associated with each ``transformation'' operation.
Thankfully, since these domains are also typically codified (or easily
codifiable), this is doable. Through capturing many series of incomplex
transformations of knowledge into other forms, we are able to sequence and
compose them until an ultimate large, normally complex and complicated,
transformation is formed. Finally, through creating a directed discussion of
ideas, we can form a stable knowledge-base from which we can draw information.
One large feasible goal from it is to draw out usable software artifacts from it
by poking and prodding in all areas needed until we can deterministically form
them, as our needs demand.

The essence of this ideology lies in naturally obtaining mechanization
techniques through formalizing \textit{everything}. The ideology forces us to
question modern software development practices: cognitive stress and normal
errors aside, if a developer doesn't \textit{truly understand} the knowledge
they are encoding in a software product, then it should be normal to expect
logical issues and an ``imperfect'' program. Taking cognitive stress into
consideration, there should be considerably less as knowledge only need be
transcribed once, and re-used infinitely. Through mechanization, cognitive
stress of re-writing knowledge is alleviated afterwards for all proceeding
instances. A formalized-knowledge-first approach to software development should
highlight areas of issue (poor understanding), never produce bugs, and create
software with the same quality as the encoded knowledge.

\intodo{The ideology also relates to the question of ``which programming
    language do I pick for my project?''. This is a moot question under this
    ideology. There is no one (1) single language that we even use to describe
    everything, so how can we possibly write all kinds of software for all kinds
    of knowledge? In a sense, the ideology ``leans into it'' by saying you
    shouldn't be limited to choosing just one (1).}

\intodo{The ideology demands that we unify modern domain knowledge. Almost
    naturally, we obtain mechanization as a side effect. In other words, it
    demands that we connect all of our compilers into one single coherent
    mega-compiler system.}

\intodo{One of the large issues with modern software development is a disconnect
    in knowledge, which is why this paradigm is the natural answer. The
    ideology is, exactly, ``unify all your knowledge.'' Everything gained by
    unifying our knowledge is a natural side effect, which we happen to care
    about.}

\intodo{Through describing their requirements in a coherent manner (e.g., some
    language), completely non-technical product owners may, and will, still be
    key figures in the production of the product.}


\chapter{Drasil}
\label{chap:drasil}
\footnotetext[1]{\url{https://jacquescarette.github.io/Drasil/}}

\begin{mdleftbar}
      ``Drasil is a framework for generating families of software artifacts from
      a coherent knowledge base, following its mantra, ``Generate All The
      Things!''. Drasil uses a series of variably sized \ACFP{dsl} to describe
      various fragments of knowledge that domain experts and users alike may use
      to piece together fragments of knowledge into a coherent ``story''.
      Through forming some coherent ``story'' in a domain captured by Drasil, a
      representational software artifact may be generated. Drasil currently
      focuses on \ACF{scs}, following Smith and Lai's \ACF{srs} template as
      described in \cite{SmithAndLai2005}. Behind the scenes of the \acs{srs}, a
      mathematical language is used to describe various theories, and have
      representational software constructed via compiling to \ACF{gool}
      \cite{Carette2019}. Through encoding knowledge in Drasil, an increase in
      productivity (and maintainability) in building reliable and traceable
      software artifacts is observed \cite{SzymczakEtAl2016}, specifically in
      \acs{scs} \cite{Smith2018}. Drasil's source code (Haskell), case studies,
      and documentation studies can be found on its
      \porthreftm{website}{https://jacquescarette.github.io/Drasil/}.''
      \cite{Balaci2021Poster}
\end{mdleftbar}

\section{An Exploration}

Originally known as Literate Scientific Software (LSS) \todo{cite}, Drasil is an
exploration of this ideology. Drasil's largest domain of knowledge covered
originates from LSS: scientific computing software (SCS). \porthref{Dr. Jacques
      Carette}{https://www.cas.mcmaster.ca/~carette/} and \porthref{Dr. Spencer
      Smith}{https://www.cas.mcmaster.ca/~smiths/} are the principal investigators of
Drasil. Deeply embedded in Haskell \todo{ref} (and currently built against
\acs{ghc} \todo{ref} 8.8.4), Drasil currently focuses on building tooling around
8 case studies, each following the formalized \ACF{srs} template
\cite{SmithAndLai2005}:

\intodo{Rewrite the below paragraph...}

With a focus on building Scientific Computing Software,
Drasil\thinspace\cite{Drasil2021} is an exploration of the ideology described in
\autoref{chap:ideology}. Rather than building one's software project in any
single or combination of general purpose programming language, the usage of a
sequence of domain-specific languages together in Drasil-based projects can be
used to describe common undergraduate level physics models and problems and
describe the target program that simulates said models.

\intodo{Last sentence of above: be more generic, and then mention that it is
      \textit{currently} focused on physics, etc.}


\intodo{Add citations to each row of the below:}

\caseStudiesTable

Drasil is currently capable of generating usable software through compiling to
\ACF{gool}, which is capable of producing Java, C++, Python, C\#
\cite{MacLachlan2020}, and Swift (not discussed in MacLachlan's Master's thesis,
but created by him as well, and available similarly). Drasil contains renderers
for HTML, Makefile, basic Markdown (enough for README), GraphViz DOT (graph
description language) \cite{Gansner1993}, plaintext documents, and \LaTeX\ /
\TeX. Drasil's source code is publicly available on
\porthref{GitHub}{https://github.com/JacquesCarette/Drasil}, and Drasil's
documentation
(\porthref{user-facing}{https://jacquescarette.github.io/Drasil/docs/index.html},
and
\porthref{internal}{https://jacquescarette.github.io/Drasil/docs/full/index.html})
is available on the Drasil project
\porthref{homepage}{https://jacquescarette.github.io/Drasil/}. A public Drasil
wiki is hosted on the \porthref{same GitHub
      project}{https://github.com/JacquesCarette/Drasil/wiki}, containing information
on potential future Drasil projects, Drasil-related papers, a
\porthref{developer workspace configuration and ``quick start''
      guide}{https://github.com/JacquesCarette/Drasil/wiki/New-Workspace-Setup}, and a
guide for \porthref{building your own project with
      Drasil}{https://github.com/JacquesCarette/Drasil/wiki/Creating-Your-Project-in-Drasil}.

\section{Methodology}

\begin{itemize}

      \item Primarily focused on undergraduate-level science (primarily physics)
            problems

      \item Generating scientific software artifacts
            \begin{itemize}
                  \item Develop a stable knowledge base for physics problems

                  \item Develop a stable framework for laying cookie-cutter
                        problems.

                  \item Make it as simple as using a projectional editor.

            \end{itemize}

      \item Approach to encoding knowledge -- e.g., ``bottom-up''

\end{itemize}

\section{State of Architecture}

\begin{itemize}

      \item Knowledge Encodings \& Organization
            \begin{itemize}
                  \item Chunks
                  \item ChunkDB
            \end{itemize}

      \item SmithEtAl template for SRS documents

      \item Haskell/GHC 8.8.4

      \item Uses a series of \ACFP{dsl} to build up the knowledge-base and
            ``story'' of a scene/project, until sufficient knowledge is built
            such that a generator may take the whole story and generate
            representational software artifacts.
            \intodo{Should add an example of the encoding of a theory.}

      \item Generates \acs{oo} programs; \acs{gool} is used to generate code in
            C++/C, Java, Python, Swift, and C\#.

      \item Doxygen, Markdown, HTML, TeX, Makefile, CSS, and more
            non-programming software artifacts.

      \item Creates a ``calculation path'' for a series of $x = f(a,b,c,...)$
            formed equations (e.g., simple LHS = complex RHS), and ODEs
            supported by python libraries.

      \item \intodo{table of Case Study/Generates SRS/Code/Complete
                  \checkmark{}s to fill in the boxes}

      \item Multiple code generating examples (each with their own
            requirements):
            \begin{itemize}
                  \item GlassBR
                  \item NoPCM
                  \item PDController
                  \item Projectile
            \end{itemize}

      \item The calculation path relies (implicitly) on:
            \begin{itemize}

                  \item The Exprs of the RelationConcept being of the form: $x =
                              f(a,b,c,...)$

                  \item The ODEs being described by the expression language, but
                        we do not rely on being able to generate representing
                        computational code due to needing extra external
                        information (e.g., we require a supplementary packet of
                        information [ODEInfo]).

            \end{itemize}

\end{itemize}

\intodo{Short-term problems -- leading into topics}




\chapter{Framing Theories}
\label{chap:modelkinds}
\chapter{Classifying Theories}
\label{chap:modelkinds}

\begin{writingdirectives}
    \item What are theories used for in Drasil?
    \item How are they captured in Drasil?
    \item Current problems? Solution?
\end{writingdirectives}

In this chapter, we will focus on improving inspection and interpretation
capabilities of theories in Drasil. Specifically, with focus on interpreting
them for generating software artifacts.

\section{Transforming Theories to Code}
\label{chap:modelkinds:sec:transforming-theories-to-code}

As seen in \refOriginalJavaProjectileMain{}, it relies on a few methods
calculating the related ``instance models'' on its behalf. Those methods are a
specific \textit{interpretation}\footnote{Or ``view.''} of the theory knowledge.
Specifically, it is a \textit{calculation-focused} interpretation. For example,
to calculate \(p_\text{land}\), \refOriginalJavaProjectileMain{} relies on the
method \inlineCode{java}{func_p_land} to calculate the landing position.
\refOriginalJavaProjectilePLandMethod{} is one possible exported method for
calculating \(p_\text{land}\) given \(v_\text{launch}\), \(\theta\), and
\(\mathbf{g}\).

\originalJavaProjectilePLandMethod{}

Of course, to generate this method in \refOriginalJavaProjectilePLandMethod{},
Drasil relies on a sufficient capture of its underlying knowledge, and a means
of transforming said knowledge to ``code'' (e.g., sufficient information that
answers: What do we want to define? How can it be converted to code?). This
capture of theories is done using Drasils representation of an ``Instance
Model.'' To build it, Drasil uses information gathered from what users fed in
about it, via \refOriginalLandPosTheoryDefinition{}.

\originalLandPosTheoryDefinition{}

Now, let's unpack \refOriginalLandPosTheoryDefinition{}. \inlineHs{landPosIM} is
an instance of an \InstanceModel{}\footnote{Drasils encoding of the ``Instance
    Models.''}, containing the equational definition component
(\inlineHs{landPosRC}), and meta-level information about the theory, including
constraint ranges, notes, and a derivation\footnote{Mostly omitted here for the
    sake of conserving space.}.

In order to transform \refOriginalLandPosTheoryDefinition{} into
\refOriginalJavaProjectilePLandMethod{}, \textit{interpretation}\footnote{More
    accurately, this interpretation is ``domain-specific interpretation''
    \cite{Czarnecki2005}.} occurs on \inlineHs{landPosRC}
(\refOriginalRelToQDHaskell{}).

\originalRelToQDHaskell{}

\refOriginalRelToQDHaskell{} shows the definition of \relToQD{}, a function that
attempts to convert arbitrary relations into coherent ``quantity
definitions''\footnote{Or, ``variable definitions.''} (\QDefinition{}), which
Drasils ``\acs{srs} to code'' generator relies on. \inlineHs{landPosRC} is an
instance of a \RelationConcept{} (\refOriginalRelationConcept{}), a coupled
mathematical relation (encoded using \Relation{}s\footnote{\Relation{} is
    Drasils \acs{dsl} encoding arbitrary mathematical relations. We will talk more
    about this later. For now, this is enough.}), natural language description of
the relation, and descriptive name.

\originalRelationConcept{}

\relToQD{} is used on \textit{all} of the captured \InstanceModel{}s of a case
study to find code it can generate. However, these \InstanceModel{}s carry
arbitrary \Relation{}s, which are arbitrary mathematical relations. As such, an
issue arises: since \relToQD{} only works with ``variable defining''
relations\footnote{In some sense, calling these relations ``variable defining''
    is also an issue of itself. We're overloading \(=\) to mean definition.}, we're
limited to only one \textit{kind} of theory. Even with this restriction, we're
further limited to only a specific formation of that theory (e.g., relations of
the form \(x = f(a, b, c, \ldots{})\))!

\subsection{Problems}
\label{chap:modelkinds:sec:transforming-theories-to-code:subsec:problems}

In the sense that we really want to be able to use \relToQD{} (or something like
it) to transform arbitrary well-understood \textit{theories} into code fragments
for Drasils code generator, it has at least 3 problems:
\begin{enumerate}
    \item[\namedlabel{mk:issue:1}{P1}] it only handles one theory kind:
        variable definitions,
    \item[\namedlabel{mk:issue:2}{P2}] and for those definitions, it requires a
        specific form, thereby limiting users to very specific usage, views, and
        transcription,
    \item[\namedlabel{mk:issue:3}{P3}] and it implicitly assumes that all inputs
        will be of this theory kind, or else it causes a panic.
\end{enumerate}

As a result of \ref{mk:issue:1}, we aren't able to encode adequately all the
theories we're interested in using, and want to generate representational code
of. In particular, as Drasil is heavily guided by physics-focused case studies,
\acsp{ode} are desired! When we want to use \acsp{ode} in the solution of a
problem, extra information is required. For example, we might need to give
Drasil (and/or developers) information about a desired approximation formula
with particular ``settings.'' Drasil does circumvent this issue for \acsp{ode},
but we would like to reconcile the half-measures and push all necessary
information back in to the theory encodings.

Assuming we wanted to describe the theory of a line, there are many ways we can
describe the equation: polynomial (\(a \cdot{} x + b \cdot{} y + c = 0\)),
slope-intercept form (\(y = m \cdot{} x + b\)), point-slope form (\(y_1 - y_2 =
m(x_1 - x_2)\)), and so on. However, as a result of \ref{mk:issue:2}, we are
forced to use the ``simple'' slope-intercept form, even though we are aware of
other forms and may prefer to describe it in other forms.

As a result of \ref{mk:issue:3}, when Drasils users are encoding theories in
Drasil, they might be misled to think that any of their encoded theories is
fully-understood to Drasil and actually usable as part of the generated solution.

Together, these issues arise because of a lack of sufficient \textit{depth} and
\textit{breadth} in the contained knowledge of the theory encodings. Because the
theory encodings rely on ``flat'' information (e.g., the relations),
transforming them programmatically is challenging. Of course, if two programmers
were to read this information in the \acs{srs}, they might be able to use it.
However, the programmers understand the context of the theory, and are able to
\textit{recognize} (from any form of a theory) if and how it can be transcribed
as a computation. Now, imagine if we wanted to  So, how can we mitigate all of
these issues?

Just as we may discuss the specifics of implementing any particular theory in
our manual implementation of a software artifact, we assume prior learned
knowledge about mathematical expressions, such as which ones we know we can
somehow translate into code. In order to mitigate these issues, we must further
capture this background knowledge because raw relations carry too little
information about how to transform into coded. So, now, more concretely, we ask:
how could we have avoided these individual problems?

\begin{longtable}[c]{>{\raggedright}p{0.3\linewidth}>{\raggedright\arraybackslash}p{0.54\linewidth}}
    \textit{To avoid~\ldots{}} & \textit{we needed~\ldots{}}                                    \\
    \ref{mk:issue:1}              & knowledge about more \textit{kinds} of theories,            \\
    \ref{mk:issue:2}              & to decompose the relations into its set of
    logical components and capture information about how instances can be
    transformed into various forms,                                                             \\
    \ref{mk:issue:3}              & a signifier for each theory \textit{kind} we're interested. \\
\end{longtable}







\imptodo{Continue writing here!}

\section{\textquotedblleft{}Classify All The Theories\textquotedblright{}}

Issues occurring due to weak knowledge capture may be resolved through strong
knowledge capture.

Beginning with the existing case studies of Drasil, we will attempt to classify
our existing knowledge better.

We aim to make \RelationConcept{} a ``view'' of other more information-dense
encodings.

In other words, we replace \Expr{} as a knowledge container, and restrict its
usage to strictly ``mathematical expressions'', as opposed to ``expressions''
and information about models/theories.

One notable change is that we will require the new theory knowledge containers
to be able to fully re-create the original shallow/raw \Expr{}s as a property of
the new theory encodings.

The once meta-level knowledge of the theories, lost in the Haskell
implementation, becomes exposed and understood to Drasil.

Ultimately, this is done through replacing \RelationConcept{} usage with
\ModelKind{}\footnote{\ModelKind{} is built upon Dr. Jacques Carettes prototype
    of an earlier version of \ModelKinds{}}, an aggregation of existing
Drasil-related knowledge of mathematical theories. \ModelKind{} is defined using
a \acs{gadt}, with one (1) type parameter. The type parameter is currently used
to determine whether the model is ``fully refined''/``grounded'' or not, and,
hence, usable in code generation.

\refCurrentModelKindsHaskell{} displays the creation of \ModelKind{} and
\ModelKinds{}.

Please note that this aggregation is based purely on the existing model examples
in the existing Drasil case studies, and the existing models are incomplete in
the larger scope.

Each \textit{kind of model} we find only has one requirement: that it should
carry enough information to, and provide a means of, recreating the original
expression from which they were abstracted out of.

This requirement is essentially that of ``viewing'' it in the expression
language, but it also tends to also add flexibility in how many ways that we can
``view'' the information in different forms.

In the Haskell code, this requirement is enforced through requiring them all to
instantiate the \Express{} typeclass (\refCurrentExpressHaskell{}).

\currentModelKindsHaskell{}

In the above \ModelKinds{} definition, there are two (2) TODO notes that you may
disregard.

The first one is merely a note for analyzing ``well-understood'' copies of our
existing \acsp{ode}, and the second one refers to models that haven't yet been
fully analyzed for how they will be used (other than for display).

\subsection{Quantity Definitions}

\currentQDefinitionHaskell{}

Assume \(y = x\) is transcribed as a \RelationConcept{}: while \(y = x\) might
conventionally be seen as ``y is equal to x'', we might want, in our model, for
it to be understood as ``x is defined by y'' but displayed differently.

Here, \(=\) is overloaded as ``definition'', instead of what \(=\) was defined
as in \Expr{}, as an ``equality'' operator.

To resolve this overloading and weak knowledge capture of definitions, we create
\EquationalModel{}s: theories that contain information about definitions of
symbols, built using a \QDefinition{} (\refCurrentQDefinitionHaskell{}).

If an \EquationalModel{} deals with theoretical symbols and is defined using
either a \ModelExpr{} or an \Expr{}, it may be used in Theory Models and General
Definitions.

If an \EquationalModel{} is defined using an \Expr{} and deals with only the
concrete (non-abstract) symbols, then the \EquationalModel{} is usable for code
generation.

At the moment, there is no information attached to symbols yet regarding whether
they are abstract or instanced, so that portion of the rule is not enforced.

\eztodo{Example of an EquationalModel/QDefinition in Haskell code, the SRS, and
    the generated code.}


\chapter{Expression Creation and Usage}
\label{chap:typedExpr}
\chapter{Typing the Expression Language}
\label{chap:typedExpr}

Mathematical expressions are one of the most prominent components of any
abstracted concept from scientific software artifacts. With a pencil and paper,
our mistakes might go unnoticed, because they are never ``hard'' validated by
any machine, but ``soft'' validated by us and other readers. In other words,
there is no clear \textit{validity assertion} when we traditionally write
expressions on paper. While \Cref{chap:modelkinds} focused on understanding how
mathematical expressions could be dissected and transformed into ``code,'' it
neglected to discuss which expressions it could even begin to dissect and
transform \textemdash{} that is to say, which expressions are ``valid?'' The
objective of this chapter is to create a system of type rules, and enforce
well-formedness/typedness\footnote{Please note that I will be using
      ``well-formed'' and ``well-typed'' interchangeably.} through them, for Drasils
mathematical expression languages.

\section{Recap of Drasils Math-related Expression Languages}

To recap, at this point, we have three (3) relevant and used ``mathematical
expression'' languages.

\subsection{One for \textquotedblleft{}Simple\textquotedblright{} Mathematics}

\Expr{} is a mathematical expression languages whose vocabulary is intended to
always have a definite value. In other words, with little to no extra work on
your end, you should be able to directly input these expressions in your
standard calculator (perhaps with a bit of work to handle vectors, functions,
etc.) to evaluate them.

\subsection{One For \textquotedblleft{}Code\textquotedblright{}}

\CodeExpr{} is a heavily mathematics-focused expression language with a few
extra features over \Expr{} for \acs{gool}/``code.'' The vocabulary should be
nearly directly usable in \acs{gool} for outputting to general-purpose
object-oriented programming languages. \CodeExpr{} is a superset of \Expr{}:
\(\Expr{} \subseteq{} \CodeExpr{}\).

\subsection{And One For General Mathematics}

\ModelExpr{} is the classical mathematics we know and love. It contains nearly
everything we know (up to what we've encoded thus far) and is intended to be a
descriptive language, with no particular restrictions on its terms (other than
that they should at least be describable on pencil and paper too). \ModelExpr{}
is also a superset of \Expr{}: \(\Expr{} \subseteq{} \ModelExpr{}\).

However, \ModelExpr{}s terms are unlikely to appear in \CodeExpr{} due to their
indescribable nature in computable \acs{oo} ``code.''

As of right now, these languages have proven themselves to be effective
encodings, weakly proven through Drasils case studies being able to produce
working software artifacts. However, they are not without issue. Notably, at the
moment, Drasil does not have any readily-available type information about their
constructions. This lack of type information hampers the ``reliability'' aspect
of the code generator because the generator is unable to restrict its output
artifacts to those which are directly usable. In order for the generated
artifacts to be directly usable, they must be \textit{well-typed} programs. In
other words, we need to make sure that the generated expressions and programs
conform to the \textit{type rules} of the respective interpreters and/or
compilers.

\section{Type Safety}

Before compilation/execution, programming language compilers and interpreters
check input programs against a logical \textit{type system} to ensure that the
steps never perform invalid instructions, where program evaluation may become
impossible\footnote{Some instructions/operations may be ``nonsense!''}. For
example, to avoid nonsensical instructions, such as \(1 + \texttt{true}\), where
parameter \textit{type} mismatches occur. In this example, a computer (like us)
have no reasonable way to understand how to add \(1\) and the truth value
\texttt{true}. Type systems provide \textit{types} (such as
\(\mathbb{Z}\)/\texttt{integer} and \(\mathbb{B}\)/\texttt{boolean}) that are
assigned to each \textit{term} (such as \(1\), \(+\), and \texttt{true}), and a
set of \textit{typing rules} to restrict how terms interact and form. Types are
information about the structure of terms. \textit{Types} are typically
meta-level information. Terms are the value/``primitive'' data of a programming
language, such as numerics, functions/methods, operators, and modules. A series
of \textit{inference rules} makes up the type rules of a system. Here,
\textit{type safety} is approximately an assurance of \textit{preservation} and
\textit{progress} formed through the typing rules \cite{Harper2016}.

\intodo{Discuss ``preservation'' and ``progress,'' and how, together, they make
      up ``safety'' as discussed in \cite{Harper2016}. Also discuss what it
      means for things to be well-typed.}

\subsection{A Simple Language}

For example, if we had a small ``simple'' language, \textbf{L}, that contains
terms for integer and boolean values, and addition, ``less than'' comparison,
conjunction, and if-then-else (ternary operators), we might write out the typing
rules as follows:

\begin{equation}
      \left.
      \infer{n : \bb{Z}}{}
      \right.
      \qquad
      \text{(where \(n\) is any integer.)}
      \label{eq:exTR:int}
\end{equation}

\begin{equation}
      \left.
      \infer{\texttt{true} : \bb{B}}{}
      \right.
      \qquad
      \text{True}
      \label{eq:exTR:true}
\end{equation}

\begin{equation}
      \left.
      \infer{\texttt{false} : \bb{B}}{}
      \right.
      \qquad
      \text{False}
      \label{eq:exTR:false}
\end{equation}

\begin{equation}
      \left.
      \infer{(a\ \texttt{+}\ b) : \bb{Z}}
            {a : \bb{Z}  &  b : \bb{Z}}
      \right.
      \qquad
      \text{Addition}
      \label{eq:exTR:addition}
\end{equation}

\begin{equation}
      \left.
      \infer{(a\ \texttt{<}\ b) : \bb{B}}
            {a : \bb{Z}  &  b : \bb{Z}}
      \right.
      \qquad
      \text{``Less than'' comparison}
      \label{eq:exTR:lessThan}
\end{equation}

\begin{equation}
      \left.
      \infer{(a\ \texttt{\wedge}\ b) : \bb{B}}
            {a : \bb{B}  &  b : \bb{B}}
      \right.
      \qquad
      \text{Conjunction}
      \label{eq:exTR:conjunction}
\end{equation}

\begin{equation}
      \left.
      \infer{(\texttt{if}\ b\ \texttt{then}\ x\ \texttt{else}\ y) : \tau}
            {b : \bb{B}  &  x : \tau  &  y : \tau}
      \right.
      \qquad
      \text{if-then-else (ternary ``if'')}
      \label{eq:exTR:ifThenElse}
\end{equation}

\intodo{Continue writing here!}

\intodo{Discuss how we can add type information to Drasils expression languages,
      and the pros/cons of the solutions. For example, should type enforcement
      done at the construction-level or post-facto processed?}

\intodo{Discuss the properties of a good solution.}

\intodo{Re-write the typing rules without the Haskell code references (keep it
      mathematical/theoretical).}

\intodo{Re-write the typing rules with the Haskell code style.}

\intodo{Discuss our proposed and performed solution.}

\begin{enumerate}

      \item Ok, so we have these expression languages. They're great!

      \item However, they admit invalid expressions.

      \item Invalid expressions are bad because:

            \begin{enumerate}

                  \item they can't be properly dissected, for us to learn from
                        them.

                  \item they will cause problems later, in the artifacts we
                        generate.

                  \item they make no real discernible ``sense.'' They might make
                        sense to anyone in particular, but, broadly, they will
                        be gibberish, because they do not obey a globally agreed
                        upon rule set.

                  \item at the moment, to ensure that our generated artifacts
                        are directly usable in compilation and usage, we need to
                        manually ensure that they ``type-check.'' However,
                        Drasil has no clear understanding of what ``type-check''
                        means! For a few expressions, this manual methodology
                        might be ``okay,'' but it surely does not scale well
                        when add expression generation, nor human error through
                        scale in expression creation/usage.

            \end{enumerate}

      \item Ok, so what information are we missing? What do we need to ``teach''
            Drasil about in order for us to ensure that all of our expressions
            are well-formed (whatever that might mean)?

            \begin{enumerate}

                  \item First, we need to discuss: what does ``well-formedness''
                        mean?

                  \item What are ``typing rules?'', and how can we create a set
                        of typing rules?

                  \item What are our needed typing judgments?

            \end{enumerate}

      \item Now that we know what is needed of the expression language, how do
            we want to implement it in Drasil?

\end{enumerate}

\section{Background: Problem}

\begin{itemize}

      \item Writing invalid expressions is possible.
            \begin{itemize}

                  \item On paper, writing invalid expressions is as easy as
                        making a typo, but complete gibberish can also be
                        written. We rely on manually checking expressions to
                        ensure that they are ``correct''. As the number of
                        expressions grows, the cost of manually checking grows
                        rapidly, and changes result in costly setbacks. Imagine
                        systems with 10, 100, and 1000+ expressions, the cost
                        grows rapidly.

                  \item With computers, we can systematically check the validity
                        of expressions by imposing various kinds of
                        restrictions.

            \end{itemize}

      \item Mentally tracking expression creations to ensure they follow the
            implicit rules of the expression language is too difficult, and
            leads to mental strain.

      \item Compiling to ``lower languages'' requires special type checking
            before compiling to them. For example, the Swift code generator has
            to ensure that there are no ambiguously typed numerals as the types
            of numerics are not overloaded in Swift.

      \item Dynamically checking for invalid expression states is possible, but
            difficult and would result in increasingly difficult term tracking
            as terms in the expression language grow/are added.

      \item In general, being able to express invalid expressions causes large
            burden and mental overhead.

\end{itemize}


\section{Requirements \& properties of a good solution}

\begin{itemize}

      \item Invalid expressions should not be representable in the various
            expression languages (i.e., the expression types should strictly
            indicate valid expression constructions), without loss of
            generality.

      \item Invalid expression formation attempts should be statically found and
            reported by the compiler, at compile-time. This will move the
            previously runtime errors to compile-time.

      \item Invalid expression cases should not need to be considered when
            working (e.g., case-ing) with expressions.

      \item ``Safety = Preservation + Progress'' (\cite{Harper2016}, Ch.6)

\end{itemize}

\section{Solution}

\begin{itemize}

      \item Use TTF encodings of the smart constructors to lessen the cognitive
            load of handling at least 3 different expression languages.

      \item Statically type all 3 variants of Expr through GADTs.

\end{itemize}

\subsection{Syntax}

\subsubsection{Current}

An idealized version of the current syntax.

\startSyntaxTable
    \newsyntaxRow{Type}{\tau}{Integer}{\bb{Z}}{Integer numbers}
    \syntaxRow{Real}{\bb{R}}{Real numbers}
    \syntaxRow{String}{String}{Text}
    \syntaxRow{Bool}{\bb{B}}{Truth values (true/false)}
    \syntaxRow{Vector($\tau$)}{[\tau]}{Vectors}
     \\

    \newsyntaxRow{Literal}{l}{Integer[$n$]}{n}{Integer number}
    \syntaxRow{Real[$r$]}{r}{Real number}
    \syntaxRow{String[$s$]}{``s''}{Text}
    \syntaxRow{Bool[$b$]}{b}{Boolean value}
    \syntaxRow{Vector($l_1...l_n$)}{<l_1, ..., l_n>}{Vectors}
     \\

    \newsyntaxRow{UnaryOp}{u}{Not}{\lnot \_}{Logical negation}
    \syntaxRow{Neg}{- \_}{Numeric negation}
    \syntaxRow{...}{}{}
     \\

    \newsyntaxRow{BinaryOp}{\oplus}{Sub}{\_ - \_}{Subtraction}
    \syntaxRow{Pow}{\_^\_}{Powers}
    \syntaxRow{...}{}{}
     \\
    
    \newsyntaxRow{AssocBinOp}{\otimes}{Add}{\_ + \_}{Addition}
    \syntaxRow{Mul}{\_ \times \_}{Multiplication}
    \syntaxRow{...}{}{}
     \\

    \newsyntaxRow{Expr}{e}{Literal($l$)}{l}{Literal values}
    \syntaxRow{Vector($e_1...e_n$)}{<e_1, ..., e_n>}{Vectors}
    % TODO: Symbols : \syntaxRow{C(u)}
    % TODO: Function "Calls"
    \syntaxRow{UnaryOp($u$,$e$)}{u(e)}{Unary operations}
    \syntaxRow{BinaryOp($\oplus$,$e_1$,$e_2$)}{e_1 \oplus e_2}{Binary operations}
    \syntaxRow{AssocOp($\otimes$, $e_1...e_n$)}{e_1 \otimes ... \otimes e_n}{Associative binary operations}
    \syntaxRow{Case($e_{1c}e_{1e}...e_{nc}e_{ne}$)}{if\ e_{1c}\ then\ e_{2e}\ elif\ e_{2c}\ ...}{If-then-else-if-then-else (Switch-like statements)}
    % TODO: BigBinOp
    % TODO: RealI / Is In Interval

\closeSyntaxTable


\subsection{Typing Rules}

\subsubsection{Literal}

\begin{enumerate}

    \item Integers:
        \[ \infer{\ofTy{Integer[i]}{Literal Integer}}{\ofTy{i}{Integer}} \]

    \item Strings (Text):
        \[ \infer{\ofTy{Str[s]}{Literal String}}{\ofTy{s}{String}} \]

    \item Real numbers:
        \[ \infer{\ofTy{Dbl[d]}{Literal Real}}{\ofTy{d}{Double}} \]

    \item Whole numbered reals (\(\bb{Z} \subset{} \bb{R}\)): 
        \[ \infer{\ofTy{ExactDbl[d]}{Literal Real}}{\ofTy{d}{Integer}} \]

    \item Percentages:
        \[ \infer{\ofTy{Perc[n,d]}{Literal Real}}{\ofTy{n}{Integer} & \ofTy{d}{Integer}} \]

\end{enumerate}


\subsubsection{Miscellaneous}


\begin{enumerate}

    \item Completeness:
        \newrule{}
            {\ofTy{Complete[]}{Completeness}}
        
        \newrule{}
            {\ofTy{Incomplete[]}{Completeness}}

    \item AssocOp:
        \begin{enumerate}
            \item Numerics:
                \newrule{\numericTy{x}}
                    {\ofTy{Add[]}{AssocOp x}}
        
                \newrule{\numericTy{x}}
                    {\ofTy{Mul[]}{AssocOp x}}
    
            \item Bool:
                \newrule{}
                    {\ofTy{And[]}{AssocOp Bool}}
        
                \newrule{}
                    {\ofTy{Or[]}{AssocOp Bool}}
        \end{enumerate}

    \item UnaryOp:
        \begin{enumerate}
            \item Numerics:
                \todo{Discuss Numerics-($\Tau$) and Numerics-With-Negation-($\Tau$)}
                \newrule{\negNumericTy{x}}
                    {\ofTy{Neg[]}{UnaryOp x x}}

                \newrule{\negNumericTy{x}}
                    {\ofTy{Abs[]}{UnaryOp x x}}
                
                \newrule{\numericTy{x}}
                    {\ofTy{Exp[]}{UnaryOp x Real}}
                
                For Log, Ln, Sin, Cos, Tan, Sec, Csc, Cot, Arcsin, Arccos, Arctan, and Sqrt, please use the following template, replacing ``$\$TRG$'' with the desired operator:
                \newlblrule{}
                    {\ofTy{\$TRG[]}{UnaryOp Real Real}}{eqn:unOpTemplate}

                \newrule{}
                    {\ofTy{RtoI[]}{UnaryOp Real Integer}}
                
                \newrule{}
                    {\ofTy{ItoR[]}{UnaryOp Integer Real}}
            
                \newrule{}
                    {\ofTy{Floor[]}{UnaryOp Real Integer}}

                \newrule{}
                    {\ofTy{Ceil[]}{UnaryOp Real Integer}}
                
                \newrule{}
                    {\ofTy{Round[]}{UnaryOp Real Integer}}
                
                \newrule{}
                    {\ofTy{Trunc[]}{UnaryOp Real Integer}}
                
            \item Vectors:
                \newrule{\negNumericTy{x}}
                    {\ofTy{NegV[]}{UnaryOp [x] [x]}}

                \newrule{\numericTy{x}}
                    {\ofTy{Norm[]}{UnaryOp [x] Real}}

                \newrule{\ty{x}}
                    {\ofTy{Dim[]}{UnaryOp [x] Integer}}

            \item Booleans:
                \newrule{}
                    {\ofTy{Not[]}{UnaryOp Bool Bool}}

        \end{enumerate}

    \item BinaryOp:
        \begin{enumerate}
            \item Arithmetic: % TODO: Should we have FracI and FracR? 
                \newrule{}
                    {\ofTy{FracR[]}{BinaryOp Real Real Real}}

            \item Bool:
                \newrule{}
                    {\ofTy{Impl[]}{BinaryOp Bool Bool Bool}}
                
                \newrule{}
                    {\ofTy{Iff[]}{BinaryOp Bool Bool Bool}}
                
            \item Equality:
                \newrule{\ty{x}}
                    {\ofTy{Eq[]}{BinaryOp x x Bool}}
        
                \newrule{\ty{x}}
                    {\ofTy{NEq[]}{BinaryOp x x Bool}}
            
            \item Ordering:
                \newrule{\numericTy{x}}
                    {\ofTy{Lt[]}{BinaryOp x x Bool}}
        
                \newrule{\numericTy{x}}
                    {\ofTy{Gt[]}{BinaryOp x x Bool}}
            
                \newrule{\numericTy{x}}
                    {\ofTy{LEq[]}{BinaryOp x x Bool}}
        
                \newrule{\numericTy{x}}
                    {\ofTy{GEq[]}{BinaryOp x x Bool}}
            
            \item Indexing: % TODO: Everything related to vectors is up for debate, we can redesign them however we see fit. It shouldn't add much complexity.
                \newrule{\ty{x}}
                    {\ofTy{Index[]}{BinaryOp [x] Integer x}}
            
            \item Vectors: \todo{discuss vectors in general}
                \newrule{\numericTy{x}}
                    {\ofTy{Cross[]}{BinaryOp [x] [x] [x]}}
        
                \newrule{\numericTy{x}}
                    {\ofTy{Dot[]}{BinaryOp [x] [x] x}}
                
                \newrule{\numericTy{x}}
                    {\ofTy{Scale[]}{BinaryOp [x] x [x]}}

        \end{enumerate}
    
    \item RTopology:
        \newrule{}
            {\ofTy{Discrete[]}{RTopology}}

        \newrule{}
            {\ofTy{Continuous[]}{RTopology}}
    
    \item DomainDesc: % TODO: Why does the topology appear as a type constructor argument, and type signature argument?
        \newrule{\ofTy{top}{$\tau_1$} & \ofTy{bot}{$\tau_2$} & \ofTy{s}{Symbol} & \ofTy{rtop}{RTopology}}
            {\ofTy{BoundedDD[s, rtop, top, bot]}{DomainDesc Discrete $\tau_1$ $\tau_2$}}

        \newrule{\ty{topT} & \ty{botT} & \ofTy{s}{Symbol} & \ofTy{rtop}{RTopology}}
            {\ofTy{AllDD[s, rtop]}{DomainDesc Continuous topT botT}}

    \item Inclusive:
        \newrule{}
            {\ofTy{Inc[]}{Inclusive}}

        \newrule{}
            {\ofTy{Exc[]}{Inclusive}}

    \item RealInterval:
        \newrule{\ty{a} & \ty{b} & \ofTy{top}{(Inclusive, a)} & \ofTy{bot}{(Inclusive, b)}}
            {\ofTy{Bounded[top, bot]}{RealInterval a b}}

        \newrule{\ty{a} & \ty{b} & \ofTy{top}{(Inclusive, a)}}
            {\ofTy{UpTo[top]}{RealInterval a b}}

        \newrule{\ty{a} & \ty{b} & \ofTy{bot}{(Inclusive, b)}}
            {\ofTy{UpFrom[bot]}{RealInterval a b}}

\end{enumerate}


\subsubsection{Expr}

\begin{haskell}{Expression Language}{curExpr}{https://github.com/JacquesCarette/Drasil/blob/dc3674274edb00b1ae0d63e04ba03729e1dbc6f9/code/drasil-lang/lib/Language/Drasil/Expr/Lang.hs\#L81-L135}
-- | Expression language where all terms are supposed to be 'well understood'
--   (i.e., have a definite meaning). Right now, this coincides with
--   "having a definite value", but should not be restricted to that.
data Expr where
  -- | Brings a literal into the expression language.
  Lit :: Literal -> Expr
  -- | Takes an associative arithmetic operator with a list of expressions.
  AssocA   :: AssocArithOper -> [Expr] -> Expr
  -- | Takes an associative boolean operator with a list of expressions.
  AssocB   :: AssocBoolOper  -> [Expr] -> Expr
  -- | C stands for "Chunk", for referring to a chunk in an expression.
  --   Implicitly assumes that the chunk has a symbol.
  C        :: UID -> Expr
  -- | A function call accepts a list of parameters and a list of named parameters.
  --   For example
  --
  --   * F(x) is (FCall F [x] []).
  --   * F(x,y) would be (FCall F [x,y]).
  --   * F(x,n=y) would be (FCall F [x] [(n,y)]).
  FCall    :: UID -> [Expr] -> [(UID, Expr)] -> Expr
  -- | For multi-case expressions, each pair represents one case.
  Case     :: Completeness -> [(Expr, Relation)] -> Expr
  -- | Represents a matrix of expressions.
  Matrix   :: [[Expr]] -> Expr
  -- | Unary operation for most functions (eg. sin, cos, log, etc.).
  UnaryOp       :: UFunc -> Expr -> Expr
  -- | Unary operation for @Bool -> Bool@ operations.
  UnaryOpB      :: UFuncB -> Expr -> Expr
  -- | Unary operation for @Vector -> Vector@ operations.
  UnaryOpVV     :: UFuncVV -> Expr -> Expr
  -- | Unary operation for @Vector -> Number@ operations.
  UnaryOpVN     :: UFuncVN -> Expr -> Expr
  -- | Binary operator for arithmetic between expressions (fractional, power, and subtraction).
  ArithBinaryOp :: ArithBinOp -> Expr -> Expr -> Expr
  -- | Binary operator for boolean operators (implies, iff).
  BoolBinaryOp  :: BoolBinOp -> Expr -> Expr -> Expr
  -- | Binary operator for equality between expressions.
  EqBinaryOp    :: EqBinOp -> Expr -> Expr -> Expr
  -- | Binary operator for indexing two expressions.
  LABinaryOp    :: LABinOp -> Expr -> Expr -> Expr
  -- | Binary operator for ordering expressions (less than, greater than, etc.).
  OrdBinaryOp   :: OrdBinOp -> Expr -> Expr -> Expr
  -- | Binary operator for @Vector x Vector -> Vector@ operations (cross product).
  VVVBinaryOp   :: VVVBinOp -> Expr -> Expr -> Expr
  -- | Binary operator for @Vector x Vector -> Number@ operations (dot product).
  VVNBinaryOp   :: VVNBinOp -> Expr -> Expr -> Expr
  -- | Operators are generalized arithmetic operators over a 'DomainDesc'
  --   of an 'Expr'.  Could be called BigOp.
  --   ex: Summation is represented via 'Add' over a discrete domain.
  Operator :: AssocArithOper -> DiscreteDomainDesc Expr Expr -> Expr -> Expr
  -- | A different kind of 'IsIn'. A 'UID' is an element of an interval.
  RealI    :: UID -> RealInterval Expr Expr -> Expr
\end{haskell}


\subsubsection{ModelExpr}

\begin{haskell}{ModelExpr Language}{curModelExpr}{https://github.com/JacquesCarette/Drasil/blob/ab9e091dabd81685ddef86b0d218582c9f75cb20/code/drasil-lang/lib/Language/Drasil/ModelExpr/Lang.hs\#L82-L151}
-- | Expression language where all terms are supposed to have a meaning, but
--   that meaning may not be that of a definite value. For example,
--   specification expressions, especially with quantifiers, belong here.
data ModelExpr where
  -- | Brings a literal into the expression language.
  Lit       :: Literal -> ModelExpr
  
  -- | Introduce Space values into the expression language.
  Spc       :: Space -> ModelExpr
  
  -- | Takes an associative arithmetic operator with a list of expressions.
  AssocA    :: AssocArithOper -> [ModelExpr] -> ModelExpr
  -- | Takes an associative boolean operator with a list of expressions.
  AssocB    :: AssocBoolOper  -> [ModelExpr] -> ModelExpr
  -- | Derivative syntax is:
  --   Type ('Part'ial or 'Total') -> principal part of change -> with respect to
  --   For example: Deriv Part y x1 would be (dy/dx1).
  Deriv     :: Integer -> DerivType -> ModelExpr -> UID -> ModelExpr
  -- | C stands for "Chunk", for referring to a chunk in an expression.
  --   Implicitly assumes that the chunk has a symbol.
  C         :: UID -> ModelExpr
  -- | A function call accepts a list of parameters and a list of named parameters.
  --   For example
  --
  --   * F(x) is (FCall F [x] []).
  --   * F(x,y) would be (FCall F [x,y]).
  --   * F(x,n=y) would be (FCall F [x] [(n,y)]).
  FCall     :: UID -> [ModelExpr] -> [(UID, ModelExpr)] -> ModelExpr
  -- | For multi-case expressions, each pair represents one case.
  Case      :: Completeness -> [(ModelExpr, ModelExpr)] -> ModelExpr
  -- | Represents a matrix of expressions.
  Matrix    :: [[ModelExpr]] -> ModelExpr
  
  -- | Unary operation for most functions (eg. sin, cos, log, etc.).
  UnaryOp       :: UFunc -> ModelExpr -> ModelExpr
  -- | Unary operation for @Bool -> Bool@ operations.
  UnaryOpB      :: UFuncB -> ModelExpr -> ModelExpr
  -- | Unary operation for @Vector -> Vector@ operations.
  UnaryOpVV     :: UFuncVV -> ModelExpr -> ModelExpr
  -- | Unary operation for @Vector -> Number@ operations.
  UnaryOpVN     :: UFuncVN -> ModelExpr -> ModelExpr
  
  
  -- | Binary operator for arithmetic between expressions (fractional, power, and subtraction).
  ArithBinaryOp :: ArithBinOp -> ModelExpr -> ModelExpr -> ModelExpr
  -- | Binary operator for boolean operators (implies, iff).
  BoolBinaryOp  :: BoolBinOp -> ModelExpr -> ModelExpr -> ModelExpr
  -- | Binary operator for equality between expressions.
  EqBinaryOp    :: EqBinOp -> ModelExpr -> ModelExpr -> ModelExpr
  -- | Binary operator for indexing two expressions.
  LABinaryOp    :: LABinOp -> ModelExpr -> ModelExpr -> ModelExpr
  -- | Binary operator for ordering expressions (less than, greater than, etc.).
  OrdBinaryOp   :: OrdBinOp -> ModelExpr -> ModelExpr -> ModelExpr
  -- | Space-related binary operations.
  SpaceBinaryOp :: SpaceBinOp -> ModelExpr -> ModelExpr -> ModelExpr
  -- | Statement-related binary operations.
  StatBinaryOp  :: StatBinOp -> ModelExpr -> ModelExpr -> ModelExpr
  -- | Binary operator for @Vector x Vector -> Vector@ operations (cross product).
  VVVBinaryOp   :: VVVBinOp -> ModelExpr -> ModelExpr -> ModelExpr
  -- | Binary operator for @Vector x Vector -> Number@ operations (dot product).
  VVNBinaryOp   :: VVNBinOp -> ModelExpr -> ModelExpr -> ModelExpr
  
  
  -- | Operators are generalized arithmetic operators over a 'DomainDesc'
  --   of an 'Expr'.  Could be called BigOp.
  --   ex: Summation is represented via 'Add' over a discrete domain.
  Operator :: AssocArithOper -> DomainDesc t ModelExpr ModelExpr -> ModelExpr -> ModelExpr
  -- | A different kind of 'IsIn'. A 'UID' is an element of an interval.
  RealI    :: UID -> RealInterval ModelExpr ModelExpr -> ModelExpr
  
  -- | Universal quantification
  ForAll   :: UID -> Space -> ModelExpr -> ModelExpr
\end{haskell}


\subsubsection{CodeExpr}

\begin{enumerate}

    \item \[ \infer{A}{B & C} \]

\end{enumerate}



\chapter{Knowledge Management}
\label{chap:knowledgeMgmt}
\begin{itemize}
    \item UIDs
          \intodo{Expand}
    \item Multiple maps from UIDs to single types
          \intodo{Expand}
    \item Problems occur:
          \begin{itemize}
              \item UID collisions
              \item Difficult to ascertain what a specific chunk type is from a
                    UID
              \item ``ChunkDB'' is not a stable core across Drasil-like projects
                    (ones that thrive on the same ``knowledge-based programming''
                    ideology). ChunkDBs are essentially the ``scope'' of a system.
          \end{itemize}
    \item Solution:
          \begin{itemize}
              \item Merge the maps!
              \item The key would be the same; a UID.
              \item The value type?
              \item An existentially quantified Data.Typeable.Typeable!
              \item e.g., ``data Chunk = forall a. Typeable => Chunk a''
          \end{itemize}
    \item But wait! We're missing a few things from chunks:
          \begin{itemize}
              \item What knowledge does the chunk rely on already having been
                    ``registered'' in the database and ready?
              \item They should have UIDs; where's our guarantee?
              \item Debugging will be difficult; need an interface to dump all
                    information of a chunk quickly.
          \end{itemize}
    \item Ok, revise: ``data Chunk = forall a. (Typeable a, HasUID a,
          HasChunkRefs a, Dumpable a) => Chunk a''
    \item Ok, much better now.
    \item Or is it? Still many problems!
          \begin{itemize}
              \item How do we explain ``Data.Typeable''?
              \item And ``HasUID''?
              \item And ``HasChunkRefs''?
              \item And ``Dumpable''?
          \end{itemize}
    \item Well, at the very least, now we're able to merge the ``chunk'' maps and
          fix many of the pre-existing problems (we're almost there!). However,
          now we're relying too much on Haskell. How do we explain those parts?
    \item  \intodo{Also, what are UIDs? TBD!}
\end{itemize}

\section{Future Work}

\subsection{Encodings}

\begin{itemize}
    \item With the above new definition of ``chunks'', they still remain a very
          vague idea, and still \textit{deeply embedded} (a place to recognize an
          encoding might be appropriate!) in Haskell.
    \item What are the kinds of chunks that can exist? What can be in a chunk,
          and what are we missing from the existing list of chunks?
    \item The problem with that is that we lose a lot of information by writing
          Haskell, and leaving the knowledge in the form of Haskell.
    \item We need to de-embed all chunks so that we can obtain a tangible
          understanding of them.
    \item Through de-embedding the chunks, we will also be forced to de-embed
          everything with it. This is including the ways in which we transform
          and generate ``new''-ish knowledge (not necessarily new types/kinds of
          knowledge, but new instances of types).
\end{itemize}


\chapter{Future Work}
\label{chap:futureWork}
\chapter{Future Work}
\label{chap:futureWork}

\intodo{Rewrite point form notes in Future Work chapter.}

\begin{itemize}

      \item This document will contain knowledge regarding the Expression language
            that is shown in Haskell code, but not quite in our encodings. To
            further improve Drasil, one of the next ``obvious'' steps is to
            transcribe the knowledge involved with writing any language down as
            well. An excruciating amount of knowledge is everywhere.

      \item The unit and dimension related to numbers is another project on its
            own. It will need to be added to calculate the units of operations and
            ensure that representations are appropriate for their (precision vs
            accuracy as Dr. Smith mentioned).

\end{itemize}


\chapter{Conclusion}
\label{chap:conclusion}
\intodo{Write a conclusion chapter.}


%------------------------------------------------------------------------------
% Bibliography
%------------------------------------------------------------------------------
\printbibliography[heading=bibintoc]

%------------------------------------------------------------------------------
% Appendix
%------------------------------------------------------------------------------

\backmatter

\appendix
\chapter{Appendix}
\originalRelation{}

\originalFewExprSmartConstructorsHaskell{}

\currentExprTTFHaskell{}

\currentModelExprHaskell{}

\currentModelExprTTFHaskell{}




\end{document}
