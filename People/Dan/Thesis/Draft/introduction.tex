\chapter{Introduction}
\ds{Introduce the term softifact somewhere in here}
Documentation is good\citep{??}, yet it is not often prioritized on software
projects. Code and other software artifacts say the same thing, but to different
audiences - if they didn't, they would be describing different systems.

Take, for example, a software requirements document. It is a human-readable
abstraction of *what* the software is supposed to do. Whereas a design document
is a human-readable version of *how* the software is supposed to fulfill its
requirements. The source code itself is a computer-readable list of instructions
combining *what* must be done and, in many languages, *how* that is to be
accomplished.

[Put in figures of an example from GlassBR/Projectile here, showing SRS, DD, and
code versions of the same knowledge]

[Figure] shows an example of the same information represented in several
different views (requirements, detailed design, and source code). We aim to
take advantage of the inherent redundancy across these views to distill a single
source of information, thus removing the need to manually duplicate information
across software artifacts.

Manually writing and maintaining a full range of software artifacts (i.e.
multiple documents for different audiences plus the source code) is
redundant and tedious. Factor in deadlines, changing requirements, and other
common issues faced during development and you have a perfect storm for
inter-artifact syncronization issues.

How can we avoid having our artifacts fall out of sync with each other?
Some would argue "just write code!" And that is exactly what a number of other
approaches have tried. Documentation generators like Doxygen, Javadoc, Pandoc,
and more take a code-centric view of the problem. Typically, they work by having
natural-language descriptions and/or explanations written as specially delimited
comments in the code which are later automatically compiled into a
human-readable document.

While these approaches definitely have their place and can come in quite handy,
they do not solve the underlying redundancy problem. The developers are still
forced to manually write descriptions of systems in both code and comments.
They also do not generate all software artifacts - commonly they are used to
generate only API documentation targeted towards developers or user manuals.

We propose a new framework, Drasil, alongside a knowledge-centric view of
software, to help take advantage of inherent redundancy, while avoiding manual
duplication and synchronization problems. Our approach looks at what underlies
the problems we solve using software and capturing that "common" or "core"
knowledge. We then use that knowledge to generate our software artifacts, thus
gaining the benefits inherent to the generation process: lack of manual
duplication, one source to maintain, and 'free' traceability of information.

\section{Value in the mundane}
??

\section{Scope}
We are well aware of the ambitious nature of attempting to solve the problem of
manual duplication and unnecessary redundancy across all possible software
systems. Frankly, it would be highly impractical to attempt to solve such a
broad spectrum of problems. Each software domain poses its own challenges,
alongside specific benefits and drawbacks. 

Our work on Drasil is most relevant to software that is well-understood and 
undergoes frequent change (maintenance). Good candidates for development using
Drasil are long-lived (10+ years) software projects with artifacts of interest
to multiple stakeholders. With that in mind, we have decided to focus on
scientific computing (SC) software. Specifically, we are looking at software 
that follows the pattern 'input -> process -> output'.

SC software has a strong fundamental underpinning of well-understood concepts.
It also has the benefit of seldomly changing, and when it does, existing models
are not necessarily invalidated. For example, rigid-body problems in physics are
well-understood and the underlying modeling equations are unlikely to change.
However, should they change, the current models will likely remain as good
approximations under a specific set of assumptions. For instance, who hasn't
heard 'assume each body is a sphere' during a physics lecture?

\section{Roadmap}

\section{Contributions \& Publications}

\ds{After so much time working here, I think I've finally realized one of the 
true contributions of this thesis/Drasil. Not only the framework itself 
(which is still awesome), but also the process of breaking everything down and 
truly understanding softifacts at a deep level to operationalize our 
understanding of SE / system design in a way that makes all of this generation 
possible. With that in mind Drasil is just one means to that end.}
