\chapter{Introduction}
\label{chap:introduction}

\begin{writingdirectives}

      \item \textit{Based on the content of a video by Dr. Cecile Badenhorst
            (\url{https://www.youtube.com/watch?v=c2oGY1c51jc}) and a post about
            writing introductions by UNSW:
            \url{https://www.student.unsw.edu.au/introductions}}

      \item Move 1: Establishing a research territory by:
      \begin{itemize}

            \item showing research area is important, interesting, and
                  incomplete

            \item reviewing previous research

      \end{itemize}

      \item Move 2: Establishing a niche by noting gaps in previous research.

      \item Move 3: Occupying the niche by:
      \begin{itemize}

            \item outlining purpose

            \item listing research questions

            \item announcing principal findings

            \item stating the value of the previous research

      \end{itemize}

      \item General Structure:
      \begin{itemize}

            \item Introduction:
                  \begin{itemize}

                        \item Jazzy information to get reader hooked

                        \item States purpose of chapter

                        \item Roadmap of what will be discussed in chapter

                  \end{itemize}

            \item Background: context of research problem, sets up the need for
                  research and relevance

            \item PPSQ: should be within first 3 pages of thesis, after intro +
                  background information.

            \item Research design and context: description of where the research
                  takes place (Drasil), introducing methodology briefly

            \item Assumptions, limitations, scope of research, and expected
                  outcomes: what do we need from this work

            \item Overview of chapters

      \end{itemize}

      \item Last Paragraph: summarize key points of chapter, link to next
      chapter

      \item What is the context of this research? What is it about?

      \item What problem does this research tackle?

      \item Why is the research problem important/significant?

      \item What previous research exists?

      \item What is the purpose of this research? What are the goals?

      \item What did the author contribute?
      \wqanswer{\Cref{sec:intro:contributions}}

      \item What does this thesis contain? \wqanswer{\Cref{sec:intro:outline}}

\end{writingdirectives}

The usual means of building software involves multiple artifacts (such as
specifications and code) that contain duplicate information that is also
supposed to be linked (\textit{traceability}). Drasil aims to use a generative
approach to de-duplicate this information and make traceability more immediate.
Drasil currently uses a stable scientific knowledge-base to generate families of
\ACF{scs} conforming to a \ACF{srs}\cite{SmithAndLai2005}.

\iffalse
      % Potential replacement for the above paragraph
      The usual means of building software involves multiple artifacts (such as
      specifications and code) that contain duplicate information that is
      supposed to be linked (\textit{traceability}). Drasil aims to capture the
      information and knowledge necessary to create these artifacts, and
      re-generate the artifacts, making traceability more immediate.

      % A less-specific introduction
      A common thread of knowledge links together all software artifacts.
      Software developers pull on this thread to produce software artifacts, but
      the thread is lost on the authors, isn't effectively shared, and artifacts
      quickly becomes out of date as more is understood or requirements changed.
      Typically, this knowledge is duplicated across many artifacts (such as
      specifications and code). Through codifying domain knowledge,
      Drasil\cite{Drasil2021} aims to use a generative approach to
      \textit{de-duplicate} this information, and generate software artifacts
      that are \textit{traceable}, \textit{reusable}, \textit{maintainable}, and
      \textit{consistent}, against a stable body of knowledge. Drasil currently
      focuses on \ACF{scs} conforming to a specific \ACF{srs}
      template\cite{SmithAndLai2005}.
\fi

We will explore capturing mathematical theories frequently pulled from in
building \acs{scs}, make mathematical expressions typed, restrict expression
usage to appropriate contexts, and scale Drasils knowledge database.

\section{Problem Statement}
\label{sec:intro:problemStatement}

Drasil has de-duplicated knowledge across \acs{scs} artifacts relevant to
specifications and code. Through codifying knowledge and creating coherent
``stories'', we are able to generate a wide variety of software artifacts (e.g.,
\acs{oo} programs via \acs{gool} with guided usage via Makefiles, and
requirements specifications [HTML and TeX]). This codified knowledge was
de-duplicated from an originating set of artifacts via bottom-up gathering,
however, we should be able to use the same knowledge to generate more artifacts
in different languages, flavours, and with more options. However, each desired
artifact language has its own way of encoding information (such as mathematical
expressions). This leaves us needing to teach Drasil more about the targeted
languages (and, at times, about the existing codified knowledge) in order to
reliably generate usable artifacts. Mathematical expression and theory encoding
becomes a key point of interest for us because they are used in across the board
(e.g., derivations, code, constraints, definitions, etc.). Drasil relies on a
single universal untyped mathematical language to describe general mathematical
knowledge (theories). As a result, transforming theories into other forms (such
as code) is complex, inflexible, and unreliable because the encodings lack
information about the meaningful structure of the theories and how and when they
are usable or valid. Continuing, as we encode more \textit{types} of information
in Drasil, we face difficulties in placing parameterized types into Drasils
active knowledge database and need it to scale.

\iffalse
      As more theories are codified and typed, Drasils knowledge database faces
      difficulties in scaling since it relies on a single unique map for each
      type of knowledge, resulting in an ever-growing list of maps and a
      tediously precise means of knowledge collection and reference.
\fi

\section{Research Questions}
\label{sec:intro:researchquestions}

\begin{enumerate}

      \item[\namedlabel{rq:one}{RQ1}] Drasil has a language of simple
            mathematical expressions that are used in multiple contexts. But not
            all expressions are valid in all contexts. How do we fix that?

      \item[\namedlabel{rq:two}{RQ2}] Drasil's current encoding of ``theories''
            are essentially black boxes. We would like to be able to use some
            structural information present in the short list of the ``kinds'' of
            theories that show up in scientific computing. How do we codify
            that?

      \item[\namedlabel{rq:three}{RQ3}] How can we ensure that our language(s)
            of simple mathematical expressions admits only valid expressions?

      \item[\namedlabel{rq:four}{RQ4}] Our current ``typed'' approach to
            collecting different kinds of data is hard to extend. How can we
            make it easier to extend?

\end{enumerate}

\section{Contributions of the Author}
\label{sec:intro:contributions}

Drasil has existed since 2014, and has already seen success in its case studies,
which are used to guide the development of Drasil. Drasils focus on \acs{scs}
relies on knowledge of mathematical theories and language, for which Drasil has
a working understanding of before this work. However, some case studies were
unable to participate in code generation due to a lack of flexible theory
information (\ref{rq:two}), or just being inapplicable. This work contributes to
structuring theory information and allowing for future developers to encode more
kinds of theories and their relationships with other things (discussed in
\Cref{chap:modelkinds}). The solution builds on a prototype by Dr. Jacques
Carette\todo{Cite Dr. Carettes ModelKinds prototype.} that facilitates
structured theories to define relationships between ``code'' and ``theories.''

Theories rely on mathematical expressions as well. We commonly differ the usable
set of language in different contexts (you are free to write a lot more on your
pencil and paper derivations than on your typical calculator). To obtain
information about the expressibility in different contexts, we divide the
expression language using a \acs{ttf} \cite{Carette2009} encoding, with a
\acsp{gadt} backend for structural edits (\Cref{chap:modelkinds}). However,
``expressibility'' also relies on the expressions adhering to a precise
syntactic set of rules. As such, we build a system of typing rules for the
expression language (\Cref{chap:typedExpr}).

Finally, to enable capturing data with type parameters and generally scale
Drasils knowledge database (\ref{rq:four}), this work merges the typed database
collections into a single untyped, yet type-preserving, database (discussed in
\Cref{chap:storingChunks}).

\intodo{I will reference at least two major points of time in the Drasil source
      code repository (measured by their hash). The ``original'' code refers to
      the code as it was written before I was onboarded to Drasil. The
      ``current'' code might include changes from others from the time I've
      started working on Drasil, including, but not limited to, code formatting,
      code commenting, and extensions.}

\section{Thesis Outline}
\label{sec:intro:outline}

In \Cref{chap:ideology}, we discuss the focal ideology underpinning this work
and Drasil. \Cref{chap:drasil} describes Drasil, the host project carrying the
fruits of this work. \Cref{chap:modelkinds} discusses how theories are encoded
in Drasil (\ref{rq:two}), the issues associated with using a single universal
mathematical language to describe theories (\ref{rq:one}), and how we can
resolve these problems. \Cref{chap:typedExpr} describes issues associated with
the formation of mathematical expressions (\ref{rq:three}).
\Cref{chap:storingChunks} focuses on how Drasil stores information, and how it
can be scaled.
