\documentclass[10pt,twoside,onecolumn,openany,letterpaper]{memoir}
%\usepackage{createspace}
%\usepackage[size=pocket,noicc]{createspace}
\usepackage[paperwidth=7.50in, paperheight=10in]{geometry}
\usepackage[T1]{fontenc}
%\usepackage[latin1]{inputenc}
\usepackage[english]{babel}
%\usepackage{tgtermes}

\usepackage{newpxtext}
\usepackage{newpxmath}
\usepackage[protrusion=true,expansion=true]{microtype}

\begin{document}

\title{Generate Everything!}
%\orcid{0000-0001-8993-9804}
%\affiliation{\department{Computing and Software}
%  \institution{McMaster University}
%  \country{Canada}
%}
%\email{carette@mcmaster.ca}
\author{Jacques Carette \and Spencer Smith}
%\email{smiths@mcmaster.ca}

%\begin{abstract}
%We generate it all. What's all? Indeed. How? Neat.
%\end{abstract}

%\keywords{code generation, document generation, knowledge capture,
%  software engineering, scientific software}

\maketitle

\chapter{Introduction}

Things to remember:
\begin{itemize}
\item Generate Everything: What could it mean?
\item What is Software?
\item scope: all scientific software
\item \textbf{all} artifacts (enumerate them!)
\item current process for producing artifacts has lots of embedded (tacit)
  knowledge which must be captured
\item bottom-up discovery -- make sure to always have a working system on hand
\end{itemize}

Good quotes:
\begin{itemize}
\item metaprograms are just programs
\item models outside an integrated toolchain are insufficiently useful
\end{itemize}

Somewhere in there will emerge some kind of idealized development process.

\chapter{Generate?}
\chapter{Everything?}
\chapter{Generate!}
\chapter{Everything!}
\chapter{Conclusion}

\end{document}
