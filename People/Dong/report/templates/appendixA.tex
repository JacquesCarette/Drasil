\chapter{Your Appendix}
\label{appendix_a}

Your appendix goes here.
