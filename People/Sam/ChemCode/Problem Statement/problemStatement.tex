\documentclass{article}

\usepackage{tabularx}
\usepackage{booktabs}

\title{Problem Statement and Goals\\\progname}

\author{\authname}

\date{January 20, 2023}

%% Comments

\usepackage{color}

\newif\ifcomments\commentstrue

\ifcomments
\newcommand{\authornote}[3]{\textcolor{#1}{[#3 ---#2]}}
\newcommand{\todo}[1]{\textcolor{red}{[TODO: #1]}}
\else
\newcommand{\authornote}[3]{}
\newcommand{\todo}[1]{}
\fi

\newcommand{\wss}[1]{\authornote{blue}{SS}{#1}}
\newcommand{\an}[1]{\authornote{magenta}{Author}{#1}}

%% Common Parts

\newcommand{\progname}{ChemCode} 
\newcommand{\authname}{Samuel Crawford}

\usepackage{hyperref}
    \hypersetup{colorlinks=true, linkcolor=blue, citecolor=blue, filecolor=blue,
                urlcolor=blue, unicode=false}
    \urlstyle{same}
                                

\begin{document}

\maketitle

\begin{table}[hp]
\caption{Revision History} \label{TblRevisionHistory}
\begin{tabularx}{\textwidth}{llX}
\toprule
\textbf{Date} & \textbf{Developer(s)} & \textbf{Change}\\
\midrule
Jan. 18, 2023 & Sam & Create document and fill in Problem, Stakeholders, and
Goals sections\\
Jan. 19, 2023 & Sam & Format for Drasil upload\\
\bottomrule
\end{tabularx}
\end{table}

\section{Problem Statement}

\wss{You should check your problem statement with the
\href{https://github.com/smiths/capTemplate/blob/main/docs/Checklists/ProbState-Checklist.pdf}
{problem statement checklist}.}
\wss{You can change the section headings, as long as you include the required
information.}

\subsection{Problem}

Chemistry is a broad field that studies matter and its interactions
\cite[pp.~8-9]{gordon_chm101_2023}, primarily through chemical reactions. During
a chemical reaction, bonds between some substances break and new ones are formed
to create new substances; these reactions are often written as chemical
equations \cite{lund_introduction_2023}. Despite new chemicals being created,
all atoms from the initial substances, or ``reactants", must be present in the
final substances, or ``products" because of the Law of Conservation of Matter
\cite{lund_introduction_2023}. This means that for a chemical equation to be
useful, it must be balanced by changing the coefficients of the substances
involved in the reaction \cite{lund_introduction_2023}. Additionally, since
molecules only exist in whole numbers (since dividing a molecule changes its
composition into new types of molecules), these coefficients must be whole
numbers, and by convention should be as small as possible
\cite{lund_introduction_2023}.

While these equations can be balanced by hand through the process of ``balancing
by inspection" \cite{lund_introduction_2023}, this can be time-consuming, prone
to error, and inefficient, especially for more complicated chemical reactions.
For each element present in the reaction, an equality can be written for the
number of elements in each substance, with the reactants on one side and the
products on the other, using the coefficients of each substance as the variables
\cite{hamid_balancing_2019}. These equalities then form a system of linear
equations that can be solved through various methods to yield a relation between
each coefficient, which can then be manipulated to find the require whole
numbers \cite{lund_introduction_2023, hamid_balancing_2019}. This method can
also identify reactions that are ``infeasible" and balance reactions involving
fractional oxidation states \cite{hamid_balancing_2019}, which ``are used to
describe the distribution of electrons in a molecule"
\cite{unacademy_fractional_2023}.

\subsection{Inputs and Outputs}

\wss{Characterize the problem in terms of ``high level'' inputs and outputs.  
Use abstraction so that you can avoid details.}

\subsection{Stakeholders}
The main stakeholder of this project is Dr.~Spencer Smith, the instructor for
the CAS 741 Development of Scientific Computing Software course for which this
project is being completed. Dr.~Smith and Dr.~Jacques Carette are in charge of
the Drasil project that \progname~seeks to extend, so the implementation and
process of getting there are of significance to them. Likewise, any future
developers of Drasil, including myself \sjc{Can I use the first person?}, are
potential stakeholders of this project, since they may use features added to
Drasil, such as ideas about chemistry or systems of linear equations. Jason
Balaci, another CAS 741 and Drasil contributor, is of particular mention, since
there may be some overlap between our project, so we may be collaborating
throughout this project.

More generally, anyone in the field of chemistry in at least a high-school level
may be a stakeholder of this project, as they may use this tool in their work.

\subsection{Environment}

\wss{Hardware and software}

\section{Goals}

The goals of this project are to develop a program that:

\begin{itemize}
    \item Can balance chemical equations using systems of linear equations.
\end{itemize}

\section{Stretch Goals}

\bibliographystyle{ieeetr}
\bibliography{sources}

\end{document}