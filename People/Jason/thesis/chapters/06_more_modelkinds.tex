\chapter{More Theory Kinds}
\label{chap:more-theory-kinds}

\imptodo{Continue writing here!}









\section{\textquotedblleft{}Classify All The Theories\textquotedblright{}}

\subsection{Constraints}

\currentConstraintSetHaskell{}

Theories that define expressions that constrain models in some way are defined
using \EquationalConstraints{}, which use \ConstraintSet{}s under the hood
(\refCurrentConstraintSetHaskell{})\todo{Describe what a ConstraintSet actually
	is.}.
	
At the moment, these are not used in code generation, and are pending design for
usage in code generation as the translation from them into software is unclear
and may be interpreted differently by readers.

This model can, and should, eventually also be usable in code generation once
there is interest in creating runtime assertions of symbols.

\eztodo{Example of an EquationalConstraints/ConstraintSet in Haskell code, and
	the SRS.}

\subsection{Definition Realms}

\currentDefiningExprHaskell{}

\currentMultiDefnHaskell{}

The Theory Models and General Definition Models are unrefined, and may contain
multiple ways for a particular theoretical symbol to be defined. This is
captured in Drasil by \EquationalRealm{}s: modelled after realms\qtodo{Cite Dr.
Carette and Yasmines paper?}.

This model kind is intended specifically for retaining information about
conscious choices made along the way to create Instance Models.

An \EquationalRealm{} is based on a \MultiDefn{}
(\refCurrentMultiDefnHaskell{}), and is intended for forming \QDefinition{}s
through choosing definitions (encoded as \DefiningExpr{}s,
\refCurrentDefiningExprHaskell{}), through refinements, from the \MultiDefn{}s.

\eztodo{Example of an EquationalRealm/MultiDefn in Haskell code, and the SRS.}

\subsection{Differential Equations}

\DEModel{} is the simple \RelationConcept{}-style capture of differential
equation-related theories, and only exists as a placeholder until all
differential equation models are converted into one of the other \ModelKinds{}.

For related solving of differential equations, \DEModel{} implicitly relies on a
developer writing a related \ODEInfo{}\todo{ref Current ODEInfo} packet which
the code generator can use to solve the system.

Actively being reconstructed in Drasil \cite{Chen2022MEng} \footnote{Shortly
after I implemented \ModelKind{}, Dong continued working on exploring
\DEModel{}s and created \NewDEModel{} as a result.}, \NewDEModel{} is the
replacement for \DEModel{}, aiming to expose more information about
well-understood differential equations and related areas.

\NewDEModel{} will eventually be renamed to \DEModel{} once all existing
\DEModel{} models have been upgraded.

\intodo{Example of a DEModel/NewDEModel and a RelationConcept in Haskell code,
	the SRS, and the related ODEInfo and code.}

\subsection{Leftovers}

\imptodo{Leftover model kinds.}

\section{Theories Undiscussed}

\ModelKinds{} is an enumeration of the currently handled model types.

Each ``model'' is meant to expose information to the Haskell compiler and for
other fragments of knowledge to make use of.

\ModelKinds{} is obviously an incomplete enumeration, and will grow as the need
for more kinds of models arises.













\section{Success}
\label{chap:more-theory-kinds:sec:success}

\ModelKind{} already has some success in enabling more theories to be encoded in
Drasil and used for code generation.

Dong Chens Master's involved exploring \acsp{ode} in Drasil and using them to
generate code that solves them \cite{Chen2022MEng}.

Through his work, we have already observed some success because more case
studies are already capable of generating code (see
\refCaseStudiesCodeTableAfterDongsWork{}).

Namely, \acs{dblpendulum} has working code generation for 4 languages (Python,
Java, C/C\(++\), and C\#), and \acs{pdcontroller} now also supports Java,
C/C\(++\), and C\# in addition to Python.

\caseStudiesCodeTableAfterDongsWork{}

