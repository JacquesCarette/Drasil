\chapter{Introduction}
\ds{Introduce the term softifact somewhere in here}

\ds{Give examples of what audience writing for - know classical software 
engineering}

\ds{Pain points and problems come first, then solution is docs, then problems 
with writing docs, then the rest of this.}
Documentation is good\citep{??}, yet it is not often prioritized on software
projects. Code and other software artifacts say the same thing, but to different
audiences - if they didn't, they would be describing different systems.

Take, for example, a software requirements document. It is a human-readable
abstraction of \textit{what} the software is supposed to do. Whereas a design 
document is a human-readable version of \textit{how} the software is supposed 
to fulfill its requirements. The source code itself is a computer-readable list 
of instructions combining \textit{what} must be done and, in many languages, 
\textit{how} that is to be accomplished.

[Put in figures of an example from \gb{}/Projectile here, showing SRS, DD, and
code versions of the same knowledge]

[Figure] shows an example of the same information represented in several
different views (requirements, detailed design, and source code). We aim to
take advantage of the inherent redundancy across these views to distill a single
source of information, thus removing the need to manually duplicate information
across software artifacts.

Manually writing and maintaining a full range of software artifacts (i.e.
multiple documents for different audiences plus the source code) is
redundant and tedious. Factor in deadlines, changing requirements, and other
common issues faced during development and you have a perfect storm for
inter-artifact syncronization issues.

How can we avoid having our artifacts fall out of sync with each other?
Some would argue ``just write code!" And that is exactly what a number of other
approaches have tried. Documentation generators like Doxygen, Javadoc, Pandoc,
and more take a code-centric view of the problem. Typically, they work by having
natural-language descriptions and/or explanations written as specially delimited
comments in the code which are later automatically compiled into a
human-readable document.

While these approaches definitely have their place and can come in quite handy,
they do not solve the underlying redundancy problem. The developers are still
forced to manually write descriptions of systems in both code and comments.
They also do not generate all software artifacts - commonly they are used to
generate only API documentation targeted towards developers or user manuals.

\ds{If we can truly understand the differences and commonalities between 
software artifacts, we can develop tools to capitalize on them and streamline 
the process of creating and maintaining software artifacts. For that we first 
need to break down common artifacts to develop an understanding of their 
contents, intended audience, and how they differ whether on intra- or 
inter-project scales}.

We propose a new framework, Drasil, alongside a knowledge-centric view of
software, to help take advantage of inherent redundancy, while avoiding 
\ds{/reducing} manual duplication and synchronization problems. Our approach 
looks at what underlies the problems we solve using software and capturing that 
``common" or ``core" knowledge. We then use that knowledge to generate our 
software artifacts, thus gaining the benefits inherent to the generation 
process: lack of manual duplication, one source to maintain, and 'free' 
traceability of information.

\section{Value in the mundane}
??
\ds{What we're doing is not big and flashy, instead it is focused on the tedium 
we deal with in our day-to-day lives as Software Engineers. We are focused on 
improving the development experience holistically. The small annoyances can be 
more frustrating than solving larger problems. Ex. The feeling of hours spent 
banging heads against walls for an off-by-one or a typo that didn't get caught, 
versus the sublime joy of finding a novel, interesting, scalable, and 
maintainable solution to a problem.}

\ds {More about the little pain points should go here - i.e. having to go 
manually update artifacts after modifying code, particularly in regulated 
industries}

\section{Problem Statement and Motivation}
Despite the recognized value of documentation and software artifacts (\sfs{}), 
the process of creating and maintaining them is fraught with redundancy, 
tedium, and potential risks of inconsistency. As software evolves, artifacts 
(requirements documents, design documents, source code, etc.) often fall out of 
sync, leading to errors, increased maintenance costs, and reduced trust in 
documentation. This thesis addresses the challenge: \emph{How can we 
systematically reduce unnecessary redundancy and improve consistency across 
software artifacts, especially in domains where correctness and traceability 
are critical?}

\section{Scope}
\label{sec:scope}
We are well aware of the ambitious nature of attempting to solve the problem of
manual duplication and unnecessary redundancy across all possible software
systems. Frankly, it would be highly impractical to attempt to solve such a
broad spectrum of problems. Each software domain poses its own challenges,
alongside specific benefits and drawbacks. 

Our work on Drasil is most relevant to software that is well-understood and 
undergoes frequent change (maintenance). Good candidates for development using
Drasil are long-lived (10+ years) software projects with artifacts of interest
to multiple stakeholders. With that in mind, we have decided to focus on
scientific computing (SC) software. Specifically, we are looking at software 
that follows the pattern \emph{input $\rightarrow$ process $\rightarrow$ 
output}.

SC software has a strong fundamental underpinning of well-understood concepts.
It also has the benefit of seldomly changing, and when it does, existing models
are not necessarily invalidated. For example, rigid-body problems in physics are
well-understood and the underlying modeling equations are unlikely to change.
However, should they change, the current models will likely remain as good
approximations under a specific set of assumptions. For instance, who hasn't
heard `assume each body is a sphere' during a physics lecture?

SC software could also benefit from buy-in to good software development 
practices as many SC software developers put the emphasis on science and not 
development~\citep{Kelly2007}. Rather than following rigid, process-heavy 
approaches deemed unfavourable~\citep{CarverEtAl2007}, developers of SC 
software choose to use knowledge acquisition driven~\citep{Kelly2015}, 
agile~\citep{Segal2005, CarverEtAl2007, AckroydEtAl2008, 
EasterbrookAndJohns2009}, or amethododical~\citep{Kelly2013} processes instead.

\section{Roadmap}

The remainder of this thesis is organized as follows:
\begin{itemize}
    \item \textbf{Chapter~\ref{c:background}: Background} reviews the 
    literature on software artifacts, reuse, literate programming, and 
    generative programming.
    \item \textbf{Chapter~\ref{c:process}: Our Process} details the methods 
    used to analyze artifacts and identify redundancy.
    \item \textbf{Chapter~\ref{c:drasil}: The Drasil Framework} presents the 
    design and implementation of Drasil.
    \item \textbf{Chapter~\ref{c:results}: Results} discusses the outcomes of 
    applying Drasil to case studies.
    \item \textbf{Chapter~\ref{c:future}: Future Work} outlines potential 
    extensions and open questions.
    \item \textbf{Chapter~\ref{c:conclusion}: Conclusion} summarizes the 
    findings and their implications.
\end{itemize}

\section{Contributions}

The main contributions of this thesis are:
\begin{itemize}
    \item A systematic analysis of redundancy and inconsistency in software 
    artifacts, with a focus on scientific computing.
    \item The design and implementation of the Drasil framework, which enables 
    knowledge-centric generation of multiple software artifacts from a single 
    source of truth.
    \item A demonstration of Drasil's effectiveness through reimplementation of 
    several case studies, highlighting improvements in consistency, 
    maintainability, and traceability.
    \item A discussion of the limitations, challenges, and future directions 
    for knowledge-centric software artifact generation.
\end{itemize}

\ds{After so much time working here, I think I've finally realized one of the 
true contributions of this thesis/Drasil. Not only the framework itself 
(which is still awesome), but also the process of breaking everything down and 
truly understanding softifacts at a deep level to operationalize our 
understanding of SE / system design in a way that makes all of this generation 
possible. With that in mind Drasil is just one means to that end.}

\ds{Minor point that came up in conversation: Parnas' paper on how/why to fake 
rational design. Our tool lets people fake it. It's all about change, there's 
no perfect understanding at the beginning and we need to change things on as we 
go. Drasil allows us to change everything to fake the rational design process 
at every step along the way.}

\ds{Continuous integration / refactoring are nothing new, but the way we used 
them ensured we were always at a steady-state where everything worked.}

\ds{Note that some of the code may not have been written by me directly, but 
was developed by the Drasil team}
