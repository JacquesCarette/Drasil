\documentclass[sigconf,review,anonymous=true]{acmart}

%% Copyright information
\setcopyright{acmcopyright} % TODO: copyright?
\copyrightyear{2022}
\acmYear{2022}
\acmDOI{10.1145/1122445.1122456} % TODO: DOI?

%% These commands are for a PROCEEDINGS abstract or paper.
\acmConference[Pittsburgh '22]{Pittsburgh '22: NIER - New Ideas and Emerging
Results (ICSE 2022)}{May 21--29, 2022}{Pittsburgh, PA, USA}
\acmBooktitle{Pittsburgh '22: NIER - New Ideas and Emerging Results (ICSE 2022),
May 21--29, 2022, Pittsburgh, PA, USA} \acmISBN{978-1-4503-XXXX-X/18/06}
% TODO: ISBN

%% Packages

% \usepackage[paperwidth=8.50in, paperheight=11in]{geometry} %% option clash
\usepackage[T1]{fontenc}
%\usepackage[latin1]{inputenc}


% \usepackage[protrusion=true,expansion=true]{microtype} %% option clash

% for inline lists, and label computation
\usepackage{calc}
\usepackage[inline,shortlabels]{enumitem}
% for nice internal links
\usepackage{hyperref}

\usepackage{listings}
\usepackage[framemethod=TikZ]{mdframed}

\mdfdefinestyle{MyFrame}{%
  linecolor=blue,
  outerlinewidth=2pt,
  roundcorner=10pt,
  innertopmargin=\baselineskip,
  innerbottommargin=\baselineskip,
  innerrightmargin=10pt,
  innerleftmargin=10pt
}

\newcommand{\CC}{C\nolinebreak\hspace{-.05em}\raisebox{.4ex}{\small\bf +}\nolinebreak\hspace{-.10em}\raisebox{.4ex}{\small\bf +}}

\makeatletter
\newcommand{\verbatimfont}[1]{\renewcommand{\verbatim@font}{\ttfamily#1}}
\makeatother


\newcommand{\heading}[1]{\noindent{\large {\textbf{#1}\ }}}
\begin{document}

\lstset{language=haskell, basicstyle=\scriptsize, breaklines=true,
  showspaces=false, showstringspaces=false, breakatwhitespace=true, texcl=true,
  escapeinside={\%*}{*)}}

\title{When Capturing Knowledge Improves Productivity}

\author{Jacques Carette}
\orcid{0000-0001-8993-9804}
\affiliation{
  \department{Computing and Software}
  \streetaddress{1280 Main Street West}
  \institution{McMaster University}
  \city{Hamilton}
  \state{Ontario}
  \postcode{L8S 4L8}
  \country{Canada}}
\email{carette@mcmaster.ca}

\author{Spencer Smith}
\orcid{0000-0002-0760-0987}
\affiliation{
  \department{Computing and Software}
  \streetaddress{1280 Main Street West}
  \institution{McMaster University}
  \city{Hamilton}
  \state{Ontario}
  \postcode{L8S 4L8}
  \country{Canada}}
\email{smiths@mcmaster.ca}

\author{Jason Balaci}
%\orcid{0000-0002-0760-0987}
\affiliation{
  \department{Computing and Software}
  \streetaddress{1280 Main Street West}
  \institution{McMaster University}
  \city{Hamilton}
  \state{Ontario}
  \postcode{L8S 4L8}
  \country{Canada}}
\email{balacij@mcmaster.ca}

\begin{abstract}
  Current software development is often quite code-centric and aimed at
  short-term deliverables, due to various contextual forces. We're interested in
  contexts where different forces are at play.  \textbf{Well understood domains}
  and \textbf{long-lived software} provide such an opportunity. By
  applying generative techniques, aggressive knowledge capture has the real
  potential to greatly increase long-term productivity.

  Key is to recognize that currently hand-written software artifacts contain
  considerable knowledge duplication. With proper tooling and appropriate
  codification of domain knowledge, greatly increasing productivity is
  feasible. We present an example of what this looks like, and just some of
  the benefits (reuse, traceability, change management) thus gained.
\end{abstract}


% Generator:   http://dl.acm.org/ccs.cfm
% TODO: This is a temporary CCSXML and \ccsdesc list
\begin{CCSXML}
<ccs2012>
   <concept>
       <concept_id>10011007.10011006.10011066.10011070</concept_id>
       <concept_desc>Software and its engineering~Application specific development environments</concept_desc>
       <concept_significance>300</concept_significance>
       </concept>
   <concept>
       <concept_id>10011007.10011074.10011075.10011076</concept_id>
       <concept_desc>Software and its engineering~Requirements analysis</concept_desc>
       <concept_significance>300</concept_significance>
       </concept>
   <concept>
       <concept_id>10011007.10011006.10011060.10011690</concept_id>
       <concept_desc>Software and its engineering~Specification languages</concept_desc>
       <concept_significance>300</concept_significance>
       </concept>
   <concept>
       <concept_id>10011007.10011074.10011092.10011782</concept_id>
       <concept_desc>Software and its engineering~Automatic programming</concept_desc>
       <concept_significance>500</concept_significance>
       </concept>
 </ccs2012>
\end{CCSXML}

\ccsdesc[300]{Software and its engineering~Application specific development environments}
\ccsdesc[300]{Software and its engineering~Requirements analysis}
\ccsdesc[300]{Software and its engineering~Specification languages}
\ccsdesc[500]{Software and its engineering~Automatic programming}
%% End of generated code

\keywords{code generation, document generation, knowledge capture,
  software engineering}

% set up math environment
\newtheorem{defn}{Definition}

\maketitle

\section{The Context}
\subsection{``Well understood'' Software?}\label{ch:wellUnderstood}

\begin{defn}
A software domain is \emph{well understood} if
\begin{enumerate}
\item its Domain Knowledge (DK) is codified,
\item the computational interpretation of the DK is clear, and
\item writing code to perform said computations is well understood.
\end{enumerate}
\end{defn}

By \emph{codified}, we mean that the knowledge exists in standard form in a
variety of textbooks. For example, many engineering domains use
ordinary differential equations as models, the quantities of
interest are known, given standard names and standard units. In other words,
standard vocabulary has been established over time and the body of knowledge is
uncontroversial.

We can refine these high level ideas, using the same numbering,
although the refinement should be
understood more holistically.
\begin{enumerate}
\item Models in the DK \emph{can be} written formally.
\item Models in the DK \emph{can be} turned into functional relations by
 existing mathematical steps.
\item Turning these functional relations into code is an understood
 transformation.
\end{enumerate}
Most importantly, the last two parts
deeply involve \emph{choices}: What quantities are considered inputs, outputs
and parameters to make the model functional? What programming language?  What
software architecture data-structures, algorithms, etc.?

In other words,
\emph{well understood} does not imply \emph{choice free}.  Writing a small 
script to move files could just as easily be done in Bash, Python or Haskell.
In all cases, assuming fluency, the author's job is straightforward because
the domain is well understood.

\subsection{Long-lived software?}
For us, long-lived software is software that is expected to be in continuous
use and evolution for $20$ or more years. The main characteristic of
such software is the \emph{expected turnover} of key staff. This means that
all tacit knowledge about the software will be lost over time if it is not
captured.

\subsection{Productivity?}
We adapt the standard definition of productivity \cite{Boehm1987}, where inputs are labour,
but adjust the outputs to be knowledge and user satisfaction, where user
satisfaction acts as a proxy for effective quality
%(see~\cite{SmithAndCarette2020arXiv} for more)
.
This explicit emphasis on all knowledge produced, rather than just the
operationalizable knowledge (aka code) 
implies that human-reusable knowledge, i.e.\ documentation, is crucial.
This is why the \emph{long-lived} context is important.

\subsection{Documentation}
Our definition of well understood also applies to \textbf{documentation} aimed
at humans.  Explicitly:
\begin{enumerate}
\item The meaning of the models is understood at a human-pedagogical
level, i.e.\ it is explainable.
\item Combining models is explainable. Thus \emph{transformers}
simultaneously operate on mathematical representations
and on explanations. This requires that English descriptions also be
captured in the same manner as the formal-mathematical knowledge.
\item Similarly, the \emph{transformers} that arise from making software
oriented decisions should be captured with a similar mechanism, and also include
English explanations.
\end{enumerate}

We dub these \emph{triform theories}, as a nod to \emph{biform
theories}~\cite{Farmer2007}. We couple 
\begin{enumerate*}
\item an axiomatic description,
\item a computational description, and
\item an English description
\end{enumerate*}
of a concept.

%\subsection{Choices}
%
%Different kinds of choices are embedded in the different kinds of knowledge.
%They can show up simply as \emph{parameters}, for example the acceleration due
%to the gravity constant associated with a planet. They can also shows up as
%different transformers, for example turning $F - m\cdot a = 0$ into $F\left(m,
%a\right) = m\cdot a$, i.e. from a conservation law into a computation.
%For motion computation that same conservation law is often rewritten as
%$a\left(m,F\right) = F/m$ as part of solving $a = \ddot{x}$ to obtain a position
%($x$) as a function of time ($t$). We also get choices of phrasing, which are
%equivalent but may be more adequate in a given context.

\subsection{Softifacts}

Software currently consists of a whole host of artifacts: requirements,
specifications, user manual, unit tests, system tests, usability tests,
build scripts, READMEs, license documents, process documents, as well as
code. We use the word \emph{softifacts} for this collection.

Whenever appropriate, we use standards and templates for each of the
generated artifacts. For requirements, we use a variant~\cite{SmithEtAl2007}
%~\cite{SmithAndLai2005,
of the IEEE~\cite{IEEE1998}
and Volere templates~\cite{RobertsonAndRobertson1999Vol}.

\subsection{Examples of context}

When are these conditions fulfilled? One example is
\emph{research software} in science and engineering. While the results of
running various simulations is entirely new, the underlying models and
how to simulate them are indeed well-known. One particularly long-lived
example is embedded software for space probes (like Pioneer 10).

\section{A New Development Process}\label{ch:process}

Given appropriate infrastructure, what would be an \emph{idealized process}
(akin to Parnas' ideas of faking a rational design process \cite{Parnas1986})?

\begin{enumerate}
\item\label{it:problem} Have a task to achieve where
\emph{software} can play a central part in the solution.
\item\label{it:understood} The underlying problem domain
is \emph{well understood}.
\item\label{it:probdesc} Describe the problem:
  \begin{enumerate}
  \item Find the base knowledge (theory) in the pre-existing library
    or, failing that, write it if it does not yet exist,
    for instance the naturally occurring known quantities and
    associated constraints.
  \item Assemble the ingredients into a coherent narrative,
  \item Describe the characteristics of a good solution,
  \item Come up with basic examples (to test correctness, intuitions, etc).
  \end{enumerate}
\item\label{it:refine} Describe, by successive refinement transformations, how
the above can be turned into a
deterministic\footnote{A current meta-design choice.}
input-output process.
  \begin{enumerate}
  \item Some refinements will involve \emph{specialization}
    (eg. from $n$-dimensional to $2$-dimensional, assuming no
    friction, etc).  These \emph{choices} and their \emph{rationale} need to be
    documented, as a a crucial part of the solution.  Whether these choices
    are (un)likely to change in the future should be recorded.
  \item Choices tend to be dependent, and thus (partially) ordered.
   \emph{Decisions} frequently enable or reveal downstream choices.
  \end{enumerate}
\item\label{it:tocode} Describe how the process from step
\ref{it:refine} can be turned into code. The same kinds of choice
can occur here.
\item\label{it:recipe} Turn the steps (i.e.\ from items~\ref{it:refine} and
\ref{it:tocode}) into a \emph{recipe}, aka program, that weaves together all the
information into a variety of artifacts (documentation, code, build scripts,
test cases, etc). These can be read, or executed, or \ldots\, as appropriate.
\end{enumerate}

While this last step might appear somewhat magical, it isn't. The whole point of
defining \emph{well understood} is to enable it! A \emph{suitable}
knowledge encoding is key. This is usually tacit knowledge that
entirely resides in developers' heads.

What is missing is an explicit \emph{information architecture} of each of
the necessary artifact. In other words, what information is necessary to
enable the mechanized generation of each artifact? It turns out that many
of them are quite straightforward.

Often steps~\ref{it:problem} and~\ref{it:probdesc} are skipped; this is
part of the \textbf{tacit knowledge} of a lot
of software.  Our process requires that this knowledge be made explicit,
a fundamental step in \emph{Knowledge Management}~\cite{Dalkir2011}.

% TODO: insert graphical illustration of the funnel from information to
% softifacts.

\section{An Example}\label{ch:example}

We have built the needed infrastructure. It consists of 60KLoc of Haskell
implementing a series of interacting Domain Specific Languages (DSLs) for
knowledge encodings, mathematical expressions, theories, English fragments,
code generation and document generation.\footnote{We will provide a link
if the paper is accepted.} A full description would take too much space.
%because of the double blind review process, we can't really cite Drasil here
Instead, we provide an illustrative example.

We will focus on information capture and the artifacts we can generate. For
concreteness, we'll use a single example from our suite: GlassBR, used to
assess the risk for glass facades subject to blast loading.
The requirements are based on an American Standard Test Method (ASTM) standard
\cite{BeasonEtAl1998, ASTM2009}. GlassBR was originally a Visual
Basic code/spreadsheet created by colleagues in a civil engineering research
lab.  We added their domain knowledge to our framework, along with recipes to
generate relevant artifacts.  Not only can we generate code for the necessary
calculations (in \CC, C\#, Java, Python and Swift), we added documentation that
was not in the original (Software Requirements Specification, doxygen, README.md
and a Makefile). Moreover, our implementation is actually a family of
implementations, since some design decisions are explicitly exposed as
changeable variabilities, as described below.

\begin{figure*}[t]
  \centering
  \includegraphics[width=\linewidth]{assets/DrasilSupportsChange-right-portrait-overlapped-ungrouped-11ptFont-squished-noSmith-v1-300dpi.png}
  \caption{Colors and shapes mapping from captured knowledge to generated
  artifacts.}
  \Description{Colors and shapes mapping from captured knowledge to generated
  artifacts.}
  \label{Fig_DrasilAndChange}
\end{figure*}

The transformation of captured knowledge is illustrated in
Figure~\ref{Fig_DrasilAndChange}. This is read starting from the upper
right box. Each piece of information in this figure has its own
shape and colour (orange-cloud, pink lozenge, etc). It should be immediately clear
that all pieces of information reappear in multiple places in the generated
artifacts. For example, the name of the software (GlassBR) ends up
appearing more than 80 times in the generated softifacts (in the folder
structure, requirements, README, Makefile and source code). Changing this
name would traditionally be extremely difficult; we can achieve this by
modifying a single place, and regenerating.

The first box shows the directory structure of the currently generated
softifacts; continuing clockwise, we see examples of Makefiles for
the Java and Python versions, parts of the fully documented, 
generated code for the main computation in those languages, user
instructions for running the code, and the processed \LaTeX for the
requirements.

The name GlassBR is probably the simplest example of what we mean by
\emph{knowledge}: here, the concept ``program name'' is internally defined, and
its \emph{value} is used throughout. A more complex example is the assumption that the ``Load Distribution Factor'' (LDF) is constant
(pink lozenge). If this needs to be modified to instead be an input, the generated software
will now have LDF as an input variable.  We also capture design decisions,
such as whether to log all calculations, whether to in-line constants rather
than show them symbolically, etc. The knowledge for GlassBR can also be reused
in different projects.

We now give example conceptual encodings corresponding to steps of the process.

\heading{Step 1: Task} Compute the probability that a particular
pane of (special) glass will break if an explosive is detonated at
a given distance. This could be in the context of building or an armored
vehicle.

\heading{Step 2: Understood}
The details are extensively documented in~\cite{BeasonEtAl1998, ASTM2009}.

\heading{Step 3a: Base Knowledge}
A recurring idea is the different types of \texttt{Glass}:
\begin{center}
  \begin{tabular}{|l|l|l|l|}
    \hline
    \textbf{Concept} & \textbf{Term (Name)} & \textbf{Abbrev.} & \textbf{Domain} \\ \hline
    \texttt{fullyT} & Fully Tempered & FT & \texttt{[Glass]} \\ \hline
    \texttt{heatS} & Heat Strengthened & HS & \texttt{[Glass]} \\ \hline
    \texttt{iGlass} & Insulating Glass & IG & \texttt{[Glass]} \\ \hline
    \texttt{lGlass} & Laminated Glass & LG & \texttt{[Glass]} \\ \hline
    \texttt{glassTypeFac} & Glass Type Factor & GTF & \texttt{[Glass]} \\ \hline
  \end{tabular}
\end{center}

The ``Risk of Failure'' is definable, using the following template for a
\textit{data definition}:
\begin{center}
  \begin{tabular}{|l|l|}
    \hline
    \textbf{Label} & Risk of Failure \\ \hline
%    \textbf{Symbol} & $B$ \\ \hline
%    \textbf{Units} & Unitless \\ \hline
    \textbf{Equation} & $B = \frac{k}{(ab)^{m-1}}(Eh^2)^m\mathit{LDF}e^J$ \\ \hline
    \textbf{Description} & \vbox{
      \hbox{\strut $B$ is the Risk of Failure (Unitless)}
      \hbox{\strut $k$ is the surface flaw parameter ($\frac{m^{12}}{N^7}$)}
      \hbox{\strut $a$ \& $b$ are the plate length \& width (\textit{m})}
      \hbox{\strut $...$}
% Do we want to have all of the variables listed? - No - there isn't room
%      \hbox{\strut $m$ is the surface flaw parameter ($\frac{m^{12}}{N^7}$)}
%      \hbox{\strut $E$ is the modulus of elasticity of glass (\textit{Pa})}
%      \hbox{\strut $h$ is the minimum thickness (\textit{m})}
%      \hbox{\strut $LDF$ is the load duration factor (Unitless)}
%      \hbox{\strut $J$ is the stress distribution factor (Unitless)}
    } \\ \hline
    \textbf{Source} & \cite{ASTM2009}, \cite{BeasonEtAl1998} \\ \hline 
% The Campidelli reference was removed because we cannot really cite personal
% communication and not reveal our identities
  \end{tabular}
\end{center}

% explosion - concept chunk
% glass slag - named chunk
% degree - concept chunk
% user input - named chunk
\heading{Step 3b: Coherent Narrative}
The descriptions in GlassBR are produced using an experimental language 
using specialized markup
for describing relations between knowledge. For example, the goal of GlassBR 
(``Predict-Glass-Withstands-Explosion'') is to ``Analyze and predict whether 
the \textit{glass slab} under consideration will be able to withstand the 
\textbf{explosion} of a certain \textbf{degree} which is calculated based on
\textit{user input}.'', where italicized names are ``named ideas'', 
and bold-faced names are ``concept chunks'' (named ideas with a domain of
related ideas). We call this goal a ``concept instance'' (a concept chunk
applied in some way). This language lets us
perform various static analyses on our artifacts.

\heading{Step 3c: Characteristics of a Good Solution}
One of our outputs is a probability, which can be checked to be between $0$ and
$1$ (not shown).

\heading{Step 4: Specialization of Theories}
In the GlassBR example, the main specialization occurs because manufacturers standardize glass
thickness, so the thickness parameter is not free to vary,
but must take on specific values.

The phenomenological nature of the GlassBR example is atypical in this way; most research software examples involve more significant
specialization, such as partial differential equations to ordinary,
elimination of variables, use of closed-forms instead of implicit equations,
and so on.

%% Commenting this out, as it didn't really represent 4b
%
%\heading{Step 4b: Choices}
%
%We create a closed harmonic system containing all related knowledge (in
%particular, the grounded theories), which is as simple as collecting them into a
%single system.
%
%\begin{lstlisting}
%iMods :: [InstanceModel]
%iMods = [pbIsSafe, lrIsSafe]
%
%si :: SystemInformation
%si = SI {
%_sys = glassBR, _kind = Doc.srs, _authors = [nikitha],
%_purpose = purpDoc glassBR Verbose, _quants = symbolsForTable,
%_concepts = [] :: [DefinedQuantityDict], _instModels = iMods,
%_datadefs = GB.dataDefs, _configFiles = configFp,
%_inputs = inputs, _outputs = outputs,
%_defSequence = qDefns, _constraints = constrained,
%_constants = constants, _sysinfodb = symbMap,
%_usedinfodb = usedDB, refdb = refDB
%}
%\end{lstlisting}

\heading{Step 5: Code-level Choices}
We can choose output programming language, how ``modular'' the generated code
is, whether we want programs or libraries, the level of logging and comments,
etc. Here we show the actual code we use for this, as it is reasonably
direct.

\begin{lstlisting}
code :: CodeSpec
code = codeSpec fullSI choices allMods

choices :: Choices
choices = defaultChoices {
 lang = [Python, Cpp, CSharp, Java, Swift], 
 modularity = Modular Separated,
 impType = Program, logFile = "log.txt", 
 logging = [LogVar, LogFunc],
 comments = [CommentFunc, CommentClass, CommentMod], 
 doxVerbosity = Quiet,
 dates = Hide, 
 onSfwrConstraint = Exception, onPhysConstraint = Exception,
 inputStructure = Bundled, 
 constStructure = Inline, constRepr = Const
}
\end{lstlisting}

\heading{Step 6: Recipe to Generate Artifacts}

The various pieces of knowledge specified in the previous steps are
assembled into \emph{recipes} for generating the artifacts, for example
specification, code, dependency diagram and log of all choices used.  Each one
of these has its own DSL that encodes the parts of a specification, the
requirements for code\footnote{which has been already published
and will be cited later}, etc.

The venerable \texttt{Makefile} language is a good example of a DSL for
specifying build scripts, but where our ideas are more properly compared
with the kind of thinking behind \textit{Rattle}~\cite{mitchell:rattle_18_nov_2020}.

And finally, we can put all of these things together, and generate
\textbf{everything}:
\begin{lstlisting}
main :: IO()
main = do
  setLocaleEncoding utf8
  gen (DocSpec (docChoices SRS [HTML, TeX]) "GlassBR_SRS") srs printSetting
  genCode choices code
  genDot fullSI
  genLog fullSI printSetting
\end{lstlisting}

\section{Concluding Remarks} \label{ch:concluding_remarks}

To summarize:
\\

\begin{mdframed}[style=MyFrame]
\begin{itemize}[leftmargin=\widthof{[\textbf{Opportunity}]}]
\item[\textbf{Opportunity}]: well-understood + long-lived
\item[\textbf{Tools}]: knowledge capture + DSLs + generators
\item[\textbf{Result}]: \emph{long-term productivity gain}
\end{itemize}
\end{mdframed}

The reason we get productivity gains is because there is in fact an
incredible amount of \emph{knowledge duplication} present in software,
especially in well understood domains. There, building software really
ought to be a matter of assembly-line style engineering. Furthermore,
there is also inherent knowledge duplication between the softifacts of
a project because \emph{they are about the same topic}.

This means that if we spend some up-front time capturing the fundamental
knowledge of specific domains (such as mechanics of rigid body motion,
dynamics of soil, trains, etc), most development time can then be later
spent on the \emph{specifics} of the given requirements.

As a side-effect, something we have not been able to illustrate in the 
short space of this paper, we can obtain traceability and consistency, by
construction.

There are further ideas that co-exist smoothly with our framework, most notably
software families and correctness.  Our process will remove human errors from
generating and maintaining documentation, code, test cases and build
environments, since these mundane details are handled by the generator.
Furthermore, our work makes it easy to experiment with ``what if'' scenarios,
which make it easy to explicitly track the
ramifications of a proposed change.  

With the right up-front investment, we can have sustainable software
development because stable knowledge is separated from 
local context, which is where most of the rapidly changing assumptions
and requirements reside.

\newpage
%% Bibliography
\bibliographystyle{ACM-Reference-Format}
\bibliography{References,extras}

\end{document}

% Good quotes:
% \begin{itemize}
% \item metaprograms are just programs
% \item models outside an integrated toolchain are insufficiently useful
% \end{itemize}

%% Was in 4a
To illustrate the specialization of theories, we will look outside of the
GlassBr example because, due to its phenomenological nature, GlassBr only has
trivial specializations.  To illustrate this step we will turn to another
example: a solar water heating system incorporating phase change material
\cite{DouviEtAl2021} (based on original FORTRAN code from colleagues in
mechanical engineering).  Since this is a heat transfer problem, the initial
theory is the general form of the conservation of thermal energy. This general
form is a partial differential equation in 3D space and
time~\cite{BirdEtAl2002}.  To specialize the theory to the solar water heating
tank, we introduced the following assumptions:

\begin{itemize}

\item The water in the tank is fully mixed.

\item The density of water has no spatial variation.

\item The specific heat capacity of water has no spatial variation.

\end{itemize}

The refined theory is an ordinary differential equation that depends only on
time.  This new theory is parameterized by material properties and the area and
thermal flux between two adjacent bodies. This theory in turn is specialized
into two more theories: one where the adjacent bodies are the tank and the
heating coil, and one where the bodies are the water and a phase change
material.

By systematically refining the theory, and generating documentation, human
beings can validate the assumptions and the mathematical model. By capturing the
knowledge, changes can be made to realize a family of mathematical models.
Moreover, the knowledge can be reused for other heat transfer problems.

