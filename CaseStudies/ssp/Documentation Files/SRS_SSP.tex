\documentclass[12pt]{article}
\usepackage{colortbl}
\usepackage{tabularx}
\usepackage{longtable}
\usepackage{comment}
\usepackage{amsmath}
\usepackage{amssymb}
\usepackage{nccmath}
\usepackage{multirow}
%\usepackage{mathtools}
\usepackage{geometry}
\usepackage{booktabs}
\usepackage{xr}
\usepackage{enumitem}
\usepackage{siunitx}
\usepackage{graphicx}
\usepackage{caption}

\definecolor{AFB}{rgb}{0.36, 0.54, 0.66}
\definecolor{Brass}{rgb}{0.71, 0.65, 0.26}
\definecolor{Amethyst}{rgb}{0.6, 0.4, 0.8}

\usepackage{xr}
\usepackage{hyperref}

\hypersetup{
    bookmarks=true,         % show bookmarks bar?
      colorlinks=true,       % false: boxed links; true: colored links
    linkcolor=red,          % color of internal links (change box
                            % color with linkbordercolor)
    citecolor=green,        % color of links to bibliography
    filecolor=magenta,      % color of file links
    urlcolor=cyan           % color of external links
}
\newcommand{\NN}[1]{{\color{red}#1}}
\newcommand{\WSS}[1]{{\color{blue}#1}}

%% Comments
\newif\ifcomments\commentstrue

\ifcomments
\newcommand{\authornote}[3]{\textcolor{#1}{[#3 ---#2]}}
\newcommand{\todo}[1]{\textcolor{red}{[TODO: #1]}}
\else
\newcommand{\authornote}[3]{}
\newcommand{\todo}[1]{}
\fi

\newcommand{\wss}[1]{\authornote{magenta}{SS}{#1}}
\newcommand{\hf}[1]{\authornote{cyan}{HF}{#1}}
\newcommand{\cjl}[1]{\authornote{green}{CL}{#1}}

\newcommand{\colAwidth}{0.13\textwidth}
\newcommand{\colBwidth}{0.82\textwidth}
\newcommand{\colCwidth}{0.1\textwidth}
\newcommand{\colDwidth}{0.05\textwidth}
\newcommand{\colEwidth}{0.8\textwidth}
\newcommand{\colFwidth}{0.17\textwidth}
\newcommand{\colGwidth}{0.5\textwidth}
\newcommand{\colHwidth}{0.28\textwidth}
\newcounter{assumpnum} %Assumption Number
\newcommand{\atheassumpnum}{A\theassumpnum}
\newcommand{\aref}[1]{A\ref{#1}}
\newcounter{goalnum} %Goal Number
\newcommand{\gthegoalnum}{GS\thegoalnum}
\newcommand{\gsref}[1]{GS\ref{#1}}
\newcounter{theorynum} %Theory Number
\newcommand{\tthetheorynum}{T\thetheorynum}
\newcommand{\tref}[1]{T\ref{#1}}
\renewcommand{\arraystretch}{1}
\newcounter{instnum} %Instance Number
\newcommand{\itheinstnum}{IM\theinstnum}
\newcommand{\iref}[1]{IM\ref{#1}}
\newcounter{datadefnum} %Datadefinition Number
\newcommand{\ddthedatadefnum}{DD\thedatadefnum}
\newcommand{\ddref}[1]{DD\ref{#1}}
\newcounter{defnum} %Definition Number
\newcommand{\dthedefnum}{GD\thedefnum}
\newcommand{\dref}[1]{GD\ref{#1}}
\newcounter{reqnum} %Requirement Number
\newcommand{\rthereqnum}{R\thereqnum}
\newcommand{\rref}[1]{R\ref{#1}}
\newcounter{lcnum} %Likely change number
\newcommand{\lthelcnum}{LC\thelcnum}
\newcommand{\lcref}[1]{LC\ref{#1}}
\newcounter{fnum} %Figure number
\newcommand{\fthefnum}{Fig\thefnum}
\newcommand{\fref}[1]{Fig\ref{#1}}
\newcounter{tablenum} %Table number
\newcommand{\tablethetablenum}{Table\thetablenum}
\newcommand{\tableref}[1]{Table\ref{#1}}

\newcommand{\forceindent}{\parindent=1em\indent\parindent=0pt\relax}

%\oddsidemargin -1000mm
%\evensidemargin -1000mm
%\textwidth 160mm
%\textheight 300mm
\newgeometry{margin=2cm}

\externaldocument[MIS-]{MIS_SSP}
\externaldocument[MG-]{MG_SSP}

\begin{document}

\title{Software Requirements Specification for Slope Stability Analysis}
\author{Henry Frankis\\McMaster University}
\date{\today}
	
\maketitle
\tableofcontents

\section{Reference Material}

\subsection{Table of Units}

Units of the Physical properties of the soil that are 
of interest when examining slope stability problems. 
\newline

\renewcommand{\arraystretch}{1.2}
\setlength{\tabcolsep}{20pt}
\begin{tabular}{  l  l  l  }
\hline
\textbf{Physical Property} & \textbf{Name} & \textbf{Symbol} \\
\hline
force & Newton & \si{\newton} \\
length & meter & \si{\meter}  \\
pressure & Pascal & $\si{\pascal}=\si{\newton\per\square\meter}$ \\
angle & degree & \si{\degree}  \\
\hline
\end{tabular}
\renewcommand{\arraystretch}{1}


\subsection{Table of Symbols}

A  collection of  the symbols,  that will  be used  in the  models and
equations of  the program  are summarized in  the table  below. Values
with a  subscript $i$ implies that  the value will be  taken at
and analyzed  at a slice or  slice interface composing the  total slip
mass.

\renewcommand{\arraystretch}{1.6}
\setlength{\tabcolsep}{20pt}
\begin{longtable}{  l  l  p{8.5cm}  }
\hline
\textbf{Symbol} & \textbf{Unit} & \textbf{Description} \\
\hline
${\varphi'}$ & \si{\degree} & Effective angle of friction. \\

${\text{c}'}$ & \si{\pascal} & Effective cohesion. \\

${\gamma}$ & \si{\kilo\newton\per\meter\cubed} & Unit weight of dry
soil / ground layer.  \\

${\gamma}_\text{Sat}$ & \si{\kilo\newton\per\meter\cubed} & Unit
weight of saturated soil / ground layer. \\

${\gamma}_\text{w}$ & \si{\kilo\newton\per\meter\cubed} & Unit weight
of water. \\

$\text{E}$ & \si{\pascal} & Elastic modulus.\\

${\nu}$ & ${/}$ & Poisson's ratio. \\

${(x,y)}$ & \si{\meter} & Cartesian position coordinates. $y$ is
considered parallel to the direction of the force of gravity and $x$
is considered perpendicular to $y$.\\

$y_\text{wt,i}$ & \si{\meter} & The $y$-ordinate, or height of the
water table at $x_\text{i}$. Can refer to either slice $i$ midpoint,
or slice interface $i$. \\

$y_\text{us,i}$ & \si{\meter} & The $y$-ordinate, or height of the top
of the slope at $x_\text{i}$. Can refer to either slice $i$ midpoint,
or slice interface $i$. \\

$y_\text{slip,i}$ & \si{\meter} & The $y$-ordinate, or height of the
slip surface at $x_\text{i}$. Can refer to either slice $i$ midpoint,
or slice interface $i$. \\

$x_\text{i}$ & \si{\meter} & The $x$-ordinate. Can refer to either
slice $\text{i}$ midpoint, or slice interface $i$. \\

$ \left( \left\{x_\text{cs}\right\}, \left\{y_\text{cs}\right\}
\right)$ & \si{\meter} & The set of $x$ and $y$ coordinates that
describe the vertexes of the critical slip surface.\\

$\text{FS}$ & ${/}$ & Global Factor of Safety. Metric describing the
stability of a surface in a slope. \\

$\text{FS}_\text{Loc,i}$ & ${/}$ & Local Factor of Safety. Factor of
Safety specific to a slice. For slice index $i$.\\

${S_{\text{i}}}$ & \si{\newton} & Mobilized shear force. Shear forces
that cause instability in a slice. For slice index $i$.\\

${P_{\text{i}}}$ & \si{\newton} & Shear resistance. Mohr Coulomb
frictional force that describes the limit of mobilized shear force the
slice can withstand before failure. For slice index $i$.\\

${T_{\text{i}}}$ & \si{\newton} & Mobilized shear force, without the
influence of interslice forces. For slice index $i$.\\

${R_{\text{i}}}$ & \si{\newton} & Shear resistance, without the
influence of interslice forces. For slice index $i$.\\

${W_{\text{i}}}$ & \si{\newton} & Weight. Downward force caused by
gravity the mass of slice $i$ exerts. For slice index ${\text{i}}$. \\

${K_{\text{c}}}$ & ${/}$ & Earthquake load factor. Proportionality
factor of force that weight pushes outwards. Caused by seismic earth
movements. \\

${{H}_{\text{i}}}$ & \si{\newton} & Interslice water force exerted in
the $x$-ordinate direction between adjacent slices. For interslice
index $i$ \\

${\Delta{H}_{\text{i}}}$ & \si{\newton} & Difference between
interslice forces on acting in the $x$-ordinate direction of the slice
on each side. For slice index $i$. Refers to net force
${{H}_{\text{i}}-{H}_{\text{i-1}}}$ \\

${E}_{\text{i}}$ & \si{\newton} & Interslice normal force being
exerted between adjacent slices. For interslice index $i$. \\

${X}_{\text{i}}$ & \si{\newton} & Interslice shear force being exerted
between adjacent slices.  For interslice index $i$. \\

${U_{\text{b,i}}}$ & \si{\newton} & Base hydrostatic force. Force from
water pressure within the slice. For slice index $i$.\\

${U_{\text{t,i}}}$ & \si{\newton} & Surface hydrostatic force. Force
from water pressure acting into the slice from standing water on the
slope surface. For slice index $i$.\\

$N_{\text{i}}$ & \si{\newton} & Total reactive force for a soil
surface subject to a body resting on it. \\

$N'_{\text{i}}$ & \si{\newton} & Effective normal force of a soil
surface, subtracting pore water reactive force from total reactive
force. \\

$N*_{\text{i}}$ & \si{\newton} & Effective normal force of a soil
surface, neglecting the influence of interslice forces. \\

$Q_\text{i}$ & \si{\newton} & An imposed surface load. A downward
force acting into the surface from midpoint of slice $i$. \\

${\alpha_{\text{i}}}$ & \si{\degree} & Angle of the base of the mass
relative to the horizontal. For slice index $i$. \\

${\beta_{\text{i}}}$ & \si{\degree} & Angle of the surface of the mass
relative to the horizontal. For slice index $i$. \\

${\omega_{\text{i}}}$ & \si{\degree} & Angle of imposed surface load
acting into the surface relative to the vertical. For slice index
$i$. \\

${\lambda}$ & ${/}$ & Ratio between Interslice normal and shear
forces. Applied to all interslices. \\

${f_{\text{i}}}$ & ${/}$ & function for inclination of interslice
forces. A scaling function for magnitude of interslice forces as a
function of the x coordinate. Can be constant or a half-sine. Value of
function at interslice index $i$.  \\

${b_{\text{i}}}$ & \si{\meter} & Base width of the slice in the
$x$-ordinate direction only. For slice $i$. \\

$\ell_{\text{b,i}}$ & \si{\meter} & Total length of the base of a
slice. For slice index $i$. \\

$\ell_{\text{s,i}}$ & \si{\meter}& Length of an interslice surface,
from slip base to slope surface in a vertical line from an interslice
vertex. For interslice index $\text{i}$. \\

${h_{\text{i}}}$ & \si{\meter} & Midpoint height. Distance from the
slip base to the slope surface in a vertical line from the midpoint of
the slice. For slice $i$. \\

${\text{n}}$ & ${/}$ & Number of slices the slip mass has been divided
into. \\

${F}$ & \si{\newton} & A generic force. Assumed 1D allowing a
scalar. \\

${M}$ & \si{\newton\meter} & Moment of a body. Assumed 2D allowing a
scalar. \\

$\Upsilon$ & {/} & A generic minimization function or algorithm. \\

$\delta$ & $\text{m}$ & Generic displacement of a body. \\

$K$ & \si{newton\per\meter} & Stiffness. How much a body resists
displacement when subject to a force. \\

$K_{\text{st,i}}$ & \si{\pascal} & Shear stiffness of an interslice
surface, without length adjustment. For interslice index $i$.  \\

$K_{\text{bt,i}}$ & \si{\pascal} & Shear stiffness of a slice base
surface, without length adjustment. For slice index $i$. \\

$K_{\text{sn,i}}$ & \si{\pascal} & Normal stiffness of an interslice
surface, without length adjustment. For interslice index $i$.  \\

$K_{\text{bn,i}}$ & \si{\pascal} & Normal stiffness of a slice base
surface, without length adjustment .For slice index $i$. \\

$K_{\text{tr}}$ & \si{\pascal} & Residual shear stiffness. \\

$K_{\text{no}}$ & \si{\pascal} & Residual normal stiffness. \\

$\delta u_{\text{i}}$ & \si{\meter} & Shear displacement of a
slice. For slice index $i$. \\

$\delta v_{\text{i}}$ & \si{\meter}& Normal displacement of a
slice. For slice index $i$. \\

$\delta x_{\text{i}}$ & \si{\meter} & Displacement of a slice in the
$x$-ordinate direction. For slice index $i$. \\

$\delta y_{\text{i}}$ & \si{\meter} & Displacement of a slice in the
$y$-ordinate direction. For slice index $i$. \\

\hline
\end{longtable}
\renewcommand{\arraystretch}{1}


\subsection{Abbreviations and Acronyms}

\renewcommand{\arraystretch}{1.2}
\begin{tabular}{l l} 
  \toprule		
  \textbf{symbol} & \textbf{description}\\
  \midrule 
  A & Assumption\\
  DD & Data Definition\\
  GD & General Definition\\
  GS & Goal Statement\\
  IM & Instance Model\\
  LC & Likely Change\\
  PS & Physical System Description\\
  R & Requirement\\
  SRS & Software Requirements Specification\\
  SSA & Slope Stability Analysis\\
  T & Theoretical Model\\
  \bottomrule
\end{tabular}\\

\section{Introduction}

A slope of Geological mass, composed of soil and rock, is subject to
the influence of gravity on the mass. For an instable slope this can
cause instability in the form of soil/rock movement.  The effects of
soil/rock movement can range from inconvenient to seriously hazardous,
resulting in significant life and economic loses. Slope stability is
of interest both when analyzing natural slopes, and when designing an
excavated slope. Slope stability analysis is the assessment of the
safety of a slope, identifying the surface most likely to experience
slip and an index of it's relative stability known as the factor of
safety.

~\newline~\newline The following section provides an overview of the
Software Requirements Specification (SRS) for a slope stability
analysis problem.  The developed program will be referred to as
Slope Stability Analysis program (SSA).  This section explains the purpose of
this document, the scope of the system, the organization of the
document and the characteristics of the intended readers.

\subsection{Purpose}

The SSA program is slope stability analysis program.  The program
determines the critical slip surface, and it's respective factor of
safety as a method of assessing the stability of a slope design. The
program is intended to be used as an educational tool, introducing
slope stability issues, analysis software, and the design of a safe
slope.

This document will be used as a starting point for subsequent
development phases, including writing the design specification and the
software verification and validation plan.  The design document will
show how the requirements are to be realized, including decisions on
the numerical algorithms and programming environment.  The
verification and validation plan will show the steps that will be used
to increase confidence in the software documentation and the
implementation.  Although the SRS fits in a series of documents that
follow the so-called waterfall model, the actual development process
is not constrained in any way.  Even when the process is not
waterfall, as Parnas and Clements~\cite{ParnasAndClements1986} point
out, the most logical way to present the documentation is still to
``fake'' a rational design process.

\subsection{Scope of Requirements} 

The scope of the requirements is limited to stability analysis of a 2
dimensional slope, composed of homogeneous soil layers. Given
appropriate inputs the code for SSA will identify the most likely
failure surface within the possible input range, and find the factor
of safety for the slope and displacement of soil that will occur on
the slope.

\subsection{Organization of Document}

The organization of this document follows the template for an SRS for
scientific computing software proposed by~\cite{Koothoor2013} and
\cite{SmithAndLai2005}.  The presentation follows the standard pattern
of presenting goals, theories, definitions, and assumptions.  For
readers that would like a more bottom up approach, they can start
reading the instance models in Section \ref{sec_instance} and trace
back to find any additional information they require.  The instance
models provide the set of algebraic equations that must be solved
iteratively to perform a Morgenstern Price Analysis, and the system of
equations that must be solved for Rigid Finite Element Analysis.

The goal statements are refined to the theoretical models, and
theoretical models (Section \ref{sec_theoretical}) to the instance
models (Section \ref{sec_instance}).

%\subsection{Intended Audience}

\section{General System Description}

This section provides general information about the system, identifies
the interfaces between the system and its environment, and describes
the user characteristics and the system constraints.

%\subsection{System Context}

\subsection{User Characteristics}

The end user of SSA should have an understanding of undergraduate
Level 1 Calculus and Physics, and be familiar with soil and material
properties.

\subsection{System Constraints}

There are no system constraints.

\section{Specific System Description}

This section first presents the problem description, which gives a
high-level view of the problem to be solved.  This is followed by the
solution characteristics specification, which presents the
assumptions, theories, definitions and finally the instance models
that model the slope. The program implements two solution methods to
analyze the slope with. Models with
\textbf{\textcolor{Amethyst}{purple}} table headings refer to a
Morgenstern Price solution. Models with
\textbf{\textcolor{AFB}{blue}} table headings refer to a RFEM
solution. Models with \textbf{\textcolor{Brass}{brass}} table
headings are models that refer to both solutions.


\subsection{Problem Description} \label{Sec_pd}

SSA is a computer program developed to evaluate the factor of safety
of a slopes slip surface and, calculate the displacement the slope
will experience.

\subsubsection{Terminology}

\begin{itemize}
\item {\textit{Factor of safety:} Stability metric. How likely a slip
  surface is to experience failure through slipping.}
  
\item {\textit{Critical slip surface:} Slip surface of the slope that
  has the lowest global factor of safety, and therefore most likely to
  experience failure.}

\item {\textit{Stress:} Forces that are exerted between planes
  internal to a larger body subject to external loading.}
  
\item {\textit{Strain:} Stress forces that result in deformation of
  the body/plane.}
  
\item {\textit{Normal Force:} A force applied perpendicular to the
  plane of the material.}
  
\item {\textit{Shear Force:} A force applied parallel to the plane of
  the material.}
  
\item {\textit{Tension:} A stress the causes displacement of the body
  away from it's center.}
  
\item {\textit{Compression:} A stress the causes displacement of the
  body towards it's center.}
  
\item {\textit{Plane Strain:} The resultant stresses in one of the
  directions of a 3 dimensional material can be approximated as
  0. Results when the length of one dimension of the body dominates
  the others. Stresses in the dominate dimensions direction are the
  ones that can be approximated as 0.}
  
\end{itemize}

\subsubsection{Physical System Description} \label{sec_system}

Analysis of the slope is performed by looking at properties of the
slope as a series of slice elements. Some properties are interslice
properties, and some are slice or slice base properties.  The index
convention for referencing which interslice or slice is being used is
shown in \fref{Fig_Index}.

\begin{itemize}
\item Interslice properties convention is noted by $\text{j}$. The end
  Interslice properties are usually not of interest, therefore use the
  interslice properties from $1\leq\text{i}\leq n-1$.

\item Slice properties convention is noted by $\text{i}$.
\end{itemize}


\begin{figure}[h!] \refstepcounter{fnum}  \label{Fig_Index}
\begin{center}
{
\setlength{\unitlength}{6cm}
\begin{picture}(2,1)
% BORDER %
\thinlines
\put(0,0){\line(0,1){1}}
\put(0,1){\line(1,0){2}}
\put(2,1){\line(0,-1){1}}
\put(2,0){\line(-1,0){2}}
% SLIP SURFACE %
\linethickness{1mm}
\qbezier(0.2, 0.9)(0.5, 0.3)(1.8, 0.1)
% SLOPE %
\linethickness{0.1mm}
\put(0.1,0.9){\line(1,0){0.5}}
\put(0.6,0.9){\line(3,-2){1.2}}
\put(1.8,0.1){\line(1,0){0.1}}
% SLICES %
\put(0.2,0.9){\line(0,1){0}}
\put(0.6,0.484){\line(0,1){0.416}}
\put(0.992,0.2938){\line(0,1){0.3401}}
\put(1.4005,0.1764){\line(0,1){0.19}}
\put(1.8,0.1){\line(0,1){0}}
% j LABELS %
\put(0.15,0.92){$j=0$}
\put(0.55,0.92){$j=1$}
\put(1.3985,0.3863){$j=n-1$}
\put(1.78,0.13){$j=n$}
% j LABELS %
\put(0.38,0.75){$i=1$}
\put(0.74,0.55){$i=2$}
\put(1.1,0.34){$i=n-1$}
\put(1.48,0.185){$i=n$}
\end{picture}
}
\caption{Index convention for numbering slice and interslice force
  variables}
 \end{center}
\end{figure}


~\newline\noindent A free body diagram of the forces acting on the
slice is displayed in \fref{Fig_Forces}.

\begin{figure}[h!] \refstepcounter{fnum}  \label{Fig_Forces}
\begin{center}
{
 \includegraphics[width=0.7\textwidth]{ForceDiagram.png}
}
\caption{Forces acting on a slice}
\end{center}
\end{figure}



\subsubsection{Goal statements}

Given the geometry of the water table, the geometry of the layers
composing the plane of a slope, and the material properties of the
layers.

\begin{itemize}
\item [G\refstepcounter{goalnum}\thegoalnum: \label{G_FS}]
  {Evaluate local and global factors of safety along a given slip
    surface.}
  
\item [G\refstepcounter{goalnum}\thegoalnum: \label{G_Critical}]
  {Identify the critical slip surface for the slope, with the lowest
    Factor of Safety.}
  
\item [G\refstepcounter{goalnum}\thegoalnum: \label{G_Displacement}]
  {Determine the displacement of the slope.}
\end{itemize}

\subsection{Solution Characteristics Specification}

The instance models that govern SSA are presented in
Subsection~\ref{sec_instance}.  The information to understand the
meaning of the instance models and their derivation is also presented,
so that the instance models can be verified.

\subsubsection{Assumptions}

This section simplifies the original problem and helps in developing
the theoretical model by filling in the missing information for the
physical system. The numbers given in the square brackets refer to the
data definition, or the instance model, in which the respective
assumption is used.

\begin{enumerate}[label=A\arabic*:,ref={\arabic*}]
\item [A\refstepcounter{assumpnum}\theassumpnum: \label{A_Concave}] The
  slip surface is concave with respect to the slope surface. The (x,y)
  coordinates of the failure surface follow a monotonic function.

\item [A\refstepcounter{assumpnum}\theassumpnum: \label{A_Input}] The
  geometry of the slope, and the material properties of the soil
  layers are given as inputs.

\item [A\refstepcounter{assumpnum}\theassumpnum: \label{A_Homo}] The
  different layers of the soil are homogeneous, with consistent soil
  properties throughout, and independent of dry or saturated
  conditions, with the exception of unit weight.

\item [A\refstepcounter{assumpnum}\theassumpnum: \label{A_Isotropic}]
  Soil layers are treated as if they have isotropic properties.
  
\item [A\refstepcounter{assumpnum}\theassumpnum: \label{A_Base}]
  Interslice normal and shear forces have a linear relationship,
  proportional to a constant $\left({\lambda}\right)$ and an
  interslice force function $\left({f}\right)$ depending on x
  position.

\item [A\refstepcounter{assumpnum}\theassumpnum: \label{A_Interslice}]
  Slice to base normal and shear forces have a linear relationship,
  dependent on the factor of safety $\left({FS}\right)$, and the
  Coulomb sliding law.

\item
  [A\refstepcounter{assumpnum}\theassumpnum: \label{A_StressStrain}]
  The stress-strain curve for interslice relationships is linear with
  a constant slope.
  
\item [A\refstepcounter{assumpnum}\theassumpnum: \label{A_2D}] The
  slope and slip surface extends far into and out of the geometry ($z$
  coordinate). This implies plane strain conditions, making 2D
  analysis appropriate.

\item [A\refstepcounter{assumpnum}\theassumpnum: \label{A_Lin}] The
  effective normal stress is large enough that the resistive shear to
  effective normal stress relationship can be approximated as a linear
  relationship.

\item [A\refstepcounter{assumpnum}\theassumpnum: \label{A_Straight}]
  The surface and base of a slice between interslice nodes are
  approximated as straight lines.
\end{enumerate}

\subsubsection{Theoretical Models} \label{sec_theoretical}

This section focuses on the general equations and laws that SSA is based
on.

~\newline

% ----------------------------- %
%        Begin TM FS           %
% ----------------------------- %
\noindent
\begin{minipage}{\textwidth}
\renewcommand*{\arraystretch}{1.5}
\begin{tabular}{| p{1.5cm} | p{14cm}|}
  
  \hline \rowcolor{Amethyst} Number&
  T\refstepcounter{theorynum}\thetheorynum \label{TM_FS}\\
  
  \hline Label&\bf Factor of Safety\\
  
  \hline Equation& \( \text{FS} = \frac{P}{S} \) \\
  
  \hline Description & The stability metric of the slope, known as the
  factor of safety $\text{FS}$, is determined by the ratio of the
  shear force at the base of the slope $S$ (\dref{GD_MobShear}), and
  the resistive shear $P$ (\tref{TM_Fmc}). \\
 
  \hline Source & \cite{FredlundKrahn}\\

  \hline Ref.\ By & \iref{IM_FS}, \dref{GD_MobShear} \\

  \hline
\end{tabular}
\end{minipage}\\
% ----------------------------- %
%        End TM FS              %
% ----------------------------- %

~\newline

% ----------------------------- %
%        Begin TM Equilibrium            %
% ----------------------------- %
\noindent
\begin{minipage}{\textwidth}
\renewcommand*{\arraystretch}{1.5}
\begin{tabular}{| p{1.5cm} | p{14cm}|}
  
  \hline \rowcolor{Amethyst} Number&
  T\refstepcounter{theorynum}\thetheorynum \label{TM_Eqm}\\
  
  \hline
  Label&\bf Equilibrium\\
  
  \hline Equation& \( \displaystyle\sum {F}_{\text{x}} = 0 ~\newline
  \displaystyle\sum F_{\text{y}} = 0 ~\newline \displaystyle\sum M = 0
  \)\\

  \hline Description & For a body in static equilibrium the net
  forces, and net moments acting on the body will cancel out. Assuming
  a 2D problem (\aref{A_2D}) the net $x$-ordinate
  $\left(F_{\text{x}}\right)$ and $y$-ordinate
  $\left(F_{\text{y}}\right)$ scalar components will be equal to
  0. All forces and their distance from the chosen point of rotation
  will create a net moment equal to 0, also able to be analyzed as a
  scalar in a 2D problem. \\
  
  \hline Source & \cite{FredlundKrahn}\\
  
  \hline Ref.\ By & \dref{GD_Fx}, \dref{GD_Fy}, \dref{GD_M},
  \iref{IM_Lambda} \\
  
  \hline
\end{tabular}
\end{minipage}\\
% ----------------------------- %
%        End TM Equilibrium   %
% ----------------------------- %

~\newline

% ----------------------------- %
%        Begin TM Mohr Coulomb            %
% ----------------------------- %
\noindent
\begin{minipage}{\textwidth}
\renewcommand*{\arraystretch}{1.5}
\begin{tabular}{| p{1.5cm} | p{14cm}|}
  
  \hline \rowcolor{Brass} Number&
  T\refstepcounter{theorynum}\thetheorynum \label{TM_Fmc}\\
  
  \hline Label&\bf Mohr-Coulomb Shear Strength\\
  
  \hline Equation& \( P = \sigma \cdot \tan\left( \varphi' \right) + c
  \) \\
  
  \hline Description & For a soil under stress it will exert a shear
  resistive strength based on the Coulomb sliding law.  The resistive
  shear is the maximum amount of shear a surface can experience while
  remaining rigid, analogous to a maximum normal force.  In this model
  the shear force $P$ is proportional to the product of the normal
  stress on the plane $\sigma$ with it's static friction, in the
  angular form $\tan\left( \varphi' \right)=U_{\text{s}}$.  The $P$
  versus $\sigma$ relationship is not truly linear, but assuming the
  effective normal force is strong enough it can be approximated with
  a linear fit (\aref{A_Lin}), where the cohesion $c$ represents the
  $P$ intercept of the fitted line.\\

  \hline Source & \cite{FredlundKrahn}\\
  
  \hline Ref.\ By & \dref{GD_P}, \dref{GD_MobShear}, \ddref{DD_R},
  \ddref{DD_T}, \iref{IM_RFEMFS}\\
  
  \hline
\end{tabular}
\end{minipage}\\
% ----------------------------- %
%        End TM Mohr Coulomb              %
% ----------------------------- %

~\newline

% ----------------------------- %
%        Begin TM Effective Stress            %
% ----------------------------- %
\noindent
\begin{minipage}{\textwidth}
\renewcommand*{\arraystretch}{1.5}
\begin{tabular}{| p{1.5cm} | p{14cm}|}
  
  \hline \rowcolor{Amethyst} Number&
  T\refstepcounter{theorynum}\thetheorynum \label{TM_EffStress}\\
  
  \hline Label&\bf Effective Stress\\
  
  \hline Equation& \( \sigma' =\sigma - \mu \) \\
  
  \hline Description & $\sigma$ is the total stress a soil mass needs
  to maintain itself as a rigid collection of particles. The source of
  the stress can be provided by the soil skeleton $\sigma'$, or by the
  pore pressure from water within the soil $\mu$. The stress from the
  soil skeleton is known as the effective stress $\sigma'$ and is the
  difference between the total stress $\sigma$ and the pore stress
  $\mu$. \\

  \hline Source & \cite{FredlundKrahn}\\
  
  \hline Ref.\ By & \dref{GD_P}, \dref{GD_MobShear}, \ddref{DD_R},
  \ddref{DD_T}, \iref{IM_E}\\
  
  \hline
\end{tabular}
\end{minipage}
% ----------------------------- %
%        End TM Effective Stress             %
% ----------------------------- %

~\newline

% ----------------------------- %
%      Begin TM Hookes         %
% ----------------------------- %
\noindent
\begin{minipage}{\textwidth}
\renewcommand*{\arraystretch}{1.5}
\begin{tabular}{| p{1.5cm} | p{14cm}|}
  
  \hline \rowcolor{AFB} Number&
  T\refstepcounter{theorynum}\thetheorynum \label{TM_Hooke}\\
  
  \hline Label&\bf Hooke's Law\\
  
  \hline Equation& \( {F}= {K} \cdot {\delta} \) \\
  
  \hline Description & Stiffness $K$ is the resistance a body offers
  to deformation by displacement $\delta$ when subject to a force $F$,
  along the same direction. A body with high stiffness will experience
  little deformation when subject to a force. \\

  \hline Source & \cite{StolleGuo}\\

  \hline Ref.\ By & \dref{GD_NetForce}, \dref{GD_Hookes},
  \ddref{DD_Stiff}, \iref{IM_RFEM}, \iref{IM_RFEMFS}\\
  
  \hline
\end{tabular}
\end{minipage}\\


% ----------------------------- %
%        End TM Hookes         %
% ----------------------------- %

\subsubsection{General Definitions} \label{sec_gendef}

This section collects the laws and equations that will be used in
deriving the data definitions, which in turn are used to build the
instance models.

~\newline

% ------------------------------------------ %
%        Begin GD Force Equilibrium X     %
% ------------------------------------------ %
\noindent
\begin{minipage}{\textwidth}
\renewcommand*{\arraystretch}{1.5}
\begin{tabular}{| p{1.5cm} | p{14cm}|}
  
  \hline \rowcolor{Amethyst} Number&
  GD\refstepcounter{defnum}\thedefnum \label{GD_Fx}\\
  
  \hline Label&\bf Normal Force Equilibrium\\
  
  \hline Equation& \( N_{\text{i}} \; = \begin{array}{l} \left[
      W_{\text{i}} - X_{\text{i-1}} + X_{\text{i}} +
      {U_{\text{t,i}}}\;{\cos\left(\beta_{\text{i}}\right)} +
      Q_{\text{i}}\;{\cos\left(\omega_{\text{i}}\right)}
      \right]\cos\left(\alpha_{\text{i}}\right) \\ + \left[
      {-K_{\text{c}}}\;{W_{\text{i}}} - E_{\text{i}} + E_{\text{i-1}}
      - H_{\text{i}} + H_{\text{i-1}} +
      {U_{\text{t,i}}}\;{\sin\left(\beta_{\text{i}}\right)} +
      Q_{\text{i}}\;{\sin\left(\omega_{\text{i}}\right)}
      \right]\sin\left(\alpha_{\text{i}}\right) \end{array} \) \\
 
  \hline Description & For a slice of mass in the slope the force
  equilibrium to satisfy \tref{TM_Eqm} in the direction perpendicular
  to the base surface of the slice. Rearranged to solve for the normal
  force of the surface $N_{\text{i}}$. Force equilibrium is derived
  from the free body diagram of \fref{Fig_Forces} in
  section~\ref{sec_system}. Index $\text{i}$ refers to the values of
  the properties for slice/interslices following convention in
  \fref{Fig_Index} in section~\ref{sec_system}. Force variable
  definitions can be found in \ddref{DD_W} to \ddref{DD_EX}.  \\

  \hline Source & \cite{ZhuEtAl2005}\\
  
  \hline Ref.\ By & \ddref{DD_R}, \ddref{DD_T}, \iref{IM_E}\\
  
  \hline
\end{tabular}
\end{minipage}\\
% ------------------------------------------ %
%        End  GD Force Equilibrium X       %
% ------------------------------------------ %

~\newline

% ------------------------------------------ %
%        Begin GD Force Equilibrium Y     %
% ------------------------------------------ %
\noindent
\begin{minipage}{\textwidth}
\renewcommand*{\arraystretch}{1.5}
\begin{tabular}{| p{1.5cm} | p{14cm}|}
  
  \hline \rowcolor{Amethyst} Number&
  GD\refstepcounter{defnum}\thedefnum \label{GD_Fy}\\
  
  \hline Label&\bf Base Shear Force Equilibrium\\
  
  \hline Equation& \( S_{\text{i}} = \begin{array}{l} \left[
      W_{\text{i}} -X_{\text{i-1}} + X_{\text{i}} +
      {U_{\text{t,i}}}\;{\cos\left(\beta_{\text{i}}\right)} +
      Q_{\text{i}}\;{\cos\left(\omega_{\text{i}}\right)}
      \right]\sin\left(\alpha_{\text{i}}\right) \\ - \left[
      {-K_{\text{c}}}\;{W_{\text{i}}} - E_{\text{i}} + E_{\text{i-1}}
      - H_{\text{i}} + H_{\text{i-1}} +
      {U_{\text{t,i}}}\;{\sin\left(\beta_{\text{i}}\right)} +
      Q_{\text{i}}\;{\cos\left(\omega_{\text{i}}\right)}
      \right]\cos\left(\alpha_{\text{i}}\right) \end{array} \) \\
  
  \hline Description & For a slice of mass in the slope the force
  equilibrium to satisfy \tref{TM_Eqm} in the direction parallel to
  the base surface of the slice. Rearranged to solve for the shear
  force acting on the base $S_{\text{i}}$. Force equilibrium is
  derived from the free body diagram of \fref{Fig_Forces} in
  section~\ref{sec_system}. Index $\text{i}$ refers to the values of
  the properties for slice/interslices following convention in
  \fref{Fig_Index} in section~\ref{sec_system}. Force variable
  definitions can be found in \ddref{DD_W} to \ddref{DD_EX}.  \\

  \hline Source & \cite{ZhuEtAl2005}\\
  
  \hline Ref.\ By & \ddref{DD_R}, \ddref{DD_T},\iref{IM_E}\\
  
  \hline
\end{tabular}
\end{minipage}\\
% ------------------------------------------ %
%        End  GD Force Equilibrium Y       %
% ------------------------------------------ %

~\newline

% ----------------------------- %
%        Begin GD Resistive Shear   %
% ----------------------------- %
\noindent
\begin{minipage}{\textwidth}
\renewcommand*{\arraystretch}{1.5}
\begin{tabular}{| p{1.5cm} | p{14cm}|}
  
  \hline \rowcolor{Amethyst} Number&
  GD\refstepcounter{defnum}\thedefnum \label{GD_P}\\
  
  \hline Label&\bf Resistive Shear \\
  
  \hline Equation& \( P_{\text{i}} = N'_{\text{i}} \cdot \tan\left(
  \varphi'_{\text{i}} \right) + c' \cdot b_{\text{i}} \cdot
  \sec\left(\alpha_{\text{i}}\right) \) \\
  
  \hline Description & The Mohr-Coulomb resistive shear strength of a
  slice $P_{\text{i}}$ is adjusted to account for the effective normal
  $\sigma'=N'=N-U_{\text{b}}$ of a soil from \tref{TM_EffStress}.
  Also and the cohesion is adjusted to account for the length $\ell$
  of the plane where the normal occurs, where $\ell_{\text{b,i}} =
  b_{\text{i}} \cdot \sec\left(\alpha\right)$, and $b_\text{i}$ is the
  x width of the base.  therefore $c = c' \cdot b_{\text{i}} \cdot
  \sec\left(\alpha_{\text{i}}\right)$. \\

  \hline Source & \cite{ZhuEtAl2005}\\
  
  \hline Ref.\ By & \dref{GD_MobShear}, \ddref{DD_R}, \ddref{DD_T}\\
  
  \hline
\end{tabular}
\end{minipage}\\
% ----------------------------- %
%        End GD Resistive Shear             %
% ----------------------------- %

~\newline

% ----------------------------- %
%  Begin GD Mobile Shear   %
% ----------------------------- %
\noindent
\begin{minipage}{\textwidth}
\renewcommand*{\arraystretch}{1.5}
\begin{tabular}{| p{1.5cm} | p{14cm}|}
  
  \hline \rowcolor{Amethyst} Number&
  GD\refstepcounter{defnum}\thedefnum \label{GD_MobShear}\\
  
  \hline Label&\bf Mobile Shear \\
  
  \hline Equation & \( S_{\text{i}} = \frac{ P_{\text{i}} }{ \text{FS}
  } = \frac { N'_{\text{i}} \cdot \tan\left( \varphi'_{\text{i}}
    \right) + c' \cdot b_{\text{i}} \cdot
    \sec\left(\alpha_{\text{i}}\right) }{\text{FS}} \) \\
  
  \hline Description & From the definition of the Factor of Safety in
  \tref{TM_FS}, and the new definition of $P_{\text{i}}$, a new
  relation for the net mobile shear force of the slice $T_{\text{i}}$
  is found as the resistive shear $P_{\text{i}}$ (\dref{GD_P}) divided
  by the factor of safety $\text{FS}$. \\

  \hline Source & \cite{ZhuEtAl2005}\\
  
  \hline Ref.\ By & \ddref{DD_R}, \ddref{DD_T} \\
  
  \hline
\end{tabular}
\end{minipage}\\
% ----------------------------- %
%        End GD Mobile Shear             %
% ----------------------------- %

~\newline

% ------------------------------------------ %
%        Begin GD  Interslice Normal/Shear Relationship    %
% ------------------------------------------ %
\noindent
\begin{minipage}{\textwidth}
\renewcommand*{\arraystretch}{1.5}
\begin{tabular}{| p{1.5cm} | p{14cm}|}
  
  \hline \rowcolor{Amethyst} Number&
  GD\refstepcounter{defnum}\thedefnum \label{GD_X}\\
  
  \hline Label&\bf Interslice Normal/Shear Relationship\\
  
  \hline Equation& \( X_\text{i} = \lambda \cdot f_\text{i} \cdot
  E_\text{i} \) \\

  \hline Description & The assumption for the Morgenstern Price method
  (\aref{A_Base}) that the interslice shear force $X_\text{i}$ is
  proportional to the interslice normal force $E_\text{i}$ by a
  proportionality constant $\lambda$, and a predetermined scaling
  function $f$, that changes the proportionality as a function of the
  $x$-ordinate position of the interslice. $f$ is typically either a
  half-sine along the slip surface, or a constant.  \\

  \hline Source & \cite{ZhuEtAl2005}\\
  
  \hline Ref.\ By & \ddref{DD_R}, \ddref{DD_T}, \iref{IM_FS},
  \iref{IM_Lambda}, \iref{IM_E}\\
  
  \hline
\end{tabular}
\end{minipage}\\
% ------------------------------------------ %
%        End  GD Interslice Normal/Shear Relationship       %
% ------------------------------------------ %

~\newline

% ------------------------------------------ %
%        Begin GD Moment Equilibrium    %
% ------------------------------------------ %
\noindent
\begin{minipage}{\textwidth}
\renewcommand*{\arraystretch}{1.5}
\begin{tabular}{| p{1.5cm} | p{14cm}|}
  
  \hline \rowcolor{Amethyst} Number&
  GD\refstepcounter{defnum}\thedefnum \label{GD_M}\\
  
  \hline Label&\bf Moment Equilibrium\\
  
  \hline Equation& \( 0 = \begin{array}{l} - {E}_{\text{i}} \left[
      {z_{\text{i}}} - \frac{b_{\text{i}}}{2} {
        \tan\left(\alpha_{\text{i}}\right)} \right] + {E}_{\text{i-1}}
    \left[ {z_{\text{i-1}}} + \frac{b_{\text{i}}}{2} {
        \tan\left(\alpha_{\text{i}}\right)} \right] -
    H_{\text{i}}\left[ z_{\text{w,i}} - \frac{b_{\text{i}}}{2} {
        \tan\left(\alpha_{\text{i}}\right)} \right] \\[5pt] +
    H_{\text{i-1}}\left[ z_{\text{w,i-1}} + \frac{b_{\text{i}}}{2} {
        \tan\left(\alpha_{\text{i}}\right)} \right] -
    \frac{b_{\text{i}}}{2} \left( X_{\text{i}} + X_{\text{i-1}}
    \right) + K_{\text{c}} W_{\text{i}} \frac{h_{\text{i}}}{2} -
    U_{\text{t,i}} \sin\left(\beta_{\text{i}}\right) h_{\text{i}} \\ -
    Q_{\text{i}}\;{\sin\left(\omega_{\text{i}}\right)}
    h_{\text{i}} \end{array} \) \\

  \hline Description & For a slice of mass in the slope the moment
  equilibrium to satisfy \tref{TM_Eqm} in the direction perpendicular
  to the base surface of the slice. Moment equilibrium is derived from
  the free body diagram of \fref{Fig_Forces} in
  section~\ref{sec_system}. Index $\text{i}$ refers to the values of
  the properties for slice/interslices following convention in
  \fref{Fig_Index} in section~\ref{sec_system}.  Variable
  definitions can be found in \ddref{DD_W} to \ddref{DD_EX}.  \\

  \hline Source & \cite{ZhuEtAl2005}\\
  
  \hline Ref.\ By & \iref{IM_Lambda}\\
  
  \hline
\end{tabular}
\end{minipage}\\
% ------------------------------------------ %
%        End  GD Moment Equilibrium       %
% ------------------------------------------ %

~\newline

% ------------------------------------------ %
%        Begin GD Net Force      %
% ------------------------------------------ %
\noindent
\begin{minipage}{\textwidth}
\renewcommand*{\arraystretch}{1.5}
\begin{tabular}{| p{1.5cm} | p{14cm}|}
  
  \hline \rowcolor{AFB} Number&
  GD\refstepcounter{defnum}\thedefnum \label{GD_NetForce}\\
  
  \hline Label&\bf Net Force\\
  
  \hline Equation & \(F_{\text{x,i}} = -\Delta H_\text{i} -K_{\text{c}}
  \cdot W_\text{i} - U_{\text{b,i}} \sin\left(\alpha_\text{i}\right) +
  U_{\text{t,i}} \sin\left(\beta\right) + Q_\text{i}
  \sin\left(\omega_\text{i}\right)\)
  
~\newline \(F_{\text{y,i}} = -W_\text{i} + U_{\text{b,i}}
  \cos\left(\alpha_\text{i}\right) - U_{\text{t,i}}
  \cos\left(\beta_\text{i}\right) - Q_\text{i}
  \cos\left(\omega_\text{i}\right) \)\\

  \hline Description & The net sum of forces acting on a slice for the
  RFEM model. The forces that create an applied load on the slice.
  $F_\text{x,i}$ refers to the load in the direction perpendicular to
  the direction of the force of gravity for slice $i$, while
  $F_\text{y,i}$ refers to the load in the direction parallel to the
  force of gravity for slice $i$.  Forces are found in the free
  body diagram of \fref{Fig_Forces} in section~\ref{sec_system}. In
  this model the elements are not exerting forces on each other, so
  the interslice forces $E$ and $X$ are not a part of the model. Index
  $\text{i}$ refers to the values of the properties for
  slice/interslices following convention in \fref{Fig_Index} in
  section~\ref{sec_system}. Force variable definitions can be found in
  \ddref{DD_W} to \ddref{DD_Q}.\\

  \hline Source & \cite{ZhuEtAl2005}\\
  
  \hline Ref.\ By & \ddref{DD_Eqm},
  \iref{IM_RFEM}\\
  
  \hline
\end{tabular}
\end{minipage}\\

% ------------------------------------------ %
%        End  GD Net Force       %
% ------------------------------------------ %

~\newline

% ------------------------------------------ %
%        Begin GD Hookes      %
% ------------------------------------------ %
\noindent
\begin{minipage}{\textwidth}
\renewcommand*{\arraystretch}{1.5}
\begin{tabular}{| p{3cm} | p{12.5cm}|}
  
  \hline \rowcolor{AFB} Number&
  GD\refstepcounter{defnum}\thedefnum \label{GD_Hookes}\\
  
  \hline Label&\bf Hooke's Law 2D\\
  
  \hline Equation&

  \( \left[\begin{array}{l} p_\text{t,i} \\ p_\text{n,i} \end{array}
    \right] = \left[ \begin{array}{l l} K_\text{t,i} & 0 \\ 0 &
      K_\text{n,i}\end{array} \right] \left[\begin{array}{l} \delta
      t_\text{i} \\ \delta n_\text{i}
    \end{array} \right] \) \\

  \hline Description & A 2D component implementation of Hooke's law as
  seen in \tref{TM_Hooke}. $\delta n_\text{i}$ is the displacement of
  the element normal to the surface and $\delta t_\text{i}$ is the
  displacement of the element parallel to the surface. $p_\text{n,i}$,
  is the net pressure acting normal to the surface, and $p_\text{t,i}$
  is the net pressure acting parallel to the surface. Pressure is used
  in place of force as the surface has not been normalized for it's
  length. The stiffness values $K_\text{n,i}$ and $K_\text{t,i}$ are
  then the resistance to displacement in the respective directions
  defined as in \ddref{DD_Stiff}. The pressure forces would be the
  result of applied loads on the mass, the product of the stiffness
  elements with the displacement would be the mass's reactive force
  that creates equilibrium with the applied forces after reaching the
  equilibrium displacement. \\

  \hline Source & \cite{StolleGuo}\\
  
  \hline Ref.\ By & \ddref{DD_KMats},
  \iref{IM_RFEM}\\
  
  \hline
\end{tabular}
\end{minipage}\\

% ------------------------------------------ %
%        End  GD Hookes       %
% ------------------------------------------ %

~\newline

% ------------------------------------------ %
%        Begin GD Displacement Vectors      %
% ------------------------------------------ %
\noindent
\begin{minipage}{\textwidth}
\renewcommand*{\arraystretch}{1.5}
\begin{tabular}{| p{3cm} | p{12.5cm}|}
  
  \hline \rowcolor{AFB} Number&
  GD\refstepcounter{defnum}\thedefnum \label{GD_DispVecs}\\
  
  \hline Label&\bf Displacement Vectors\\
  
  \hline Equation&

  \( \bar{\delta}_\text{i} = \left[ \begin{array}{l} \delta x_\text{i}
      \\ \delta y_\text{i} \end{array} \right]\)
  ~\newline ~\newline
  \( \bar{\epsilon}_\text{i} = \left[ \begin{array}{l} \delta
      u_\text{i} \\ \delta v_\text{i} \end{array} \right]
  = \left[ \begin{array}{l l} \cos\left(\alpha_\text{i}\right) &
      \sin\left(\alpha_\text{i}\right)\\ -\sin\left(\alpha_\text{i}\right)
      & \cos\left(\alpha_\text{i}\right)
    \end{array}\right] \bar{\delta_\text{i}} \)\\

      \hline Description & Vectors describing the displacement of
      slice $\text{i}$. $\bar{\delta}_\text{i}$ is the displacement in
      the unrotated coordinate system, where $\delta x_\text{i}$ is
      the displacement of the slice perpendicular to the direction of
      gravity, and $\delta y_\text{i}$ is the displacement of the
      slice parallel to the force of
      gravity. $\bar{\epsilon}_\text{i}$ is the displacement in the
      rotated coordinate system, where $\delta u_\text{i}$ is the
      displacement of the slice parallel to the slice base, and
      $\delta y_\text{i}$ is the displacement of the slice
      perpendicular to the slice base. $\bar{\epsilon_\text{i}}$ can
      also be found by rotating $\bar{\delta_\text{i}}$ clockwise by
      the base angle $\alpha$ through a rotation matrix as shown.\\

  \hline Source & \cite{StolleGuo}\\
  
  \hline Ref.\ By & \ddref{DD_KMats},
  \iref{IM_RFEM}, \iref{IM_RFEMFS}\\
  
  \hline
\end{tabular}
\end{minipage}\\

% ------------------------------------------ %
%        End  GD Displacement Vectors       %
% ------------------------------------------ %

\subsubsection{Data Definition} \label{sec_datadef}

This section collects and defines all the data needed to build the
instance models. Definitions \ddref{DD_W} to \ddref{DD_Q} are the
force variables that can be solved by direct analysis of given
inputs. The interslice forces \ddref{DD_EX} are force variables that
must be written in terms of \ddref{DD_W} to \ddref{DD_Q} to solve.
~\newline

% ------------------------------------------ %
%               Begin DD W    %
% ------------------------------------------ %

\noindent
\begin{minipage}{\textwidth}
\renewcommand*{\arraystretch}{1.6}
\begin{tabular}{| p{1.5cm} | p{14cm} |}
  
\hline \rowcolor{Brass} Number&
DD\refstepcounter{datadefnum}\thedatadefnum \label{DD_W}\\

\hline Label& \bf Slice Weight \\

\hline
Equation & 
\( W_\text{i} =  b_\text{i} \cdot \left\{  \begin{array}{l l}
  \left[ y_\text{us,i} - y_\text{slip,i} \right] \cdot
  \gamma_\text{Sat} & \text{If } y_\text{wt,i} \geq y_\text{us,i}
  \\ \left[ y_\text{us,i} - y_\text{wt,i} \right] \cdot \gamma +
  \left[ y_\text{wt,i} - y_\text{slip,i} \right] \cdot
  \gamma_\text{Sat} & \text{If } y_\text{us,i} > y_\text{wt,i} >
  y_\text{slip,i} \\ \left[ y_\text{us,i} - y_\text{slip,i} \right]
  \cdot \gamma & \text{If } y_\text{wt,i} \leq y_\text{slip,i} \\
 \end{array} \right. \) \\

\hline Description & The weight of the slice $W$ is how much force the
soil mass exerts into the slip surface.  The piecewise function is the
unit weight of the slice. Wet regions are determined by
($y_\text{wt,i} > y_\text{slip,i}$), where $y_\text{wt,i}$ is the
height of the water table and $y_\text{slip,i}$ is the height of the
slip surface, at the midpoint of slice $\text{i}$. Dry regions are
determined by ($y_\text{us,i} > y_\text{slip,i}$), where
$y_\text{us,i}$ is the height of the slice surface at the midpoint of
slice $\text{i}$. The height of weight regions is multiplied by weight
of wet soil $\gamma_\text{Sat}$, and dry regions are multiplied by the
weight of dry soil $\gamma$. the sum is the total unit weight of the
slice. The unit weight is multiplied with the width of the slice
$b_\text{i}$(\ddref{DD_b}) to get the total weight of the slice.\\

\hline Sources& \cite{FredlundKrahn}\\

\hline Ref.\ By & \ddref{DD_R}, \ddref{DD_T}, \iref{IM_FS},
\iref{IM_Lambda}, \iref{IM_E}\\

\hline
\end{tabular}
\end{minipage}\\

% ------------------------------------------ %
%               End DD W        %
% ------------------------------------------ %

~\newline

% ------------------------------------------ %
%               Begin DD Ub    %
% ------------------------------------------ %

\noindent
\begin{minipage}{\textwidth}
\renewcommand*{\arraystretch}{1.6}
\begin{tabular}{| p{1.5cm} | p{14cm} |}
  
\hline \rowcolor{Brass} Number&
DD\refstepcounter{datadefnum}\thedatadefnum \label{DD_Ub}\\

\hline Label& \bf Base Water Force \\

\hline
Equation & 
\( U_\text{b,i} =  \ell_\text{b,i} \cdot \left\{  \begin{array}{l l}
\left[ y_\text{wt,i} - y_\text{slip,i} \right] \cdot \gamma_\text{w} &
\text{If } y_\text{wt,i} > y_\text{slip,i} \\ 0 & \text{If }
y_\text{wt,i} \leq y_\text{slip,i} \\
 \end{array} \right. \)
\\

\hline Description & The base water force $U_\text{b,i}$ is how much
force the water contributes to the total normal force, the variable
$\mu$ from \tref{TM_EffStress}. If the slice contains water
$y_\text{wt,i} > y_\text{slip,i}$, where $y_\text{wt,i}$ is the height
of the water table and $y_\text{slip,i}$ is the height of the slip
surface at the midpoint of slice $\text{i}$,then the water will
contribute to the normal force of the slice. The unit base water force
is the product of the height of the water above the slip surface with
the unit weight of water $\gamma_\text{w}$. The unit base water force
is normalized by multiplying by the length of the base of the slice
$\ell_\text{b,i}$, to get the total base water force.  \\

\hline Sources& \cite{FredlundKrahn}\\

\hline Ref.\ By & \ddref{DD_R}, \ddref{DD_T}, \iref{IM_FS},
\iref{IM_Lambda}, \iref{IM_E}\\

\hline
\end{tabular}
\end{minipage}\\

% ------------------------------------------ %
%               End DD Ub        %
% ------------------------------------------ %

~\newline

% ------------------------------------------ %
%               Begin DD Ut    %
% ------------------------------------------ %

\noindent
\begin{minipage}{\textwidth}
\renewcommand*{\arraystretch}{1.6}
\begin{tabular}{| p{1.5cm} | p{14cm} |}
  
\hline \rowcolor{Brass} Number&
DD\refstepcounter{datadefnum}\thedatadefnum \label{DD_Ut}\\

\hline Label& \bf Surface Water Force \\

\hline
Equation & 
\( U_\text{t,i} = \ell_\text{s,i} \cdot \left\{  \begin{array}{l l}
  \left[ y_\text{wt,i} - y_\text{us,i} \right] \cdot \gamma_\text{w} &
  \text{If } y_\text{wt,i} > y_\text{us,i} \\ 0 & \text{If }
  y_\text{wt,i} \leq y_\text{us,i} \\
 \end{array} \right. \) \\

\hline Description & The surface water force $U_\text{t,i}$, exerting
a force downwards into the slice. Occurs when standing water is
resting on the surface of the slope, such that. $y_\text{wt,i} >
y_\text{us,i}$, where $y_\text{wt,i}$ is the height of the water table
and $y_\text{us,i}$ is the height of the slope surface at the midpoint
of slice $\text{i}$. The unit surface water force is the product of
the height the water is above the surface with the unit weight of
water $\gamma_\text{w}$. The unit surface water force is normalized by
multiplying by the length of the surface of the slice
$\ell_\text{s,i}$, to get the total base water force.  \\

\hline Sources & \cite{FredlundKrahn}\\

\hline Ref.\ By & \ddref{DD_R}, \ddref{DD_T}, \iref{IM_FS},
\iref{IM_Lambda}, \iref{IM_E}\\

\hline
\end{tabular}
\end{minipage}\\

% ------------------------------------------ %
%               End DD Ut        %
% ------------------------------------------ %

~\newline

% ------------------------------------------ %
%               Begin DD H    %
% ------------------------------------------ %

\noindent
\begin{minipage}{\textwidth}
\renewcommand*{\arraystretch}{1.6}
\begin{tabular}{| p{1.5cm} | p{14cm} |}
  
\hline \rowcolor{Brass} Number&
DD\refstepcounter{datadefnum}\thedatadefnum \label{DD_H}\\

\hline Label& \bf Interslice Water Forces \\

\hline Equation & \( H_\text{i} = \left\{ \begin{array}{l l}
  \frac{\left[ y_\text{us,i} - y_\text{slip,i} \right] ^2 }{2} \cdot
  \gamma_\text{Sat} + \left[ y_\text{wt,i} - y_\text{us,i} \right] ^2
  \cdot \gamma_\text{Sat} & \text{If } y_\text{wt,i} \geq
  y_\text{us,i} \\ \frac{\left[ y_\text{wt,i} - y_\text{slip,i}
      \right] ^2 }{2} \cdot \gamma_\text{Sat} & \text{If }
  y_\text{us,i} > y_\text{wt,i} > y_\text{slip,i} \\ 0 & \text{If }
  y_\text{wt,i} \leq y_\text{slip,i} \\ \end{array} \right. \)\\

\hline Description & The interslice water force $H_\text{i}$ is a
force created by water acting horizontally from an interslice
$\text{i}$ onto adjacent slice.  \\

\hline Sources & \cite{FredlundKrahn}\\

\hline Ref.\ By & \ddref{DD_R}, \ddref{DD_T}, \iref{IM_FS},
\iref{IM_Lambda}, \iref{IM_E}\\

\hline
\end{tabular}
\end{minipage}\\

% ------------------------------------------ %
%               End DD H        %
% ------------------------------------------ %

~\newline

% ------------------------------------------ %
%               Begin DD Angles   %
% ------------------------------------------ %

\noindent
\begin{minipage}{\textwidth}
\renewcommand*{\arraystretch}{1.6}
\begin{tabular}{| p{1.5cm} | p{14cm} |}
  
\hline \rowcolor{Brass} Number&
DD\refstepcounter{datadefnum}\thedatadefnum \label{DD_Angles}\\

\hline Label& \bf Angles \\

\hline
Equation & 
\( \alpha_\text{i} = \frac{y_\text{slip,i} -
  y_\text{slip,i-1}}{x_\text{slip,i} - x_\text{slip,i-1}} \),
\( \beta_\text{i} = \frac{y_\text{us,i} -
  y_\text{us,i-1}}{x_\text{us,i} - x_\text{us,i-1}} \)\\

\hline
Description & The angle the slip surface (slice base) and slope
surface (slice surface) make with the horizontal.  Uses approximation
\aref{A_Straight} that the base and surface of slices between nodes
are straight lines.  \\

\hline Sources& \cite{FredlundKrahn}\\

\hline Ref.\ By & \ddref{DD_R}, \ddref{DD_T}, \iref{IM_FS},
\iref{IM_Lambda}, \iref{IM_E}\\

\hline
\end{tabular}
\end{minipage}\\

% ------------------------------------------ %
%               End DD Angles        %
% ------------------------------------------ %

~\newline

% ------------------------------------------ %
%               Begin DD b   %
% ------------------------------------------ %

\noindent
\begin{minipage}{\textwidth}
\renewcommand*{\arraystretch}{1.6}
\begin{tabular}{| p{1.5cm} | p{14cm} |}
  
\hline \rowcolor{Brass} Number&
DD\refstepcounter{datadefnum}\thedatadefnum \label{DD_b}\\

\hline Label& \bf Lengths \\

\hline
Equation & 
\( b_\text{i} = x_\text{slip,i} - x_\text{slip,i-1} \),
\( \ell_\text{b,i} = b_\text{i} \cdot \sec\left(\alpha_\text{i}\right)
\)
\( \ell_\text{s,i} = b_\text{i} \cdot \sec\left(\beta_\text{i}\right)
\)\\

\hline Description & The width of the slice $b_\text{i}$ is the
difference in x-ordinates between $x_\text{slip}$ nodes surrounding
the slice.  Using the slice base angle $\alpha_\text{i}$ and surface
angle $\beta_\text{i}$ from \ddref{DD_Angles}, and Pythagoreans
theorem the total length of the slice base $\ell_\text{b,i}$ or slice
surface $\ell_\text{s,i}$ can be calculated.  \\

\hline Sources& \cite{FredlundKrahn}\\

\hline Ref.\ By & \ddref{DD_R}, \ddref{DD_T}, \iref{IM_FS},
\iref{IM_Lambda}, \iref{IM_E}\\

\hline
\end{tabular}
\end{minipage}\\

% ------------------------------------------ %
%               End DD b        %
% ------------------------------------------ %

~\newline

% ------------------------------------------ %
%               Begin DD Kc  %
% ------------------------------------------ %

\noindent
\begin{minipage}{\textwidth}
\renewcommand*{\arraystretch}{1.6}
\begin{tabular}{| p{1.5cm} | p{14cm} |}
  
\hline \rowcolor{Brass} Number&
DD\refstepcounter{datadefnum}\thedatadefnum \label{DD_Kc}\\

\hline Label& \bf Seismic Load Factor\\

\hline Equation & \(K_\text{E,i} = K_\text{c} W_\text{i} \) \\

\hline Description & Input $K_\text{c}$ is the seismic load factor. It
represents the proportion of weight that slice $\text{i}$ will exert
outward as a result of horizontal motion of the earth dues to
earthquakes.  The force $K_\text{E,i}$ will be the product of the load
factor $K_\text{c}$ and the weight of the slice $W_\text{i}$ from
\ddref{DD_W}.  \\

\hline Sources& \cite{FredlundKrahn}\\

\hline Ref.\ By & \ddref{DD_R}, \ddref{DD_T}, \iref{IM_FS},
\iref{IM_Lambda}, \iref{IM_E}\\

\hline
\end{tabular}
\end{minipage}\\

% ------------------------------------------ %
%               End DD Kc        %
% ------------------------------------------ %

~\newline

% ------------------------------------------ %
%               Begin DD Q  %
% ------------------------------------------ %

\noindent
\begin{minipage}{\textwidth}
\renewcommand*{\arraystretch}{1.6}
\begin{tabular}{| p{1.5cm} | p{14cm} |}
  
\hline \rowcolor{Brass} Number&
DD\refstepcounter{datadefnum}\thedatadefnum \label{DD_Q}\\

\hline Label& \bf Surface Loads \\

\hline Equation & \( Q_\text{i}, \omega_\text{i} \)\\

\hline Description & Inputs $Q_\text{i}$ and $\omega_\text{i}$ are the
results of a slope surface load from an external weight such as a
building on the slope. $Q_\text{i}$ is the magnitude of the surface
load being exerted on slice $\text{i}$ and $\omega$ is the angle the
force is being exerted at relative to the vertical (line parallel to
the direction of the force of gravity).  \\

\hline Sources& \cite{ZhuEtAl2005}\\

\hline Ref.\ By & \ddref{DD_R}, \ddref{DD_T}, \iref{IM_FS},
\iref{IM_Lambda}, \iref{IM_E}\\

\hline
\end{tabular}
\end{minipage}\\

% ------------------------------------------ %
%               End DD Q        %
% ------------------------------------------ %

~\newline

% ------------------------------------------ %
%               Begin DD E,X  %
% ------------------------------------------ %

\noindent
\begin{minipage}{\textwidth}
\renewcommand*{\arraystretch}{1.6}
\begin{tabular}{| p{1.5cm} | p{14cm} |}
  
\hline \rowcolor{Amethyst} Number&
DD\refstepcounter{datadefnum}\thedatadefnum \label{DD_EX}\\

\hline Label& \bf Interslice Forces \\

\hline Equation & \( E_\text{i}, X_\text{i} = \lambda \cdot f_\text{i}
\cdot E_\text{i} \) \\

\hline Description & The interslice forces are the normal and shear
forces occurring on a slice at it's interfaces as a result of the
internal stress on the slices of mass from the weight of the adjacent
slice.  $E_\text{i}$ is the normal force exerted by interslice
$\text{i}$, and $X_\text{i}$ is the shear force exerted by interslice
$\text{i}$. The value of $X_\text{i}$ is determined by it's
relationship to $E_\text{i}$ in \dref{GD_X}. $E_\text{i}$ is one of
the three solution variables, determined by \iref{IM_E}.  \\

\hline Sources& \cite{ZhuEtAl2005}\\

\hline Ref.\ By & \ddref{DD_R}, \ddref{DD_T}, \iref{IM_FS},
\iref{IM_Lambda}, \iref{IM_E}\\

\hline
\end{tabular}
\end{minipage}\\

% ------------------------------------------ %
%               End DD E,X        %
% ------------------------------------------ %

~\newline

% ------------------------------------------ %
%               Begin DD R    %
% ------------------------------------------ %

\noindent
\begin{minipage}{\textwidth}
\renewcommand*{\arraystretch}{1.6}
\begin{tabular}{| p{1.5cm} | p{14cm} |}
  
\hline \rowcolor{Amethyst} Number&
DD\refstepcounter{datadefnum}\thedatadefnum \label{DD_R}\\

\hline Label& \bf Resistive Shear, Without Interslice Forces \\

\hline
Equation & 
\( R_{\text{i}}= \begin{array}{l}
  \left( \begin{array}{l}
    \left[ W_{\text{i}} + U_{\text{t,i}}
      \cos\left(\beta_{\text{i}}\right) + Q_{\text{i}}
      \cos\left(\omega_{\text{i}}\right) \right]
    \cos\left(\alpha_{\text{i}}\right) \\
+ \left[ - K_{\text{c}} W_{\text{i}} - \Delta H_{\text{i}} +
  U_{\text{t,i}} \sin\left(\beta_{\text{i}}\right) + Q_{\text{i}}
  \sin\left(\omega_{\text{i}}\right) \right]
\sin\left(\alpha_{\text{i}}\right) - U_{\text{b,i}} \end{array}
  \right) \cdot \tan\left(\varphi'\right) \\
+ c'_{\text{i}} \cdot b_{\text{i}} \cdot
\sec\left(\alpha_{\text{i}}\right) \end{array} \)\\

\hline Description & The resistive shear of \dref{GD_P}, with the
normal force $N_\text{i}$ defined in terms of the physical properties
of \ddref{DD_W} to \ddref{DD_EX}, without considering the
effects of the interslice forces $E$ and $X$.  \\

\hline Sources& \cite{ZhuEtAl2005}\\

\hline Ref.\ By & \iref{IM_FS}\\

\hline
\end{tabular}
\end{minipage}\\

% ------------------------------------------ %
%               End DD R         %
% ------------------------------------------ %

\subsubsection*{Resistive Shear Force, Without the
  Influence of Interslice Forces Derivation}

\noindent
The resistive shear force of a slice is defined as $P_\text{i}$ in
\dref{GD_P}.  The effective normal in the equation for $P_\text{i}$ of
the soil is defined in the perpendicular force equilibrium of a slice
from \dref{GD_Fy}, Using the effective normal $N'_\text{i}$ of
\tref{TM_EffStress} shown in equation~(\ref{eq:N'}).

\begin{equation} \label{eq:N'}
 N'_{\text{i}} \; = \begin{array}{l}
   \left[ W_{\text{i}} - X_{\text{i-1}} + X_{\text{i}} +
     {U_{\text{t,i}}}\;{\cos\left(\beta_{\text{i}}\right)} +
     Q_{\text{i}}\;{\cos\left(\omega_{\text{i}}\right)}
     \right]\cos\left(\alpha_{\text{i}}\right) \\ + \left[
     {-K_{\text{c}}}\;{W_{\text{i}}} - E_{\text{i}} + E_{\text{i-1}} -
     H_{\text{i}} + H_{\text{i-1}} +
     {U_{\text{t,i}}}\;{\sin\left(\beta_{\text{i}}\right)} +
     Q_{\text{i}}\;{\sin\left(\omega_{\text{i}}\right)}
     \right]\sin\left(\alpha_{\text{i}}\right) \\ -
   U_{\text{b,i}} \end{array}
 \end{equation}

\noindent
The values of the interslice forces $E$ and $X$ in the equation are
unknown, while the other values are found from the physical force
definitions of \ddref{DD_W} to \ddref{DD_EX}.  Consider a force
equilibrium without the affect of interslice
forces, to obtain a solvable value as done for $N^*_\text{i}$ in
equation~(\ref{eq:N*}).

\begin{equation} \label{eq:N*}
 N^*_{\text{i}} \; = \begin{array}{l}
   \left[ W_{\text{i}} +
     {U_{\text{t,i}}}\;{\cos\left(\beta_{\text{i}}\right)} +
     Q_{\text{i}}\;{\cos\left(\omega_{\text{i}}\right)}
     \right]\cos\left(\alpha_{\text{i}}\right) \\ + \left[
     {-K_{\text{c}}}\;{W_{\text{i}}}- H_{\text{i}} + H_{\text{i-1}} +
     {U_{\text{t,i}}}\;{\sin\left(\beta_{\text{i}}\right)} +
     Q_{\text{i}}\;{\sin\left(\omega_{\text{i}}\right)}
     \right]\sin\left(\alpha_{\text{i}}\right) -
   U_{\text{b,i}} \end{array}
\end{equation}

\noindent
Using $N^*_\text{i}$, a resistive shear force neglecting the influence
of interslice forces can be solved for in terms of all known values as
done in equation~(\ref{eq:R}).

\begin{equation*}
R_\text{i} = N^*_\text{i} \tan\left(\varphi'\right) + c'_\text{i}
\cdot b'_\text{i} \sec\left(\alpha_text{i}'\right)
\end{equation*}

\begin{equation}\label{eq:R}   R_{\text{i}} \; =
  \left( \begin{array}{l} \left[ W_{\text{i}} + U_{\text{t,i}}
      \cos\left(\beta_{\text{i}}\right) + Q_{\text{i}}
      \cos\left(\omega_{\text{i}}\right) \right]
    \cos\left(\alpha_{\text{i}}\right) \\ + \left[ - K_{\text{c}}
      W_{\text{i}} - \Delta H_{\text{i}} + U_{\text{t,i}}
      \sin\left(\beta_{\text{i}}\right) + Q_{\text{i}}
      \sin\left(\omega_{\text{i}}\right) \right]
    \sin\left(\alpha_{\text{i}}\right) - U_{\text{b,i}} \end{array}
  \right) \cdot \tan\left(\varphi'\right) + c'_{\text{i}} \cdot
  b_{\text{i}} \cdot \sec\left(\alpha_{\text{i}}\right)
 \end{equation}

~\newline

% ------------------------------------------ %
%               Begin DD T    %
% ------------------------------------------ %

\noindent
\begin{minipage}{\textwidth}
\renewcommand*{\arraystretch}{1.6}
\begin{tabular}{| p{1.5cm} | p{14cm} |}
  
\hline
\rowcolor{Amethyst}
Number& DD\refstepcounter{datadefnum}\thedatadefnum \label{DD_T}\\

\hline
Label& \bf Mobile Shear, Without Interslice Forces \\

\hline
Equation &
\( T_{\text{i}} = \begin{array}{l} \left[ W_{\text{i}} +
    U_{\text{t,i}} \cos\left(\beta_{\text{i}}\right) + Q_{\text{i}}
    \cos\left(\omega_{\text{i}}\right) \right]
  \sin\left(\alpha_{\text{i}}\right)\\ - \left[ - K_{\text{c}} W_{\text{i}} -
    \Delta H_{\text{i}} + U_{\text{t,i}}
    \sin\left(\beta_{\text{i}}\right) + Q_{\text{i}}
    \sin\left(\omega_{\text{i}}\right) \right]
  \cos\left(\alpha_{\text{i}}\right) \end{array} \) \\

\hline Description & The mobile shear of \dref{GD_Fy}, defined in
terms of the physical properties of \ddref{DD_W}, to \ddref{DD_EX} without
considering the effects of the interslice forces $E$ and $X$.\\

\hline
Sources& \cite{ZhuEtAl2005}\\

\hline Ref.\ By & \iref{IM_FS}\\

\hline
\end{tabular}
\end{minipage}\\

% ------------------------------------------ %
%               End DD T         %
% ------------------------------------------ %

\subsubsection*{Mobile Shear Force, Without the
  Influence of Interslice Forces Derivation}

\noindent
The mobile shear force acting on a slice is defined as $S_\text{i}$
from the force equilibrium in \dref{GD_Fy},also shown in
equation~(\ref{eq:Si}).

\begin{equation}  \label{eq:Si}
  S_{\text{i}} = \begin{array}{l} \left[ W_{\text{i}} -X_{\text{i-1}}
      + X_{\text{i}} +
      {U_{\text{t,i}}}\;{\cos\left(\beta_{\text{i}}\right)} +
      Q_{\text{i}}\;{\cos\left(\omega_{\text{i}}\right)}
      \right]\sin\left(\alpha_{\text{i}}\right) \\ - \left[
      {-K_{\text{c}}}\;{W_{\text{i}}} - E_{\text{i}} + E_{\text{i-1}}
      - H_{\text{i}} + H_{\text{i-1}} +
      {U_{\text{t,i}}}\;{\sin\left(\beta_{\text{i}}\right)} +
      Q_{\text{i}}\;{\cos\left(\omega_{\text{i}}\right)}
      \right]\cos\left(\alpha_{\text{i}}\right) \end{array}
\end{equation}

\noindent
The equation is unsolvable, containing the unknown interslice normal
force $E$ and shear force $X$.  Consider a force equilibrium without
the affect of interslice forces, to obtain the mobile shear force
without the influence of interslice forces $T$, as done in
equation~(\ref{eq:T}).n

\begin{equation}  \label{eq:T} T_{\text{i}} =
  \begin{array}{l}  
\left[ W_{\text{i}} + U_{\text{t,i}} \cos\left(\beta_{\text{i}}\right)
  + Q_{\text{i}} \cos\left(\omega_{\text{i}}\right) \right]
\sin\left(\alpha_{\text{i}}\right) \\ - \left[ - K_{\text{c}}
  W_{\text{i}} - \Delta H_{\text{i}} + U_{\text{t,i}}
  \sin\left(\beta_{\text{i}}\right) + Q_{\text{i}}
  \sin\left(\omega_{\text{i}}\right) \right]
\cos\left(\alpha_{\text{i}}\right) \end{array}
\end{equation}

\noindent
The values of $R_\text{i}$ and $T_\text{i}$ are now defined completely
in terms of the known force property values of \ddref{DD_W} to
\ddref{DD_EX}.

~\newline

% ------------------------------------------ %
%               Begin DD K Mats     %
% ------------------------------------------ %

\noindent
\begin{minipage}{\textwidth}
\renewcommand*{\arraystretch}{1.6}
\begin{tabular}{| p{1.5cm} | p{14cm} |}
  
\hline \rowcolor{AFB} Number&
DD\refstepcounter{datadefnum}\thedatadefnum \label{DD_KMats}\\

\hline Label& \bf Displacement Reaction Force \\

\hline Equation & \textbf{Interslice} ~\newline
\( \left[ \begin{array}{l} p_\text{x,i}
        \\ p_\text{y,i} \end{array}\right]
= \bar{K}_\text{s,i} \; \bar{\delta}_\text{i}
= \left[ \begin{array}{l l}
      K_\text{st,i} & 0 \\ 0 & K_\text{sn,i}\end{array} \right]
  \left[\begin{array}{l} \delta x_\text{i} \\ \delta y_\text{i}
    \end{array} \right] \)
~\newline ~\newline \textbf{Basal} ~\newline
\( \left[ \begin{array}{l} p_\text{x,i}
        \\ p_\text{y,i} \end{array}\right]
= \bar{K}_\text{b,i} \; \bar{\delta}_\text{i}
= \left[ \begin{array}{l l}
        K_\text{bA,i} & K_\text{bB,i} \\
        K_\text{bB,i} & K_\text{bA,i}
  \end{array} \right]
\left[\begin{array}{l} \delta x_\text{i} \\ \delta y_\text{i}
  \end{array} \right] \) \\


\hline Description & The force displacement relationship of
\dref{GD_Hookes} for a slice based on interslice or basal surface
forces, generalized for displacement in the unrotated coordinate
system $\bar{\delta}_\text{i}$ from \dref{GD_DispVecs}. Uses the
definitions of shear and normal stiffness from
\ddref{DD_Stiff}. $K_\text{bA,i}$, and $K_\text{bB,i}$ are effective
values for the rotated coordinate system found in
equations~(\ref{eq:KbA}) and~(\ref{eq:KbB}) respectively. \\

\hline Sources& \cite{StolleGuo}\\

\hline Ref.\ By & \iref{IM_RFEM}\\

\hline
\end{tabular}
\end{minipage}\\

% ------------------------------------------ %
%               End DD K Mats         %
% ------------------------------------------ %

\subsubsection*{Derivation of Stifness Matrixes}
\noindent
Using the force-displacement relationship of \dref{GD_Hookes} to
define stiffness matrix $\bar{K_\text{i}}$, as seen in
equation~(\ref{eq:Kmat}).

\begin{equation} \label{eq:Kmat}
  \bar{K_\text{i}} = \left[ \begin{array}{l l}
      K_\text{t,i} & 0 \\ 0 & K_\text{n,i}\end{array} \right]
\end{equation}

\noindent
For interslice surfaces the stiffness constants and displacements
refer to an unrotated coordinate system, $\bar{\delta}_\text{i}$ of
\dref{GD_DispVecs}. The interslice elements are left in their standard
coordinate system, and therefore are described by the same equation
from \dref{GD_Hookes}. Seen as $\bar{K}_\text{s,i}$ in
\ddref{DD_KMats}. $K_\text{st,i}$ is the shear element in the matrix,
and $K_\text{sn,i}$ is the normal element in the matrix, calculated as
in \ddref{DD_Stiff}.

~\newline\noindent For basal surfaces the stiffness constants and
displacements refer to a system rotated for the base angle $\alpha$
(\ddref{DD_Angles}). To analyze the effect of force-displacement
relationships occurring on both basal and interslice surfaces of an
element $i$ they must reference the same coordinate system. The
basal stiffness matrix must be rotated counter clockwise to align with
the angle of the basal surface. The base stiffness counter clockwise
rotation is applied in equation~(\ref{eq:Krot}) to the new matrix
$\bar{K^*_\text{i}}$.


\begin{equation} \label{eq:Krot}
  \begin{aligned} \bar{K^*_\text{i}} & = 
   \left[ \begin{array}{l l} \cos\left(\alpha_\text{i}\right) & -
       \sin\left(\alpha_\text{i}\right)\\ \sin\left(\alpha_\text{i}\right)
       & \cos\left(\alpha_\text{i}\right)\end{array} \right]
   \bar{K_\text{i}}\\ {} & = \left[ \begin{array}{l l} K_\text{bt,i}
       \cos\left(\alpha_\text{i}\right) & - K_\text{bn,i}
       \sin\left(\alpha_\text{i}\right)\\ K_\text{bt,i}
       \sin\left(\alpha_\text{i}\right) & K_\text{bn,i}
       \cos\left(\alpha_\text{i}\right)\end{array}\right]
  \end{aligned} \end{equation}

\noindent
The Hooke's law force displacement relationship of \dref{GD_Hookes}
applied to the base also references a displacement vector
$\bar{\epsilon}_\text{i}$ of \dref{GD_DispVecs} rotated for the base
angle angle of the slice $\alpha_\text{i}$. The basal displacement
vector $\bar{\epsilon}_\text{i}$ is rotated clockwise to align with
the interslice displacement vector $\bar{\delta}_\text{i}$, applying
the definition of $\bar{\epsilon}_\text{i}$ in terms of
$\bar{\delta}_\text{i}$ as seen in \dref{GD_DispVecs}. Using this with
base stiffness matrix $\bar{K^*}_\text{i}$, a basal force displacement
relationship in the same coordinate system as the interslice
relationship can be derived as done in equation~(\ref{eq:pRot}).
    
\begin{equation}\label{eq:pRot} \begin{aligned}
    \left[ \begin{array}{l} p_\text{bx,i}
        \\ p_\text{by,i} \end{array}\right]
    & = \bar{K^*_\text{i}} \; \bar{\epsilon} \\
      {} & = \left[ \begin{array}{l l} K_\text{bt,i}
      \cos\left(\alpha_\text{i}\right) & - K_\text{bn,i}
      \sin\left(\alpha_\text{i}\right)\\  K_\text{bt,i}
      \sin\left(\alpha_\text{i}\right) & K_\text{bn,i}
      \cos\left(\alpha_\text{i}\right)\end{array}\right]
      \left[ \begin{array}{l l} \cos\left(\alpha_\text{i}\right)
          &\sin\left(\alpha_\text{i}\right)\\
          -\sin\left(\alpha_\text{i}\right)
          & \cos\left(\alpha_\text{i}\right)\end{array}\right]
      \left[ \begin{array}{l} \delta x_\text{i}
          \\ \delta y_\text{i} \end{array} \right] \\
      {} & = \left[ \begin{array}{l l}
          K_\text{bt,i} \cos^2\left(\alpha_\text{i}\right)+
          K_\text{bn,i} \sin^2\left(\alpha_\text{i}\right)
          & \left(K_\text{bt,i} - K_\text{bn,i}\right)
          \sin\left(\alpha_\text{i}\right)
          \cos\left(\alpha_\text{i}\right)\\
          \left(K_\text{bt,i} - K_\text{bn,i}\right)
          \sin\left(\alpha_\text{i}\right)
          \cos\left(\alpha_\text{i}\right)
          & K_\text{bt,i} \cos^2\left(\alpha_\text{i}\right)+
          K_\text{bn,i} \sin^2\left(\alpha_\text{i}\right)
        \end{array} \right]
    \left[ \begin{array}{l} \delta x_\text{i}
        \\ \delta y_\text{i}
      \end{array} \right]\end{aligned} 
      \end{equation}

\noindent
The new effective base stiffness matrix $K'_\text{i}$,as derived in
equation~(\ref{eq:Krot}) is defined in equation~(\ref{eq:K'}). This is
seen as matrix $\bar{K}_\text{b,i}$ in
\dref{DD_KMats}. $K_\text{bt,i}$ is the shear element in the matrix,
and $K_\text{bn,i}$ is the normal element in the matrix, calculated as
in \ddref{DD_Stiff}. The notation is simplified by the introduction of
the constants $K_\text{bA,i}$ and $K_\text{bB,i}$, defined in
equations~(\ref{eq:KbA}) and (\ref{eq:KbB}) respectively.

\begin{equation} \label{eq:K'}
  \begin{aligned} & \bar{ K'_\text{i}}
     = \left[ \begin{array}{l l}
          K_\text{bt,i} \cos^2\left(\alpha_\text{i}\right)+
          K_\text{bn,i} \sin^2\left(\alpha_\text{i}\right)
          & \left(K_\text{bt,i} - K_\text{bn,i}\right)
          \sin\left(\alpha_\text{i}\right)
          \cos\left(\alpha_\text{i}\right)\\
          \left(K_\text{bt,i} - K_\text{bn,i}\right)
          \sin\left(\alpha_\text{i}\right)
          \cos\left(\alpha_\text{i}\right)
          & K_\text{bt,i} \cos^2\left(\alpha_\text{i}\right)+
          K_\text{bn,i} \sin^2\left(\alpha_\text{i}\right)
      \end{array} \right] \\
    & \bar{ K'_\text{i}} = \left[ \begin{array}{l l}
        K_\text{bA,i} & K_\text{bB,i} \\
        K_\text{bB,i} & K_\text{bA,i}
      \end{array} \right] \end{aligned} \end{equation}


\begin{equation} \label{eq:KbA}
  K_\text{bA,i} = K_\text{bt,i} \cos^2\left(\alpha_\text{i}\right)+
                 K_\text{bn,i} \sin^2\left(\alpha_\text{i}\right)
\end{equation}

\begin{equation} \label{eq:KbB}
  K_\text{bB,i} = \left(K_\text{bt,i} - K_\text{bn,i}\right)
                 \sin\left(\alpha_\text{i}\right)
                 \cos\left(\alpha_\text{i}\right)
\end{equation}


\noindent
A force-displacement relationship for an element $\text{i}$ can be
written in terms of displacements occurring in the unrotated
coordinate system $\bar{\delta}_\text{i}$ of \dref{GD_DispVecs} using
the matrix $K_\text{s,i}$, and $K_\text{b,i}$ as seen in
\ddref{DD_KMats}.

~\newline

% ------------------------------------------ %
%               Begin DD Eqm     %
% ------------------------------------------ %

\noindent
\begin{minipage}{\textwidth}
\renewcommand*{\arraystretch}{1.6}
\begin{tabular}{| p{1.5cm} | p{14.25cm} |}
  
\hline \rowcolor{AFB} Number&
DD\refstepcounter{datadefnum}\thedatadefnum \label{DD_Eqm}\\

\hline Label& \bf Net Force-Displacement Equilibrium \\

\hline Equation &
\( - \ell_\text{s,i-1} \cdot \bar{K}_\text{s,i-1} \cdot
\bar{\delta}_\text{i-1}
+ \left( \ell_\text{s,i-1} \cdot \bar{K}_\text{s,i-1}
+\ell_\text{b,i} \cdot \bar{K}_\text{b,i}
+\ell_\text{s,i} \cdot \bar{K}_\text{s,i} \right)
\cdot \bar{\delta}_\text{i}
- \ell_\text{s,i} \cdot \bar{K}_\text{s,i} \cdot
\bar{\delta}_\text{i+1}
= \left[ \begin{array}{l} F_\text{x,i} \\ F_\text{y,i} \end{array}
\right] \) \\

\hline Description & The total force displacement equilibrium
relationship on a slice, considering the effects of displacements by
adjacent slices. The net loads acting on a slice ($F_\text{x,i}$ and
$F_\text{y,i}$) are found from \dref{GD_NetForce}. A net load on a
slice will cause an opposing reaction force, therefore
$\bar{\delta}_\text{i}$ is positively related to the net force. The
reaction of the slice will be dampened by the movement of adjacent
slices, therefore $\bar{\delta}_\text{i-1}$ and
$\bar{\delta}_\text{i+1}$ are negatively related. $\bar{K}_\text{s,i}$
and $\bar{K}_\text{b,i}$ are defined as in \ddref{DD_KMats}. The
stiffnesses are normalized for the length of the surfaces $\ell$. \\

\hline Sources& \cite{StolleGuo}\\

\hline Ref.\ By & \iref{IM_RFEM}\\

\hline
\end{tabular}
\end{minipage}\\

% ------------------------------------------ %
%               End DD Eqm         %
% ------------------------------------------ %

~\newline

% ------------------------------------------ %
%               Begin DD Soil Stiffness     %
% ------------------------------------------ %

\noindent
\begin{minipage}{\textwidth}
\renewcommand*{\arraystretch}{1.6}
\begin{tabular}{| p{1.5cm} | p{14cm} |}
  
\hline \rowcolor{AFB} Number&
DD\refstepcounter{datadefnum}\thedatadefnum \label{DD_Stiff}\\

\hline Label& \bf Soil Stiffness \\

\hline Input & $E$ , $\nu$ , $b$ , $c$ , $\sigma$ , $\phi$ , $\kappa$
$a$ , $A$ , $u$ , $v$\\

\hline
Output & 
\( K_{\text{t}} = \frac{E}{2 \left[ 1 + \nu \right]} \frac{0.1}{b} +
\frac { c - \sigma \cdot \tan\left(\phi\right) }{ \left| \delta u
  \right| + a }\)
~\newline~\newline
 \( K_{\text{n}}= \) 
\(  \left\{
\renewcommand{\arraystretch}{2}
\begin{tabular}{ p{4cm} l} 
$ \frac { E \left[1-\nu\right] }{ \left[ 1 + \nu \right] \left[ 1 -
      2\nu + b \right] }$ & for $\text{v} \leq \text{0}$ \\
\noindent\parbox[c]{\hsize} {$ \frac {0.01 \; E \left[1-\nu\right] }{
    \left[ 1 + \nu \right] \left[ 1 - 2\nu + b \right] } +
  \frac{\kappa}{\delta v + A} $} & for $\text{v} \geq \text{0}$
\end{tabular}
\renewcommand{\arraystretch}{1}
\right. \) \\

\hline Description & Calculations are applied to interslice and basal
interfaces, which will have different material properties and
displacements. Material constants are constant for a homogeneous
layer, and a weighting scheme is used to interpolate a new material
constant for surfaces that cross multiple layers with different
material constants, noted as an effective material constant. \\
& $K_{\text{t}}$ is the shear stiffness of a slice.\\
& $K_{\text{n}}$ is the normal stiffness of a slice.\\
& $E$ is the effective Young's modulus.\\
& $c$ is the effective cohesion.\\
& $v$ is the effective Poisson's ratio. \\
& $\tan\left(\phi\right)$ is the effective static friction. \\
& $b$ is the length of the base for base stiffness, and the length
between adjacent slice midpoints for interslice stiffness.\\
& $\sigma$ is the normal stress on the surface.\\
& $\delta u$ is the shear displacement of the surface.  Relative for
interslice surfaces, and absolute for base surfaces.\\
& $\delta v$ is the normal displacement of the surface.  Relative for
interslice surfaces, and absolute for base surfaces\\
& $\kappa$, $A$, and $a$ are constants.\\

\hline Sources& \cite{StolleGuo}\\

\hline Ref.\ By & \iref{IM_RFEM}, \iref{IM_RFEMFS}\\

\hline
\end{tabular}
\end{minipage}\\

% ------------------------------------------ %
%               End DD Soil Stiffness         %
% ------------------------------------------ %


\subsubsection{Instance Models} \label{sec_instance}

This section transforms the problem defined in the
Section~\ref{Sec_pd} into one which is expressed in mathematical
terms. It uses concrete symbols defined in Section~\ref{sec_datadef}
to replace the abstract symbols in the models identified in the
Sections~\ref{sec_theoretical} and ~\ref{sec_gendef}.

~\newline\noindent The Morgenstern Price Method is a vertical slice,
limit equilibrium slope stability analysis method. Analysis is
performed by breaking the assumed failure surface into a series of
vertical slices of mass. Static equilibrium analysis using two force
equilibrium, and one moment equation as in \tref{TM_Eqm}. The problem
is statically indeterminate with only these 3 equations and one
constitutive equation (the Mohr Coulomb shear strength of
\tref{TM_Fmc}) so the assumption of \dref{GD_X} is used. Solving for
force equilibrium allows definitions of all forces in terms of the
physical properties of \ddref{DD_W} to \ddref{DD_EX}, as done in
\ddref{DD_R}, \ddref{DD_T}.

~\newline\noindent The values of the interslice normal force $E$ the
interslice normal/shear force magnitude ratio $\lambda$, and the
Factor of Safety $\text{FS}$, are unknown. Equations for the unknowns
are written in terms of only the values in \ddref{DD_W} to
\ddref{DD_EX}, the values of $R_\text{i}$, and $T_\text{i}$ in
\ddref{DD_R} and \ddref{DD_T}, and each other. The relationships
between the unknowns are non linear, and therefore explicit equations
cannot be derived and an iterative solution method is required.


~\newline

% ------------------------------------------ %
%               Begin IM Factor of safety      %
% ------------------------------------------ %

\noindent
\begin{minipage}{\textwidth}
\renewcommand*{\arraystretch}{1.6}
\begin{tabular}{| p{1.5cm} | p{14cm} |}
  
\hline \rowcolor{Amethyst} Number&
IM\refstepcounter{instnum}\theinstnum \label{IM_FS}\\

\hline Label& \bf Factor of Safety \\

\hline Input & ${\Psi_{\text{v}}}$ , ${\Phi_{\text{v}}}$ ,
${T_{\text{v}}}$ , ${R_{\text{v}}}$ \\

\hline
Output &
\( {FS}= \frac{\displaystyle\sum_{v=1}^{n-1} \left[ {R_{v}}
    \;{\displaystyle\prod_{c=i}^{n-1} \frac{\Psi_{c}}{\Phi_{c}}
    }\right] + {R_{n}} }{\displaystyle\sum_{v=1}^{n-1} \left[ {T_{v}}
    \;{\displaystyle\prod_{c=i}^{n-1} \frac{\Psi_{c}}{\Phi_{c}}
    }\right] + {T_{n}} } \)\\

\hline Description & Equation for the Factor of Safety, the ratio
between resistive and mobile shear the slip surface. The sum of values
from each slice is taken to find the total resistive and mobile shear
for the slip surface. The constants $\Phi$ and $\Psi$ convert the
resistive and mobile shear without the influence of interslice forces,
to a calculation considering the interslice forces. \\

\hline Sources& \cite{ZhuEtAl2005}\\

\hline Ref.\ By & \iref{IM_Lambda}, \iref{IM_E}\\

\hline
\end{tabular}
\end{minipage}\\

% ------------------------------------------ %
%               End IM Factor of Safety         %
% ------------------------------------------ %

\subsubsection*{Factor of Safety Derivation}

\noindent
Using equation~(\ref{eq:Interslice2}) from
section~\ref{sec:Ederivation}, rearranging, and applying the boundary
condition that $E_{\text{0}}$ and $E_{\text{n}}$ are equal to $0$ an
equation for the factor of safety is found as equation~(\ref{eq:FS}),
also seen in \iref{IM_FS}.

\begin{equation}\label{eq:FS}
  \text{FS}= \frac{\displaystyle\sum_{v=1}^{n-1} \left[ {R_{v}}
      \;{\displaystyle\prod_{c=v}^{n-1} \frac{\Psi_{c}}{\Phi_{c}}
      }\right] + {R_{n}} }{\displaystyle\sum_{v=1}^{n-1} \left[
      {T_{v}} \;{\displaystyle\prod_{c=v}^{n-1}
        \frac{\Psi_{c}}{\Phi_{c}} }\right] + {T_{n}} }
\end{equation}

\noindent
The constants $\Psi$ and $\Phi$ described in equations \ref{eq:Psi}
and \ref{eq:Phi} are functions of the unknowns: the interslice
normal/shear force ratio $\lambda$ (\iref{IM_Lambda}) and the Factor
of Safety itself $\text{FS}$ (\iref{IM_FS}).

~\newline

% ------------------------------------------ %
%               Begin IM Lambda      %
% ------------------------------------------ %

\noindent
\begin{minipage}{\textwidth}
\renewcommand*{\arraystretch}{1.6}
\begin{tabular}{| p{1.5cm} | p{14cm} |}
  
\hline \rowcolor{Amethyst} Number&
IM\refstepcounter{instnum}\theinstnum \label{IM_Lambda}\\

\hline Label& \bf Normal/Shear Force Ratio \\

\hline Input & $b_{\text{v}}$ , $E_{\text{v}}$ , $H_{\text{v}}$ ,
$\alpha_{\text{v}}$ , $h_{\text{v}}$ , $W_{\text{v}}$ ,
$U_{\text{t,v}}$ , $\beta_{\text{v}}$ , $f_{\text{v}}$ ,
${K_{\text{c}}}$ \\

\hline
Output & 
\( {C1_{\text{i}}}= \) 
\(  \left\{
\renewcommand{\arraystretch}{2}
\begin{tabular}{ p{6.5cm} r} 
  $ {{b}_{\text{1}}}\left[{{E}_{\text{1}} + {H}_{\text{1}}}
    \right]{\tan\left(\alpha_{\text{1}}\right) } $ & for $
  \text{i}={\text{1}} $ \\
\noindent\parbox[c]{\hsize} {$ {{b}_{\text{i}}} \left[
    \left({{E}_{\text{i}} + {E}_{\text{i-1}}}\right) +
    \left({{H}_{\text{i}} + {H}_{\text{i-1}}}\right)
    \right]{\tan\left(\alpha_{\text{i}}\right)} \\ +
  {{h}_{\text{i}}}\left( {K_{\text{c}}}\;{W_{\text{i}}} -
  {2}\;{U_{\text{t,i}}}\;{\sin\left(\beta_{\text{i}}\right)} -
  {2}\;{Q_{\text{i}}}\;{\cos\left(\omega_{\text{i}}\right)} \right) $}
& for $ 2\leq\text{i}\leq{\text{n-1}} $ \\ $
                {{b}_{\text{n}}}\left[{{E}_{\text{n-1}} +
                    {H}_{\text{n-1}}}\right]{\tan\left(\alpha_{\text{n-1}}\right)
                } $ & for $ \text{i}=\text{n} $ \\
\end{tabular} 
\renewcommand{\arraystretch}{1}
\right. \)
~\newline~\newline
\( {C2_{\text{i}}}= \)
\(  \left\{
\renewcommand{\arraystretch}{2}
\begin{tabular}{ p{7cm} r} 
  $ {{b}_{\text{1}}}{{E}_{\text{1}}}{f_{\text{1}}} $ & for $
  \text{i}=\text{1} $ \\ $ {{b}_{\text{i}}}\;{\left({
      {f_{\text{i}}}{{E}_{\text{i}}} +
      {f_{\text{i-1}}}{{E}_{\text{i-1}}} }\right)} $ & for $
  2\leq\text{i}\leq{\text{n-1}} $ \\ $
  {{b}_{\text{n}}}{{E}_{\text{n-1}}}{f_{\text{n-1}}} $ & for $
  \text{v}=\text{n} $ \\
\end{tabular} 
\renewcommand{\arraystretch}{1}
\right. \) 
~\newline
\( \lambda= \frac{ \displaystyle\sum_{i=1}^{n} {C1_{\text{i}}}}
   {\displaystyle\sum_{i=1}^{n} {C2_{\text{i}}}} \) \\

\hline Description & $\lambda$ is the magnitude ratio between shear
and normal forces at the interslice interfaces as the assumption of
the Morgenstern Price method in \dref{GD_X}. The inclination function
$f$ determines the relative magnitude ratio between the different
interslices, while $\lambda$ determines the magnitude. $\lambda$ uses
the sum of interslice normal and shear forces taken from each
interslice. \\

\hline Sources& \cite{ZhuEtAl2005}\\

\hline Ref.\ By & \iref{IM_FS}, \iref{IM_E} \\

\hline
\end{tabular}
\end{minipage}\\

% ------------------------------------------ %
%               End IM Lambda         %
% ------------------------------------------ %

\subsubsection*{Normal/Shear Force Ratio Derivation}

 The last static equation of \tref{TM_Eqm} the moment equilibrium of
 \dref{GD_M} about the midpoint of the base is taken, with the
 assumption of \dref{GD_X}. Results in equation~(\ref{eq:Moment}).

\begin{equation}\label{eq:Moment}
  0 = \begin{array}{l} - {E}_{\text{i}} \left[ {z_{\text{i}}} -
      \frac{b_{\text{i}}}{2} { \tan\left(\alpha_{\text{i}}\right)}
      \right] + {E}_{\text{i-1}} \left[ {z_{\text{i-1}}} +
      \frac{b_{\text{i}}}{2} { \tan\left(\alpha_{\text{i}}\right)}
      \right] - H_{\text{i}}\left[ z_{\text{w,i}} -
      \frac{b_{\text{i}}}{2} { \tan\left(\alpha_{\text{i}}\right)}
      \right] \\[5pt] + H_{\text{i-1}}\left[ z_{\text{w,i-1}} +
      \frac{b_{\text{i}}}{2} { \tan\left(\alpha_{\text{i}}\right)}
      \right] -\lambda \frac{b_{\text{i}}}{2} \left( E_{\text{i}}
    f_{\text{i}} + E_{\text{i-1}} f_{\text{i-1}} \right) +
    K_{\text{c}} W_{\text{i}} \frac{h_{\text{i}}}{2} - U_{\text{t,i}}
    \sin\left(\beta_{\text{i}}\right) h_{\text{i}} -
    Q_{\text{i}}\;{\sin\left(\omega_{\text{i}}\right)}
    h_{\text{i}} \end{array}
\end{equation} 

\noindent
Rearranging the equation in terms of $\lambda$ leads to equation
~(\ref{eq:lambda1}).

\begin{equation}\label{eq:lambda1}
  \lambda = \frac { \begin{array}{l} - {E}_{\text{i}} \left[
        {z_{\text{i}}} - \frac{b_{\text{i}}}{2} {
          \tan\left(\alpha_{\text{i}}\right)} \right] +
      {E}_{\text{i-1}} \left[ {z_{\text{i-1}}} +
        \frac{b_{\text{i}}}{2} { \tan\left(\alpha_{\text{i}}\right)}
        \right] - H_{\text{i}}\left[ z_{\text{w,i}} -
        \frac{b_{\text{i}}}{2} { \tan\left(\alpha_{\text{i}}\right)}
        \right] \\[5pt] + H_{\text{i-1}}\left[ z_{\text{w,i-1}} +
        \frac{b_{\text{i}}}{2} { \tan\left(\alpha_{\text{i}}\right)}
        \right] + K_{\text{c}} W_{\text{i}} \frac{h_{\text{i}}}{2} -
      U_{\text{t,i}} \sin\left(\beta_{\text{i}}\right) h_{\text{i}} -
      Q_{\text{i}}\;{\sin\left(\omega_{\text{i}}\right)}
      h_{\text{i}} \end{array} } { \frac{b_{\text{i}}}{2} \left[
      E_{\text{i}} f_{\text{i}} + E_{\text{i-1}} f_{\text{i-1}}
      \right] }
\end{equation}

\noindent
Taking a summation of each slice, and considering the boundary
conditions that $E_{\text{0}}$ and $E_{\text{n}}$ are equal to zero, a
general equation for the constant $\lambda$ is developed in
equation~(\ref{eq:Lambda}), also found in \iref{IM_Lambda}.

\begin{equation}\label{eq:Lambda}
\lambda= \frac{ \displaystyle\sum_{i=1}^{n} { {b_{\text{i}}} \left[
      \left({{E}_{\text{i}} + {E}_{\text{i-1}}}\right) +
      \left({{H}_{\text{i}} + {H}_{\text{i-1}}}\right)
      \right]{\tan\left(\alpha_{\text{i}}\right)} \\ +
    {{h}_{\text{i}}}\;\left[ {K_{\text{c}}}\;{W_{\text{i}}} -
      {2}\;{U_{\text{t,i}}}\;{\sin\left(\beta_{\text{i}}\right)} - {2}
      \; Q_{\text{i}}\;{\sin\left(\omega_{\text{i}}\right)} \right] }}
         {\displaystyle\sum_{i=1}^{n} { {{b}_{\text{i}}}\;{\left[{
                   {f_{i}}{{E}_{\text{i}}} +
                   {f_{\text{i-1}}}{{E}_{\text{i-1}}} }\right]} }}
\end{equation}

\noindent
Equation~(\ref{eq:Lambda}) for $\lambda$, is a function of the unknown
interslice normal force $E$ (\iref{IM_E}).

~\newline

% ------------------------------------------ %
%               Begin IM E    %
% ------------------------------------------ %

\noindent
\begin{minipage}{\textwidth}
\renewcommand*{\arraystretch}{1.6}
\begin{tabular}{| p{1.5cm} | p{14cm} |}
  
\hline \rowcolor{Amethyst} Number&
IM\refstepcounter{instnum}\theinstnum \label{IM_E}\\

\hline Label& \bf Interslice Forces \\

\hline Input & $\text{FS}$, $T_\text{i}$, $R_\text{i}$, $\Psi$,
$\Phi$\\

\hline
Output &

\( E_{\text{i}}= \) 
\(  \left\{
\renewcommand{\arraystretch}{1.75}
\begin{tabular}{ p{3cm} l} 
$ \frac{ \left(\text{FS}\right) T_{\text{i}} - R_{\text{i}} }{
    \Phi_{\text{i}} } $ & for $\text{i=1}$ \\
\noindent\parbox[c]{\hsize} {$ \frac{ \Psi_{\text{i-1}} \cdot
    E_{\text{i-1}} + \left(\text{FS}\right) \cdot T_{\text{i}} -
    R_{\text{i}} }{ \Phi_{\text{i}} } $} & for
$1\leq\text{i}\leq\text{n-1}$ \\
\noindent\parbox[c]{\hsize} {$0 $} & for $\text{i=0}$ and $\text{i=n}$
\end{tabular}
\renewcommand{\arraystretch}{1}
\right. \) \\

\hline Description & The value of the interslice normal force
$E_\text{i}$ at interface $\text{i}$. The net force the weight of the
slices adjacent to interface $\text{i}$ exert horizontally on each
other.\\

\hline Sources& \cite{ZhuEtAl2005}\\

\hline Ref.\ By & \iref{IM_FS}, \iref{IM_Lambda}\\

\hline
\end{tabular}
\end{minipage}\\

% ------------------------------------------ %
%               End IM E         %
% ------------------------------------------ %

\subsubsection*{Interslice Force Derivation} \label{sec:Ederivation}

Taking the perpendicular force equilibrium of \dref{GD_Fx} with the
effective stress definition from \tref{TM_EffStress} that
$N_{\text{i}}=N'_{\text{i}} - U_{\text{b,i}}$, and the assumption of
\dref{GD_X} the equilibrium equation can be rewritten as
equation~(\ref{eq:F_perp}).

\begin{equation}\label{eq:F_perp}   N'_{\text{i}}  = \begin{array}{l}    
    \left[ W_{\text{i}} - \lambda \cdot f_{\text{i-1}} \cdot
      E_{\text{i-1}} + \lambda \cdot f_{\text{i}} \cdot E_{\text{i}} +
      U_{\text{t,i}} {\cos\left(\beta_{\text{i}}\right)} +
      Q_{\text{i}} \cos\left(\omega_{\text{i}}\right)
      \right]\cos\left(\alpha_{\text{i}}\right) \\ + \left[
      -K_{\text{c}} W_{\text{i}} - E_{\text{i}} + E_{\text{i-1}} -
      H_{\text{i}} + H_{\text{i-1}} + U_{\text{t,i}}
      \sin\left(\beta_{\text{i}}\right) + Q_{\text{i}}
      \sin\left(\omega_{\text{i}}\right) \right]
    \sin\left(\alpha_{\text{i}}\right) - U_{\text{b,i}} \end{array}
\end{equation}

\noindent
Taking the parallel force equilibrium of \dref{GD_Fy} with the
definition of mobile shear from \dref{GD_MobShear} and the assumption
of \dref{GD_X}, the equilibrium equation can be rewritten as
equation~(\ref{eq:F_par}).


\begin{equation}  \label{eq:F_par}
  \frac{ N_{\text{i}} \tan\left(\varphi'\right) + c'_{\text{i}} \cdot
    b'_{\text{i}} \cdot \sec\left(\alpha_{\text{i}}\right) }{
    \text{FS} } = \begin{array}{l} \left[ W_{\text{i}} - \lambda \cdot
      f_{\text{i-1}} \cdot E_{\text{i-1}} + \lambda \cdot f_{\text{i}}
      \cdot E_{\text{i}} + U_{\text{t,i}}
      \cos\left(\beta_{\text{i}}\right) + Q_{\text{i}}
      \cos\left(\omega_{\text{i}}\right) \right]
    \sin\left(\alpha_{\text{i}}\right) \\ - \left[ -K_{\text{c}}
      W_{\text{i}} - E_{\text{i}} + E_{\text{i-1}} - H_{\text{i}} +
      H_{\text{i-1}} + U_{\text{t,i}} \cdot
      \sin\left(\beta_{\text{i}}\right) + Q_{\text{i}}
      \sin\left(\omega_{\text{i}}\right) \right]
    \cos\left(\alpha_{\text{i}}\right) \end{array}
\end{equation}

\noindent
Substituting the equation for $N'_{\text{i}}$ from
equation~(\ref{eq:F_perp}) into equation~(\ref{eq:F_par}) and
rearranging results in equation~(\ref{eq:Interslice1})

\begin{equation}\label{eq:Interslice1}
E_\text{i} \left[ \begin{array}{l} \left[ \lambda \cdot f_\text{i}
      \cos\left(\alpha_\text{i}\right) -
      \sin\left(\alpha_\text{i}\right) \right]
    \tan\left(\varphi'\right) \\ - \left[ \lambda \cdot f_\text{i}
      \sin\left(\alpha_\text{i}\right) +
      \cos\left(\alpha_\text{i}\right) \right]
    \left(\text{FS}\right) \end{array} \right] = E_\text{i-1}
\left[ \begin{array}{l} \left[ \lambda \cdot f_\text{i-1}
      \cos\left(\alpha_\text{i}\right) -
      \sin\left(\alpha_\text{i}\right) \right]
    \tan\left(\varphi'\right) \\ - \left[ \lambda \cdot f_\text{i-1}
      \sin\left(\alpha_\text{i}\right) +
      \cos\left(\alpha_\text{i}\right) \right]
    \left(\text{FS}\right) \end{array} \right] +
\left(\text{FS}\right) \cdot T_\text{i} - R_\text{i}
\end{equation}

\noindent
Where $R_\text{i}$ and $T_\text{i}$ are the resistive and mobile shear
of the slice, without the influence of interslice forces $E$ and $X$,
as defined in \ddref{DD_R} and \ddref{DD_T}.  Making use of the
constants $\phi$ and $\Psi$ with full equations found below in
equations~(\ref{eq:Phi}) and~(\ref{eq:Psi}) respectively, then
equation ~(\ref{eq:Interslice1}) can be simplified to equation
~(\ref{eq:Interslice2}), also seen in \iref{IM_E}.

\begin{equation}\label{eq:Phi} \begin{aligned}
\Phi_{\text{i}}={} & \left[ \lambda \cdot f_{\text{i}}
  \cos\left(\alpha_{\text{i}}\right) -
  \sin\left(\alpha_{\text{i}}\right) \right]\left[
  \tan\left({\varphi'_{\text{i}}}\right) \right] - \left[ \lambda
  \cdot f_{\text{i}} \sin\left(\alpha_{\text{i}}\right) +
  \cos\left(\alpha_{\text{i}}\right) \right]\left(\text{FS}\right)
\\[5pt] & \text{Where i is the local slice of mass for }
1\leq\text{i}\leq\text{n-1}
\end{aligned}\end{equation}

~\newline 
\begin{equation}\label{eq:Psi}\begin{aligned}
\Psi_{\text{i}} ={}& \left[ \lambda \cdot f_{\text{i}}
  \cos\left(\alpha_{\text{i+1}}\right) -
  \sin\left(\alpha_{\text{i+1}}\right) \right]\left[
  \tan\left(\varphi'\right) \right] - \left[ \lambda \cdot
  f_{\text{i}} \sin\left(\alpha_{\text{i+1}}\right) +
  \cos\left(\alpha_{\text{i+1}}\right) \right] \left( \text{FS}
\right) \\[5pt] & \text{Where i is the local slice of mass for }
1\leq\text{i}\leq\text{n-1}
 \end{aligned}\end{equation}
~\newline
\begin{equation}\label{eq:Interslice2}
{E}_{\text{i}} = \frac{{\Psi_{\text{i-1}}}\;{{E}_{\text{i-1}}} +
  \left({{FS}}\right)\;{T_{\text{i}}} -
       {R_{\text{i}}}}{\Phi_{\text{i}}}
\end{equation}

\noindent
The constants $\Psi$ and $\Phi$ in equation~(\ref{eq:Interslice2}) for
$E_\text{i}$ is a function of the unknown values, the interslice
normal/shear force ratio $\lambda$ (\iref{IM_Lambda}), and the Factor
of Safety $\text{FS}$ (\iref{IM_FS}).

~\newline

% ------------------------------------------ %
%               Begin IM Force Displacement      %
% ------------------------------------------ %

\noindent
\begin{minipage}{\textwidth}
\renewcommand*{\arraystretch}{1.6}
\begin{tabular}{| p{1.5cm} | p{14cm} |}
  
\hline \rowcolor{AFB} Number&
IM\refstepcounter{instnum}\theinstnum \label{IM_RFEM}\\

\hline Label& \bf Force Displacement Equilibrium \\

\hline Input & $E$ , $\nu$ , $b$ , $c$ , $\sigma$ , $\phi$ , $\kappa$
$a$ , $A$ , $u$ , $v$\\

\hline
Output &

\textbf{X Equilibrium} ~\newline
 \( \begin{array}{l} -\Delta H_\text{i} -K_{\text{c}} \cdot W_\text{i} -
U_{\text{b,i}} \sin\left(\alpha_\text{i}\right) \\ + U_{\text{t,i}}
\sin\left(\beta_\text{i}\right) + Q_\text{i}
\sin\left(\omega_\text{i}\right) \end{array} =  \begin{array}{l}  
\left[ \delta x_{\text{i-1}} \right] \left( - \ell_{\text{s,i-1}}
K_{\text{sn,i-1}} \right) \\ + \left[ \delta x_{\text{i}} \right]
\left( - \ell_{\text{s,i-1}} K_{\text{sn,i-1}} + \ell_{\text{s,i}}
K_{\text{sn,i}} + \ell_{\text{b,i}} K_{\text{bA,i}} \right) \\ +
\left[ \delta x_{\text{i+1}} \right] \left( - \ell_{\text{s,i}}
K_{\text{sn,i}} \right) + \left[ \delta y_{\text{i}} \right] \left( -
\ell_{\text{b,i}} K_{\text{bB,i}} \right)
\end{array} \) ~\newline ~\newline
\textbf{Y Equilibrium} ~\newline
\( \begin{array}{l} -W_\text{i} + U_{\text{b,i}}
\cos\left(\alpha_\text{i}\right)\\ - U_{\text{t,i}}
\cos\left(\beta_\text{i}\right) - Q_{\text{i}}
\cos\left(\omega_\text{i}\right)\end{array} = \begin{array}{l}  
\left[ \delta y_{\text{i-1}} \right] \left( - \ell_{\text{s,i-1}}
K_{\text{st,i-1}} \right) \\ + \left[ \delta y_{\text{i}} \right]
\left( - \ell_{\text{s,i-1}} K_{\text{st,i-1}} + \ell_{\text{s,i}}
K_{\text{st,i}} + \ell_{\text{b,i}} K_{\text{bA,i}} \right) \\ +
\left[ \delta y_{\text{i+1}} \right] \left( - \ell_{\text{s,i}}
K_{\text{st,i}} \right) + \left[ \delta x_{\text{i}} \right] \left( -
\ell_{\text{b,i}} K_{\text{bB,i}} \right) \end{array}\) \\

\hline Description & One set of force displacement equilibrium
equations in the x and y directions. There is of equations for each
element. System of equations solved for displacements ($\delta x$, and
$\delta y$) \\
& $\Delta H_{\text{i}} = H_{\text{i}}-H_{\text{i-1}}$ is the net
hydrostatic force across a slice.\\
& $K_{\text{c}}$ is the earthquake load factor.\\
& $W_{\text{i}}$ is the weight of the slice.\\
& $U_{\text{b,i}}$ is the pore water pressure acting on the slice
base.\\
& $U_{\text{t,i}}$ is the pore water pressure acting on the slice
surface. \\
& $\alpha_{\text{i}}$ is the angle of the base with the horizontal. \\
& $\beta_{\text{i}}$ is the angle of the surface with the horizontal\\
& $\delta x_{\text{i}}$ is the x displacement of slice $\text{i}$\\
& $\delta y_{\text{i}}$ is the y displacement of slice $\text{i}$\\
& $ \ell_{\text{s,i}}$ is the length of the interslice surface
$\text{i}$\\
& $ \ell_{\text{s,i}}$ is the length of the base surface $\text{i}$\\
& $ K_{\text{st,i}}$ is the interslice shear stiffness at surface
$\text{i}$.\\
& $ K_{\text{st,i-1}}$ is the interslice normal stiffness at surface
$\text{i}$.\\
& $ K_{\text{bA,i}}$, and $ K_{\text{bB,i}}$ are the base stiffness
values for slice $\text{i}$.\\

\hline Sources& \cite{StolleGuo}\\

\hline Ref.\ By & \iref{IM_RFEMFS} \\

\hline
\end{tabular}
\end{minipage}\\

% ------------------------------------------ %
%               End IM Force Displacement         %
% ------------------------------------------ %

\subsubsection*{Rigid Finite Element
  Displacement Derivation}

Using the net force-displacement equilibrium equation of a slice from
\ddref{DD_Eqm}, with the definitions of the stiffness matrices from
\ddref{DD_KMats}, and the force definitions from \dref{GD_NetForce}, a
broken down force-displacement equilibrium equation can be
derived. Equation~(\ref{eq:NetFx}) gives the broken down equation in
the $x$ direction, and equation~(\ref{eq:NetFy}) gives the broken down
equation in the $y$ direction.

\begin{equation}\label{eq:NetFx}
-\Delta H_\text{i} -K_{\text{c}} \cdot W_\text{i} - U_{\text{b,i}}
\sin\left(\alpha_\text{i}\right) + U_{\text{t,i}}
\sin\left(\beta_\text{i}\right) + Q_\text{i}
\sin\left(\omega_\text{i}\right) = \begin{array}{l} \left[ \delta
    x_{\text{i-1}} \right] \left( - \ell_{\text{s,i-1}}
  K_{\text{sn,i-1}} \right) \\ + \left[ \delta x_{\text{i}} \right]
  \left( - \ell_{\text{s,i-1}} K_{\text{sn,i-1}} + \ell_{\text{s,i}}
  K_{\text{sn,i}} + \ell_{\text{b,i}} K_{\text{bA,i}} \right) \\ +
  \left[ \delta x_{\text{i+1}} \right] \left( - \ell_{\text{s,i}}
  K_{\text{sn,i}} \right) + \left[ \delta y_{\text{i}} \right] \left(
  - \ell_{\text{b,i}} K_{\text{bB,i}} \right)
  \end{array}
\end{equation}


\begin{equation}\label{eq:NetFy}
-W_\text{i} + U_{\text{b,i}} \cos\left(\alpha_\text{i}\right) -
U_{\text{t,i}} \cos\left(\beta_\text{i}\right) - Q_{\text{i}}
\cos\left(\omega_\text{i}\right) = \begin{array}{l} \left[ \delta
    y_{\text{i-1}} \right] \left( - \ell_{\text{s,i-1}}
  K_{\text{st,i-1}} \right) \\ + \left[ \delta y_{\text{i}} \right]
  \left( - \ell_{\text{s,i-1}} K_{\text{st,i-1}} + \ell_{\text{s,i}}
  K_{\text{st,i}} + \ell_{\text{b,i}} K_{\text{bA,i}} \right) \\ +
  \left[ \delta y_{\text{i+1}} \right] \left( - \ell_{\text{s,i}}
  K_{\text{st,i}} \right) + \left[ \delta x_{\text{i}} \right] \left(
  - \ell_{\text{b,i}} K_{\text{bB,i}} \right)
 \end{array}
\end{equation}

\noindent
Using the known input assumption of \aref{A_Input}, the force variable
definitions of \ddref{DD_W} to \ddref{DD_Q} on the left side of the
equations can be solved for. The only unknown in the variables to
solve for the stiffness values from \ddref{DD_Stiff} is the
displacements. Therefore taking the equation from each slice a set of
$2 \cdot n$ equations, with $2 \cdot n$ unknown displacements in the
$x$ and $y$ directions of each slice can be derived. Solutions for
the displacements of each slice can then be found. The use of
displacement in the definition of the stiffness values makes the
equation implicit, which means an iterative solution method, with an
initial guess for the displacements in the stiffness values is
required.

~\newline

% ------------------------------------------ %
%               Begin IM RFEM FS     %
% ------------------------------------------ %
\noindent
\begin{minipage}{\textwidth}
\renewcommand*{\arraystretch}{1.6}
\begin{tabular}{| p{1.5cm} | p{14cm} |}
  
\hline \rowcolor{AFB} Number&
IM\refstepcounter{instnum}\theinstnum \label{IM_RFEMFS}\\

\hline Label& \bf RFEM Factor of Safety \\

\hline Input & $c$, $\ell_{\text{b}}$, $\delta u$ , $\delta v$,
$\varphi'$, $K_{\text{bt,i}}$, $K_{\text{bn,i}}$ \\

\hline Output & \( \begin{aligned} & FS_{\text{Loc,i}} = \frac{c -
    K_{\text{bn,i}} \cdot \delta v_{\text{i}} \cdot
    \tan\left(\varphi'_{\text{i}}\right)}{K_{\text{bt,i}} \cdot \delta
    u_{\text{i}}} \\[7.5pt] & FS = \frac{\displaystyle\sum_{i=1}^{n}
    \ell_{\text{b,i}} \left[ c - K_{\text{bn,i}} \cdot \delta
      v_{\text{i}} \cdot \tan\left(\varphi'_{\text{i}}\right) \right]}
  {\displaystyle\sum_{i=1}^{n} \ell_{\text{b,i}} \left[
      K_{\text{bt,i}} \cdot \delta u_{\text{i}} \right]
} \end{aligned} \) \\

\hline
Description 
& $FS_{\text{Loc,i}}$ Factor of Safety for slice $\text{i}$.\\
& $FS$ Factor of Safety for the entire slip surface.\\
& $c$ is the cohesion of slice ${\text{i}}$'s base.\\
& $\varphi'_{\text{i}}$ is the effective angle of friction of slice
${\text{i}}$'s base.\\
& $\delta v_{\text{i}}$ is the normal displacement of slice
$\text{i}$\\
& $\delta u_{\text{i}}$ is the shear displacement of slice
$\text{i}$\\
& $ \ell_{\text{b,i}}$ is the length of the base of slice $\text{i}$\\
& $ K_{\text{bt,i}}$ is the base shear stiffness at surface
$\text{i}$.\\
& $ K_{\text{bn,i}}$ is the base normal stiffness at surface
$\text{i}$.\\
& $ \text{n} $ is the number of slices in the slip surface.\\

\hline Sources& \cite{StolleGuo} \\

\hline
\end{tabular}
\end{minipage}\\

% ------------------------------------------ %
%               End IM RFEM FS         %
% ------------------------------------------ %

\subsubsection*{Rigid Finite Element
  Factor of Safety Derivation}

\noindent
RFEM analysis can also be used to calculate the Factor of safety for
the slope.  For a slice element $\text{i}$ the displacements $\delta
x_{\text{i}}$ and $\delta y_{\text{i}}$, are solved from the system of
equations in \iref{IM_RFEM}. The definition of
$\bar{\epsilon}_\text{i}$ as the rotation of the displacement vector
$\bar{\delta}_\text{i}$ is seen in\dref{GD_DispVecs} .

This is used to find the displacements of the slice parallel to the
base of the slice $\delta u$ in equation~(\ref{eq:DispPar}) and normal
to the base of the slice $\delta v$ in equation~(\ref{eq:DispPerp}).

\begin{equation}\label{eq:DispPar}
\delta u_{\text{i}} = \cos\left(\alpha_{\text{i}}\right) \delta
x_{\text{i}} + \sin\left(\alpha_{\text{i}}\right) \delta y_{\text{i}}
\end{equation}

\begin{equation}\label{eq:DispPerp}
\delta v_{\text{i}} = -\sin\left(\alpha_{\text{i}}\right) \delta
x_{\text{i}} + \cos\left(\alpha_{\text{i}}\right) \delta y_{\text{i}}
\end{equation}

\noindent
With the definition of normal stiffness from \ddref{DD_Stiff} to find
the normal stiffness of the base $K_\text{bn,i}$, and the now known
base displacement perpendicular to the surface $\delta v_{\text{i}}$
from equation~(\ref{eq:DispPerp}), the normal base stress can be
calculated from the force-displacement relationship of
\tref{TM_Hooke}. Stress $\sigma$ is used in place of force $F$ as the
stiffness hasn't been normalized for the length of the base. Results in
equation~(\ref{eq:Sigma}).

\begin{equation}\label{eq:Sigma}
\sigma_{\text{i}} = K_{\text{bn,i}} \cdot \delta v_{\text{i}}
\end{equation}

\noindent
The resistive shear to calculate the factor of safety $FS$ in is found
from the Mohr Coulomb resistive strength of soil in
\tref{TM_Fmc}. Using the normal stress $\sigma$ from
equation~(\ref{eq:Sigma}) as the stress the resistive shear of the
slice can be calculated from calculated in equation~(\ref{eq:S}).

\begin{equation}\label{eq:S}
S_{\text{i}} = c - \sigma_{\text{i}} \cdot \tan\left(\varphi'\right)
\end{equation}

\noindent
previously the value of the base shear stiffness $K_\text{bt,i}$ as
seen in equation~(\ref{eq:ShearStiff}) was unsolvable because the
normal stress $\sigma_\text{i}$ was unknown. With the definition of
$\sigma_\text{i}$ from equation~(\ref{eq:Sigma}) and the definition of
displacement shear to the base $\delta u_\text{i}$ from
equation~(\ref{eq:DispPerp}), the value of $K_\text{bt,i}$ becomes
solvable.

\begin{equation}\label{eq:ShearStiff}
K_{\text{bt,i}} = \frac{E_\text{i}}{2 \left[ 1 + \nu_\text{i} \right]}
\frac{0.1}{b_\text{i}} + \frac { c_\text{i} - \sigma_\text{i} \cdot
  \tan\left(\phi_\text{i}\right) }{ \left| \delta u_\text{i} \right| +
  a }
\end{equation}

\noindent
With shear stiffness $K_{\text{bt,i}}$ calculated in
equation~(\ref{eq:ShearStiff}) and shear displacement $\delta
u_{\text{i}}$ calculated in equation~(\ref{eq:DispPar}) values now
known the shear stress acting on the base of a slice $\tau$ can be
calculated using \tref{TM_Hooke}, as done in equation~(\ref{eq:Tau}).
Again stress $\tau$ is used in place of force $F$ as the stiffness
hasn't been normalized for the length of the base.

\begin{equation}\label{eq:Tau}
\tau_{\text{i}} = K_{\text{bt,i}} \cdot \delta u_{\text{i}}
\end{equation}

\noindent
The shear stress on the base $\tau$ acts as the mobile shear acting on
the base. Using the definition Factor of Safety equation from
\tref{TM_FS}, with the definitions of resistive shear strength of a
slice $S_\text{i}$ from equation~(\ref{eq:S}) and mobile shear on a
slice $\tau$ from equation~(\ref{eq:Tau}) the factor of safety for a
slice $FS_{\text{Loc,i}}$ can be found from as seen in
equation~(\ref{eq:FSLoc}), and \iref{IM_RFEMFS}.

\begin{equation}\label{eq:FSLoc}
FS_{\text{Loc,i}} = \frac{S_{\text{i}}}{\tau_{\text{i}}} = \frac{c -
  K_{\text{bn,i}} \cdot \delta v_{\text{i}} \cdot
  \tan\left(\varphi'_{\text{i}}\right)}{K_{\text{bt,i}} \cdot \delta
  u_{\text{i}}}
\end{equation}

\noindent The global Factor of Safety is then the ratio of the
summation of the resistive and mobile shears for each slice, with a
weighting for the length of the slices base. Shown in
equation~(\ref{eq:FSGlob}), and \iref{IM_RFEMFS}.

\begin{equation}\label{eq:FSGlob}
FS = \frac{\displaystyle\sum_{i=1}^{n} \ell_{\text{i}} \cdot
  S_{\text{i}}}{\displaystyle\sum_{i=1}^{n} \ell_{\text{i}} \cdot
  \tau_{\text{i}}} = \frac{\displaystyle\sum_{i=1}^{n}
  \ell_{\text{b,i}} \left[ c - K_{\text{bn,i}} \cdot \delta
    v_{\text{i}} \cdot \tan\left(\varphi'_{\text{i}}\right) \right]}
{\displaystyle\sum_{i=1}^{n} \ell_{\text{b,i}} \left[ K_{\text{bt,i}}
    \cdot \delta u_{\text{i}} \right]}
\end{equation}

~\newline

% ------------------------------------------ %
%               Begin IM Minimization    %
% ------------------------------------------ %

\noindent
\begin{minipage}{\textwidth}
\renewcommand*{\arraystretch}{1.6}
\begin{tabular}{| p{1.5cm} | p{14cm} |}
  
\hline \rowcolor{Brass} Number&
IM\refstepcounter{instnum}\theinstnum \label{IM_Min}\\

\hline Label& \bf Critical Slip Identification \\

\hline Input & The geometry of the water table, the geometry of the
layers composing the plane of a slope, and the material properties of
the layers. \\

\hline Output & \(  \text{FS}_\text{Min}
 = \Upsilon\left( \left\{x_\text{cs},y_\text{cs}\right\},\text{Input}
 \right) \)
 \\

\hline Description & Given the necessary slope inputs, a minimization
algorithm or function $\Upsilon$, will identify the critical slip
surface of the slope, with the critical slip coordinates
$\left\{x_\text{cs},y_\text{cs}\right\}$ and the minimum factor of
safety $\text{FS}_\text{Min}$ that results.\\

\hline Sources& \cite{LiEtAl}\\

\hline
\end{tabular}
\end{minipage}\\

% ------------------------------------------ %
%               End IM Minimization         %
% ------------------------------------------ %

\subsubsection{Data Constraints} \label{sec_DataConstraints}    

Table~\ref{TblInputVar} and \ref{TblOutputVar} show the data
constraints on the input and output variables, respectively.  The
column physical constraints gives the physical limitations on the
range of values that can be taken by the variable.  The constraints
are conservative, to give the user of the model the flexibility to
experiment with unusual situations.  The column of typical values is
intended to provide a feel for a common scenario.  The uncertainty
column provides an estimate of the confidence with which the physical
quantities can be measured.  This information would be part of the
input if one were performing an uncertainty quantification exercise.

\newpage

%\begin{table}[!h]
%\caption{Input Variables} 
\renewcommand{\arraystretch}{1.5}
\noindent \begin{longtable}{p{2.4cm} p{5.8cm} p{1.2cm} c}
  \toprule  \label{TblInputVar}
  \textbf{Var} & \textbf{Physical Constraints} & \textbf{Typical
    Value} & \textbf{Uncertainty}\\ \midrule
  $(x,y)$ of water table vertices's & Consecutive vertexes have
  increasing x values. All layers start and end vertices's go to the
  same x values. & N/A & / \\
  $(x,y)$ of slip vertices's & Consecutive vertexes have increasing x
  values. All layers start and end vertices's go to the same x
  values. & N/A & / \\
  $(x,y)$ of slope vertices's (*) & Consecutive vertexes have
  increasing x values. All layers start and end vertices's go to the
  same x values. & N/A & / \\
  $E$ (*) & $E > 0$ & 15000 & /\\
  $c$ (*) & $c >0$ & 10 & /\\
  $v$ (*) & $ 0 < v < 1 $ & 0.4 & \\
  $\varphi'$ (*) & $ 0 < \varphi < 90 $ & 25 & / \\
  $\gamma$ (*) & $\gamma > 0$ & 20 & / \\
  $\gamma_{\text{Sat}}$ (*) & $\gamma_{\text{Sat}} > 0 $ & 20 & /
  \\
  $\gamma_{\text{Wat}}$ & $\gamma_{\text{Wat}} > 0 $ & 9.8 & / \\
  \bottomrule
\end{longtable}
%\end{table}

\noindent \begin{description}
\item[(*)] Input coordinates needed for each layer.
\end{description}

%\begin{table}[!h]
%\caption{Output Variables} 
\renewcommand{\arraystretch}{1.2} 
\noindent \begin{longtable}{l l} 
  \toprule \label{TblOutputVar}
  \textbf{Var} & \textbf{Physical Constraints} \\
  \midrule 
  $FS$ & $FS>0$ \\
  $(x,y)$ Slip vertices's &  Vertices's monotonic \\
  $\delta x$ &  \\
  $\delta y$ &  \\
  \bottomrule
\end{longtable}
%\end{table}

\section{Requirements}

This section provides the functional requirements, the business tasks
that the software is expected to complete, and the nonfunctional
requirements, the qualities that the software is expected to exhibit.

\subsection{Functional Requirements}

\noindent \begin{itemize}

\item[R\refstepcounter{reqnum}\thereqnum \label{R_Inputs}:] Read the
  input file, and store the data. Necessary input data summarized in
  \tableref{Table:Inputs}. [\aref{A_Input}, \aref{A_Homo}]
  
  \renewcommand{\arraystretch}{1.5}
  %\noindent \begin{tabularx}{1.0\textwidth}{l l X}
  \noindent \begin{longtable}{l l p{12cm}} \toprule \textbf{symbol} &
    \refstepcounter{tablenum}  \label{Table:Inputs}
    \textbf{unit} & \textbf{description}\\ \midrule
    $\left(x,y\right)$ & $\text{m}$ & x and y coordinates for vertices
    of the slope layers, and for the water table if one exists.
    Assumed straight line fits between vertexes.\\
    $E$ & $\text{kPa}$ & Young's modulus for each layer of the
    slope.\\
    $c$ & $\text{kPa}$ & Cohesion for each slope layer. \\
    $v$ & $ / $ & Poisson's ratio for each soil layer. \\
    $\varphi$ & $\text{deg}$ & Effective angle of friction for each
    slope layer. \\
    $\gamma$ & $\frac{\text{kN}}{\text{m}^3}$ & Unit weight of dry
    soil / ground layer for each slope layer. \\
    $\gamma_{\text{Sat}}$ & $\frac{\text{kN}}{\text{m}^3}$ & Unit
    weight of saturated soil / ground layer for each slope
    layer. \\
    $\gamma_{\text{Wat}}$ & $\frac{\text{kN}}{\text{m}^3}$ & Unit
    weight of water. \\ \bottomrule
\end{longtable}

\item[R\refstepcounter{reqnum}\thereqnum \label{R_InitGen}:] Generate
  potential critical slip surface's for the input slope. 

\item[R\refstepcounter{reqnum}\thereqnum \label{R_KinAdm}:] Test the
  slip surfaces to determine if they are physically realizable based
  on a set of pass or fail criteria. [\aref{A_Concave}]

\item[R\refstepcounter{reqnum}\thereqnum \label{R_Slice}:] Prepare the
  slip surfaces for a method of slices or limit equilibrium analysis.

\item[R\refstepcounter{reqnum}\thereqnum \label{R_FS}:] Calculate the
  factors of safety of the slip surfaces.

\item[R\refstepcounter{reqnum}\thereqnum \label{R_Weight}:] Rank and
  weight the slopes based on their factor of safety, such that a slip
  surface with a smaller factor of safety has a larger weighting.

\item[R\refstepcounter{reqnum}\thereqnum \label{R_NewGen}:] Generate
  new potential critical slip surfaces based on previously analysed
  slip surfaces with low factors of safety.

\item[R\refstepcounter{reqnum}\thereqnum \label{R_Minimize}:] Repeat
  requirements \rref{R_KinAdm} to \rref{R_NewGen} until the
  minimum factor of safety remains approximately the same over a
  predetermined number of repetitions. Identify the slip surface that
  generates the minimum factor of safety as the critical slip surface.

\item[R\refstepcounter{reqnum}\thereqnum \label{R_PrepOutput}:]
  Prepare the critical slip surface for method of slices or limit
  equilibrium analysis.

\item[R\refstepcounter{reqnum}\thereqnum \label{R_Analyze}:] Calculate
  the factor of safety of the critical slip surface using the
  Morgenstern price method. Also calculate the local and global
  factors of safety for the critical slip using the RFEM method, and
  the displacement of the slice elements using the RFEM method.

\item[R\refstepcounter{reqnum}\thereqnum \label{R_Output}:] Display
  the critical slip surface and the slice element displacements
  graphically. Give the values of the factors of safety calculated by
  both methods, and the local factors of safety calculated by the RFEM
  method of analysis.
  
\end{itemize}


\subsection{Nonfunctional Requirements}

SSA is intended to be an educational tool, therefore accuracy and
performance speed are secondary program priorities to correctness,
understandability, reusability, and maintainability.

\section{Likely Changes}    

%\noindent \begin{itemize}
%\end{itemize}

\bibliographystyle {plain}
\bibliography {Bib_SSP}

\end{document}
