Developing sustainable research software is painful.  Pain points include:
not enough time for documentation and testing; insufficient experience
with development or technology; frequent changes.  One treatment is
literate programming: developers write programs with human understandability as
the first goal. For example Org-mode with Babel allows mixing code
(in multiple languages), documentation, test results, and output.
Co-locating code, documentation, and tests helps developers keep them in sync.
The consequences of changes become easier to see.

Another pain treatment is code generation, i.e., programs that
write code.  For example, Matlab and Maple have code
generation facilities. There are also tools suites like Spiral for digital
signal processing, and FEniCS (Finite Element and Computational Software).
A generator better embodies complex theory, allowing programs to be
written more rapidly, and regenerated quickly when changes occur.

In our holistic approach to pain management, we combine literate programming and
code generation via a pervasively generative approach.  It is about way more
than code: we also generate documentation, test scripts, test results, build
files, reports, and other resources. We build a knowledge base of models for
physics, computing, mathematics, and documentation, and then write "recipes"
that weave together this knowledge to generate the desired artifacts.

The holistic approach addresses multiple pain points.  Our knowledge-first
approach lets scientists focus on science rather than software. Our
infrastructure captures common practise about software development and
technology idiosyncrasies, alleviating that pain point as well.  As the recipes
are high-level programs that embody meaningful choices, they are relatively
easy to change.
