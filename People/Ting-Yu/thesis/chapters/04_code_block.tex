\chapter{Code Block}
unit of contents: linear vs nested

To deal with the second type of metadata, we need to answer the following 
questions: i) what is the type of the cell, markdown or code? ii) how do we 
know where to end a cell and start a new one? That is, when to ``cut" the 
contents?

The first question is easy to answer as there are no codes in SRS; contents are 
either images or texts. Therefore, we can simply make all cells of type 
markdown. 

As for the second question, we would like to first know what is a cell, 
conceptually? A working definition helps us understand what should do for 
implementation. In the discussion, we define a cell as a kind of ``unit of 
contents" \cite{cellsseparation}.  Thinking of what would be the ``unit" of 
contents, the intuition might be paragraphs or sections. In Drasil document 
language, SRS is made up of sections (Code~\ref{code:docdesc}), we can take 
advantage of it and make each section a cell. This solution do work. 
problem more the intuition answer of it is to have each section as a cell.

\begin{listing}
	\caption{Source Code for DocDesc}
	\label{code:docdesc}
	\begin{lstlisting}[language=haskell1]
		type DocDesc = [DocSection]
		
		data DocSection = TableOfContents
		| RefSec RefSec
		| IntroSec IntroSec
		| StkhldrSec StkhldrSec
		| GSDSec GSDSec
		| SSDSec SSDSec
		| ReqrmntSec ReqrmntSec
		| LCsSec LCsSec
		| UCsSec UCsSec
		| TraceabilitySec TraceabilitySec
		| AuxConstntSec AuxConstntSec
		| Bibliography
		| AppndxSec AppndxSec
		| OffShelfSolnsSec OffShelfSolnsSec
	\end{lstlisting}
\end{listing}