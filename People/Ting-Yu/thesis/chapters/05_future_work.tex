\chapter{Future Work} \label{chap:futureWork}
\section{JSON Printer Improvement}
While the current JSON printer can satisfy our need of generating Jupyter 
Notebook documents, there are a few issues need to be addressed. 

\section{Design Lesson Plan Content Type} \label{chap:contentType}
In Chapter~\ref{chap:codeBlock}, we discussed the potential limitations of the 
current \macblue{LayoutObj} for the structure of lesson plans. Most of the 
existing layout objects are designed for SRS data types such as 
\textbf{Definition}. To better accommodate the content types found in lesson 
plans, we could define a new set of \macblue{LayoutObj}s that are specific to 
these types of contents, such as a model that includes step-by-step 
instructions, since many lessons include these instructions. By doing so, we 
could ensure that each content type is handled explicitly by the appropriate 
\macblue{LayoutObj}, and we could create a separate cell for each type of 
content as discussed earlier. This approach would make it easier to split the 
content into logical units of information, and it would also make the resulting 
notebook more modular and easier to navigate.

\section{Develop the Structure of Lesson Plans}
The current structure of lesson plans includes several chapters such as 
learning objectives, case problems, and examples, and each chapter is made up 
of a list of contents. However, this structure needs improvement to better fit 
the architecture of each chapter. By gaining a better understanding of our 
lesson plans and the structure of each chapter, we can incorporate the newly 
designed specific content types (as discussed in \ref{chap:contentType}) into 
each chapter. For example, the Case Problem chapter should include the model of 
procedure analysis, which includes step-by-step instructions. Having a more 
detailed and adaptable structure of lesson plans would enable greater 
consistency and efficiency in creating and delivering content. Furthermore, it 
would make it easier to capture the key elements and knowledge of each lesson.