% Modified from https://www.overleaf.com/latex/templates/stockholm-university-poster-template/xsrhnggqmcsx
% The author, Qi Dang, based their work on that of Jiyu Chen, as mentioned above.
% Modified by Jason Balaci
% Last Modified: 2022-04-12

\documentclass[20pt,margin=1in,innermargin=-1in,blockverticalspace=-0.1in]{tikzposter}
\geometry{paperwidth=36in,paperheight=24in}

\usepackage{xparse}

\usepackage[canadian]{babel}

\usepackage[utf8]{inputenc}
\usepackage[T1]{fontenc}

\usepackage{blindtext}

\usepackage{amsmath}
\usepackage{amsfonts}
\usepackage{amsthm}
\usepackage{amssymb}
\usepackage{mathrsfs}
\usepackage{graphicx}
\usepackage{adjustbox}
\usepackage{enumitem}
\usepackage{csquotes}
\usepackage[backend=biber,style=numeric]{biblatex}

\usepackage{tikz}
\usetikzlibrary{cd}
\usepackage{quiver}
\usetikzlibrary{babel} % Make sure quiver/tikz uses babel

\usepackage{mwe} % for placeholder images

\addbibresource{references.bib}

\usepackage{McMasterTheme}
\tikzposterlatexaffectionproofoff
\usetheme{McMasterTheme}
\usecolorstyle{McMasterStyle}
\usetitlestyle{Filled}

\usepackage[scaled]{helvet}
\renewcommand\familydefault{\sfdefault} 
\usepackage[T1]{fontenc}

\title{Capturing Mathematical Knowledge in Drasil: the Case of Theories}
\author{Jason Balaci (balacij), Dr. Jacques Carette (carette)}
\institute{Department of Computing and Software, McMaster University}
\titlegraphic{\includegraphics[width=0.16\textwidth]{assets/mcm-cas_left-wht_eps.eps}}
\projectlogo{\includegraphics[width=0.06\textwidth]{assets/Drasil Logo White.png}}

% begin document
\begin{document}

\maketitle

\centering
\begin{columns}
    %----------------------------------
    %- First Column
    %----------------------------------
    \column{0.32}

    \block{What is Drasil?}{ Drasil is a framework for generating families of
        software artifacts from a coherent knowledge base, following its mantra,
        ``Generate All The Things!''. Drasil uses a series of variably sized
        Domain-Specific Languages (DSLs) to describe various fragments of
        knowledge that domain experts and users alike may use to piece together
        fragments of knowledge into a coherent ``story''. Through forming some
        coherent ``story'' in a domain captured by Drasil, a representational
        software artifact may be generated. Drasil currently focuses on
        Scientific Computing Software (SCS), following Smith and Lai's Software
        Requirements Specifications (SRS) template as described in
        \cite{SmithAndLai2005}. Behind the scenes of the SRS, a mathematical
        language is used to describe various theories, and have representational
        software constructed via compiling to Generic Object-Oriented Language
        (GOOL) \cite{Carette2020GOOL}. Through encoding knowledge in Drasil, an
        increase in productivity (and maintainability) in building reliable and
        traceable software artifacts is observed \cite{SzymczakEtAl2016},
        specifically in SCS \cite{Smith2018}. Drasil's source code (Haskell),
        case studies, and documentation studies can be found on its website;
        \url{https://jacquescarette.github.io/Drasil/}.}

    \block{Research Motivation/Problem}{
        \begin{itemize}
            \item \textit{Theories} are constructed using a natural English
                  description and a single term from a single universal
                  mathematical expression language.
                  \begin{itemize}
                      \item Expressions must be precisely written in a manner
                            ``digestible'' to the code generator so that a
                            representational software code snippet can be
                            constructed.
                      \item Not all expressions have definite values and are
                            immediately usable in all programming languages.
                  \end{itemize}
            \item Only a handful of the case studies generate code,
                  because...
                  \begin{itemize}
                      \item Understanding of the expressions is weak and brittle
                            as they don't expose sufficient information about
                            the theories.
                      \item Software artifacts are validated, we must obey rigid
                            rules of other languages.
                      \item Cognitive load of writing expressions in precise
                            manners to accommodate the code generator would
                            increase.
                  \end{itemize}
        \end{itemize}
    }

    %----------------------------------
    %- Second Column
    %----------------------------------
    \column{0.36}

    \block{Mathematical Knowledge Flow}{
        \vspace{-1em}
        % Originally created with https://q.uiver.app/?q=WzAsNyxbOCw0XSxbMSwwLCJUaGVvcmllcyAoUmVsYXRpb25Db25jZXB0KSJdLFswLDAsIkV4cHJlc3Npb25zIChFeHByKSJdLFsyLDAsIk1hdGhlbWF0aWNhbCBLbm93bGVkZ2UiXSxbMiwxLCJRdWFudGl0eSBEZWZpbml0aW9ucyAoUURlZm5zKSJdLFsyLDIsIkNvZGUgKEdPT0wpIl0sWzEsMSwiU2NpZW50aWZpYyBSZXF1aXJlbWVudHMgU3BlY2lmaWNhdGlvbnMgKFNSUykiXSxbMiwxXSxbMSwzXSxbMyw0XSxbNCw1XSxbMSw2XV0=
        % But requires some post-processing
        \[\begin{tikzcd}[cramped,sep=small,align=center,ampersand replacement=\&]
                {\parbox{0.2\linewidth}{\centering Expression (\textit{Expr}) \\$+$ Description}}
                \& {\parbox{0.24\linewidth}{\centering Theory \\(\textit{RelationConcept})}}
                \& |[draw=orange, cloud, aspect=2.8, inner sep=0pt, line width=4]| {\parbox{0.18\linewidth}{\centering Mathematical Knowledge}} \\

                {\parbox{0.25\linewidth}{\centering Scientific Requirements Specifications (SRS)}}
                \& {\parbox{0.1\linewidth}{\centering Code (GOOL)}}
                \& {\parbox{0.2\linewidth}{\centering Quantity Definition (\textit{QDefn})}} \\

                \arrow[line width=2, color=green, from=1-1, to=1-2]
                \arrow[line width=2, squiggly, color=red, from=1-2, to=1-3]
                \arrow[line width=2, squiggly, color=red, from=1-3, to=2-3]
                \arrow[line width=2, squiggly, color=red, from=2-3, to=2-2]
                \arrow[line width=2, color=green, from=1-2, to=2-1]
            \end{tikzcd}\]
        \vspace{-2.5em}

        \small
        \begin{tabular}{llllll}
            \textcolor{green}{$\rightarrow$}    & Stable transformation   &
            \textcolor{red}{$\rightsquigarrow$} & Unstable transformation & \tikz{\node[cloud, aspect=3, draw=orange] (c) at (0,0) {};} & Partially/implicitly captured \\
        \end{tabular}
    }

    \block{Capturing Mathematical Knowledge}{
        \vspace{-1em}
        % Originally created with https://q.uiver.app/?q=WzAsNixbMSwxLCJNYXRoZW1hdGljYWwgS25vd2xlZGdlIChNb2RlbEtpbmRzKSJdLFsxLDIsIkNvZGUgKEdPT0wpIl0sWzAsMCwiRXF1YXRpb25hbE1vZGVsIChRRGVmbikiXSxbMSwwLCJFcXVhdGlvbmFsUmVhbG0gKE11bHRpRGVmbikiXSxbMiwwLCJldGMuIl0sWzAsMiwiU1JTIl0sWzAsMSwiT25seSBFcXVhdGlvbmFsIl0sWzIsMF0sWzMsMF0sWzQsMF0sWzAsNV1d
        % But requires some post-processing
        \[\begin{tikzcd}[cramped,sep=small,align=center,ampersand replacement=\&]
                {\parbox{0.25\linewidth}{\centering EquationalModel (\textit{QDefn Expr'/ModelExpr})}}
                \& {\parbox{0.25\linewidth}{\centering EquationalRealm (\textit{MultiDefn Expr'/ModelExpr})}}
                \& {\parbox{0.25\linewidth}{\centering etc.}} \\

                {\parbox{0.1\linewidth}{\centering SRS}}
                \& {\parbox{0.25\linewidth}{\centering Mathematical Knowledge (\textit{ModelKinds})}}
                \& {\parbox{0.1\linewidth}{\centering Code (GOOL)}} \\

                \arrow[line width=2, squiggly, color=blue, from=2-2, to=2-3]
                \arrow[line width=2, color=green, from=1-1, to=2-2]
                \arrow[line width=2, color=green, from=1-2, to=2-2]
                \arrow[line width=2, color=green, from=1-3, to=2-2]
                \arrow[line width=2, color=green, from=2-2, to=2-1]
            \end{tikzcd}\]
        \vspace{-2em}

        \small
        \begin{tabular}{llll}
            \textcolor{green}{$\rightarrow$} & Stable transformation & \textcolor{blue}{$\rightsquigarrow$} & ``Close to stable'' transformation \\
        \end{tabular}
    }

    \block{What changed?}{
        \vspace{-1em}
        \begin{itemize}
            \item More static validation! Safer generation!
            \item Expression language division: \(\textit{Expr} \longrightarrow \textit{Expr'}\ \cup\ \textit{ModelExpr}\ \cup\ \textit{CodeExpr}\)
                  \begin{itemize}
                      \item \textit{Expr'}: Restricting the language to terms
                            ``well-understood'', with a definite meaning and
                            value.
                      \item \textit{ModelExpr}: Restricting the language to
                            terms with definite meanings, but not necessarily
                            definite values.
                      \item \textit{CodeExpr}: Restricting the language to terms
                            that have a definite meaning and value to most
                            general purpose programming languages, with some
                            goodies for OOP.
                  \end{itemize}
            \item Created Typed Tagless Final (TTF) \cite{Carette2009}
                  encodings; Expression creation is just as easy as it ever was!
        \end{itemize}
    }

    %----------------------------------
    %- Third Column
    %----------------------------------
    \column{0.32}

    \block{}{
        \vspace{-1.5em}
        \begin{itemize}
            \item Created a system of classifying \textit{theories}
                  (\textit{ModelKinds})
                  \begin{itemize}
                      \item Increased the depth \& breadth of knowledge
                            contained.
                      \item First-class representation of theories, with their
                            meaningful components fully exposed. No more
                            brittle \textit{cast}-like conversion of mathematical
                            expressions (low info. density) to
                            well-understood pieces of knowledge (high
                            info. density).
                      \item Instances of theories usable in a wide variety of
                            ways can be statically \& reliably checked for
                            validity.
                      \item Creating instances of theories comes with
                            projectional editor-like ease.
                  \end{itemize}
            \item Improving productivity, stability, and flexibility in usage
            \item Current theory types:
                  \begin{itemize}
                      \item \textit{EquationalModel}: \(x = f(a, b, c, \ldots)\)
                      \item \textit{EquationalRealm}: \(x = a \lor x = b \lor x = c \lor \ldots\)
                      \item \textit{EquationalConstraints}: \(a \land b \land c \land \ldots\)
                      \item \textit{DEModel}: \(\ldots\frac{dy}{dx}\ldots\)
                  \end{itemize}
        \end{itemize}
        \vspace{-0.5em}
    }

    \block{Conclusion \& Future Work}{
        \vspace{-1.2em}
        \begin{itemize}
            \item Through capturing and classifying mathematical knowledge, we
                  earn significant gains in flexibility, usability, and productivity.
            \item \textit{In progress}: Adding extra type information to
                  expressions, allowing us to add type information to GOOL, and
                  further improve static validity rules of various mathematical
                  constructions.
            \item Teaching more theories to Drasil!
        \end{itemize}
    }

    \block{References}{
        \vspace{-1em}
        \renewcommand*{\bibfont}{\footnotesize}
        \printbibliography[heading=none]
    }

\end{columns}

\end{document}
