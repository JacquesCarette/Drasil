\documentclass[xcolor={dvipsnames}]{beamer}

\usepackage[T1]{fontenc}

\usepackage{amssymb}
\usepackage{stmaryrd}
\usepackage{amsfonts}
\usepackage{amsmath}
\usepackage{latexsym}
\usepackage{url}

\usepackage{listings}

\usepackage{mathtools}
\usepackage{calc}
\usepackage{lmodern}
\usepackage{changepage}
\usepackage{hyperref}
\usepackage{graphicx}

\usetheme{Madrid}

\setbeamertemplate{caption}{\raggedright\insertcaption\par}

\title{Adding Types and Theory Kinds to Drasil}
\author[J. Balaci]{Jason Balaci\\\small{}Under the supervision of Dr. Jacques Carette}

\institute[McMaster U.]{McMaster University}
\date{Dec. $8^{\text{th}}$, 2022}

\AtBeginSection[]
{
  \begin{frame}
    \frametitle{Table of Contents}
    \tableofcontents[currentsection]
  \end{frame}
}

% For code highlighting
\usepackage[newfloat,outputdir=build]{minted}
\usemintedstyle{colorful}

% For loading images
\usepackage{graphicx}
\graphicspath{ {./assets/img/} }


% For fancy pictures
\usepackage{tikz}
\usetikzlibrary{shapes,arrows,cd}
\usetikzlibrary{babel} % Make sure quiver/tikz uses babel
\usetikzlibrary {arrows.meta,graphs,graphdrawing}
\usegdlibrary {layered}

\usepackage{todonotes}


\newcommand{\inlineHs}[1]{\mintinline{haskell}|#1|}
% Command based on: https://tex.stackexchange.com/questions/266811/define-a-new-command-with-parameters-inside-newcommand
\newcommand{\codeName}[1]{\expandafter\newcommand\csname #1\endcsname{\inlineHs{#1}}}

\codeName{CodeExpr}
\codeName{Expr}
\codeName{ModelExpr}

\begin{document}

%------------------------------------------------------------------------------
% TITLE
%------------------------------------------------------------------------------
\frame{\titlepage}

%------------------------------------------------------------------------------
% TABLE OF CONTENTS
%------------------------------------------------------------------------------

\begin{frame}
\frametitle{Table of Contents}
\tableofcontents
\end{frame}

%------------------------------------------------------------------------------
% OVERVIEW
%------------------------------------------------------------------------------
\section{Overview}

\begin{frame}
  \frametitle{Overview}

  \begin{enumerate}
    \item Background: Drasil
    \item 4 Research Areas:
      \begin{enumerate}
        \item Capturing and interpreting theories
        \item Mathematical expressions in various contexts
        \item Well-typedness of mathematical expressions
        \item An extensible database for remembering everything
      \end{enumerate}
  \end{enumerate}

\end{frame}

%------------------------------------------------------------------------------
% BACKGROUND: DRASIL
%------------------------------------------------------------------------------
\section{Drasil}

\begin{frame}
  \frametitle{What is Drasil? How does it work?}
  \framesubtitle{\textquotedblleft{}Generate All The Things!\textquotedblright{}}

  \begin{columns}
    \begin{column}{0.45\textwidth}
      \includegraphics[width=\textwidth]{drasil_logo.png}
    \end{column}
    \hfill
    \begin{column}{0.45\textwidth}
      \begin{enumerate}
        \item Software generation suite for ``well-understood'' domains
        \item De-duplicating and capturing knowledge across software artifacts
        \item Uses a Software Requirements Specification (SRS) template to
              decompose scientific problems and generate software
      \end{enumerate}
    \end{column}
  \end{columns}

\end{frame}

\begin{frame}
  \frametitle{Example}
  \framesubtitle{SRS to Code}

  \begin{columns}
    \begin{column}{0.45\textwidth}
      Using SRS components:
      \begin{enumerate}
        \item Symbols (inputs, outputs, and everything in-between)
        \item Problem description and goals
        \item Assumptions
        \item Abstract theories
        \item Concrete theories
        \item \(\ldots{}\)
      \end{enumerate}
    \end{column}
    \hfill
    \begin{column}{0.5\textwidth}
      \includegraphics[width=\textwidth]{roughNetworkOfDomains.png}
    \end{column}
  \end{columns}

\end{frame}

%------------------------------------------------------------------------------
% THEORIES
%------------------------------------------------------------------------------
\section{Theories in Drasil}

\begin{frame}
  \frametitle{How are theories used?}

  \begin{columns}
    \begin{column}{0.45\textwidth}
      \missingfigure{Theory in Haskell}

      \missingfigure{relToQD Snippet}
    \end{column}
    \hfill
    \begin{column}{0.45\textwidth}
      \missingfigure{Gen'd and rendered in SRS}

      \missingfigure{Gen'd code}
    \end{column}
  \end{columns}
\end{frame}

\begin{frame}
  \frametitle{Decompose, classify, and encode}

  \missingfigure{ModelKinds Haskell}

  \begin{itemize}
    \item EquationalModel: \(x := f(x,y,z,\ldots{})\)
    \item EquationalConstraints: \(a \land b \land c \land \ldots{}\)
    \item EquationalRealm: \(x := f(x,y,z,\ldots{}) \lor x := g(x,y,z,\ldots{}) \lor \ldots{}\)
    \item DEModel \& NewDEModel: \(dy = f(x, y, \ldots{})\)
    \item OthModel: ?
  \end{itemize}
\end{frame}

\begin{frame}
  \frametitle{Ripple effects}

  \begin{itemize}
    \item Tangible categorization through type information
    \item Simpler creation, interaction, and analysis
    \item Flexible, more opportunity for (domain-specific) interpretation
  \end{itemize}

\end{frame}

%------------------------------------------------------------------------------
% EXPRESSIONS
%------------------------------------------------------------------------------
\section{Expressions}

\begin{frame}
  \frametitle{What are they used for?}

  In theories!

  \begin{columns}
    \begin{column}{0.3\textwidth}
      \missingfigure{}
      In code
    \end{column}
    \hfill
    \begin{column}{0.3\textwidth}
      \missingfigure{}
      In concrete theories
    \end{column}
    \hfill
    \begin{column}{0.3\textwidth}
      \missingfigure{}
      In abstract theories
    \end{column}
  \end{columns}
\end{frame}

\begin{frame}
  \frametitle{Restricting terms by context}

  \begin{enumerate}
    \item Split up the expression language by context
    \item Use a ``Typed-Tagless'' encoding to make usage seamless and
          interoperable
  \end{enumerate}

  \[\Expr{} \Rightarrow{} \Expr{} \cup{} \ModelExpr{} \cup{} \CodeExpr{}\]

\end{frame}

\begin{frame}
  \frametitle{Ripple effects}

  \begin{enumerate}
    \item Restricting terms by context
    \item Stop users from entering in un-translatable expressions
    \item Know when Equational Models are usable for code generation
  \end{enumerate}

\end{frame}

\begin{frame}
  \frametitle{Continued problems: formation}

  \missingfigure{Examples of ill-typed expressions}
\end{frame}

\begin{frame}
  \frametitle{Typing}

  \begin{enumerate}
    \item Adding optional \emph{bidirectional} type-checking
    \item Running type-checking post-facto to check all data en masse
  \end{enumerate}

  \begin{columns}
    \begin{column}{0.45\textwidth}
      \missingfigure{Typing rules}
    \end{column}
    \hfill
    \begin{column}{0.45\textwidth}
      \missingfigure{Typing rules}
    \end{column}
  \end{columns}

\end{frame}

\begin{frame}
  \frametitle{Ripple effects}

  \begin{enumerate}
    \item Found many typing issues and areas of code (e.g., vectors) that need
          more work
    \item First capture of implicit ``chunk'' constraints
    \item Realized we can move more ``checks'' to be done post-facto or
          on-demand
  \end{enumerate}

\end{frame}

%------------------------------------------------------------------------------
% DATA (CHUNKS)
%------------------------------------------------------------------------------
\section{Data in Drasil (\textquotedblleft{}Chunks\textquotedblright{})}

\begin{frame}
  \frametitle{How is data stored in Drasil?}

  \missingfigure{Image of chunks funnelled into Drasil: chunk -> (Drasil) <- chunk}

  \missingfigure{ChunkDB Haskell}

\end{frame}

\begin{frame}
  \frametitle{Scaling against new \textit{kinds} of data}

  \begin{center}
    Mask the type information!
  \end{center}

  \missingfigure{``Chunk'' data type Haskell}

  \missingfigure{Upgraded ChunkDB Haskell}
\end{frame}

\begin{frame}
  \frametitle{Ripple effects}

  \begin{enumerate}
    \item Everything is in one place!
    \item Chunk analysis capabilities:
      \begin{enumerate}
        \item (easier) type usage analytics
        \item chunk dependency tree
        \item find cyclic knowledge
        \item dumping all chunks at once
      \end{enumerate}
    \item ChunkDB arithmetic?
    \item (easier) Generalized chunk constraints!
    \item Usable ``base'' across Drasil-like projects
  \end{enumerate}
\end{frame}

%------------------------------------------------------------------------------
% FUTURE WORK
%------------------------------------------------------------------------------
\section{Future Work}

\begin{frame}
  \frametitle{Future Work}

  Point-form notes from my thesis.
\end{frame}

%------------------------------------------------------------------------------
% CONCLUSION
%------------------------------------------------------------------------------
\section{Conclusion}

\begin{frame}
  \frametitle{Concluding Remarks \& Takeaways}

  Re-stating 4 major bodies of work done and their key conclusions/takeaways
\end{frame}

\end{document}
