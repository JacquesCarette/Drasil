\documentclass[12pt]{article}
\usepackage{fontspec}
\usepackage{fullpage}
\usepackage{hyperref}
\hypersetup{bookmarks=true,colorlinks=true,linkcolor=red,citecolor=blue,filecolor=magenta,urlcolor=cyan}
\usepackage{amsmath}
\usepackage{amssymb}
\usepackage{mathtools}
\usepackage{unicode-math}
\usepackage{tabu}
\usepackage{longtable}
\usepackage{booktabs}
\usepackage{caption}
\usepackage{enumitem}
\usepackage{graphics}
\usepackage{filecontents}
\usepackage[backend=bibtex]{biblatex}
\usepackage{url}
\setmathfont{Latin Modern Math}
\newcommand{\gt}{\ensuremath >}
\newcommand{\lt}{\ensuremath <}
\global\tabulinesep=1mm
\newlist{symbDescription}{description}{1}
\setlist[symbDescription]{noitemsep, topsep=0pt, parsep=0pt, partopsep=0pt}
\bibliography{bibfile}
\title{Software Requirements Specification for Pendulum}
\author{Olu Owojaiye}
\begin{document}
\maketitle
\tableofcontents
\newpage
\section{Reference Material}
\label{Sec:RefMat}
This section records information for easy reference.

\subsection{Table of Units}
\label{Sec:ToU}
The unit system used throughout is SI (Système International d'Unités). In addition to the basic units, several derived units are also used. For each unit, \hyperref[Table:ToU]{Tab: ToU} lists the symbol, a description and the SI name.

\begin{longtable}{l l l}
\toprule
\textbf{Symbol} & \textbf{Description} & \textbf{SI Name}
\\
\midrule
\endhead
${{}^{\circ}}$ & angle & degree
\\
${\text{kg}}$ & mass & kilogram
\\
${\text{m}}$ & length & metre
\\
${\text{N}}$ & force & newton
\\
${\text{rad}}$ & angle & radian
\\
${\text{s}}$ & time & second
\\
\bottomrule
\caption{Table of Units}
\label{Table:ToU}
\end{longtable}
\subsection{Table of Symbols}
\label{Sec:ToS}
The symbols used in this document are summarized in \hyperref[Table:ToS]{Tab: ToS} along with their units. Throughout the document, symbols in bold will represent vectors, and scalars otherwise. The symbols are listed in alphabetical order. For vector quantities, the units shown are for each component of the vector.

\begin{longtable}{l l l}
\toprule
\textbf{Symbol} & \textbf{Description} & \textbf{Units}
\\
\midrule
\endhead
${a_{\text{x}}}$ & $x$-component of acceleration & $\frac{\text{m}}{\text{s}^{2}}$
\\
${a_{\text{y}}}$ & $y$-component of acceleration & $\frac{\text{m}}{\text{s}^{2}}$
\\
$\mathbf{a}$ & Acceleration & $\frac{\text{m}}{\text{s}^{2}}$
\\
$\mathbf{F}$ & Force & ${\text{N}}$
\\
$\mathbf{g}$ & Gravitational acceleration & $\frac{\text{m}}{\text{s}^{2}}$
\\
$\mathbf{\hat{i}}$ & Unit Vector & --
\\
${L_{\text{rod}}}$ & Length of rod & ${\text{m}}$
\\
$m$ & Mass & ${\text{kg}}$
\\
${{p_{\text{x}}}^{\text{i}}}$ & $x$-component of initial position & ${\text{m}}$
\\
${{p_{\text{y}}}^{\text{i}}}$ & $y$-component of initial position & ${\text{m}}$
\\
$\mathbf{p}$ & Position & ${\text{m}}$
\\
$\mathbf{T}$ & Tension & ${\text{N}}$
\\
$t$ & Time & ${\text{s}}$
\\
${v_{\text{x}}}$ & $x$-component of velocity & $\frac{\text{m}}{\text{s}}$
\\
${v_{\text{y}}}$ & $y$-component of velocity & $\frac{\text{m}}{\text{s}}$
\\
$\mathbf{v}$ & Velocity & $\frac{\text{m}}{\text{s}}$
\\
$α$ & Angular Acceleration & $\frac{\text{rad}}{\text{s}^{2}}$
\\
$θ$ & Angle of pendulum & ${{}^{\circ}}$
\\
$ω$ & Angular Velocity & $\frac{\text{rad}}{\text{s}}$
\\
\bottomrule
\caption{Table of Symbols}
\label{Table:ToS}
\end{longtable}
\subsection{Abbreviations and Acronyms}
\label{Sec:TAbbAcc}
\begin{longtable}{l l}
\toprule
\textbf{Abbreviation} & \textbf{Full Form}
\\
\midrule
\endhead
\bottomrule
\caption{Abbreviations and Acronyms}
\label{Table:TAbbAcc}
\end{longtable}
\section{Introduction}
\label{Sec:Intro}
A pendulum consists of mass attached to the end of a rod, its moving curve is highly sensitive to initial conditions Therefore, it is useful to have a program the to simulate the motion of the pendulum to exhibit the chaotic characteristics of it The program documented here is called pendulum.

The following section provides an overview of the Software Requirements Specification (SRS) for Pendulum. This section explains the purpose of this document, the scope of the requirements, the characteristics of the intended reader, and the organization of the document.

\subsection{Scope of Requirements}
\label{Sec:ReqsScope}
The scope of the requirements includes the analysis of a two-dimensional (2D) pendulum motion problem with various initial conditions..

\section{Specific System Description}
\label{Sec:SpecSystDesc}
This section first presents the problem description, which gives a high-level view of the problem to be solved. This is followed by the solution characteristics specification, which presents the assumptions, theories, and definitions that are used.

\subsection{Problem Description}
\label{Sec:ProbDesc}
A system is needed to is needed to efficiently and correctly predict the motion pendulum.

\subsubsection{Terminology and Definitions}
\label{Sec:TermDefs}
This subsection provides a list of terms that are used in the subsequent sections and their meaning, with the purpose of reducing ambiguity and making it easier to correctly understand the requirements.

\begin{itemize}
\item{Gravity: The force that attracts one physical body with mass to another.}
\item{Cartesian coordinate system: A coordinate system that specifies each point uniquely in a plane by a set of numerical coordinates, which are the signed distances to the point from two fixed perpendicular oriented lines, measured in the same unit of length (from \cite{cartesianWiki}).}
\end{itemize}
\subsubsection{Physical System Description}
\label{Sec:PhysSyst}
The physical system of Pendulum, as shown in \hyperref[Figure:Motion]{Fig:Motion}, includes the following elements:

\begin{itemize}
\item[PS1:]{The rod.}
\item[PS2:]{The mass.}
\end{itemize}
\begin{figure}
\begin{center}
\includegraphics[width=0.7\textwidth]{../../../datafiles/DblPendulum/pendulum.PNG}
\caption{The physical system}
\label{Figure:Motion}
\end{center}
\end{figure}
\subsubsection{Goal Statements}
\label{Sec:GoalStmt}
Given the the mass length of the rod, initial angle of the mass and the gravitational constant, the goal statements are:

\begin{itemize}
\item[Motion-of-the-mass:\phantomsection\label{motionMass}]{the Calculate the motion of the mass}
\end{itemize}
\subsection{Solution Characteristics Specification}
\label{Sec:SolCharSpec}
The instance models that govern Pendulum are presented in \hyperref[Sec:IMs]{Section: Instance Models}. The information to understand the meaning of the instance models and their derivation is also presented, so that the instance models can be verified.

\subsubsection{Assumptions}
\label{Sec:Assumps}
This section simplifies the original problem and helps in developing the theoretical models by filling in the missing information for the physical system. The assumptions refine the scope by providing more detail.

\begin{itemize}
\item[pend2DMotion:\phantomsection\label{pend2DMotion}]{The Pendulum motion is two-dimensional (2D).}
\item[cartCoord:\phantomsection\label{cartCoord}]{A Cartesian coordinate system is used}
\item[cartCoordRight:\phantomsection\label{cartCoordRight}]{The Cartesian coordinate system is right-handed where positive $x$-axis. and $y$-axis point right up}
\item[yAxisDir:\phantomsection\label{yAxisDir}]{The The direction of the $y$-axis is directed opposite to gravity.}
\item[startOrigin:\phantomsection\label{startOrigin}]{The Pendulum is attached to the origin.}
\end{itemize}
\subsubsection{Theoretical Models}
\label{Sec:TMs}
This section focuses on the general equations and laws that Pendulum is based on.

\vspace{\baselineskip}
\noindent
\begin{minipage}{\textwidth}
\begin{tabular}{>{\raggedright}p{0.13\textwidth}>{\raggedright\arraybackslash}p{0.82\textwidth}}
\toprule \textbf{Refname} & \textbf{TM:acceleration}
\phantomsection 
\label{TM:acceleration}
\\ \midrule \\
Label & Acceleration
        
\\ \midrule \\
Equation & \begin{displaymath}
           \mathbf{a}=\frac{\,d\mathbf{v}}{\,dt}
           \end{displaymath}
\\ \midrule \\
Description & \begin{symbDescription}
              \item{$\mathbf{a}$ is the acceleration ($\frac{\text{m}}{\text{s}^{2}}$)}
              \item{$t$ is the time (${\text{s}}$)}
              \item{$\mathbf{v}$ is the velocity ($\frac{\text{m}}{\text{s}}$)}
              \end{symbDescription}
\\ \midrule \\
Source & \cite{accelerationWiki} and \cite[(pg. 7)]{hibbeler2004}
         
\\ \midrule \\
RefBy & 
\\ \bottomrule
\end{tabular}
\end{minipage}
\vspace{\baselineskip}
\noindent
\begin{minipage}{\textwidth}
\begin{tabular}{>{\raggedright}p{0.13\textwidth}>{\raggedright\arraybackslash}p{0.82\textwidth}}
\toprule \textbf{Refname} & \textbf{TM:velocity}
\phantomsection 
\label{TM:velocity}
\\ \midrule \\
Label & Velocity
        
\\ \midrule \\
Equation & \begin{displaymath}
           \mathbf{v}=\frac{\,d\mathbf{p}}{\,dt}
           \end{displaymath}
\\ \midrule \\
Description & \begin{symbDescription}
              \item{$\mathbf{v}$ is the velocity ($\frac{\text{m}}{\text{s}}$)}
              \item{$t$ is the time (${\text{s}}$)}
              \item{$\mathbf{p}$ is the position (${\text{m}}$)}
              \end{symbDescription}
\\ \midrule \\
Source & \cite{velocityWiki} and \cite[(pg. 6)]{hibbeler2004}
         
\\ \midrule \\
RefBy & 
\\ \bottomrule
\end{tabular}
\end{minipage}
\vspace{\baselineskip}
\noindent
\begin{minipage}{\textwidth}
\begin{tabular}{>{\raggedright}p{0.13\textwidth}>{\raggedright\arraybackslash}p{0.82\textwidth}}
\toprule \textbf{Refname} & \textbf{TM:NewtonSecLawMot}
\phantomsection 
\label{TM:NewtonSecLawMot}
\\ \midrule \\
Label & Newton's second law of motion
        
\\ \midrule \\
Equation & \begin{displaymath}
           \mathbf{F}=m \mathbf{a}
           \end{displaymath}
\\ \midrule \\
Description & \begin{symbDescription}
              \item{$\mathbf{F}$ is the force (${\text{N}}$)}
              \item{$m$ is the mass (${\text{kg}}$)}
              \item{$\mathbf{a}$ is the acceleration ($\frac{\text{m}}{\text{s}^{2}}$)}
              \end{symbDescription}
\\ \midrule \\
Notes & The net force $\mathbf{F}$ on a body is proportional to the acceleration $\mathbf{a}$ of the body, where $m$ denotes the mass of the body as the constant of proportionality.
        
\\ \midrule \\
Source & --
         
\\ \midrule \\
RefBy & \hyperref[IM:calOfAngularAcceleration]{IM: calOfAngularAcceleration}
        
\\ \bottomrule
\end{tabular}
\end{minipage}
\subsubsection{General Definitions}
\label{Sec:GDs}
This section collects the laws and equations that will be used to build the instance models.

\vspace{\baselineskip}
\noindent
\begin{minipage}{\textwidth}
\begin{tabular}{>{\raggedright}p{0.13\textwidth}>{\raggedright\arraybackslash}p{0.82\textwidth}}
\toprule \textbf{Refname} & \textbf{GD:velocityIX}
\phantomsection 
\label{GD:velocityIX}
\\ \midrule \\
Label & The $x$-component of velocity of the pendulum
        
\\ \midrule \\
Units & $\frac{\text{m}}{\text{s}}$
        
\\ \midrule \\
Equation & \begin{displaymath}
           {v_{\text{x}}}=ω {L_{\text{rod}}} \cos\left(θ\right)
           \end{displaymath}
\\ \midrule \\
Description & \begin{symbDescription}
              \item{${v_{\text{x}}}$ is the $x$-component of velocity ($\frac{\text{m}}{\text{s}}$)}
              \item{$ω$ is the angular velocity ($\frac{\text{rad}}{\text{s}}$)}
              \item{${L_{\text{rod}}}$ is the length of rod (${\text{m}}$)}
              \item{$θ$ is the angle of pendulum (${{}^{\circ}}$)}
              \end{symbDescription}
\\ \midrule \\
Source & --
         
\\ \midrule \\
RefBy & 
\\ \bottomrule
\end{tabular}
\end{minipage}
\paragraph{Detailed derivation of $x$-component velocity:}
\label{GD:velocityIXDeriv}
\begin{displaymath}
{v_{\text{x}}}=ω {L_{\text{rod}}} \cos\left(θ\right)
\end{displaymath}
\vspace{\baselineskip}
\noindent
\begin{minipage}{\textwidth}
\begin{tabular}{>{\raggedright}p{0.13\textwidth}>{\raggedright\arraybackslash}p{0.82\textwidth}}
\toprule \textbf{Refname} & \textbf{GD:velocityIY}
\phantomsection 
\label{GD:velocityIY}
\\ \midrule \\
Label & The $y$-component of velocity of the pendulum
        
\\ \midrule \\
Units & $\frac{\text{m}}{\text{s}}$
        
\\ \midrule \\
Equation & \begin{displaymath}
           {v_{\text{y}}}=ω {L_{\text{rod}}} \cos\left(θ\right)
           \end{displaymath}
\\ \midrule \\
Description & \begin{symbDescription}
              \item{${v_{\text{y}}}$ is the $y$-component of velocity ($\frac{\text{m}}{\text{s}}$)}
              \item{$ω$ is the angular velocity ($\frac{\text{rad}}{\text{s}}$)}
              \item{${L_{\text{rod}}}$ is the length of rod (${\text{m}}$)}
              \item{$θ$ is the angle of pendulum (${{}^{\circ}}$)}
              \end{symbDescription}
\\ \midrule \\
Source & --
         
\\ \midrule \\
RefBy & 
\\ \bottomrule
\end{tabular}
\end{minipage}
\paragraph{Detailed derivation of $y$-component velocity:}
\label{GD:velocityIYDeriv}
\begin{displaymath}
{v_{\text{y}}}=ω {L_{\text{rod}}} \cos\left(θ\right)
\end{displaymath}
\vspace{\baselineskip}
\noindent
\begin{minipage}{\textwidth}
\begin{tabular}{>{\raggedright}p{0.13\textwidth}>{\raggedright\arraybackslash}p{0.82\textwidth}}
\toprule \textbf{Refname} & \textbf{GD:accelerationIX}
\phantomsection 
\label{GD:accelerationIX}
\\ \midrule \\
Label & The $x$-component of acceleration of the pendulum
        
\\ \midrule \\
Units & $\frac{\text{m}}{\text{s}^{2}}$
        
\\ \midrule \\
Equation & \begin{displaymath}
           {a_{\text{x}}}=-ω {L_{\text{rod}}} \sin\left(θ\right)+α {L_{\text{rod}}} \cos\left(θ\right)
           \end{displaymath}
\\ \midrule \\
Description & \begin{symbDescription}
              \item{${a_{\text{x}}}$ is the $x$-component of acceleration ($\frac{\text{m}}{\text{s}^{2}}$)}
              \item{$ω$ is the angular velocity ($\frac{\text{rad}}{\text{s}}$)}
              \item{${L_{\text{rod}}}$ is the length of rod (${\text{m}}$)}
              \item{$θ$ is the angle of pendulum (${{}^{\circ}}$)}
              \item{$α$ is the angular acceleration ($\frac{\text{rad}}{\text{s}^{2}}$)}
              \end{symbDescription}
\\ \midrule \\
Source & --
         
\\ \midrule \\
RefBy & 
\\ \bottomrule
\end{tabular}
\end{minipage}
\paragraph{Detailed derivation of $x$-component acceleration:}
\label{GD:accelerationIXDeriv}
\begin{displaymath}
{a_{\text{x}}}=-ω {L_{\text{rod}}} \sin\left(θ\right)+α {L_{\text{rod}}} \cos\left(θ\right)
\end{displaymath}
\vspace{\baselineskip}
\noindent
\begin{minipage}{\textwidth}
\begin{tabular}{>{\raggedright}p{0.13\textwidth}>{\raggedright\arraybackslash}p{0.82\textwidth}}
\toprule \textbf{Refname} & \textbf{GD:accelerationIY}
\phantomsection 
\label{GD:accelerationIY}
\\ \midrule \\
Label & The $y$-component of acceleration of the pendulum
        
\\ \midrule \\
Units & $\frac{\text{m}}{\text{s}^{2}}$
        
\\ \midrule \\
Equation & \begin{displaymath}
           {a_{\text{y}}}=ω {L_{\text{rod}}} \cos\left(θ\right)+α {L_{\text{rod}}} \sin\left(θ\right)
           \end{displaymath}
\\ \midrule \\
Description & \begin{symbDescription}
              \item{${a_{\text{y}}}$ is the $y$-component of acceleration ($\frac{\text{m}}{\text{s}^{2}}$)}
              \item{$ω$ is the angular velocity ($\frac{\text{rad}}{\text{s}}$)}
              \item{${L_{\text{rod}}}$ is the length of rod (${\text{m}}$)}
              \item{$θ$ is the angle of pendulum (${{}^{\circ}}$)}
              \item{$α$ is the angular acceleration ($\frac{\text{rad}}{\text{s}^{2}}$)}
              \end{symbDescription}
\\ \midrule \\
Source & --
         
\\ \midrule \\
RefBy & 
\\ \bottomrule
\end{tabular}
\end{minipage}
\paragraph{Detailed derivation of $y$-component acceleration:}
\label{GD:accelerationIYDeriv}
\begin{displaymath}
{a_{\text{y}}}=ω {L_{\text{rod}}} \cos\left(θ\right)+α {L_{\text{rod}}} \sin\left(θ\right)
\end{displaymath}
\vspace{\baselineskip}
\noindent
\begin{minipage}{\textwidth}
\begin{tabular}{>{\raggedright}p{0.13\textwidth}>{\raggedright\arraybackslash}p{0.82\textwidth}}
\toprule \textbf{Refname} & \textbf{GD:hForceOnPendulum}
\phantomsection 
\label{GD:hForceOnPendulum}
\\ \midrule \\
Label & Horizontal force on the pendulum
        
\\ \midrule \\
Units & ${\text{N}}$
        
\\ \midrule \\
Equation & \begin{displaymath}
           \mathbf{F}=m {a_{\text{x}}}=-\mathbf{T} \sin\left(θ\right)
           \end{displaymath}
\\ \midrule \\
Description & \begin{symbDescription}
              \item{$\mathbf{F}$ is the force (${\text{N}}$)}
              \item{$m$ is the mass (${\text{kg}}$)}
              \item{${a_{\text{x}}}$ is the $x$-component of acceleration ($\frac{\text{m}}{\text{s}^{2}}$)}
              \item{$\mathbf{T}$ is the tension (${\text{N}}$)}
              \item{$θ$ is the angle of pendulum (${{}^{\circ}}$)}
              \end{symbDescription}
\\ \midrule \\
Source & --
         
\\ \midrule \\
RefBy & 
\\ \bottomrule
\end{tabular}
\end{minipage}
\paragraph{Detailed derivation of force pendulum:}
\label{GD:hForceOnPendulumDeriv}
\begin{displaymath}
\mathbf{F}=m {a_{\text{x}}}=-\mathbf{T} \sin\left(θ\right)
\end{displaymath}
\vspace{\baselineskip}
\noindent
\begin{minipage}{\textwidth}
\begin{tabular}{>{\raggedright}p{0.13\textwidth}>{\raggedright\arraybackslash}p{0.82\textwidth}}
\toprule \textbf{Refname} & \textbf{GD:vForceOnPendulum}
\phantomsection 
\label{GD:vForceOnPendulum}
\\ \midrule \\
Label & Vertical force on the pendulum
        
\\ \midrule \\
Units & ${\text{N}}$
        
\\ \midrule \\
Equation & \begin{displaymath}
           \mathbf{F}=m {a_{\text{y}}}=\mathbf{T} \cos\left(θ\right)-m \mathbf{g}
           \end{displaymath}
\\ \midrule \\
Description & \begin{symbDescription}
              \item{$\mathbf{F}$ is the force (${\text{N}}$)}
              \item{$m$ is the mass (${\text{kg}}$)}
              \item{${a_{\text{y}}}$ is the $y$-component of acceleration ($\frac{\text{m}}{\text{s}^{2}}$)}
              \item{$\mathbf{T}$ is the tension (${\text{N}}$)}
              \item{$θ$ is the angle of pendulum (${{}^{\circ}}$)}
              \item{$\mathbf{g}$ is the gravitational acceleration ($\frac{\text{m}}{\text{s}^{2}}$)}
              \end{symbDescription}
\\ \midrule \\
Source & --
         
\\ \midrule \\
RefBy & 
\\ \bottomrule
\end{tabular}
\end{minipage}
\paragraph{Detailed derivation of force pendulum:}
\label{GD:vForceOnPendulumDeriv}
\begin{displaymath}
\mathbf{F}=m {a_{\text{y}}}=\mathbf{T} \cos\left(θ\right)-m \mathbf{g}
\end{displaymath}
\subsubsection{Data Definitions}
\label{Sec:DDs}
This section collects and defines all the data needed to build the instance models.

\vspace{\baselineskip}
\noindent
\begin{minipage}{\textwidth}
\begin{tabular}{>{\raggedright}p{0.13\textwidth}>{\raggedright\arraybackslash}p{0.82\textwidth}}
\toprule \textbf{Refname} & \textbf{DD:positionIX}
\phantomsection 
\label{DD:positionIX}
\\ \midrule \\
Label & $x$-component of initial position
        
\\ \midrule \\
Symbol & ${{p_{\text{x}}}^{\text{i}}}$
         
\\ \midrule \\
Units & ${\text{m}}$
        
\\ \midrule \\
Equation & \begin{displaymath}
           {{p_{\text{x}}}^{\text{i}}}={L_{\text{rod}}} \sin\left(θ\right)
           \end{displaymath}
\\ \midrule \\
Description & \begin{symbDescription}
              \item{${{p_{\text{x}}}^{\text{i}}}$ is the $x$-component of initial position (${\text{m}}$)}
              \item{${L_{\text{rod}}}$ is the length of rod (${\text{m}}$)}
              \item{$θ$ is the angle of pendulum (${{}^{\circ}}$)}
              \end{symbDescription}
\\ \midrule \\
Notes & ${{p_{\text{x}}}^{\text{i}}}$ is the horizontal position
        
        ${{p_{\text{x}}}^{\text{i}}}$ is shown in \hyperref[Figure:Motion]{Fig:Motion}.
        
\\ \midrule \\
Source & --
         
\\ \midrule \\
RefBy & 
\\ \bottomrule
\end{tabular}
\end{minipage}

\vspace{\baselineskip}
\noindent
\begin{minipage}{\textwidth}
\begin{tabular}{>{\raggedright}p{0.13\textwidth}>{\raggedright\arraybackslash}p{0.82\textwidth}}
\toprule \textbf{Refname} & \textbf{DD:positionIY}
\phantomsection 
\label{DD:positionIY}
\\ \midrule \\
Label & $y$-component of initial position
        
\\ \midrule \\
Symbol & ${{p_{\text{y}}}^{\text{i}}}$
         
\\ \midrule \\
Units & ${\text{m}}$
        
\\ \midrule \\
Equation & \begin{displaymath}
           {{p_{\text{y}}}^{\text{i}}}={L_{\text{rod}}} \cos\left(θ\right)
           \end{displaymath}
\\ \midrule \\
Description & \begin{symbDescription}
              \item{${{p_{\text{y}}}^{\text{i}}}$ is the $y$-component of initial position (${\text{m}}$)}
              \item{${L_{\text{rod}}}$ is the length of rod (${\text{m}}$)}
              \item{$θ$ is the angle of pendulum (${{}^{\circ}}$)}
              \end{symbDescription}
\\ \midrule \\
Notes & ${{p_{\text{y}}}^{\text{i}}}$ is the vertical position
        
        ${{p_{\text{y}}}^{\text{i}}}$ is shown in \hyperref[Figure:Motion]{Fig:Motion}.
        
\\ \midrule \\
Source & --
         
\\ \midrule \\
RefBy & 
\\ \bottomrule
\end{tabular}
\end{minipage}

\subsubsection{Instance Models}
\label{Sec:IMs}
This section transforms the problem defined in \hyperref[Sec:ProbDesc]{Section: Problem Description} into one which is expressed in mathematical terms. It uses concrete symbols defined in \hyperref[Sec:DDs]{Section: Data Definitions} to replace the abstract symbols in the models identified in \hyperref[Sec:TMs]{Section: Theoretical Models} and \hyperref[Sec:GDs]{Section: General Definitions}.

\vspace{\baselineskip}
\noindent
\begin{minipage}{\textwidth}
\begin{tabular}{>{\raggedright}p{0.13\textwidth}>{\raggedright\arraybackslash}p{0.82\textwidth}}
\toprule \textbf{Refname} & \textbf{IM:calOfAngularAcceleration}
\phantomsection 
\label{IM:calOfAngularAcceleration}
\\ \midrule \\
Label & Calculation of angular acceleration
        
\\ \midrule \\
Input & ${L_{\text{rod}}}$, $θ$
        
\\ \midrule \\
Output & $α$
         
\\ \midrule \\
Input Constraints & \begin{displaymath}
                    {L_{\text{rod}}}\gt{}0
                    \end{displaymath}
                    \begin{displaymath}
                    θ\gt{}0
                    \end{displaymath}
\\ \midrule \\
Output Constraints & 
\\ \midrule \\
Equation & \begin{displaymath}
           α=-\frac{\mathbf{g}}{{L_{\text{rod}}}} \sin\left(θ\right)
           \end{displaymath}
\\ \midrule \\
Description & \begin{symbDescription}
              \item{$α$ is the angular acceleration ($\frac{\text{rad}}{\text{s}^{2}}$)}
              \item{$\mathbf{g}$ is the gravitational acceleration ($\frac{\text{m}}{\text{s}^{2}}$)}
              \item{${L_{\text{rod}}}$ is the length of rod (${\text{m}}$)}
              \item{$θ$ is the angle of pendulum (${{}^{\circ}}$)}
              \end{symbDescription}
\\ \midrule \\
Notes & The constraint $θ\gt{}0$ is required
        
\\ \midrule \\
Source & --
         
\\ \midrule \\
RefBy & \hyperref[outputValues]{FR: Output-Values} and \hyperref[calcAngPos]{FR: Calculate-Angular-Position-Of-Mass}
        
\\ \bottomrule
\end{tabular}
\end{minipage}
\paragraph{Detailed derivation of angular acceleration:}
\label{IM:calOfAngularAccelerationDeriv}
Using the Newton's Law in \hyperref[TM:NewtonSecLawMot]{TM: NewtonSecLawMot}, we have:

\begin{displaymath}
\mathbf{F}=m \mathbf{a}
\end{displaymath}
Where $\mathbf{F}$ denotes the force, $m$ denotes the mass and $\mathbf{a}$ denotes the acceleration Therefore:

\begin{displaymath}
-m \mathbf{g} \sin\left(θ\right)=m α
\end{displaymath}
Then we have:

\begin{displaymath}
-\mathbf{g} \sin\left(θ\right)=α
\end{displaymath}
the when the rod makes an $θ$ with the vertical, displacement of the mass is given by.

\begin{displaymath}
\mathbf{T} \cos\left(θ\right) \mathbf{\hat{j}}-\mathbf{T} \sin\left(θ\right)-m \mathbf{g} \mathbf{\hat{j}}=m {L_{\text{rod}}} \left(α \cos\left(θ\right) \mathbf{\hat{j}}-ω^{2} \sin\left(θ\right)+α \sin\left(θ\right) \mathbf{\hat{j}}+ω^{2} \cos\left(θ\right) \mathbf{\hat{j}}\right)
\end{displaymath}
With the trig identity $\cos\left(θ\right)+\sin\left(θ\right)=1$:

\begin{displaymath}
\mathbf{T} \cos\left(θ\right) \mathbf{\hat{j}}-\mathbf{T} \sin\left(θ\right)-m \mathbf{g} \mathbf{\hat{j}}=m {L_{\text{rod}}} \left(α \cos\left(θ\right) \mathbf{\hat{j}}-ω^{2} \sin\left(θ\right)+α \sin\left(θ\right) \mathbf{\hat{j}}+ω^{2} \cos\left(θ\right) \mathbf{\hat{j}}\right)
\end{displaymath}
\subsubsection{Data Constraints}
\label{Sec:DataConstraints}
\hyperref[Table:InDataConstraints]{Table:InDataConstraints} shows the data constraints on the input variables. The column for physical constraints gives the physical limitations on the range of values that can be taken by the variable. The uncertainty column provides an estimate of the confidence with which the physical quantities can be measured. This information would be part of the input if one were performing an uncertainty quantification exercise. The constraints are conservative, to give the user of the model the flexibility to experiment with unusual situations. The column of typical values is intended to provide a feel for a common scenario.

\begin{longtable}{l l l l}
\toprule
\textbf{Var} & \textbf{Physical Constraints} & \textbf{Typical Value} & \textbf{Uncert.}
\\
\midrule
\endhead
$\mathbf{g}$ & $\mathbf{g}\gt{}0$ & $9.8$ $\frac{\text{m}}{\text{s}^{2}}$ & 10$\%$
\\
${L_{\text{rod}}}$ & ${L_{\text{rod}}}\gt{}0$ & $44.2$ ${\text{m}}$ & 10$\%$
\\
$m$ & $m\gt{}0$ & $56.2$ ${\text{kg}}$ & 10$\%$
\\
$θ$ & $θ\gt{}0$ & $2.1$ ${{}^{\circ}}$ & 10$\%$
\\
\bottomrule
\caption{Input Data Constraints}
\label{Table:InDataConstraints}
\end{longtable}
\subsubsection{Properties of a Correct Solution}
\label{Sec:CorSolProps}
\hyperref[Table:OutDataConstraints]{Table:OutDataConstraints} shows the data constraints on the output variables. The column for physical constraints gives the physical limitations on the range of values that can be taken by the variable.

\begin{longtable}{l l}
\toprule
\textbf{Var} & \textbf{Physical Constraints}
\\
\midrule
\endhead
$α$ & $α\gt{}0$
\\
\bottomrule
\caption{Output Data Constraints}
\label{Table:OutDataConstraints}
\end{longtable}
\section{Requirements}
\label{Sec:Requirements}
This section provides the functional requirements, the tasks and behaviours that the software is expected to complete, and the non-functional requirements, the qualities that the software is expected to exhibit.

\subsection{Functional Requirements}
\label{Sec:FRs}
This section provides the functional requirements, the tasks and behaviours that the software is expected to complete.

\begin{itemize}
\item[Input-Values:\phantomsection\label{inputValues}]{Input the values from \hyperref[Table:ReqInputs]{Table:ReqInputs}.}
\item[Verify-Input-Values:\phantomsection\label{verifyInptVals}]{Check the entered input values to ensure that they do not exceed the data constraints mentioned in \hyperref[Sec:DataConstraints]{Section: Data Constraints}. If any of the input values are out of bounds, an error message is displayed and the calculations stop.}
\item[Calculate-Angular-Position-Of-Mass:\phantomsection\label{calcAngPos}]{Calculate the following values: $α$ (from \hyperref[IM:calOfAngularAcceleration]{IM: calOfAngularAcceleration}) and $θ$ (from \hyperref[IM:calOfAngularAcceleration]{IM: calOfAngularAcceleration}).}
\item[Output-Values:\phantomsection\label{outputValues}]{Output ${L_{\text{rod}}}$ (from \hyperref[IM:calOfAngularAcceleration]{IM: calOfAngularAcceleration}) and ${L_{\text{rod}}}$ (from \hyperref[IM:calOfAngularAcceleration]{IM: calOfAngularAcceleration}).}
\end{itemize}
\begin{longtable}{l l l}
\toprule
\textbf{Symbol} & \textbf{Description} & \textbf{Units}
\\
\midrule
\endhead
$\mathbf{F}$ & Force & ${\text{N}}$
\\
${L_{\text{rod}}}$ & Length of rod & ${\text{m}}$
\\
\bottomrule
\caption{Required Inputs following \hyperref[inputValues]{FR: Input-Values}}
\label{Table:ReqInputs}
\end{longtable}
\subsection{Non-Functional Requirements}
\label{Sec:NFRs}
This section provides the non-functional requirements, the qualities that the software is expected to exhibit.

\begin{itemize}
\end{itemize}
\section{Values of Auxiliary Constants}
\label{Sec:AuxConstants}
There are no auxiliary constants.

\section{References}
\label{Sec:References}
\begin{filecontents*}{bibfile.bib}
\end{filecontents*}
\nocite{*}
\bibstyle{ieeetr}
\printbibliography[heading=none]
\end{document}
