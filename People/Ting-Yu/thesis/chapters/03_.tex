\chapter{Projectile Lesson} \label{chap:casestudy}
With the addition of a JSON printer capable of generating Jupyter Notebooks, we 
are now looking to expand Drasil's application by generating educational 
documents. As discussed in Chapter~\ref{chap:intro}, Jupyter Notebooks are 
commonly used in teaching engineering courses due to their characteristics and 
advantages. One of the educational practices to enhance education is conducting 
lesson plans \cite{cicek2013effective, wong2018first}, which provide a 
guide for structuring daily activities in each class period. A lesson plan 
outlines the learning objectives, methods and procedures for achieving them, 
and the measurement of how student progress. Because they are more accessible 
than academic papers, lesson plans are an ideal starting point for generating 
educational documents in Drasil. Also, we are able to work with real examples 
in a lesson plan. This chapter will cover the structure of a lesson plan, how 
we define the language of lesson plans in Drasil, and a new case study on 
Projectile Lesson.

\section{Language of Lesson Plans}
To generate a new type of document, lesson plans, in Drasil, we must define its 
language first. Drasil's document language has SRS, and we are creating a 
language for lesson plans. As discussed in Chapter~\ref{chap:nbprinter}, a 
Drasil document has a title, authors, and sections, which hold the contents 
of the document. The definition of a document is defined in 
\textbf{drasil-lang} as 
shown in Code~\ref{code:drasil-lang-document} \footnote{ShowToC is 
ShowTableOfContents in the source code, which is to determine whether to show 
the table of contents in the document.}, where \macblue{Document} is the type 
for SRS document and \macblue{Notebook} is for Jupyter Notebook, specifically 
lesson plans at this moment. The reason why we define them separately is 
because we print the SRS and lesson plans differently. We are able to pattern 
match the way we print the document in the printer. 

\begin{listing}[h]
	\caption{Pseudocode for Definition of Document}
	\label{code:drasil-lang-document}
	\begin{lstlisting}[language=haskell1]
	data Document = Document Title Author ShowToC [Section]
								| Notebook Title Author [Section]
	\end{lstlisting}
\end{listing}


Before defining the language for lesson plans, we need to understand the 
components and categorize the knowledge to create a universal structure within 
Drasil. We analyzed the similarities and differences of elements in textbook 
chapters in 
\href{https://github.com/smiths/caseStudies/blob/master/CaseStudies/projectile/projectileLesson/AboutProjectileLesson.pdf}{Discussion
 of Projectile Lesson: What and	Why} using online resources. Based on our 
analysis, we narrowed down the elements and defined a structure that fits our 
lesson plans the most. It's worth noting that this structure may be subject to 
future modifications to better suit our needs. Following is the structure of
our lesson plans:
\begin{itemize}
	\item Introduction: an introduction of the lesson plan or the topic.
	\item Learning Objectives: what students can do or will learn after the 
	lesson.  
	\item Review: a recap of what has been covered previously.
	\item A Case Problem: a case problem that link the topic to a real world 
	problem.
	\item Example: an example of the case problem.
	\item Summary: a summary of the lesson plan.
	\item Bibliography: references that support the lesson plan.
	\item Appendix: additional resources or information of the lesson.
\end{itemize}

With the lesson plan structure in place, we can now define helper types and 
functions to create the document language for generating lesson plans. Our 
first step is to define the types and data for the lesson and its chapters. 
Code~\ref{code:core} is the core language of the lesson plan. A 
\macblue{LsnDesc} type represents a lesson description (line 3), which consists 
of lesson chapters (line 5), including an introduction, learning objectives, 
review, case problem, example, summary, bibliography, and appendix. The details 
of each chapter are defined in line 14-33. At present,  the contents of each 
chapter are the only defined elements as the chapter structure has not yet been 
fully understood. We intend to further develop the chapter structure in the 
future. \todo{link} 

\begin{listing}[h!]
	\caption{Source Code for Notebook Core Language}
	\label{code:core}
	\begin{lstlisting}[language=haskell1]
		module Drasil.DocumentLanguage.Notebook.Core where
		
		type LsnDesc = [LsnChapter]

		data LsnChapter = Intro Intro
										| LearnObj LearnObj
										| Review Review
										| CaseProb CaseProb
										| Example Example
										| Smmry Smmry
										| BibSec
										| Apndx Apndx
		
		-- ** Introduction
		newtype Intro = IntrodProg [Contents]
		
		-- ** Learning Objectives
		newtype LearnObj = LrnObjProg [Contents]
		
		-- ** Review Chapter
		newtype Review = ReviewProg [Contents]
		
		-- ** A Case Problem
		newtype CaseProb = CaseProbProg [Contents]
		
		-- ** Examples of the lesson
		newtype Example = ExampleProg [Contents]
		
		-- ** Summary
		newtype Smmry = SmmryProg [Contents]
		
		-- ** Appendix
		newtype Apndx = ApndxProg [Contents]
	\end{lstlisting}
\end{listing}

Code~\ref{code:LsnDecl} shows the function, \macred{mkLsnDesc}, for creating 
the lesson description. \macblue{LsnDecl} is the lesson plan declaration that 
made up of all necessary chapters, while \macblue{LsnDesc} is 
the type we defined in Code~\ref{code:core} line 3, which is a more usable form 
than \macblue{LsnDecl} for generating documents.

\begin{listing}[h]
	\caption{Source Code for mkLsnDesc}
	\label{code:LsnDecl}
	\begin{lstlisting}[language=haskell1]
		module Drasil.DocumentLanguage.Notebook.LsnDecl where
		
		type LsnDecl  = [LsnChapter]
		
		data LsnChapter = Intro NB.Intro
										| LearnObj NB.LearnObj
										| Review NB.Review
										| CaseProb NB.CaseProb
										| Example NB.Example
										| Smmry NB.Smmry
										| BibSec
										| Apndx NB.Apndx
		
		mkLsnDesc :: SystemInformation -> LsnDecl -> NB.LsnDesc
		mkLsnDesc _ = map sec where
			sec :: LsnChapter -> NB.LsnChapter
			sec (Intro i)     = NB.Intro i
			sec (LearnObj l)  = NB.LearnObj l
			sec (Review r)    = NB.Review r  
			sec (CaseProb c)  = NB.CaseProb c
			sec (Example e)   = NB.Example e  
			sec (Smmry s)     = NB.Smmry s
			sec BibSec        = NB.BibSec
			sec (Apndx a)     = NB.Apndx a
	\end{lstlisting}
\end{listing}

The core language of the notebook is \macblue{LsnDesc} and 
\macblue{Lsnchapter} since a notebook is created from a lesson description, 
which is composed of lesson chapters. We take the lesson description and system 
information to form a notebook (Code~).

\begin{listing}[h]
	\caption{Source Code for mkNb}
	\label{code:mkNb}
	\begin{lstlisting}[language=haskell1]
		mkNb :: LsnDecl -> (IdeaDict -> IdeaDict -> Sentence) 
					-> SystemInformation -> Document
		mkNb dd comb si@SI {_sys = sys, _kind = kind, _authors = authors} =
		Notebook (nw kind `comb` nw sys) (foldlList Comma List $ map (S . name) 
		authors) $
		mkSections si l where
		l = mkLsnDesc si dd
	\end{lstlisting}
\end{listing}


all fuctions are located in \textbf{drasil-docLang}, a table overview
projectile overview: the reason we choose, introducstion
jupyter notebook structure (linear)
unit of contents -> linear vs nested

how we defined document structure in Drasil (kind of walk through the step? The 
process that the language was built)

