% -*- coding: utf-8; -*-
% vim: set fileencoding=utf-8 :
\documentclass[english,submission]{programming}
%% First parameter: the language is 'english'.
%% Second parameter: use 'submission' for initial submission, remove it for camera-ready (see 5.1)

\usepackage[backend=biber]{biblatex}
\addbibresource{../Bib/drasil.bib}
\addbibresource{../Bib/codegen.bib}

\usepackage{listings}  % for making code pretty

%%%%%%%%%%%%%%%%%%
%% These data MUST be filled for your submission. (see 5.3)
\paperdetails{
  %% perspective options are: art, sciencetheoretical, scienceempirical, engineering.
  %% Choose exactly the one that best describes this work. (see 2.1)
  perspective=art,
  %% State one or more areas, separated by a comma. (see 2.2)
  %% Please see list of areas in http://programming-journal.org/cfp/
  %% The list is open-ended, so use other areas if yours is/are not listed.
  area={Modeling and Modularity, Generative Programming},
  %% You may choose the license for your paper (see 3.)
  %% License options include: cc-by (default), cc-by-nc
  license=cc-by
}
%%%%%%%%%%%%%%%%%%

%%%%%%%%%%%%%%%%%%
%% These data are provided by the editors. May be left out on submission.
%\paperdetails{
%  submitted=2016-08-10,
%  published=2016-10-11,
%  year=2016,
%  volume=1,
%  issue=1,
%  articlenumber=1,
%}
%%%%%%%%%%%%%%%%%%


\begin{document}

\title{Generating code fit for humans} % Working title
% \subtitle{}
\titlerunning{Generating code fit for humans}

\author[a]{Jacques Carette}
\author[a]{Spencer Smith}
\affiliation[a]{Department of Computing and Software, McMaster University, Canada}

\keywords{Code generation, documentation, layout, knowledge capture}

%%%%%%%%%%%%%%%%%%%%%%%%%%%%%
% Please go to https://dl.acm.org/ccs/ccs.cfm and generate your Classification
% System [view CCS TeX Code] stanz and copy _all of it_ to this place.
%% From HERE
% To HERE
%%%%%%%%%%%%%%%%%%%%%%%

\maketitle

% Please always include the abstract.
% The abstract MUST be written according to the directives stated in 
% http://programming-journal.org/submission/
% Failure to adhere to the abstract directives may result in the paper
% being returned to the authors.
\begin{abstract}
  Context:

  Inquiry:

  Approach:

  Knowledge:

  Grounding:

  Importance:

\end{abstract}

\section{Introduction}
\label{sec:intro}

It is an instructive exercise to take a piece of well-written, well-documented
piece of software and \emph{carefully explain} the provenance and justification
for all the parts -- not just the code, but also the documentation, the tests, the
build scripts, the README, the requirements and design document, and so on.
One of the striking things of such an explanation is how much repetition it contains.
Post-facto, this isn't surprising: all these ``pieces'' belong to a single software
package that accomplishes a particular task.

What is more surprising is that if such an exercise is conducted multiple times
on software from different domains, how much of the ``skeleton'' of the explanation
remains the same. Again, this is because some parts don't change: what artifacts
are present and what ``knowledge'' is encoded by each is shared by all these different
pieces of software. What changes are the domain-specific aspects.

Another surprise is just how succinct the (careful!) explanation sometimes is, at
least when this explanation is for people who are both domain experts and computing
professionals. This reflects the idea that there is an accepted, concise, body of
knowledge being re-used, and the explanation is largely how to weave and expand that
knowledge from its human-encoded version to a version suitable for machine-oriented
processing.

Thus we thought: what if we pick a particularly well-understood component (such as
well-documented code), can we turn this explanatory process into a description
language for documented-code assembly, assuming we already had a back-end for
multi-language code generation?

As we're in a generative setting, we often solve such problems by creating a
\emph{Domain Specific Language} (DSL), or indeed multiple DSLs, that let us
capture the fundamental \emph{Domain Knowledge} (DK) embodied in the problem domains.
We can then write interpreters and/or transpilers that translate from these DSLs
into forms usually hand-written by humans.

We structure the papers as a rational reconstruction of an idealized discovery
process: in Section~\ref{sec:problem} we give increasingly detailed illustrations
of the problem that needs to be solved. The examples are chosen to succinctly
illustrate the principal design points that will occur later. Section~\ref{sec:analysis}
analyses the examples to draw out the Domain Knowledge that needs to be encoded.
Section~\ref{sec:organize} shows out the DK can be organized, leading to the design
of various DSLs. Actually generating the artifacts is done by what we call
\emph{recipes}\footnote{but should perhaps just call programs}, and these are documented
in Section~\ref{sec:recipes}. The amount of code supporting our work is surprisingly
compact; in Section~\ref{sec:tech} we describe the various technologies we've used
to achieve that. People familiar with this domain will recognize our goals as being
far from new, as there was extensive work in the $1980$s and $90$s with similar aim,
which we document in Section~\ref{sec:relwork}, along with more recent relevant work
as well as various work that have inspired us. Finally we conclude in Section~\ref{sec:conc},
in part commenting on why we think resurrecting these ``old fashioned'' ideas is
worthwhile.

%%
% Spencer commented on the introduction as follows:
% I’m not sure I like the central conceit of the introduction - the idea that the
% reader should imagine carefully explaining the provenance and justification for
% all of the software artifacts.  Will the reader really be able to imagine that?
% This isn’t something that people commonly do.  I’m not sure we even do this.
% Since we don’t have rationale information we don’t really have justification
% information, especially not in our code.  I would think provenance information
% is traceability to the original source of the part.  We have traceability
% between the parts, but other than saying the author, we don’t have a mechanism
% for capturing the source of the code.  What if a requirement comes from a
% statement by a stakeholder, we don’t have a means to credit that stakeholder
% with that statement.  What if the code comes from another similar project; we
% don’t have a means to express that.  
%
% Jacques' follow-up comments:
% - that intro is 'top down' - maybe bottom-up would be better (Spencer suggests
%     the top down view could go in the conclusion)
% - the top down view is meant to capture the kinds of things that a structured
%     code review might capture, not more than that.
\section{The problem}
\label{sec:problem}

Subsection of intro? But give examples, well illustrated. See outline.txt.

% - give examples of increasingly fit-for-humans code
%   - first example:
%     - multiple versions of code:
%       - no white-space, obscure identifiers, non-idiomatic, no comment
%       - + laid out
%       - + idiomatic (GOOL get us to here)
%       - + real identifiers
%       - + proper documentation
%     - probably give all of the above in one big figure
%     - accompanying text will give hints as to what's needed to achieve each bit
%   - second example:
%     - in multiple languages (probably without documentation, for space reasons? have that online?)
%     - maybe use HGHC here?
%   - third example:
%     - with design choices changed (again, maybe no comments)

\section{Analysis}
\label{sec:analysis}

Analyse the actual content of the examples. See what makes them tick.

\section{Organize}
\label{sec:organize}

The content previously outlined is not a jumble.

\section{Recipes}
\label{sec:recipes}

The programs that weave together things.

\section{Technology}
\label{sec:tech}

GOOL, pretty-printing. Probably also tagless and classy lenses.

\section{Related Work}
\label{sec:relwork}

See the Drasil Wiki page for lots.

\section{Conclusion}
\label{sec:conc}

\acks
Our zillions of students who built large parts of the software.

%\appendix
%\section{A Famous Filler Text}

\begin{comment}
\section{Submission Checklist}
\smaller
\begin{itemize}
\renewcommand*\labelitemi{\ensuremath{\square}}
\item All authors are listed, with affiliation
\item The abstract is not longer than 500 words.
\item The abstract states, in this order, \emph{Context},
  \emph{Inquiry}, \emph{Approach}, \emph{Knowledge}, \emph{Grounding}, and
  \emph{Importance} of the submission.
\item The ACM CCS 2012 classification is included.
\item A list of relevant keywords is provided.
\item One \emph{perspective} has been chosen.
\item One or more \emph{area(s) of submission} have been chosen.
\item The license has been acknowledged or changed to fit.
\item The title page does not spill onto the second page.
\item The running title on even pages' heads is not wider than the text block.
\item The running list of author on odd pages' heads is not wider than the text
  block.
\item The only text encoding used is \textsc{utf-8}.
\item The fonts, margins, and spacings are unchanged.
\item All \textsc{url}s are marked using the \lstinline|\url| command.
\item All units are separated from their values with a half space, consistently
  (see ~\cite{siunitx}).
\item Bold font is not used (or only in exceptional circumstances).
\item Underlining is not used (no exceptions).
\item Bold or bright colors are only ever used sparingly.
\item The captions of (a) tables, (b) listings, and (c) algorithms are
  \emph{above} the content, flush left.
\item The captions of (a) figures, (b) charts, (c) combined content, and
  similar are \emph{below} the content, flush left.
\item %
  \begin{minipage}[t]{.4\linewidth}
    Tables adhere to best practice layout~\cite{booktabs}. They do not contain
  vertical lines and only few horizontal lines.
  \end{minipage}
  {\renewcommand{\arraystretch}{.6}
    \hspace{2em} \textcolor{DarkGreen}{Yes} \hspace{1em}
      \begin{tabular}[t]{ll}\noalign{\vspace*{-2.2ex}}
        \toprule
        \emph{a} & \emph{b}\\
        \midrule
        x & abc \\
        y & def \\
        \bottomrule
      \end{tabular}
      \hspace{2em} \textcolor{DarkRed}{No} \hspace{1em}
      \begin{tabular}[t]{|l|l|}\noalign{\vspace*{-1.6ex}}
        \hline
        \emph{a} & \emph{b}\\
        \hline\hline
        x & abc \\\hline
        y & def \\\hline
      \end{tabular}}
\item All fonts are embedded. This includes the fonts for any included
  (\textsc{pdf}-)graphics.
\item All graphics are either vector graphics or have at least 300\,dpi.
  \item Acknowledgements, including statement of funding bodies or projects, is
  placed before the references, using the \lstinline|\acks| command.
\item Citations in the text are preceded by a non-breaking space.
  (\(\rightarrow\) Citations do not appear at the beginning of a line)
\item The bibliography lists all cited works, all authors are given (no
  \emph{et. al.}), the full names of the authors are used.
\item All bibliography entries have a \textsc{doi} (unless not applicable, such as for
  slides or unpublished work).
\item All bibliography entries show an \textsc{isbn}, if they have one assigned.
\item All bibliography entries that represent web sites have a \textsc{url} and
  an \emph{access date}.
\item No bibliography entries have duplicate information in \textsc{url} and
  \textsc{doi}.
\end{itemize}
\end{comment}
% 
\printbibliography

\end{document}

% Local Variables:
% TeX-engine: pdflatex
% End:
