\prefacesection{Abstract}

Scientific Computing (SC) involves analyzing and simulating complex scientific 
and engineering problems using computing techniques and tools. To improve the 
understandability, maintainability, and reproducibility of SC software,  
documentation should be an integral part of the development process. Jupyter 
Notebook is a popular tool for developing SC software documentation, and is 
also used to enhance teaching and learning efficiency in engineering education 
due to its flexibility and benefits, such as the ability to combine text and 
code. Despite the importance of documentation, it is often missing or poorly 
executed in SC software because it is time-consuming. 

Drasil is a framework that aims to improve the efficiency of documentation 
development. By encoding each piece of information for scientific problems once 
and generating the document automatically, Drasil saves time in the 
documentation development process. We are interested in generating Jupyter 
Notebooks in Drasil to expand its applications, including generating 
educational documents.

To achieve this, we implement a JSON printer capable of generating Drasil 
software artifacts, such as Software Requirement Specifications (SRS), in 
notebook format. This enables us to generate Jupyter Notebooks in Drasil, and 
generate educational documents, starting with lesson plans. We develop the 
structure of our lesson plans and designed the language of lesson plans in 
Drasil. Additionally, Jupyter Notebooks seamlessly integrate different content 
types with code, making them ideal for data research. We explore two different 
approaches for splitting the contents. These approaches involve splitting 
content either by sections or by content types. The goal of these approaches is 
to effectively combine text and code of our Drasil-generated Jupyter Notebooks.